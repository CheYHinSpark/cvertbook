\documentclass{ctexbook}
\XeTeXinputencoding "utf8"

\usepackage{zhspacing}
\zhspacing

% 字体设置
\def\myfeat{vertical:+vert:+vhal}
\newfontfamily\song[RawFeature={\myfeat}]{Songti TC}
\newfontfamily\kai[RawFeature={\myfeat}]{Kaiti TC}
\let\mainfont\song
\let\gezhufont\kai

\let\zhfont\mainfont
%\let\zhpunctfont\song

% 行间距
\linespread{1.3}

% TODO: 页码设置
%\usepackage{xCJKnumb}
%\usepackage{fancyhdr}
%\pagestyle{fancyplain}
%\fancyhf{}
%\renewcommand{\headrulewidth}{0pt}
%\def\myrightmark{}
%\fancyhead[L]{\myrightmark}
%\fancyhead[R]{\xCJKdigits{\thepage}}

% 横向页面
\usepackage{everypage}
\AddEverypageHook{\special{pdf: put @thispage <</Rotate 90>>}}

% 页面布局
\usepackage{geometry}
\geometry{left=3cm,right=3cm,top=5cm,bottom=4cm}

% 割注设置
\usepackage{gezhu}
\setgezhulines{2}	%割注行数
\everygezhu{\linespread{1}\let\zhfont\gezhufont\fontsize{18}{18}\selectfont}
\setgezhuraise{-9pt}	%割注浮动

% 自定义命令
\def\myauthor#1{\endgraf\setgezhulines{1}\gezhu{#1}\endgraf\setgezhulines{2}}

\begin{document}

% 正文字号
\fontsize{36}{36}
\selectfont

% 首行缩进
\parindent=0pt

\begin{withgezhu}
般若波羅蜜多心經
\myauthor{唐三藏法師玄奘譯}
\myauthor{明憨山大師述}

\gezhu{此經題稱般若者何乃梵語也此云智慧稱波羅蜜多者何亦梵語也此云到彼岸謂生死苦趣猶如大海而眾生情想無涯無明不覺識浪奔騰起惑造業流轉生死苦果無窮不能得度故云此岸惟吾佛以大智慧光明照破情塵煩惱永斷諸苦皆盡二死永亡直超苦海高證涅槃故云彼岸所言心者正是大智慧到彼岸之心殆非世人肉團妄想之心也良由世人不知本有智慧光明之心但認妄想攀緣影子而以依附血肉之團者為真心所以執此血肉之軀以為我有故依之造作種種惡業念念流浪曾無一念回光返照而自覺者日積月累從生至死從死至生無非是業無非是苦何由得度惟吾佛聖人能自覺本真智慧照破五蘊身心本來不有當體全空故頓超彼岸直渡苦海因愍迷者而復以此自證法門而開導之欲使人人皆自覺悟智慧本有妄想元虛身心皆空世界如化不造眾惡遠離生死咸出苦海至涅槃樂故說此經經即聖人之言教所謂終古之常法也}

\newpage

觀自在菩薩 行深般若波羅蜜多時 照見五蘊皆空 度一切苦厄
\gezhu{菩薩即能修之人甚深般若即所修之法照見五蘊皆空則修之之方度一切苦厄則修之實效也以此菩薩從佛聞此甚深般若即思而修之以智慧觀返照五蘊內外一空身心世界洞然無物忽然超越世出世間永離諸苦得大自在由是觀之菩薩既能以此得度足知人人皆可依之而修矣是故世尊特告尊者以示觀音之妙行欲曉諸人人也吾人苟能作如是觀若一念頓悟自心本有智慧光明如此廣大靈通徹照五蘊元空四大非有有何苦而不度又何業累之牽纏人我是非之強辯窮通得失之較計富貴貧賤之可嬰心者哉此上乃菩薩學般若之實效也言五蘊者即色受想行識耳然照乃能觀之智五蘊即所觀之境皆空則實效也}
舍利子
\gezhu{此佛弟子之名也然舍利亦梵語此云鶖也此鳥目最明利其母目如之故以為名此尊者乃鶖之子也故云舍利子在佛弟子中居智慧第一而此般若法門最為甚深非大智慧者不能領悟故特告之所謂可與智者道也}
色不異空 空不異色 色即是空 空即是色 受想行識 亦復如是
\gezhu{此正對鶖子釋前五蘊皆空之意而五蘊中先舉色蘊而言者色乃人之身相也以其此身人人執之以為己有乃堅固妄想之所凝結所謂我執之根本最為難破者今入觀之初先觀此身四大假合本來不有當體全空內外洞然不為此身之所籠罩則生死去來了無罣礙名色蘊破色蘊若破則彼四蘊可漸次深觀例此而推矣而言色不異空者此句破凡夫之常見也良由凡夫但認色身執為真實將謂是常而作千秋百歲之計殊不知此身虛假不實為生老病死四相所遷念念不停以至老死畢竟無常終歸於空此猶屬生滅之空尚未盡理良以四大幻色元不異於真空耳凡夫不知故曉之曰色不異空謂色身本不異於真空也空不異色者此句破外道二乘斷滅之見也因外道修行不知身從業生業從心生三世循環輪轉不息由不達三世因果報應之理乃謂人死之後清氣歸天濁氣歸地一靈真性還乎太虛苟如此說則絕無報應之理而作善者為徒勞作惡者為得計矣以性歸太虛則善惡無徵幾於淪滅豈不幸哉孔子言曰游魂為變故知鬼神之情狀此正謂死而不亡者乃輪迴報應之理昭然也而世人不察橫為斷滅謬之甚耳然二乘雖依佛教而修由不達三界唯心萬法唯識不了生死如幻如化將謂三界之相以為實有故觀三界如牢獄厭四生如桎梏不起一念度生之心沈空滯寂淪於寂滅故曉之曰空不異色謂真空本不異於幻色非是離色斷滅之空正顯般若乃實相真空耳何也以般若真空如大圓鏡一切幻色如鏡中像苟知像不離鏡則知空不異色矣此正破二乘離色斷滅之空及外道豁達之空也又恐世人將色空二字話為兩橛不能平等一如而觀故又和會之曰色即是空空即是色耳苟如此觀知色不異空則無聲色貨利可貪亦無五欲塵勞可戀此則頓度凡夫之苦也苟知空不異色則不起滅定而現諸威儀不動本際而作度生事業居空而萬行沸騰涉有而一道清淨此則頓超外道二乘之執也苟知色空平等一如則念念度生不見生之可度心心求佛不見佛果可求所謂圓成一心無智無得此則超越菩薩而頓登佛地彼岸者也即此色蘊一法能作如是觀則其四蘊應念圓明正如一根既返源六根成解脫故云受想行識亦復如是也誠能如是則諸苦頓斷佛果可至彼岸非遙只在當人一念觀心成就耳如此之法豈非甚深者哉}
舍利子 是諸法空相 不生不滅 不垢不淨 不增不減
\gezhu{此又恐世人以生滅心錯認真空實相般若之法而作生滅垢淨增減之解故召尊者以曉之曰所言真空之實相者不是生滅垢淨增減之法也且生滅垢淨增減者乃眾生情見之法耳而我般若真空實相之體湛然清淨猶若虛空乃出情之法也豈然之哉故以不字不之謂五蘊諸法即是真空實相一一皆離此諸過也}
是故空中無色 無受想行識 無眼耳鼻舌身意 無色聲香味觸法
無眼界 乃至無意識界 無無明 亦無無明盡 乃至無老死 亦無老死盡 無苦集滅道 無智亦無得
\gezhu{此乃通釋般若所以離過之意謂般若真空所以永離諸過者以此中清淨無物故無五蘊之跡不但無五蘊亦無六根不但無六根亦無六塵不但無六塵亦無六識斯則根塵識界皆凡夫法般若真空總皆離之故都云無此則離凡夫法也然般若中不但無凡夫法亦無聖人法以四諦十二因緣六度等皆出世三乘聖人之法也苦集滅道四諦以厭苦斷集慕滅修道乃聲聞法也無明緣行行緣識識緣名色名色緣六入六入緣觸觸緣受受緣愛愛緣取取緣有有緣生生緣老死乃十二因緣流轉門即苦集二諦無明盡至老死盡乃還滅門即滅道二諦此緣覺所觀法也般若體中本皆無之極而推之不但無二乘法亦無菩薩法何也智即觀智乃六度之智慧能求之心得即佛果乃所求之境然菩薩修行以智為首下化眾生只為上求佛果良以佛境如空無所依若以有所得心而求之皆非真也以般若真空體中本無此事故曰無智亦無得無得乃真得方得為究竟耳}
以無所得故 菩提薩埵 依般若波羅蜜多故 心無罣礙 無罣礙故 無有恐怖 遠離顛倒夢想 究竟涅槃
\gezhu{良由佛果以無得而得故菩薩修行依般若而觀然一切諸法本皆空寂若依情想分別而觀則心境纏喩不能解脫處處貪著皆是罣礙若依般若真智而觀則心境皆空觸處洞然無非解脫故云依此般若故心無罣礙由心無罣礙則無生死可怖故云無有恐怖既無生死可怖則亦無佛果可求以怖生死求涅槃皆夢想顛倒之事耳圓覺云生死涅槃猶如昨夢然非般若圓觀決不能離此顛倒夢想之相既不能離顛倒夢想決不能究竟涅槃然涅槃亦梵語此云寂滅又云圓寂謂圓除五住寂滅永安乃佛所歸之極果也意謂能離聖凡之情者方能證入涅槃耳菩薩修行捨此決非真修也}
三世諸佛 依般若波羅蜜多故 得阿耨多羅三藐三菩提 故知般若波羅蜜多 是大神咒 是大明咒 是無上咒 是無等等咒 能除一切苦 真實不虛
\gezhu{謂不但菩薩依此般若而修即三世諸佛莫不皆依此般若得成無上正等正覺之果故云三世諸佛依般若波羅蜜多故得阿耨多羅三藐三菩提此梵語也阿云無耨多羅云上三云正藐云等菩提云覺乃佛果之極稱也由此而觀故知般若波羅蜜多能驅生死煩惱之魔故云是大神咒能破生死長夜癡暗故云是大明咒世出世間無有一法過般若者故云是無上咒般若為諸佛母出生一切無量功德故世出世間無物與等惟此能等一切故云是無等等咒所言咒者非別有咒即此般若便是然既曰般若而又名咒者何也極言神效之速耳如軍中之密令能默然奉行者無不決勝般若能破生死魔軍決勝如此又如甘露飲之者能不死而般若有味之者則頓除生死大患故云能除一切苦而言真實不虛者以示佛語不妄欲人諦信不疑決定修行為要也}
故說般若波羅蜜多咒 即說咒曰
\gezhu{由其般若實有除苦得樂之功所以即說密咒使人默持以取速效耳}
揭諦揭諦 波羅揭諦 波羅僧揭諦 菩提薩婆訶
\gezhu{此梵語也前文為顯說般若此咒為密說般若不容意解但直默誦其收功之速正在忘情絕解不思議之力耳然此般若所以收功之速者乃人人本有之心光諸佛證之以為神通妙用眾生迷之以作妄想塵勞所以日用而不知自昧本真枉受辛苦可不哀哉苟能頓悟本有當下迴光返照一念熏修則生死情關忽然隳裂正如千年暗室一燈能破更不別求方便耳吾人有志出生死者舍此決無舟筏矣所謂滔滔苦海中般若為舟航冥冥長夜中般若為燈燭今夫人者驅馳險道泛濫苦海甘心而不求此者吾不知其所歸矣雖然般若如宵練遇物即斷物斷而不自知非神聖者不能用況小丈夫哉}
\end{withgezhu}
\end{document}

