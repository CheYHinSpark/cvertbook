\documentclass{article}
\usepackage{CJK, CJKpunct}
\usepackage{CJKulem}
\usepackage{gezhu}
\makeatletter
%\gezhu@lines=3
\let\gezhu@debugbox\gezhu@fbox
\begin{document}
\begin{CJK*}{GBK}{song}


\makeatletter
\gezhu@makespecials
\loop
\noindent\hrulefill\par
\the\hsize\par
\begin{withgezhu}
	\parskip=30pt

世祖光武皇帝讳秀,字文叔,|{礼“{祖有功而宗有德}”,光武中兴,故庙称世祖。谥法
:“能绍前业曰光,克定祸乱曰武。”伏侯古今注曰:“秀之字曰茂。伯、仲、叔、季
,兄弟之次。长兄伯升,次仲,故字文叔焉。”}南阳蔡阳人,|{南阳,郡,今邓州县
也。蔡阳,县,故城在今随州枣阳县西南。}高祖九世之孙也,出自景帝生长沙定王发
。|{长沙,郡,今潭州县也。}发生舂陵节侯买,|{舂陵,乡名,本属零陵泠道县,在
今永州唐兴县北,元帝时徙南阳,仍号舂陵,故城在今随州枣阳县东。事具宗室四王传
。}买生郁林太守外,|{郁林,郡,今贵州县。前书曰:“郡守,秦官。秩二千石。景
帝更名太守。”}外生钜鹿都尉回,|{钜鹿,郡,今邢州县也。前书曰:“都尉,本{郡
尉},秦官也。掌佐守,典武职,秩比二千石。景帝更名都尉。”}回生南顿令钦,|{南
顿,县,属汝南郡,故城在今陈州项城县西。前书曰:“令、长,皆秦官也。万户以上
为令,秩千石至六百石;不满万户为长,秩五百石至三百石。”}钦生光武。光武年九
岁而孤,养于叔父良。

身长七尺三寸,美须眉,大口,隆准,日角。|{隆,高也。许负云:“鼻头为准。”郑
玄尚书中候注云:“日角谓庭中骨起,状如日。”}性勤于稼穑,|{种曰稼,敛曰穑。}
而兄伯升好侠养士,常非笑光武事田业,比之高祖兄仲。|{仲,合阳侯喜也,能为产业
。见前书。}王莽天凤中,|{王莽始建国六年改为天凤。}乃之长安,受尚书,略通大义
。|{东观记曰:“受尚书于中大夫庐江许子威。资用乏,与同舍生韩子合钱买驴,令从
者僦,以给诸公费。”}

莽末,天下连岁灾蝗,寇盗锋起。|{言贼锋锐竞起。字或作“蜂”,谕多也。}地皇三年
,|{天凤六年改为地皇。}南阳荒饥,|{《韩诗外传》曰:“一谷不升曰歉,二谷不升曰
饥,三谷不升曰馑,四谷不升曰荒,五谷不升曰大侵。”}诸家宾客多为小盗。光武避吏
新野,|{新野属南阳郡,今邓州县。《续汉书》曰:“伯升宾客劫人,上避吏于新野邓晨
家。”}因卖谷于宛。|{《东观记》曰:“时南阳旱饥,而上田独收。”宛,县,属南阳
郡,故城今邓州南阳县也。}宛人李通等以图谶说光武云:“刘氏复起,李氏为辅。”|{
图,《河图》也。谶,符命之书。谶,验也。言为王者受命之征验也。《易·坤灵图》曰
:“汉之臣李阳也。”}光武初不敢当,然独念兄伯升素结轻客,必举大事,且王莽败亡已
兆,天下方乱,遂与定谋,于是乃市兵弩。十月,与李通从弟轶等起于宛,时年二十八。

十一月,有星孛于张。|{《前书音义》曰:“孛星光芒短,蓬然。张,南方宿也。”《续
汉志》曰:“张为周地。星孛于张,东南行即翼、轸之分。翼、轸,楚地,是楚地将有兵
乱。后一年正月,光武起兵舂陵,攻南阳,斩阜、赐等,杀其士众数万人。光武都洛阳,
居周地,除秽布新之象。”}光武遂将宾客还舂陵。时伯升已会众起兵。初,诸家子弟恐
惧,皆亡逃自匿,曰“伯升杀我”。及见光武绛衣大冠,|{董巴《舆服志》曰:“大冠者
,谓武冠,武官冠之。”《东观记》曰:“上时绛衣大冠,将军服也。”}皆惊曰“谨厚
者亦复为之”,乃稍自安。伯升于是招新市、平林兵,|{新市,县,属江夏郡,故城在今
郢州富水县东北。平林,地名,在今随州随县东北。}与其帅王凤、陈牧西击长聚。|{《
广雅》曰:“聚,居也,音慈谕反。”《前书音义》曰:“小于乡曰聚。”}光武初骑牛
,杀新野尉乃得马。|{《前书》曰,尉,秦官,秩四百石至二百石也。}进屠唐子乡,|{
《例》曰:“多所诛杀曰屠。”唐子乡有唐子山,在今唐州湖阳县西南。}又杀湖阳尉。
|{湖阳属南阳郡,今唐州县也。《东观记》曰:“刘终诈称江夏吏,诱杀之。”}军中分
财物不均,众恚恨,欲反攻诸刘。光武敛宗人所得物,悉以与之,众乃悦。进拔棘阳,|{
县名,属南阳郡,在棘水之阳,古谢国也,故城在今唐州湖阳县西北。棘音己力反。}与
王莽前队大夫甄阜、|{王莽置六队,郡置大夫一人,职如太守。南阳为前队,河内为后队
,颍川为左队,弘农为右队,河东为兆队,荥阳为祈队。队音遂。}属正梁丘赐|{王莽每
队置属正一人,职如都尉。}战于小长安,|{《续汉书》曰淯阳县有小长安聚,故城在今
邓州南阳县南。}汉军大败,还保棘阳。
\end{withgezhu}
  \ifdim\hsize > 5cm
    \advance\hsize by -4pt
\repeat
\end{CJK*}
\end{document}
