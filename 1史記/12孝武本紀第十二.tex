\chapter{孝武本紀第十二}

孝武皇帝者,孝景中子也。母曰王太后。孝景四年,以皇子為膠東王。孝景七年,栗太子廢為臨江王,以膠東王為太子。孝景十六年崩,太子即位,為孝武皇帝。孝武皇帝初即位,尤敬鬼神之祀。

元年,漢興已六十餘歲矣,天下乂安,薦紳之屬皆望天子封禪改正度也。而上鄉儒術,招賢良,趙綰、王臧等以文學為公卿,欲議古立明堂城南,以朝諸侯。草巡狩封禪改歷服色事未就。會竇太后治黃老言,不好儒術,使人微得趙綰等姦利事,召案綰、臧,綰、臧自殺,諸所興為者皆廢。

後六年,竇太后崩。其明年,上徵文學之士公孫弘等。

明年,上初至雍,郊見五畤。後常三歲一郊。是時上求神君,捨之上林中蹄氏觀。神君者,長陵女子,以子死悲哀,故見神於先後宛若。宛若祠之其室,民多往祠。平原君往祠,其後子孫以尊顯。及武帝即位,則厚禮置祠之內中,聞其言,不見其人云。

是時而李少君亦以祠灶、穀道、卻老方見上,上尊之。少君者,故深澤侯入以主方。匿其年及所生長,常自謂七十,能使物,卻老。其游以方遍諸侯。無妻子。人聞其能使物及不死,更饋遺之,常餘金錢帛衣食。人皆以為不治產業而饒給,又不知其何所人,愈信,爭事之。少君資好方,善為巧發奇中。嘗從武安侯飲,坐中有年九十餘老人,少君乃言與其大父游射處,老人為兒時從其大父行,識其處,一坐盡驚。少君見上,上有故銅器,問少君。少君曰,此器齊桓公十年陳於柏寢。已而案其刻,果齊桓公器。一宮盡駭,以少君為神,數百歲人也。

少君言於上曰,祠灶則致物,致物而丹沙可化為黃金,黃金成以為飲食器則益壽,益壽而海中蓬萊僊者可見,見之以封禪則不死,黃帝是也。臣嘗游海上,見安期生,食臣棗,大如瓜。安期生僊者,通蓬萊中,合則見人,不合則隱。於是天子始親祠灶,而遣方士入海求蓬萊安期生之屬,而事化丹沙諸藥齊為黃金矣。

居久之,李少君病死。天子以為化去不死也,而使黃錘史寬舒受其方。求蓬萊安期生莫能得,而海上燕齊怪迂之方士多相效,更言神事矣。

亳人薄誘忌奏祠泰一方,曰,天神貴者泰一,泰一佐曰五帝。古者天子以春秋祭泰一東南郊,用太牢具,七日,為壇開八通之鬼道。於是天子令太祝立其祠長安東南郊,常奉祠如忌方。其後人有上書,言古者天子三年一用太牢具祠神三一,天一,地一,泰一。天子許之,令太祝領祠之忌泰一壇上,如其方。後人復有上書,言古者天子常以春秋解祠,祠黃帝用一梟破鏡,冥羊用羊,祠馬行用一青牡馬,泰一、皋山山君、地長用牛,武夷君用乾魚,陰陽使者以一牛。令祠官領之如其方,而祠於忌泰一壇旁。

其後,天子苑有白鹿,以其皮為幣,以發瑞應,造白金焉。

其明年,郊雍,獲一角獸,若麃然。有司曰,陛下肅祗郊祀,上帝報享,錫一角獸,蓋麟云。於是以薦五畤,畤加一牛以燎。賜諸侯白金,以風符應合于天地。

於是濟北王以為天子且封禪,乃上書獻泰山及其旁邑。天子受之,更以他縣償之。常山王有罪,遷,天子封其弟於真定,以續先王祀,而以常山為郡。然後五嶽皆在天子之郡。

其明年,齊人少翁以鬼神方見上。上有所幸王夫人,夫人卒,少翁以方術蓋夜致王夫人及灶鬼之貌云,天子自帷中望見焉。於是乃拜少翁為文成將軍,賞賜甚多,以客禮禮之。文成言曰,上即欲與神通,宮室被服不象神,神物不至。乃作畫雲氣車,及各以勝日駕車辟惡鬼。又作甘泉宮,中為臺室,畫天、地、泰一諸神,而置祭具以致天神。居歲餘,其方益衰,神不至。乃為帛書以飯牛,詳弗知也,言此牛腹中有奇。殺而視之,得書,書言其怪,天子疑之。有識其手書,問之人,果為偽書。於是誅文成將軍而隱之。

其後則又作柏梁、銅柱、承露僊人掌之屬矣。

文成死明年,天子病鼎湖甚,巫醫無所不致,至不愈。游水發根乃言曰,上郡有巫,病而鬼下之。上召置祠之甘泉。及病,使人問神君。神君言曰,天子毋憂病。病少愈,彊與我會甘泉。於是病愈,遂幸甘泉,病良已。大赦天下,置壽宮神君。神君最貴者大夫太一,其佐曰大禁、司命之屬,皆從之。非可得見,聞其音,與人言等。時去時來,來則風肅然也。居室帷中。時晝言,然常以夜。天子祓,然後入。因巫為主人,關飲食。所欲者言行下。又置壽宮、北宮,張羽旗,設供具,以禮神君。神君所言,上使人受書其言,命之曰畫法。其所語,世俗之所知也,毋絕殊者,而天子獨喜。其事祕,世莫知也。

其後三年,有司言元宜以天瑞命,不宜以一二數。一元曰建元,二元以長星曰元光,三元以郊得一角獸曰元狩云。

其明年冬,天子郊雍,議曰,今上帝朕親郊,而後土毋祀,則禮不答也。有司與太史公、祠官寬舒等議,天地牲角繭栗。今陛下親祀后土,后土宜於澤中圜丘為五壇,壇一黃犢太牢具,已祠盡瘞,而從祠衣上黃。於是天子遂東,始立后土祠汾陰脽上,如寬舒等議。上親望拜,如上帝禮。禮畢,天子遂至滎陽而還。過雒陽,下詔曰,三代邈絕,遠矣難存。其以三十里地封周後為周子南君,以奉先王祀焉。是歲,天子始巡郡縣,侵尋於泰山矣。

其春,樂成侯上書言欒大。欒大,膠東宮人,故嘗與文成將軍同師,已而為膠東王尚方。而樂成侯姊為康王后,毋子。康王死,他姬子立為王。而康后有淫行,與王不相中得,相危以法。康后聞文成已死,而欲自媚於上,乃遣欒大因樂成侯求見言方。天子既誅文成,後悔恨其早死,惜其方不盡,及見欒大,大悅。大為人長美,言多方略,而敢為大言,處之不疑。大言曰,臣嘗往來海中,見安期、羨門之屬。顧以為臣賤,不信臣。又以為康王諸侯耳,不足予方。臣數言康王,康王又不用臣。臣之師曰,黃金可成,而河決可塞,不死之藥可得,僊人可致也。臣恐效文成,則方士皆掩口,惡敢言方哉。上曰,文成食馬肝死耳。子誠能修其方,我何愛乎。大曰,臣師非有求人,人者求之。陛下必欲致之,則貴其使者,令有親屬,以客禮待之,勿卑,使各佩其信印,乃可使通言於神人。神人尚肯邪不邪。致尊其使,然後可致也。於是上使先驗小方,鬬旗,旗自相觸擊。

是時上方憂河決,而黃金不就,乃拜大為五利將軍。居月餘,得四金印,佩天士將軍、地土將軍、大通將軍、天道將軍印。制詔御史,昔禹疏九江,決四瀆。閒者河溢皋陸,隄繇不息。朕臨天下二十有八年,天若遺朕士而大通焉。乾稱蜚龍,鴻漸于般,意庶幾與焉。其以二千戶封地士將軍大為樂通侯。賜列侯甲第,僮千人。乘輿斥車馬帷帳器物以充其家。又以衛長公主妻之,齎金萬斤,更名其邑曰當利公主。天子親如五利之第。使者存問所給,連屬於道。自大主將相以下,皆置酒其家,獻遺之。於是天子又刻玉印曰天道將軍,使使衣羽衣,夜立白茅上,五利將軍亦衣羽衣,立白茅上受印,以示弗臣也。而佩天道者,且為天子道天神也。於是五利常夜祠其家,欲以下神。神未至而百鬼集矣,然頗能使之。其後治裝行,東入海,求其師云。大見數月,佩六印,貴振天下,而海上燕齊之閒,莫不搤捥而自言有禁方,能神僊矣。

其夏六月中,汾陰巫錦為民祠魏脽后土營旁,見地如鉤狀,掊視得鼎。鼎大異於眾鼎,文鏤毋款識,怪之,言吏。吏告河東太守勝,勝以聞。天子使使驗問巫錦得鼎無姦詐,乃以禮祠,迎鼎至甘泉,從行,上薦之。至中山,晏溫,有黃雲蓋焉。有麃過,上自射之,因以祭雲。至長安,公卿大夫皆議請尊寶鼎。天子曰,閒者河溢,歲數不登,故巡祭后土,祈為百姓育穀。今年豐廡未有報,鼎曷為出哉。有司皆曰,聞昔大帝興神鼎一,一者一統,天地萬物所系終也。黃帝作寶鼎三,象天地人也。禹收九牧之金,鑄九鼎,皆嘗鬺烹上帝鬼神。遭聖則興,遷于夏商。周德衰,宋之社亡,鼎乃淪伏而不見。頌云自堂徂基,自羊徂牛,鼐鼎及鼒,不虞不驁,胡考之休。今鼎至甘泉,光潤龍變,承休無疆。合茲中山,有黃白雲降蓋,若獸為符,路弓乘矢,集獲壇下,報祠大饗。惟受命而帝者心知其意而合德焉。鼎宜見於祖禰,藏於帝廷,以合明應。制曰,可。

入海求蓬萊者,言蓬萊不遠,而不能至者,殆不見其氣。上乃遣望氣佐侯其氣雲。

其秋,上幸雍,且郊。或曰五帝,泰一之佐也。宜立泰一而上親郊之。上疑未定。齊人公孫卿曰,今年得寶鼎,其冬辛巳朔旦冬至,與黃帝時等。卿有札書曰,黃帝得寶鼎宛侯朐,問於鬼臾區。區對曰,黃帝得寶鼎神筴,是歲己酉朔旦冬至,得天之紀,終而復始。於是黃帝迎日推筴,後率二十歲得朔旦冬至,凡二十推,三百八十年。黃帝僊登于天。卿因所忠欲奏之。所忠視其書不經,疑其妄書,謝曰,寶鼎事已決矣,尚何以為。卿因嬖人奏之。上大說,召問卿。對曰,受此書申功,申功已死。上曰,申功何人也。卿曰,申功,齊人也。與安期生通,受黃帝言,無書,獨有此鼎書。曰漢興復當黃帝之時。漢之聖者在高祖之孫且曾孫也。寶鼎出而與神通,封禪。封禪七十二王,唯黃帝得上泰山封。申功曰,漢主亦當上封,上封則能僊登天矣。黃帝時萬諸侯,而神靈之封居七千。天下名山八,而三在蠻夷,五在中國。中國華山、首山、太室、泰山、東萊,此五山黃帝之所常遊,與神會。黃帝且戰且學僊。患百姓非其道,乃斷斬非鬼神者。百餘歲然後得與神通。黃帝郊雍上帝,宿三月。鬼臾區號大鴻,死葬雍,故鴻冢是也。其後於黃帝接萬靈明廷。明廷者,甘泉也。所謂寒門者,谷口也。黃帝采首山銅,鑄鼎荊山下。鼎既成,有龍垂胡髯下迎黃帝。黃帝上騎,群臣後宮從上龍七十餘人,乃上去。餘小臣不得上,乃悉持龍髯,龍髯拔,墮黃帝之弓。百姓仰望黃帝既上天,乃抱其弓與龍胡髯號。故後世因名其處曰鼎湖,其弓曰烏號。於是天子曰,嗟乎。吾誠得如黃帝,吾視去妻子如脫屣耳。乃拜卿為郎,東使候神於太室。

上遂郊雍,至隴西,西登空桐,幸甘泉。令祠官寬舒等具泰一祠壇,壇放薄忌泰一壇,壇三垓。五帝壇環居其下,各如其方,黃帝西南,除八通鬼道。泰一所用,如雍一畤物,而加醴棗脯之屬,殺一牦牛以為俎豆牢具。而五帝獨有俎豆醴進。其下四方地,為餟食群神從者及北斗云。已祠,胙餘皆燎之。其牛色白,鹿居其中,彘在鹿中,水而洎之。祭日以牛,祭月以羊彘特。泰一祝宰則衣紫及繡。五帝各如其色,日赤,月白。

十一月辛已朔旦冬至,昧爽,天子始郊拜泰一。朝朝日,夕夕月,則揖,而見泰一如雍禮。其贊饗曰,天始以寶鼎神筴授皇帝,朔而又朔,終而復始,皇帝敬拜見焉。而衣上黃。其祠列火滿壇,壇旁烹炊具。有司云祠上有光焉。公卿言皇帝始郊見泰一雲陽,有司奉瑄玉嘉牲薦饗。是夜有美光,及晝,黃氣上屬天。太史公、祠官寬舒等曰,神靈之休,祐福兆祥,宜因此地光域立泰畤壇以明應。令太祝領,祀秋及臘閒祠。三歲天子一郊見。

其秋,為伐南越,告禱泰一,以牡荊畫幡日月北斗登龍,以象天一三星,為泰一鋒,名曰靈旗。為兵禱,則太史奉以指所伐國。而五利將軍使不敢入海,之泰山祠。上使人微隨驗,實無所見。五利妄言見其師,其方盡,多不讎。上乃誅五利。

其冬,公孫卿候神河南,見僊人跡緱氏城上,有物若雉,往來城上。天子親幸緱氏城視跡。問卿,得毋效文成、五利乎。卿曰,僊者非有求人主,人主求之。其道非少寬假,神不來。言神事,事如迂誕,積以歲乃可致。於是郡國各除道,繕治宮觀名山神祠所,以望幸矣。

其年,既滅南越,上有嬖臣李延年以好音見。上善之,下公卿議,曰,民閒祠尚有鼓舞之樂,今郊祠而無樂,豈稱乎。公卿曰,古者祀天地皆有樂,而神祇可得而禮。或曰,泰帝使素女鼓五十弦瑟,悲,帝禁不止,故破其瑟為二十五弦。於是塞南越,禱祠泰一、后土,始用樂舞,益召歌兒,作二十五弦及箜篌瑟自此起。

其來年冬,上議曰,古者先振兵澤旅,然後封禪。乃遂北巡朔方,勒兵十餘萬,還祭黃帝冢橋山,澤兵須如。上曰,吾聞黃帝不死,今有冢,何也。或對曰,黃帝已僊上天,群臣葬其衣冠。即至甘泉,為且用事泰山,先類祠泰一。

自得寶鼎,上與公卿諸生議封禪。封禪用希曠絕,莫知其儀禮,而群儒采封禪尚書、周官、王制之望祀射牛事。齊人丁公年九十餘,曰,封者,合不死之名也。秦皇帝不得上封。陛下必欲上,稍上即無風雨,遂上封矣。上於是乃令諸儒習射牛,草封禪儀。數年,至且行。天子既聞公孫卿及方士之言,黃帝以上封禪,皆致怪物與神通,欲放黃帝以嘗接神僊人蓬萊士,高世比德於九皇,而頗采儒術以文之。群儒既以不能辯明封禪事,又牽拘於詩書古文而不敢騁。上為封祠器示群儒,群儒或曰不與古同,徐偃又曰太常諸生行禮不如魯善,周霸屬圖封事,於是上絀偃、霸,盡罷諸儒弗用。

三月,遂東幸緱氏,禮登中嶽太室。從官在山下聞若有言萬歲云。問上,上不言,問下,下不言。於是以三百戶封太室奉祠,命曰崇高邑。東上泰山,山之草木葉未生,乃令人上石立之泰山顛。

上遂東巡海上,行禮祠八神。齊人之上疏言神怪奇方者以萬數,然無驗者。乃益發船,令言海中神山者數千人求蓬萊神人。公孫卿持節常先行候名山,至東萊,言夜見一人,長數丈,就之則不見,見其跡甚大,類禽獸云。群臣有言見一老父牽狗,言吾欲見巨公,已忽不見。上既見大跡,未信,及群臣有言老父,則大以為僊人也。宿留海上,與方士傳車及閒使求僊人以千數。

四月,還至奉高。上念諸儒及方士言封禪人人殊,不經,難施行。天子至梁父,禮祠地主。乙卯,令侍中儒者皮弁薦紳,射牛行事。封泰山下東方,如郊祠泰一之禮。封廣丈二尺,高九尺,其下則有玉牒書,書祕。禮畢,天子獨與侍中奉車子侯上泰山,亦有封。其事皆禁。明日,下陰道。丙辰,禪泰山下阯東北肅然山,如祭后土禮。天子皆親拜見,衣上黃而盡用樂焉。江淮閒一茅三脊為神藉。五色土益雜封。縱遠方奇獸蜚禽及白雉諸物,頗以加祠。兕旄牛犀象之屬弗用。皆至泰山然後去。封禪祠,其夜若有光,晝有白雲起封中。

天子從封禪還,坐明堂,群臣更上壽。於是制詔御史,朕以眇眇之身承至尊,兢兢焉懼弗任。維德菲薄,不明于禮樂。修祀泰一,若有象景光,屑如有望,依依震於怪物,欲止不敢,遂登封泰山,至於梁父,而後禪肅然。自新,嘉與士大夫更始,賜民百戶牛一酒十石,加年八十孤寡布帛二匹。復博、奉高、蛇丘、歷城,毋出今年租稅。其赦天下,如乙卯赦令。行所過毋有復作。事在二年前,皆勿聽治。又下詔曰,古者天子五載一巡狩,用事泰山,諸侯有朝宿地。其令諸侯各治邸泰山下。

天子既已封禪泰山,無風雨菑,而方士更言蓬萊諸神山若將可得,於是上欣然庶幾遇之,乃復東至海上望,冀遇蓬萊焉。奉車子侯暴病,一日死。上乃遂去,并海上,北至碣石,巡自遼西,歷北邊至九原。五月,返至甘泉。有司言寶鼎出為元鼎,以今年為元封元年。

其秋,有星茀于東井。後十餘日,有星茀于三能。望氣王朔言,候獨見其星出如瓠,食頃復入焉。有司言曰,陛下建漢家封禪,天其報德星雲。

其來年冬,郊雍五帝,還,拜祝祠泰一。贊饗曰,德星昭衍,厥維休祥。壽星仍出,淵耀光明。信星昭見,皇帝敬拜泰祝之饗。

其春,公孫卿言見神人東萊山,若云見天子。天子於是幸緱氏城,拜卿為中大夫。遂至東萊,宿留之數日,毋所見,見大人跡。復遣方士求神怪采芝藥以千數。是歲旱。於是天子既出毋名,乃禱萬里沙,過祠泰山。還至瓠子,自臨塞決河,留二日,沈祠而去。使二卿將卒塞決河,河徙二渠,復禹之故跡焉。

是時既滅南越,越人勇之乃言越人俗信鬼,而其祠皆見鬼,數有效。昔東甌王敬鬼,壽至百六十歲。後世謾怠,故衰秏。乃令越巫立越祝祠,安臺無壇,亦祠天神上帝百鬼,而以雞卜。上信之,越祠雞卜始用焉。

公孫卿曰,僊人可見,而上往常遽,以故不見。今陛下可為觀,如緱氏城,置脯棗,神人宜可致。且僊人好樓居。於是上令長安則作蜚廉桂觀,甘泉則作益延壽觀,使卿持節設具而候神人,乃作通天臺,置祠具其下,將招來神僊之屬。於是甘泉更置前殿,始廣諸宮室。夏,有芝生殿防內中。天子為塞河,興通天臺,若有光雲,乃下詔曰,甘泉防生芝九莖,赦天下,毋有復作。

其明年,伐朝鮮。夏,旱。公孫卿曰,黃帝時封則天旱,乾封三年。上乃下詔曰,天旱,意乾封乎。其令天下尊祠靈星焉。

其明年,上郊雍,通回中道,巡之。春,至鳴澤,從西河歸。

其明年冬,上巡南郡,至江陵而東。登禮潛之天柱山,號曰南嶽。浮江,自尋陽出樅陽,過彭蠡,祀其名山川。北至瑯邪,并海上。四月中,至奉高修封焉。

初,天子封泰山,泰山東北阯古時有明堂處,處險不敞。上欲治明堂奉高旁,未曉其制度。濟南人公玊帶上黃帝時明堂圖。明堂圖中有一殿,四面無壁,以茅蓋,通水,圜宮垣為複道,上有樓,從西南入,命曰昆侖,天子從之入,以拜祠上帝焉。於是上令奉高作明堂汶上,如帶圖。及五年修封,則祠泰一、五帝於明堂上坐,令高皇帝祠坐對之。祠后土於下房,以二十太牢。天子從昆侖道入,始拜明堂如郊禮。禮畢,燎堂下。而上又上泰山,有祕祠其顛。而泰山下祠五帝,各如其方,黃帝并赤帝,而有司侍祠焉。泰山上舉火,下悉應之。

其後二歲,十一月甲子朔旦冬至,推歷者以本統。天子親至泰山,以十一月甲子朔旦冬至日祠上帝明堂,每修封禪。其贊饗曰,天增授皇帝泰元神筴,周而復始。皇帝敬拜泰一。東至海上,考入海及方士求神者,莫驗,然益遣,冀遇之。

十一月乙酉,柏梁災。十二月甲午朔,上親禪高里,祠后土。臨渤海,將以望祠蓬萊之屬,冀至殊庭焉。

上還,以柏梁災故,朝受計甘泉。公孫卿曰,黃帝就青靈臺,十二日燒,黃帝乃治明庭。明庭,甘泉也。方士多言古帝王有都甘泉者。其後天子又朝諸侯甘泉,甘泉作諸侯邸。勇之乃曰,越俗有火災,復起屋必以大,用勝服之。於是作建章宮,度為千門萬戶。前殿度高未央,其東則鳳闕,高二十餘丈。其西則唐中,數十里虎圈。其北治大池,漸臺高二十餘丈,名曰泰液池,中有蓬萊、方丈、瀛洲、壺梁,象海中神山龜魚之屬。其南有玉堂、璧門、大鳥之屬。乃立神明臺、井幹樓,度五十餘丈,輦道相屬焉。

夏,漢改歷,以正月為歲首,而色上黃,官名更印章以五字。因為太初元年。是歲,西伐大宛。蝗大起。丁夫人、雒陽虞初等以方祠詛匈奴、大宛焉。

其明年,有司言雍五畤無牢熟具,芬芳不備。乃命祠官進畤犢牢具,五色食所勝,而以木禺馬代駒焉。獨五帝用駒,行親郊用駒。及諸名山川用駒者,悉以木禺馬代。行過,乃用駒。他禮如故。

其明年,東巡海上,考神僊之屬,未有驗者。方士有言黃帝時為五城十二樓,以候神人於執期,命曰迎年。上許作之如方,名曰明年。上親禮祠上帝,衣上黃焉。

公玊帶曰,黃帝時雖封泰山,然風后、封鉅、岐伯令黃帝封東泰山,禪凡山合符,然後不死焉。天子既令設祠具,至東泰山,東泰山卑小,不稱其聲,乃令祠官禮之,而不封禪焉。其後令帶奉祠候神物。夏,遂還泰山,修五年之禮如前,而加禪祠石閭。石閭者,在泰山下阯南方,方士多言此僊人之閭也,故上親禪焉。

其後五年,復至泰山修封,還過祭常山。

今天子所興祠,泰一、后土,三年親郊祠,建漢家封禪,五年一修封。薄忌泰一及三一、冥羊、馬行、赤星,五,寬舒之祠官以歲時致禮。凡六祠,皆太祝領之。至如八神諸神,明年、凡山他名祠,行過則祀,去則已。方士所興祠,各自主,其人終則已,祠官弗主。他祠皆如其故。今上封禪,其後十二歲而還,遍於五岳、四瀆矣。而方士之候祠神人,入海求蓬萊,終無有驗。而公孫卿之候神者,猶以大人跡為解,無其效。天子益怠厭方士之怪迂語矣,然終羈縻弗絕,冀遇其真。自此之後,方士言祠神者彌眾,然其效可睹矣。

太史公曰,余從巡祭天地諸神名山川而封禪焉。入壽宮侍祠神語,究觀方士祠官之言,於是退而論次自古以來用事於鬼神者,具見其表裏。後有君子,得以覽焉。至若俎豆珪幣之詳,獻酬之禮,則有司存焉。