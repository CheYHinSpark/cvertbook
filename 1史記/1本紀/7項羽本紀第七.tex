\chapter{項羽本紀第七}

項籍者,下相人也,字羽。初起時,年二十四。其季父項梁,梁父即楚將項燕,為秦將王翦所戮者也。項氏世世為楚將,封於項,故姓項氏。

項籍少時,學書不成,去學劍,又不成。項梁怒之。籍曰,書足以記名姓而已。劍一人敵,不足學,學萬人敵。於是項梁乃教籍兵法,籍大喜,略知其意,又不肯竟學。項梁嘗有櫟陽逮,乃請蘄獄掾曹咎書抵櫟陽獄掾司馬欣,以故事得已。項梁殺人,與籍避仇於吳中。吳中賢士大夫皆出項梁下。每吳中有大繇役及喪,項梁常為主辦,陰以兵法部勒賓客及子弟,以是知其能。秦始皇帝游會稽,渡浙江,梁與籍俱觀。籍曰,彼可取而代也。梁掩其口,曰,毋妄言,族矣。梁以此奇籍。籍長八尺餘,力能扛鼎,才氣過人,雖吳中子弟皆已憚籍矣。

秦二世元年七月,陳涉等起大澤中。其九月,會稽守通謂梁曰,江西皆反,此亦天亡秦之時也。吾聞先即制人,後則為人所制。吾欲發兵,使公及桓楚將。是時桓楚亡在澤中。梁曰,桓楚亡,人莫知其處,獨籍知之耳。梁乃出,誡籍持劍居外待。梁復入,與守坐,曰,請召籍,使受命召桓楚。守曰,諾。梁召籍入。須臾,梁眴籍曰,可行矣。於是籍遂拔劍斬守頭。項梁持守頭,佩其印綬。門下大驚,擾亂,籍所擊殺數十百人。一府中皆慴伏,莫敢起。梁乃召故所知豪吏,諭以所為起大事,遂舉吳中兵。使人收下縣,得精兵八千人。梁部署吳中豪傑為校尉、候、司馬。有一人不得用,自言於梁。梁曰,前時某喪使公主某事,不能辦,以此不任用公。眾乃皆伏。於是梁為會稽守,籍為裨將,徇下縣。

廣陵人召平於是為陳王徇廣陵,未能下。聞陳王敗走,秦兵又且至,乃渡江矯陳王命,拜梁為楚王上柱國。曰,江東已定,急引兵西擊秦。項梁乃以八千人渡江而西。聞陳嬰已下東陽,使使欲與連和俱西。陳嬰者,故東陽令史,居縣中,素信謹,稱為長者。東陽少年殺其令,相聚數千人,欲置長,無適用,乃請陳嬰。嬰謝不能,遂彊立嬰為長,縣中從者得二萬人。少年欲立嬰便為王,異軍蒼頭特起。陳嬰母謂嬰曰,自我為汝家婦,未嘗聞汝先古之有貴者。今暴得大名,不祥。不如有所屬,事成猶得封侯,事敗易以亡,非世所指名也。嬰乃不敢為王。謂其軍吏曰,項氏世世將家,有名於楚。今欲舉大事,將非其人,不可。我倚名族,亡秦必矣。於是眾從其言,以兵屬項梁。項梁渡淮,黥布、蒲將軍亦以兵屬焉。凡六七萬人,軍下邳。

當是時,秦嘉已立景駒為楚王,軍彭城東,欲距項梁。項梁謂軍吏曰,陳王先首事,戰不利,未聞所在。今秦嘉倍陳王而立景駒,逆無道。乃進兵擊秦嘉。秦嘉軍敗走,追之至胡陵。嘉還戰一日,嘉死,軍降。景駒走死梁地。項梁已并秦嘉軍,軍胡陵,將引軍而西。章邯軍至栗,項梁使別將朱雞石、餘樊君與戰。餘樊君死。朱雞石軍敗,亡走胡陵。項梁乃引兵入薛,誅雞石。項梁前使項羽別攻襄城,襄城堅守不下。已拔,皆阬之。還報項梁。項梁聞陳王定死,召諸別將會薛計事。此時沛公亦起沛,往焉。

居鄛人范增,年七十,素居家,好奇計,往說項梁曰,陳勝敗碧當。夫秦滅六國,楚最無罪。自懷王入秦不反,楚人憐之至今,故楚南公曰楚雖三戶,亡秦必楚也。今陳勝首事,不立楚後而自立,其勢不長。今君起江東,楚蜂午之將皆爭附君者,以君世世楚將,為能復立楚之後也。於是項梁然其言,乃求楚懷王孫心民閒,為人牧羊,立以為楚懷王,從民所望也。陳嬰為楚上柱國,封五縣,與懷王都盱臺。項梁自號為武信君。

居數月,引兵攻亢父,與齊田榮、司馬龍且軍救東阿,大破秦軍於東阿。田榮即引兵歸,逐其王假。假亡走楚。假相田角亡走趙。角弟田閒故齊將,居趙不敢歸。田榮立田儋子市為齊王。項梁已破東阿下軍,遂追秦軍。數使使趣齊兵,欲與俱西。田榮曰,楚殺田假,趙殺田角、田閒,乃發兵。項梁曰,田假為與國之王,窮來從我,不忍殺之。趙亦不殺田角、田閒以市於齊。齊遂不肯發兵助楚。項梁使沛公及項羽別攻城陽,屠之。西破秦軍濮陽東,秦兵收入濮陽。沛公、項羽乃攻定陶。定陶未下,去,西略地至雝丘,大破秦軍,斬李由。還攻外黃,外黃未下。

項梁起東阿,西,北比至定陶,再破秦軍,項羽等又斬李由,益輕秦,有驕色。宋義乃諫項梁曰,戰勝而將驕卒惰者敗。今卒少惰矣,秦兵日益,臣為君畏之。項梁弗聽。乃使宋義使於齊。道遇齊使者高陵君顯,曰,公將見武信君乎。曰,然。曰,臣論武信君軍必敗。公徐行即免死,疾行則及禍。秦果悉起兵益章邯,擊楚軍,大破之定陶,項梁死。沛公、項羽去外黃攻陳留,陳留堅守不能下。沛公、項羽相與謀曰,今項梁軍破,士卒恐。乃與呂臣軍俱引兵而東。呂臣軍彭城東,項羽軍彭城西,沛公軍碭。

章邯已破項梁軍,則以為楚地兵不足憂,乃渡河擊趙,大破之。當此時,趙歇為王,陳餘為將,張耳為相,皆走入鉅鹿城。章邯令王離、涉閒圍鉅鹿,章邯軍其南,筑甬道而輸之粟。陳餘為將,將卒數萬人而軍鉅鹿之北,此所謂河北之軍也。

楚兵已破於定陶,懷王恐,從盱臺之彭城,并項羽、呂臣軍自將之。以呂臣為司徒,以其父呂青為令尹。以沛公為碭郡長,封為武安侯,將碭郡兵。

初,宋義所遇齊使者高陵君顯在楚軍,見楚王曰,宋義論武信君之軍必敗,居數日,軍果敗。兵未戰而先見敗徵,此可謂知兵矣。王召宋義與計事而大說之,因置以為上將軍,項羽為魯公,為次將,范增為末將,救趙。諸別將皆屬宋義,號為卿子冠軍。行至安陽,留四十六日不進。項羽曰,吾聞秦軍圍趙王鉅鹿,疾引兵渡河,楚擊其外,趙應其內,破秦軍必矣。宋義曰,不然。夫搏牛之虻不可以破蟣虱。今秦攻趙,戰勝則兵罷,我承其敝,不勝,則我引兵鼓行而西,必舉秦矣。故不如先鬬秦趙。夫被堅執銳,義不如公,坐而運策,公不如義。因下令軍中曰,猛如虎,很如羊,貪如狼,彊不可使者,皆斬之。乃遣其子宋襄相齊,身送之至無鹽,飲酒高會。天寒大雨,士卒凍饑。項羽曰,將戮力而攻秦,久留不行。今歲饑民貧,士卒食芋菽,軍無見糧,乃飲酒高會,不引兵渡河因趙食,與趙并力攻秦,乃曰承其敝。夫以秦之彊,攻新造之趙,其勢必舉趙。趙舉而秦彊,何敝之承。且國兵新破,王坐不安席,埽境內而專屬於將軍,國家安危,在此一舉。今不恤士卒而徇其私,非社稷之臣。項羽晨朝上將軍宋義,即其帳中斬宋義頭,出令軍中曰,宋義與齊謀反楚,楚王陰令羽誅之。當是時,諸將皆慴服,莫敢枝梧。皆曰,首立楚者,將軍家也。今將軍誅亂。乃相與共立羽為假上將軍。使人追宋義子,及之齊,殺之。使桓楚報命於懷王。懷王因使項羽為上將軍,當陽君、蒲將軍皆屬項羽。

項羽已殺卿子冠軍,威震楚國,名聞諸侯。乃遣當陽君、蒲將軍將卒二萬渡河,救鉅鹿。戰少利,陳餘復請兵。項羽乃悉引兵渡河,皆沈船,破釜甑,燒廬舍,持三日糧,以示士卒必死,無一還心。於是至則圍王離,與秦軍遇,九戰,絕其甬道,大破之,殺蘇角,虜王離。涉閒不降楚,自燒殺。當是時,楚兵冠諸侯。諸侯軍救鉅鹿下者十餘壁,莫敢縱兵。及楚擊秦,諸將皆從壁上觀。楚戰士無不一以當十,楚兵呼聲動天,諸侯軍無不人人惴恐。於是已破秦軍,項羽召見諸侯將,入轅門,無不膝行而前,莫敢仰視。項羽由是始為諸侯上將軍,諸侯皆屬焉。

章邯軍棘原,項羽軍漳南,相持未戰。秦軍數卻,二世使人讓章邯。章邯恐,使長史欣請事。至咸陽,留司馬門三日,趙高不見,有不信之心。長史欣恐,還走其軍,不敢出故道,趙高果使人追之,不及。欣至軍,報曰,趙高用事於中,下無可為者。今戰能勝,高必疾妒吾功,戰不能勝,不免於死。願將軍孰計之。陳餘亦遺章邯書曰,白起為秦將,南征鄢郢,北阬馬服,攻城略地,不可勝計,而竟賜死。蒙恬為秦將,北逐戎人,開榆中地數千里,竟斬陽周。何者。功多,秦不能盡封,因以法誅之。今將軍為秦將三歲矣,所亡失以十萬數,而諸侯并起滋益多。彼趙高素諛日久,今事急,亦恐二世誅之,故欲以法誅將軍以塞責,使人更代將軍以脫其禍。夫將軍居外久,多內卻,有功亦誅,無功亦誅。且天之亡秦,無愚智皆知之。今將軍內不能直諫,外為亡國將,孤特獨立而欲常存,豈不哀哉。將軍何不還兵與諸侯為從,約共攻秦,分王其地,南面稱孤,此孰與身伏鈇質,妻子為僇乎。章邯狐疑,陰使候始成使項羽,欲約。約未成,項羽使蒲將軍日夜引兵度三戶,軍漳南,與秦戰,再破之。項羽悉引兵擊秦軍汙水上,大破之。

章邯使人見項羽,欲約。項羽召軍吏謀曰,糧少,欲聽其約。軍吏皆曰,善。項羽乃與期洹水南殷虛上。已盟,章邯見項羽而流涕,為言趙高。項羽乃立章邯為雍王,置楚軍中。使長史欣為上將軍,將秦軍為前行。

到新安。諸侯吏卒異時故繇使屯戍過秦中,秦中吏卒遇之多無狀,及秦軍降諸侯,諸侯吏卒乘勝多奴虜使之,輕折辱秦吏卒。秦吏卒多竊言曰,章將軍等詐吾屬降諸侯,今能入關破秦,大善,即不能,諸侯虜吾屬而東,秦必盡誅吾父母妻子。諸侯微聞其計,以告項羽。項羽乃召黥布、蒲將軍計曰,秦吏卒尚眾,其心不服,至關中不聽,事必危,不如擊殺之,而獨與章邯、長史欣、都尉翳入秦。於是楚軍夜擊阬秦卒二十餘萬人新安城南。

行略定秦地。函谷關有兵守關,不得入。又聞沛公已破咸陽,項羽大怒,使當陽君等擊關。項羽遂入,至于戲西。沛公軍霸上,未得與項羽相見。沛公左司馬曹無傷使人言於項羽曰,沛公欲王關中,使子嬰為相,珍寶盡有之。項羽大怒,曰,旦日饗士卒,為擊破沛公軍。當是時,項羽兵四十萬,在新豐鴻門,沛公兵十萬,在霸上。范增說項羽曰,沛公居山東時,貪於財貨,好美姬。今入關,財物無所取,婦女無所幸,此其志不在小。吾令人望其氣,皆為龍虎,成五采,此天子氣也。急擊勿失。

楚左尹項伯者,項羽季父也,素善留侯張良。張良是時從沛公,項伯乃夜馳之沛公軍,私見張良,具告以事,欲呼張良與俱去。曰,毋從俱死也。張良曰,臣為韓王送沛公,沛公今事有急,亡去不義,不可不語。良乃入,具告沛公。沛公大驚,曰,為之柰何。張良曰,誰為大王為此計者。曰,鯫生說我曰距關,毋內諸侯,秦地可盡王也。故聽之。良曰,料大王士卒足以當項王乎。沛公默然,曰,固不如也,且為之柰何。張良曰,請往謂項伯,言沛公不敢背項王也。沛公曰,君安與項伯有故。張良曰,秦時與臣游,項伯殺人,臣活之。今事有急,故幸來告良。沛公曰孰與君少長。良曰,長於臣。沛公曰君為我呼入,吾得兄事之。張良出,要項伯。項伯即入見沛公。沛公奉卮酒為壽,約為婚姻,曰,吾入關,秋豪不敢有所近,籍吏民,封府庫,而待將軍。所以遣將守關者,備他盜之出入與非常也。日夜望將軍至,豈敢反乎。願伯具言臣之不敢倍德也。項伯許諾。謂沛公曰,旦日不可不蚤自來謝項王。沛公曰,諾。於是項伯復夜去,至軍中,具以沛公言報項王。因言曰,沛公不先破關中,公豈敢入乎。今人有大功而擊之,不義也,不如因善遇之。項王許諾。

沛公旦日從百餘騎來見項王,至鴻門,謝曰,臣與將軍戮力而攻秦,將軍戰河北,臣戰河南,然不自意能先入關破秦,得復見將軍於此。今者有小人之言,令將軍與臣有郤。項王曰,此沛公左司馬曹無傷言之,不然,籍何以至此。項王即日因留沛公與飲。項王、項伯東向坐。亞父南向坐。亞父者,范增也。沛公北向坐,張良西向侍。范增數目項王,舉所佩玉珪以示之者三,項王默然不應。范增起,出召項莊,謂曰,君王為人不忍,若入前為壽,壽畢,請以劍舞,因擊沛公於坐,殺之。不者,若屬皆且為所虜。莊則入為壽,壽畢,曰,君王與沛公飲,軍中無以為樂,請以劍舞。項王曰,諾。項莊拔劍起舞,項伯亦拔劍起舞,常以身翼蔽沛公,莊不得擊。於是張良至軍門,見樊噲。樊噲曰,今日之事何如。良曰,甚急。今者項莊拔劍舞,其意常在沛公也。噲曰,此迫矣,臣請入,與之同命。噲即帶劍擁盾入軍門。交戟之衛士欲止不內,樊噲側其盾以撞,衛士仆地,噲遂入,披帷西向立,瞋目視項王,頭髪上指,目眥盡裂。項王按劍而跽曰,客何為者。張良曰,沛公之參乘樊噲者也。項王曰,壯士,賜之卮酒。則與斗卮酒。噲拜謝,起,立而飲之。項王曰,賜之彘肩。則與一生彘肩。樊噲覆其盾於地,加彘肩上,拔劍切而啗之。項王曰,壯士,能復飲乎。樊噲曰,臣死且不避,卮酒安足辭。夫秦王有虎狼之心,殺人如不能舉,刑人如恐不勝,天下皆叛之。懷王與諸將約曰先破秦入咸陽者王之。今沛公先破秦入咸陽,豪毛不敢有所近,封閉宮室,還軍霸上,以待大王來。故遣將守關者,備他盜出入與非常也。勞苦而功高如此,未有封侯之賞,而聽細說,欲誅有功之人。此亡秦之續耳,竊為大王不取也。項王未有以應,曰,坐。樊噲從良坐。坐須臾,沛公起如廁,因招樊噲出。

沛公已出,項王使都尉陳平召沛公。沛公曰,今者出,未辭也,為之柰何。樊噲曰,大行不顧細謹,大禮不辭小讓。如今人方為刀俎,我為魚肉,何辭為。於是遂去。乃令張良留謝。良問曰,大王來何操。曰,我持白璧一雙,欲獻項王,玉斗一雙,欲與亞父,會其怒,不敢獻。公為我獻之張良曰,謹諾。當是時,項王軍在鴻門下,沛公軍在霸上,相去四十里。沛公則置車騎,脫身獨騎,與樊噲、夏侯嬰、靳彊、紀信等四人持劍盾步走,從酈山下,道芷陽閒行。沛公謂張良曰,從此道至吾軍,不過二十里耳。度我至軍中,公乃入。沛公已去,閒至軍中,張良入謝,曰,沛公不勝桮杓,不能辭。謹使臣良奉白璧一雙,再拜獻大王足下,玉斗一雙,再拜奉大將軍足下。項王曰,沛公安在。良曰,聞大王有意督過之,脫身獨去,已至軍矣。項王則受璧,置之坐上。亞父受玉斗,置之地,拔劍撞而破之,曰,唉。豎子不足與謀。奪項王天下者,必沛公也,吾屬今為之虜矣。沛公至軍,立誅殺曹無傷。

居數日,項羽引兵西屠咸陽,殺秦降王子嬰,燒秦宮室,火三月不滅,收其貨寶婦女而東。人或說項王曰,關中阻山河四塞,地肥饒,可都以霸。項王見秦宮皆以燒殘破,又心懷思欲東歸,曰,富貴不歸故鄉,如衣繡夜行,誰知之者。說者曰,人言楚人沐猴而冠耳,果然。項王聞之,烹說者。

項王使人致命懷王。懷王曰,如約。乃尊懷王為義帝。項王欲自王,先王諸將相。謂曰,天下初發難時,假立諸侯後以伐秦。然身被堅執銳首事,暴露於野三年,滅秦定天下者,皆將相諸君與籍之力也。義帝雖無功,故當分其地而王之。諸將皆曰,善。乃分天下,立諸將為侯王。項王、范增疑沛公之有天下,業已講解,又惡負約,恐諸侯叛之,乃陰謀曰,巴、蜀道險,秦之遷人皆居蜀。乃曰,巴、蜀亦關中地也。故立沛公為漢王,王巴、蜀、漢中,都南鄭。而三分關中,王秦降將以距塞漢王。項王乃立章邯為雍王,王咸陽以西,都廢丘。長史欣者,故為櫟陽獄掾,嘗有德於項梁,都尉董翳者,本勸章邯降楚。故立司馬欣為塞王,王咸陽以東至河,都櫟陽,立董翳為翟王,王上郡,都高奴。徙魏王豹為西魏王,王河東,都平陽。瑕丘申陽者,張耳嬖臣也,先下河南郡,迎楚河上,故立申陽為河南王,都雒陽。韓王成因故都,都陽翟。趙將司馬卬定河內,數有功,故立卬為殷王,王河內,都朝歌。徙趙王歇為代王。趙相張耳素賢,又從入關,故立耳為常山王,王趙地,都襄國。當陽君黥布為楚將,常冠軍,故立布為九江王,都六。鄱君吳芮率百越佐諸侯,又從入關,故立芮為衡山王,都邾。義帝柱國共敖將兵擊南郡,功多,因立敖為臨江王,都江陵。徙燕王韓廣為遼東王。燕將臧荼從楚救趙,因從入關,故立荼為燕王,都薊。徙齊王田市為膠東王。齊將田都從共救趙,因從入關,故立都為齊王,都臨菑。故秦所滅齊王建孫田安,項羽方渡河救趙,田安下濟北數城,引其兵降項羽,故立安為濟北王,都博陽。田榮者,數負項梁,又不肯將兵從楚擊秦,以故不封。成安君陳餘棄將印去,不從入關,然素聞其賢,有功於趙,聞其在南皮,故因環封三縣。番君將梅鋗功多,故封十萬戶侯。項王自立為西楚霸王,王九郡,都彭城。

漢之元年四月,諸侯罷戲下,各就國。項王出之國,使人徙義帝,曰,古之帝者地方千里,必居上游。乃使使徙義帝長沙郴縣。趣義帝行,其群臣稍稍背叛之,乃陰令衡山、臨江王擊殺之江中。韓王成無軍功,項王不使之國,與俱至彭城,廢以為侯,已又殺之。臧荼之國,因逐韓廣之遼東,廣弗聽,荼擊殺廣無終,

田榮聞項羽徙齊王市膠東,而立齊將田都為齊王,乃大怒,不肯遣齊王之膠東,因以齊反,迎擊田都。田都走楚。齊王市畏項王,乃亡之膠東就國。田榮怒,追擊殺之即墨。榮因自立為齊王,而西殺擊濟北王田安,并王三齊。榮與彭越將軍印,令反梁地。陳餘陰使張同、夏說說齊王田榮曰,項羽為天下宰,不平。今盡王故王於醜地,而王其群臣諸將善地,逐其故主趙王,乃北居代,餘以為不可。聞大王起兵,且不聽不義,願大王資餘兵,請以擊常山,以復趙王,請以國為捍蔽。齊王許之,因遣兵之趙。陳餘悉發三縣兵,與齊并力擊常山,大破之。張耳走歸漢。陳餘迎故趙王歇於代,反之趙。趙王因立陳餘為代王。

是時,漢還定三秦。項羽聞漢王皆已并關中,且東,齊、趙叛之,大怒。乃以故吳令鄭昌為韓王,以距漢。令蕭公角等擊彭越。彭越敗蕭公角等。漢使張良徇韓,乃遺項王書曰,漢王失職,欲得關中,如約即止,不敢東。又以齊、梁反書遺項王曰,齊欲與趙并滅楚。楚以此故無西意,而北擊齊。徵兵九江王布。布稱疾不往,使將將數千人行。項王由此怨布也。漢之二年冬,項羽遂北至城陽,田榮亦將兵會戰。田榮不勝,走至平原,平原民殺之。遂北燒夷齊城郭室屋,皆阬田榮降卒,系虜其老弱婦女。徇齊至北海,多所殘滅。齊人相聚而叛之。於是田榮弟田橫收齊亡卒得數萬人,反城陽。項王因留,連戰未能下。

春,漢王部五諸侯兵,凡五十六萬人,東伐楚。項王聞之,即令諸將擊齊,而自以精兵三萬人南從魯出胡陵。四月,漢皆已入彭城,收其貨寶美人,日置酒高會。項王乃西從蕭,晨擊漢軍而東,至彭城,日中,大破漢軍。漢軍皆走,相隨入穀、泗水,殺漢卒十餘萬人。漢卒皆南走山,楚又追擊至靈壁東睢水上。漢軍卻,為楚所擠,多殺,漢卒十餘萬人皆入睢水,睢水為之不流。圍漢王三匝。於是大風從西北而起,折木發屋,揚沙石,窈冥晝晦,逢迎楚軍。楚軍大亂,壞散,而漢王乃得與數十騎遁去,欲過沛,收家室而西,楚亦使人追之沛,取漢王家,家皆亡,不與漢王相見。漢王道逢得孝惠、魯元,乃載行。楚騎追漢王,漢王急,推墮孝惠、魯元車下,滕公常下收載之。如是者三。曰,雖急不可以驅,柰何棄之。於是遂得脫。求太公、呂后不相遇。審食其從太公、呂后閒行,求漢王,反遇楚軍。楚軍遂與歸,報項王,項王常置軍中。

是時呂后兄周呂侯為漢將兵居下邑,漢王閒往從之,稍稍收其士卒。至滎陽,諸敗軍皆會,蕭何亦發關中老弱未傅悉詣滎陽,復大振。楚起於彭城,常乘勝逐北,與漢戰滎陽南京、索閒,漢敗楚,楚以故不能過滎陽而西。

項王之救彭城,追漢王至滎陽,田橫亦得收齊,立田榮子廣為齊王。漢王之敗彭城,諸侯皆復與楚而背漢。漢軍滎陽,筑甬道屬之河,以取敖倉粟。漢之三年,項王數侵奪漢甬道,漢王食乏,恐,請和,割滎陽以西為漢。

項王欲聽之。歷陽侯范增曰,漢易與耳,今釋弗取,後必悔之。項王乃與范增急圍滎陽。漢王患之,乃用陳平計閒項王。項王使者來,為太牢具,舉欲進之。見使者,詳驚愕曰,吾以為亞父使者,乃反項王使者。更持去,以惡食食項王使者。使者歸報項王,項王乃疑范增與漢有私,稍奪之權。范增大怒,曰,天下事大定矣,君王自為之。願賜骸骨歸卒伍。項王許之。行未至彭城,疽發背而死。

漢將紀信說漢王曰,事已急矣,請為王誑楚為王,王可以閒出。於是漢王夜出女子滎陽東門被甲二千人,楚兵四面擊之。紀信乘黃屋車,傅左纛,曰,城中食盡,漢王降。楚軍皆呼萬歲。漢王亦與數十騎從城西門出,走成皋。項王見紀信,問,漢王安在。曰,漢王已出矣。項王燒殺紀信。

漢王使御史大夫周苛、樅公、魏豹守滎陽。周苛、樅公謀曰,反國之王,難與守城。乃共殺魏豹。楚下滎陽城,生得周苛。項王謂周苛曰,為我將,我以公為上將軍,封三萬戶。周苛罵曰,若不趣降漢,漢今虜若,若非漢敵也。項王怒,烹周苛,井殺樅公。

漢王之出滎陽,南走宛、葉,得九江王布,行收兵,復入保成皋。漢之四年,項王進兵圍成皋。漢王逃,獨與滕公出成皋北門,渡河走修武,從張耳、韓信軍。諸將稍稍得出成皋,從漢王。楚遂拔成皋,欲西。漢使兵距之鞏,令其不得西。

是時,彭越渡河擊楚東阿,殺楚將軍薛公。項王乃自東擊彭越。漢王得淮陰侯兵,欲渡河南。鄭忠說漢王,乃止壁河內。使劉賈將兵佐彭越,燒楚積聚。項王東擊破之,走彭越。漢王則引兵渡河,復取成皋,軍廣武,就敖倉食。項王已定東海來,西,與漢俱臨廣武而軍,相守數月。

當此時,彭越數反梁地,絕楚糧食,項王患之。為高俎,置太公其上,告漢王曰,今不急下,吾烹太公。漢王曰,吾與項羽俱北面受命懷王,曰約為兄弟,吾翁即若翁,必欲烹而翁,則幸分我一桮羹。項王怒,欲殺之。項伯曰,天下事未可知,且為天下者不顧家,雖殺之無益,只益禍耳。項王從之。

楚漢久相持未決,丁壯苦軍旅,老弱罷轉漕。項王謂漢王曰,天下匈匈數歲者,徒以吾兩人耳,願與漢王挑戰決雌雄,毋徒苦天下之民父子為也。漢王笑謝曰,吾寧鬬智,不能鬬力。項王令壯士出挑戰。漢有善騎射者樓煩,楚挑戰三合,樓煩輒射殺之。項王大怒,乃自被甲持戟挑戰。樓煩欲射之,項王瞋目叱之,樓煩目不敢視,手不敢發,遂走還入壁,不敢復出。漢王使人閒問之,乃項王也。漢王大驚。於是項王乃即漢王相與臨廣武閒而語。漢王數之,項王怒,欲一戰。漢王不聽,項王伏弩射中漢王。漢王傷,走入成皋。

項王聞淮陰侯已舉河北,破齊、趙,且欲擊楚,乃使龍且往擊之。淮陰侯與戰,騎將灌嬰擊之,大破楚軍,殺龍且。韓信因自立為齊王。項王聞龍且軍破,則恐,使盱臺人武涉往說淮陰侯。淮陰侯弗聽。是時,彭越復反,下梁地,絕楚糧。項王乃謂海春侯大司馬曹咎等曰,謹守成皋,則漢欲挑戰,慎勿與戰,毋令得東而已。我十五日必誅彭越,定梁地,復從將軍。乃東,行擊陳留、外黃。

外黃不下。數日,已降,項王怒,悉令男子年十五已上詣城東,欲阬之。外黃令舍人兒年十三,往說項王曰,彭越彊劫外黃,外黃恐,故且降,待大王。大王至,又皆阬之,百姓豈有歸心。從此以東,梁地十餘城皆恐,莫肯下矣。項王然其言,乃赦外黃當阬者。東至睢陽,聞之皆爭下項王。

漢果數挑楚軍戰,楚軍不出。使人辱之,五六日,大司馬怒,渡兵汜水。士卒半渡,漢擊之,大破楚軍,盡得楚國貨賂。大司馬咎、長史翳、塞王欣皆自剄汜水上。大司馬咎者,故蘄獄掾,長史欣亦故櫟陽獄吏,兩人嘗有德於項梁,是以項王信任之。當是時,項王在睢陽,聞海春侯軍敗,則引兵還。漢軍方圍鐘離眛於滎陽東,項王至,漢軍畏楚,盡走險阻。

是時,漢兵盛食多,項王兵罷食絕。漢遣陸賈說項王,請太公,項王弗聽。漢王復使侯公往說項王,項王乃與漢約,中分天下,割鴻溝以西者為漢,鴻溝而東者為楚。項王許之,即歸漢王父母妻子。軍皆呼萬歲。漢王乃封侯公為平國君。匿弗肯復見。曰,此天下辯士,所居傾國,故號為平國君。項王已約,乃引兵解而東歸。

漢欲西歸,張良、陳平說曰,漢有天下太半,而諸侯皆附之。楚兵罷食盡,此天亡楚之時也,不如因其機而遂取之。今釋弗擊,此所謂養虎自遺患也。漢王聽之。漢五年,漢王乃追項王至陽夏南,止軍,與淮陰侯韓信、建成侯彭越期會而擊楚軍。至固陵,而信、越之兵不會。楚擊漢軍,大破之。漢王復入壁,深塹而自守。謂張子房曰,諸侯不從約,為之柰何。對曰,楚兵且破,信、越未有分地,其不至固宜。君王能與共分天下,今可立致也。即不能,事未可知也。君王能自陳以東傅海,盡與韓信,睢陽以北至穀城,以與彭越,使各自為戰,則楚易敗也。漢王曰,善。於是乃發使者告韓信、彭越曰,并力擊楚。楚破,自陳以東傅海與齊王,睢陽以北至穀城與彭相國。使者至,韓信、彭越皆報曰,請今進兵。韓信乃從齊往,劉賈軍從壽春并行,屠城父,至垓下。大司馬周殷叛楚,以舒屠六,舉九江兵,隨劉賈、彭越皆會垓下,詣項王。

項王軍壁垓下,兵少食盡,漢軍及諸侯兵圍之數重。夜聞漢軍四面皆楚歌,項王乃大驚曰,漢皆已得楚乎。是何楚人之多也。項王則夜起,飲帳中。有美人名虞,常幸從,駿馬名騅,常騎之。於是項王乃悲歌慨,自為詩曰,力拔山兮氣蓋世,時不利兮騅不逝。騅不逝兮可柰何,虞兮虞兮柰若何。歌數闋,美人和之。項王泣數行下,左右皆泣,莫能仰視。

於是項王乃上馬騎,麾下壯士騎從者八百餘人,直夜潰圍南出,馳走。平明,漢軍乃覺之,令騎將灌嬰以五千騎追之。項王渡淮,騎能屬者百餘人耳。項王至陰陵,迷失道,問一田父,田父紿曰左。左,乃陷大澤中。以故漢追及之。項王乃復引兵而東,至東城,乃有二十八騎。漢騎追者數千人。項王自度不得脫。謂其騎曰,吾起兵至今八歲矣,身七十餘戰,所當者破,所擊者服,未嘗敗北,遂霸有天下。然今卒困於此,此天之亡我,非戰之罪也。今日固決死,願為諸君快戰,必三勝之,為諸君潰圍,斬將,刈旗,令諸君知天亡我,非戰之罪也。乃分其騎以為四隊,四向。漢軍圍之數重。項王謂其騎曰,吾為公取彼一將。令四面騎馳下,期山東為三處。於是項王大呼馳下,漢軍皆披靡,遂斬漢一將。是時,赤泉侯為騎將,追項王,項王瞋目而叱之,赤泉侯人馬俱驚,辟易數里與其騎會為三處。漢軍不知項王所在,乃分軍為三,復圍之。項王乃馳,復斬漢一都尉,殺數十百人,復聚其騎,亡其兩騎耳。乃謂其騎曰,何如。騎皆伏曰,如大王言。

於是項王乃欲東渡烏江。烏江亭長檥船待,謂項王曰,江東雖小,地方千里,眾數十萬人,亦足王也。願大王急渡。今獨臣有船,漢軍至,無以渡。項王笑曰,天之亡我,我何渡為。且籍與江東子弟八千人渡江而西,今無一人還,縱江東父兄憐而王我,我何面目見之。縱彼不言,籍獨不愧於心乎。乃謂亭長曰,吾知公長者。吾騎此馬五歲,所當無敵,嘗一日行千里,不忍殺之,以賜公。乃令騎皆下馬步行,持短兵接戰。獨籍所殺漢軍數百人。項王身亦被十餘創。顧見漢騎司馬呂馬童,曰,若非吾故人乎。馬童面之,指王翳曰,此項王也。項王乃曰,吾聞漢購我頭千金,邑萬戶,吾為若德。乃自刎而死。王翳取其頭,餘騎相蹂踐爭項王,相殺者數十人。最其後,郎中騎楊喜,騎司馬呂馬童,郎中呂勝、楊武各得其一體。五人共會其體,皆是。故分其地為五,封呂馬童為中水侯,封王翳為杜衍侯,封楊喜為赤泉侯,封楊武為吳防侯,封呂勝為涅陽侯。

項王已死,楚地皆降漢,獨魯不下。漢乃引天下兵欲屠之,為其守禮義,為主死節,乃持項王頭視魯,魯父兄乃降。始,楚懷王初封項籍為魯公,及其死,魯最後下,故以魯公禮葬項王穀城。漢王為發哀,泣之而去。

諸項氏枝屬,漢王皆不誅。乃封項伯為射陽侯。桃侯、平皋侯、玄武侯皆項氏,賜姓劉。

太史公曰,吾聞之周生曰舜目蓋重瞳子,又聞項羽亦重瞳子。羽豈其苗裔邪。何興之暴也。夫秦失其政,陳涉首難,豪傑蜂起,相與并爭,不可勝數。然羽非有尺寸,乘執起隴畝之中,三年,遂將五諸侯滅秦,分裂天下,而封王侯,政由羽出,號為霸王,位雖不終,近古以來未嘗有也。及羽背關懷楚,放逐義帝而自立,怨王侯叛己,難矣。自矜功伐,奮其私智而不師古,謂霸王之業,欲以力征經營天下,五年卒亡其國,身死東城,尚不覺寤而不自責,過矣。乃引天亡我,非用兵之罪也,豈不謬哉。