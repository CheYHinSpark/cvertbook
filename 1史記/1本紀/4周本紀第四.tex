\chapter{周本紀第四}

周后稷,名棄。其母有邰氏女,曰姜原。姜原為帝嚳元妃。姜原出野,見巨人跡,心忻然說,欲踐之,踐之而身動如孕者。居期而生子,以為不祥,棄之隘巷,馬牛過者皆辟不踐,徙置之林中,適會山林多人,遷之,而棄渠中冰上,飛鳥以其翼覆薦之。姜原以為神,遂收養長之。初欲棄之,因名曰棄。

棄為兒時,屹如巨人之志。其游戲,好種樹麻、菽,麻、菽美。及為成人,遂好耕農,相地之宜,宜穀者稼穡焉,民皆法則之。帝堯聞之,舉棄為農師,天下得其利,有功。帝舜曰,棄,黎民始饑,爾后稷播時百穀。封棄於邰,號曰后稷,別姓姬氏。后稷之興,在陶唐﹑虞﹑夏之際,皆有令德。

后稷卒,子不窋立。不窋末年,夏后氏政衰,去稷不務,不窋以失其官而犇戎狄之間。不窋卒,子鞠立。鞠卒,子公劉立。公劉雖在戎狄之間,復修后稷之業,務耕種,行地宜,自漆、沮度渭,取材用,行者有資,居者有畜積,民賴其慶。百姓懷之,多徙而保歸焉。周道之興自此始,故詩人歌樂思其德。公劉卒,子慶節立,國於豳。

慶節卒,子皇仆立。皇仆卒,子差弗立。差弗卒,子毀隃立。毀隃卒,子公非立。公非卒,子高圉立。高圉卒,子亞圉立。亞圉卒,子公叔祖類立。公叔祖類卒,子古公亶父立。古公亶父復修后稷、公劉之業,積德行義,國人皆戴之。薰育戎狄攻之,欲得財物,予之。已復攻,欲得地與民。民皆怒,欲戰。古公曰,有民立君,將以利之。今戎狄所為攻戰,以吾地與民。民之在我,與其在彼,何異。民欲以我故戰,殺人父子而君之,予不忍為。乃與私屬遂去豳,度漆、沮,踰梁山,止於岐下。豳人舉國扶老攜弱,盡復歸古公於岐下。及他旁國聞古公仁,亦多歸之。於是古公乃貶戎狄之俗,而營筑城郭室屋,而邑別居之。作五官有司。民皆歌樂之,頌其德。

古公有長子曰太伯,次曰虞仲。太姜生少子季歷,季歷娶太任,皆賢婦人,生昌,有聖瑞。古公曰,我世當有興者,其在昌乎。長子太伯、虞仲知古公欲立季歷以傳昌,乃二人亡如荊蠻,文身斷髪,以讓季歷。

古公卒,季歷立,是為公季。公季修古公遺道,篤於行義,諸侯順之。

公季卒,子昌立,是為西伯。西伯曰文王,遵后稷、公劉之業,則古公、公季之法,篤仁,敬老,慈少。禮下賢者,日中不暇食以待士,士以此多歸之。伯夷、叔齊在孤竹,聞西伯善養老,盍往歸之。太顛、閎夭、散宜生、鬻子、辛甲大夫之徒皆往歸之。

崇侯虎譖西伯於殷紂曰,西伯積善累德,諸侯皆向之,將不利於帝。帝紂乃囚西伯於羑里。閎夭之徒患之。乃求有莘氏美女,驪戎之文馬,有熊九駟,他奇怪物,因殷嬖臣費仲而獻之紂。紂大說,曰,此一物足以釋西伯,況其多乎。乃赦西伯,賜之弓矢斧鉞,使西伯得征伐。曰,譖西伯者,崇侯虎也。西伯乃獻洛西之地,以請紂去炮格之刑。紂許之。

西伯陰行善,諸侯皆來決平。於是虞、芮之人有獄不能決,乃如周。入界,耕者皆讓畔,民俗皆讓長。虞、芮之人未見西伯,皆慚,相謂曰,吾所爭,周人所恥,何往為,秖取辱耳。遂還,俱讓而去。諸侯聞之,曰西伯蓋受命之君。

明年,伐犬戎。明年,伐密須。明年,敗耆國。殷之祖伊聞之,懼,以告帝紂。紂曰,不有天命乎。是何能為。明年,伐邘。明年,伐崇侯虎。而作豐邑,自岐下而徙都豐。明年,西伯崩,太子發立,是為武王。

西伯蓋即位五十年。其囚羑里,蓋益易之八卦為六十四卦。詩人道西伯,蓋受命之年稱王而斷虞芮之訟。後十年而崩,謚為文王。改法度,制正朔矣。追尊古公為太王,公季為王季,蓋王瑞自太王興。

武王即位,太公望為師,周公旦為輔,召公、畢公之徒左右王,師修文王緒業。

九年,武王上祭于畢。東觀兵,至于盟津。為文王木主,載以車,中軍。武王自稱太子發,言奉文王以伐,不敢自專。乃告司馬、司徒、司空、諸節,齊栗,信哉。予無知,以先祖有德臣,小子受先功,畢立賞罰,以定其功。遂興師。師尚父號曰,總爾眾庶,與爾舟楫,後至者斬。武王渡河,中流,白魚躍入王舟中,武王俯取以祭。既渡,有火自上復于下,至于王屋,流為烏,其色赤,其聲魄云。是時,諸侯不期而會盟津者八百諸侯。諸侯皆曰,紂可伐矣。武王曰,女未知天命,未可也。乃還師歸。

居二年,聞紂昏亂暴虐滋甚,殺王子比干,囚箕子。太師疵、少師彊抱其樂器而奔周。於是武王遍告諸侯曰,殷有重罪,不可以不畢伐。乃遵文王,遂率戎車三百乘,虎賁三千人,甲士四萬五千人,以東伐紂。十一年十二月戊午,師畢渡盟津,諸侯咸會。曰,孳孳無怠。武王乃作太誓,告于眾庶,今殷王紂乃用其婦人之言,自絕于天,毀壞其三正,離逖其王父母弟,乃斷棄其先祖之樂,乃為淫聲,用變亂正聲,怡說婦人。故今予發維共行天罰。勉哉夫子,不可再,不可三。

二月甲子昧爽,武王朝至于商郊牧野,乃誓。武王左杖黃鉞,右秉白旄,以麾。曰,遠矣西土之人。武王曰,嗟。我有國冢君,司徒、司馬、司空,亞旅、師氏,千夫長、百夫長,及庸、蜀、羌、髳、微、纑、彭、濮人,稱爾戈,比爾干,立爾矛,予其誓。王曰,古人有言牝雞無晨。牝雞之晨,惟家之索。今殷王紂維婦人言是用,自棄其先祖肆祀不答,昏棄其家國,遺其王父母弟不用,乃維四方之多罪逋逃是崇是長,是信是使,俾暴虐于百姓,以姦軌于商國。今予發維共行天之罰。今日之事,不過六步七步,乃止齊焉,夫子勉哉。不過於四伐五伐六伐七伐,乃止齊焉,勉哉夫子。尚桓桓,如虎如羆,如豺如離,于商郊,不御克奔,以役西土,勉哉夫子。爾所不勉,其于爾身有戮。誓已,諸侯兵會者車四千乘,陳師牧野。

帝紂聞武王來,亦發兵七十萬人距武王。武王使師尚父與百夫致師,以大卒馳帝紂師。紂師雖眾,皆無戰之心,心欲武王亟入。紂師皆倒兵以戰,以開武王。武王馳之,紂兵皆崩畔紂。紂走,反入登于鹿臺之上,蒙衣其殊玉,自燔于火而死。武王持大白旗以麾諸侯,諸侯畢拜武王,武王乃揖諸侯,諸侯畢從。武王至商國,商國百姓咸待於郊。於是武王使群臣告語商百姓曰,上天降休。商人皆再拜稽首,武王亦答拜。遂入,至紂死所。武王自射之,三發而後下車,以輕劍擊之,以黃鉞斬紂頭,縣大白之旗。已而至紂之嬖妾二女,二女皆經自殺。武王又射三發,擊以劍,斬以玄鉞,縣其頭小白之旗。武王已乃出復軍。

其明日,除道,修社及商紂宮。及期,百夫荷罕旗以先驅。武王弟叔振鐸奉陳常車,周公旦把大鉞,畢公把小鉞,以夾武王。散宜生、太顛、閎夭皆執劍以衛武王。既入,立于社南大卒之左,左右畢從。毛叔鄭奉明水,衛康叔封布茲,召公奭贊采,師尚父牽牲。尹佚筴祝曰,殷之末孫季紂,殄廢先王明德,侮蔑神祇不祀,昏暴商邑百姓,其章顯聞于天皇上帝。於是武王再拜稽首,曰,膺更大命,革殷,受天明命。武王又再拜稽首,乃出。

封商紂子祿父殷之餘民。武王為殷初定未集,乃使其弟管叔鮮、蔡叔度相祿父治殷。已而命召公釋箕子之囚。命畢公釋百姓之囚,表商容之閭。命南宮括散鹿臺之財,發鉅橋之粟,以振貧弱萌隸。命南宮括、史佚展九鼎保玉。命閎夭封比干之墓。命宗祝享祠于軍。乃罷兵西歸。行狩,記政事,作武成。封諸侯,班賜宗彝,作分殷之器物。武王追思先聖王,乃褒封神農之後於焦,黃帝之後於祝,帝堯之後於薊,帝舜之後於陳,大禹之後於杞。於是封功臣謀士,而師尚父為首封。封尚父於營丘,曰齊。封弟周公旦於曲阜,曰魯。封召公奭於燕。封弟叔鮮於管,弟叔度於蔡。餘各以次受封。

武王徵九牧之君,登豳之阜,以望商邑。武王至于周,自夜不寐。周公旦即王所,曰,曷為不寐。王曰,告女,維天不饗殷,自發未生於今六十年,麋鹿在牧,蜚鴻滿野。天不享殷,乃今有成。維天建殷,其登名民三百六十夫,不顯亦不賓滅,以至今。我未定天保,何暇寐。王曰,定天保,依天室,悉求夫惡,貶從殷王受。日夜勞來定我西土,我維顯服,及德方明。自洛汭延于伊汭,居易毋固,其有夏之居。我南望三涂,北望嶽鄙,顧詹有河,粵詹雒、伊,毋遠天室。營周居于雒邑而後去。縱馬於華山之陽,放牛於桃林之虛,偃干戈,振兵釋旅,示天下不復用也。

武王已克殷,後二年,問箕子殷所以亡。箕子不忍言殷惡,以存亡國宜告。武王亦醜,故問以天道。

武王病。天下未集,群公懼,穆卜,周公乃祓齋,自為質,欲代武王,武王有瘳。後而崩,太子誦代立,是為成王。

成王少,周初定天下,周公恐諸侯畔周,公乃攝行政當國。管叔、蔡叔群弟疑周公,與武庚作亂,畔周。周公奉成王命,伐誅武庚、管叔,放蔡叔。以微子開代殷後,國於宋。頗收殷餘民,以封武王少弟封為衛康叔。晉唐叔得嘉穀,獻之成王,成王以歸周公于兵所。周公受禾東土,魯天子之命。初,管、蔡畔周,周公討之,三年而畢定,故初作大誥,次作微子之命,次歸禾,次嘉禾,次康誥、酒誥、梓材,其事在周公之篇。周公行政七年,成王長,周公反政成王,北面就群臣之位。

成王在豐,使召公復營洛邑,如武王之意。周公復卜申視,卒營筑,居九鼎焉。曰,此天下之中,四方入貢道里均。作召誥、洛誥。成王既遷殷遺民,周公以王命告,作多士、無佚。召公為保,周公為師,東伐淮夷,殘奄,遷其君薄泵。成王自奄歸,在宗周,作多方。既絀殷命,襲淮夷,歸在豐,作周官。興正禮樂,度制於是改,而民和睦,頌聲興。成王既伐東夷,息慎來賀,王賜榮伯作賄息慎之命。

成王將崩,懼太子釗之不任,乃命召公、畢公率諸侯以相太子而立之。成王既崩,二公率諸侯,以太子釗見於先王廟,申告以文王、武王之所以為王業之不易,務在節儉,毋多欲,以篤信臨之,作顧命。太子釗遂立,是為康王。康王即位,遍告諸侯,宣告以文武之業以申之,作康誥。故成康之際,天下安寧,刑錯四十餘年不用。康王命作策畢公分居里,成周郊,作畢命。

康王卒,子昭王瑕立。昭王之時,王道微缺。昭王南巡狩不返,卒於江上。其卒不赴告,諱之也。立昭王子滿,是為穆王。穆王即位,春秋已五十矣。王道衰微,穆王閔文武之道缺,乃命伯臩申誡太仆國之政,作臩命。復寧。

穆王將征犬戎,祭公謀父諫曰,不可。先王燿德不觀兵。夫兵戢而時動,動則威,觀則玩,玩則無震。是故周文公之頌曰,載戢干戈,載櫜弓矢,我求懿德,肆于時夏,允王保之。先王之於民也,茂正其德而厚其性,阜其財求而利其器用,明利害之鄉,以文修之,使之務利而辟害,懷德而畏威,故能保世以滋大。昔我先王世后稷以服事虞、夏。及夏之衰也,棄稷不務,我先王不窋用失其官,而自竄於戎狄之閒。不敢怠業,時序其德,遵修其緒,修其訓典,朝夕恪勤,守以敦篤,奉以忠信。奕世載德,不忝前人。至于文王、武王,昭前之光明而加之以慈和,事神保民,無不欣喜。商王帝辛大惡于民,庶民不忍,訢載武王,以致戎于商牧。是故先王非務武也,勸恤民隱而除其害也。夫先王之制,邦內甸服,邦外侯服,侯衛賓服,夷蠻要服,戎翟荒服。甸服者祭,侯服者祀,賓服者享,要服者貢,荒服者王。日祭,月祀,時享,歲貢,終王。先王之順祀也,有不祭則修意,有不祀則修言,有不享則修文,有不貢則修名,有不王則修德,序成而有不至則修刑。於是有刑不祭,伐不祀,征不享,讓不貢,告不王。於是有刑罰之辟,有攻伐之兵,有征討之備,有威讓之命,有文告之辭。布令陳辭而有不至,則增修於德,無勤民於遠。是以近無不聽,遠無不服。今自大畢、伯士之終也,犬戎氏以其職來王,天子曰予必以不享征之,且觀之兵,無乃廢先王之訓,而王幾頓乎。吾聞犬戎樹敦,率舊德而守終純固,其有以御我矣。王遂征之,得四白狼四白鹿以歸。自是荒服者不至。

諸侯有不睦者,甫侯言於王,作修刑辟。王曰,吁,來。有國有土,告汝祥刑。在今爾安百姓,何擇非其人,何敬非其刑,何居非其宜與。兩造具備,師聽五辭。五辭簡信,正於五刑。五刑不簡,正於五罰。五罰不服,正於五過。五過之疵,官獄內獄,閱實其罪,惟鈞其過。五刑之疑有赦,五罰之疑有赦,其審克之。簡信有眾,惟訊有稽。無簡不疑,共嚴天威。黥辟疑赦,其罰百率,閱實其罪。劓辟疑赦,其罰倍灑,閱實其罪。臏辟疑赦,其罰倍差,閱實其罪。宮辟疑赦,其罰五百率,閱實其罪。大辟疑赦,其罰千率,閱實其罪。墨罰之屬千,劓罰之屬千,臏罰之屬五百,宮罰之屬三百,大辟之罰其屬二百,五刑之屬三千。命曰甫刑。

穆王立五十五年,崩,子共王繄扈立。共王游於涇上,密康公從,有三女奔之。其母曰,必致之王。夫獸三為群,人三為眾,女三為粲。王田不取群,公行不下眾,王御不參一族。夫粲,美之物也。眾以美物歸女,而何德以堪之。王猶不堪,況爾之小醜乎。小醜備物,終必亡。康公不獻,一年,共王滅密。共王崩,子懿王艱立。懿王之時,王室遂衰,詩人作刺。

懿王崩,共王弟辟方立,是為孝王。孝王崩,諸侯復立懿王太子燮,是為夷王。

夷王崩,子厲王胡立。厲王即位三十年,好利,近榮夷公。大夫芮良夫諫厲王曰,王室其將卑乎。夫榮公好專利而不知大難。夫利,百物之所生也,天地之所載也,而有專之,其害多矣。天地百物皆將取焉,何可專也。所怒甚多,不備大難。以是教王,王其能久乎。夫王人者,將導利而布之上下者也。使神人百物無不得極,猶日怵惕懼怨之來也。故頌曰思文后稷,克配彼天,立我蒸民,莫匪爾極。大雅曰陳錫載周。是不布利而懼難乎,故能載周以至于今。今王學專利,其可乎。匹夫專利,猶謂之盜,王而行之,其歸鮮矣。榮公若用,周必敗也。厲王不聽,卒以榮公為卿士,用事。

王行暴虐侈傲,國人謗王。召公諫曰,民不堪命矣。王怒,得衛巫,使監謗者,以告則殺之。其謗鮮矣,諸侯不朝。三十四年,王益嚴,國人莫敢言,道路以目。厲王喜,告召公曰,吾能弭謗矣,乃不敢言。召公曰,是鄣之也。防民之口,甚於防水。水壅而潰,傷人必多,民亦如之。是故為水者決之使導,為民者宣之使言。故天子聽政,使公卿至於列士獻詩,瞽獻曲,史獻書,師箴,瞍賦,矇誦,百工諫,庶人傳語,近臣盡規,親戚補察,瞽史教誨,耆艾修之,而後王斟酌焉,是以事行而不悖。民之有口也,猶土之有山川也,財用於是乎出,猶其有原隰衍沃也,衣食於是乎生。口之宣言也,善敗於是乎興。行善而備敗,所以產財用衣食者也。夫民慮之於心而宣之於口,成而行之。若壅其口,其與能幾何。王不聽。於是國莫敢出言,三年,乃相與畔,襲厲王。厲王出奔於彘。

厲王太子靜匿召公之家,國人聞之,乃圍之。召公曰,昔吾驟諫王,王不從,以及此難也。今殺王太子,王其以我為讎而懟怒乎。夫事君者,險而不讎懟,怨而不怒,況事王乎。乃以其子代王太子,太子竟得脫。

召公、周公二相行政,號曰共和。共和十四年,厲王死于彘。太子靜長於召公家,二相乃共立之為王,是為宣王。宣王即位,二相輔之,修政,法文、武、成、康之遺風,諸侯復宗周。十二年,魯武公來朝。

宣王不修籍於千畝,虢文公諫曰不可,王弗聽。三十九年,戰于千畝,王師敗績于姜氏之戎。

宣王既亡南國之師,乃料民於太原。仲山甫諫曰,民不可料也。宣王不聽,卒料民。

四十六年,宣王崩,子幽王宮湦立。幽王二年,西周三川皆震。伯陽甫曰,周將亡矣。夫天地之氣,不失其序,若過其序,民亂之也。陽伏而不能出,陰迫而不能蒸,於是有地震。今三川實震,是陽失其所而填陰也。陽失而在陰,原必塞,原塞,國必亡。夫水土演而民用也。土無所演,民乏財用,不亡何待。昔伊、洛竭而夏亡,河竭而商亡。今周德若二代之季矣,其川原又塞,塞必竭。夫國必依山川,山崩川竭,亡國之徵也。川竭必山崩。若國亡不過十年,數之紀也。天之所棄,不過其紀。是歲也,三川竭,岐山崩。

三年,幽王嬖愛褒姒。褒姒生子伯服,幽王欲廢太子。太子母申侯女,而為后。後幽王得褒姒,愛之,欲廢申后,并去太子宜臼,以褒姒為后,以伯服為太子。周太史伯陽讀史記曰,周亡矣。昔自夏后氏之衰也,有二神龍止於夏帝庭而言曰,余,褒之二君。夏帝卜殺之與去之與止之,莫吉。卜請其漦而藏之,乃吉。於是布幣而策告之,龍亡而漦在,櫝而去之。夏亡,傳此器殷。殷亡,又傳此器周。比三代,莫敢發之,至厲王之末,發而觀之。漦流于庭,不可除。厲王使婦人裸而譟之。漦化為玄黿,以入王後宮。後宮之童妾既齔而遭之,既笄而孕,無夫而生子,懼而棄之。宣王之時童女謠曰,厭弧箕服,實亡周國。於是宣王聞之,有夫婦賣是器者,宣王使執而戮之。逃於道,而見鄉者後宮童妾所棄妖子出於路者,聞其夜啼,哀而收之,夫婦遂亡,奔於褒。褒人有罪,請入童妾所棄女子者於王以贖罪。棄女子出於褒,是為褒姒。當幽王三年,王之後宮見而愛之,生子伯服,竟廢申后及太子,以褒姒為后,伯服為太子。太史伯陽曰,禍成矣,無可奈何。

褒姒不好笑,幽王欲其笑萬方,故不笑。幽王為烽燧大鼓,有寇至則舉烽火。諸侯悉至,至而無寇,褒姒乃大笑。幽王說之,為數舉烽火。其後不信,諸侯益亦不至。

幽王以虢石父為卿,用事,國人皆怨。石父為人佞巧善諛好利,王用之。又廢申后,去太子也。申侯怒,與繒、西夷犬戎攻幽王。幽王舉烽火徵兵,兵莫至。遂殺幽王驪山下,虜褒姒,盡取周賂而去。於是諸侯乃即申侯而共立故幽王太子宜臼,是為平王,以奉周祀。

平王立,東遷于雒邑,辟戎寇。平王之時,周室衰微,諸侯彊并弱,齊、楚、秦、晉始大,政由方伯。

四十九年,魯隱公即位。

五十一年,平王崩,太子洩父蚤死,立其子林,是為桓王。桓王,平王孫也。

桓王三年,鄭莊公朝,桓王不禮。五年,鄭怨,與魯易許田。許田,天子之用事太山田也。八年,魯殺隱公,立桓公。十三年,伐鄭,鄭射傷桓王,桓王去歸。

二十三年,桓王崩,子莊王佗立。莊王四年,周公黑肩欲殺莊王而立王子克。辛伯告王,王殺周公。王子克奔燕。

十五年,莊王崩,子釐王胡齊立。釐王三年,齊桓公始霸。

五年,釐王崩,子惠王閬立。惠王二年。初,莊王嬖姬姚,生子穨,穨有寵。及惠王即位,奪其大臣以為囿,故大夫邊伯等五人作亂,謀召燕、衛師,伐惠王。惠王犇溫,已居鄭之櫟。立釐王弟穨為王。樂及徧舞,鄭、虢君怒。四年,鄭與虢君伐殺王穨,復入惠王。惠王十年,賜齊桓公為伯。

二十五年,惠王崩,子襄王鄭立。襄王母蚤死,後母曰惠后。惠后生叔帶,有寵於惠王,襄王畏之。三年,叔帶與戎、翟謀伐襄王,襄王欲誅叔帶,叔帶奔齊。齊桓公使管仲平戎于周,使隰朋平戎于晉。王以上卿禮管仲。管仲辭曰,臣賤有司也,有天子之二守國、高在。若節春秋來承王命,何以禮焉。陪臣敢辭。王曰,舅氏,余嘉乃勳,毋逆朕命。管仲卒受下卿之禮而還。九年,齊桓公卒。十二年,叔帶復歸于周。

十三年,鄭伐滑,王使游孫、伯服請滑,鄭人囚之。鄭文公怨惠王之入不與厲公爵,又怨襄王之與衛滑,故囚伯服。王怒,將以翟伐鄭。富辰諫曰,凡我周之東徙,晉、鄭焉依。子穨之亂,又鄭之由定,今以小怨棄之。王不聽。十五年,王降翟師以伐鄭。王德翟人,將以其女為后。富辰諫曰,平、桓、莊、惠皆受鄭勞,王棄親親翟,不可從。王不聽。十六年,王絀翟后,翟人來誅,殺譚伯。富辰曰,吾數諫不從。如是不出,王以我為懟乎。乃以其屬死之。

初,惠后欲立王子帶,故以黨開翟人,翟人遂入周。襄王出奔鄭,鄭居王于汜。子帶立為王,取襄王所絀翟后與居溫。十七年,襄王告急于晉,晉文公納王而誅叔帶。襄王乃賜晉文公珪鬯弓矢,為伯,以河內地與晉。二十年,晉文公召襄王,襄王會之河陽、踐土,諸侯畢朝,書諱曰天王狩于河陽。

二十四年,晉文公卒。

三十一年,秦穆公卒。

三十二年,襄王崩,子頃王壬臣立。頃王六年,崩,子匡王班立。匡王六年,崩,弟瑜立,是為定王。

定王元年,楚莊王伐陸渾之戎,次洛,使人問九鼎。王使王孫滿應設以辭,楚兵乃去。十年,楚莊王圍鄭,鄭伯降,已而復之。十六年,楚莊王卒。

二十一年,定王崩,子簡王夷立。簡王十三年,晉殺其君厲公,迎子周於周,立為悼公。

十四年,簡王崩,子靈王泄心立。靈王二十四年,齊崔杼弒其君莊公。

二十七年,靈王崩,子景王貴立。景王十八年,后太子聖而蚤卒。二十年,景王愛子朝,欲立之,會崩,子丐之黨與爭立,國人立長子猛為王,子朝攻殺猛。猛為悼王。晉人攻子朝而立丐,是為敬王。

敬王元年,晉人入敬王,子朝自立,敬王不得入,居澤。四年,晉率諸侯入敬王于周,子朝為臣,諸侯城周。十六年,子朝之徒復作亂,敬王奔于晉。十七年,晉定公遂入敬王于周。

三十九年,齊田常殺其君簡公。

四十一年,楚滅陳。孔子卒。

四十二年,敬王崩,子元王仁立。元王八年,崩,子定王介立。

定王十六年,三晉滅智伯,分有其地。

二十八年,定王崩,長子去疾立,是為哀王。哀王立三月,弟叔襲殺哀王而自立,是為思王。思王立五月,少弟嵬攻殺思王而自立,是為考王。此三王皆定王之子。

考王十五年,崩,子威烈王午立。

考王封其弟于河南,是為桓公,以續周公之官職。桓公卒,子威公代立。威公卒,子惠公代立,乃封其少子於鞏以奉王,號東周惠公。

威烈王二十三年,九鼎震。命韓、魏、趙為諸侯。

二十四年,崩,子安王驕立。是歲盜殺楚聲王。

安王立二十六年,崩,子烈王喜立。烈王二年,周太史儋見秦獻公曰,始周與秦國合而別,別五百載復合,合十七歲而霸王者出焉。

十年,烈王崩,弟扁立,是為顯王。顯王五年,賀秦獻公,獻公稱伯。九年,致文武胙於秦孝公。二十五年,秦會諸侯於周。二十六年,周致伯於秦孝公。三十三年,賀秦惠王。三十五年,致文武胙於秦惠王。四十四年,秦惠王稱王。其後諸侯皆為王。

四十八年,顯王崩,子慎靚王定立。慎靚王立六年,崩,子赧王延立。王赧時東西周分治。王赧徙都西周。

西周武公之共太子死,有五庶子,毋適立。司馬翦謂楚王曰,不如以地資公子咎,為請太子。左成曰,不可。周不聽,是公之知困而交疏於周也。不如請周君孰欲立,以微告翦,翦請令楚賀資之以地。果立公子咎為太子。

八年,秦攻宜陽,楚救之。而楚以周為秦故,將伐之。蘇代為周說楚王曰,何以周為秦之禍也。言周之為秦甚於楚者,欲令周入秦也,故謂周秦也。周知其不可解,必入於秦,此為秦取周之精者也。為王計者,周於秦因善之,不於秦亦言善之,以疏之於秦。周絕於秦,必入於郢矣。

秦借道兩周之閒,將以伐韓,周恐借之畏於韓,不借畏於秦。史厭謂周君曰,何不令人謂韓公叔曰秦之敢絕周而伐韓者,信東周也。公何不與周地,發質使之楚。秦必疑楚不信周,是韓不伐也。又謂秦曰韓彊與周地,將以疑周於秦也,周不敢不受。秦必無辭而令周不受,是受地於韓而聽於秦。

秦召西周君,西周君惡往,故令人謂韓王曰,秦召西周君,將以使攻王之南陽也,王何不出兵於南陽。周君將以為辭於秦。周君不入秦,秦必不敢踰河而攻南陽矣。

東周與西周戰,韓救西周。或為東周說韓王曰,西周故天子之國,多名器重寶。王案兵毋出,可以德東周,而西周之寶必可以盡矣。

王赧謂成君。楚圍雍氏,韓徵甲與粟於東周,東周君恐,召蘇代而告之。代曰,君何患於是。臣能使韓毋徵甲與粟於周,又能為君得高都。周君曰,子茍能,請以國聽子。代見韓相國曰,楚圍雍氏,期三月也,今五月不能拔,是楚病也。今相國乃徵甲與粟於周,是告楚病也。韓相國曰,善。使者已行矣。代曰,何不與周高都。韓相國大怒曰,吾毋徵甲與粟於周亦已多矣,何故與周高都也。代曰,與周高都,是周折而入於韓也,秦聞之必大怒忿周,即不通周使,是以獘高都得完周也。曷為不與。相國曰,善。果與周高都。

三十四年,蘇厲謂周君曰,秦破韓、魏,撲師武,北取趙藺、離石者,皆白起也。是善用兵,又有天命。今又將兵出塞攻梁,梁破則周危矣。君何不令人說白起乎。曰楚有養由基者,善射者也。去柳葉百步而射之,百發而百中之。左右觀者數千人,皆曰善射。有一夫立其旁,曰善,可教射矣。養由基怒,釋弓搤劍,曰客安能教我射乎。客曰非吾能教子支左詘右也。夫去柳葉百步而射之,百發而百中之,不以善息,少焉氣衰力倦,弓撥矢鉤,一發不中者,百發盡息。今破韓、魏,撲師武,北取趙藺、離石者,公之功多矣。今又將兵出塞,過兩周,倍韓,攻梁,一舉不得,前功盡棄。公不如稱病而無出。

四十二年,秦破華陽約。馬犯謂周君曰,請令梁城周。乃謂梁王曰,周王病若死,則犯必死矣。犯請以九鼎自入於王,王受九鼎而圖犯。梁王曰,善。遂與之卒,言戍周。因謂秦王曰,梁非戍周也,將伐周也。王試出兵境以觀之。秦果出兵。又謂梁王曰,周王病甚矣,犯請後可而復之。今王使卒之周,諸侯皆生心,後舉事且不信。不若令卒為周城,以匿事端。梁王曰,善。遂使城周。

四十五年,周君之秦客謂周最聚曰,公不若譽秦王之孝,因以應為太后養地,秦王必喜,是公有秦交。交善,周君必以為公功。交惡,勸周君入秦者必有罪矣。秦攻周,而周聚謂秦王曰,為王計者不攻周。攻周,實不足以利,聲畏天下。天下以聲畏秦,必東合於齊。兵獘於周。合天下於齊,則秦不王矣。天下欲獘秦,勸王攻周。秦與天下獘,則令不行矣。

五十八年,三晉距秦。周令其相國之秦,以秦之輕也,還其行。客謂相國曰,秦之輕重未可知也。秦欲知三國之情。公不如急見秦王曰請為王聽東方之變,秦王必重公。重公,是秦重周,周以取秦也,齊重,則固有周聚以收齊,是周常不失重國之交也。秦信周,發兵攻三晉。

五十九年,秦取韓陽城負黍,西周恐,倍秦,與諸侯約從,將天下銳師出伊闕攻秦,令秦無得通陽城。秦昭王怒,使將軍摎攻西周。西周君犇秦,頓首受罪,盡獻其邑三十六,口三萬。秦受其獻,歸其君於周。

周君、王赧卒,周民遂東亡。秦取九鼎寶器,而遷西周公於憚狐。後七歲,秦莊襄王滅東西周。東西周皆入于秦,周既不祀。

太史公曰,學者皆稱周伐紂,居洛邑,綜其實不然。武王營之,成王使召公卜居,居九鼎焉,而周復都豐、鎬。至犬戎敗幽王,周乃東徙于洛邑。所謂周公葬我於畢,畢在鎬東南杜中。秦滅周。漢興九十有餘載,天子將封泰山,東巡狩至河南,求周苗裔,封其後嘉三十里地,號曰周子南君,比列侯,以奉其先祭祀。