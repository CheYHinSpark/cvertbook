\chapter{張丞相列傳第三十六}

張丞相蒼者,陽武人也。好書律歷。秦時為御史,主柱下方書。有罪,亡歸。及沛公略地過陽武,蒼以客從攻南陽。蒼坐法當斬,解衣伏質,身長大,肥白如瓠,時王陵見而怪其美士,乃言沛公,赦勿斬。遂從西入武關,至咸陽。沛公立為漢王,入漢中,還定三秦。陳餘擊走常山王張耳,耳歸漢,漢乃以張蒼為常山守。從淮陰侯擊趙,蒼得陳餘。趙地已平,漢王以蒼為代相,備邊寇。已而徙為趙相,相趙王耳。耳卒,相趙王敖。復徙相代王。燕王臧荼反,高祖往擊之。蒼以代相從攻臧荼有功,以六年中封為北平侯,食邑千二百戶。

遷為計相,一月,更以列侯為主計四歲。是時蕭何為相國,而張蒼乃自秦時為柱下史,明習天下圖書計籍。蒼又善用算律歷,故令蒼以列侯居相府,領主郡國上計者。黥布反亡,漢立皇子長為淮南王,而張蒼相之。十四年,遷為御史大夫。

周昌者,沛人也。其從兄曰周苛,秦時皆為泗水卒史。及高祖起沛,擊破泗水守監,於是周昌、周苛自卒史從沛公,沛公以周昌為職志,周苛為客。從入關,破秦。沛公立為漢王,以周苛為御史大夫,周昌為中尉。

漢王四年,楚圍漢王滎陽急,漢王遁出去,而使周苛守滎陽城。楚破滎陽城,欲令周苛將。苛罵曰,若趣降漢王。不然,今為虜矣。項羽怒,亨周苛。於是乃拜周昌為御史大夫。常從擊破項籍。以六年中與蕭、曹等俱封,封周昌為汾陰侯,周苛子周成以父死事,封為高景侯。

昌為人彊力,敢直言,自蕭、曹等皆卑下之。昌嘗燕時入奏事,高帝方擁戚姬,昌還走,高帝逐得,騎周昌項,問曰,我何如主也。昌仰曰,陛下即桀紂之主也。於是上笑之,然尤憚周昌。及帝欲廢太子,而立戚姬子如意為太子,大臣固爭之,莫能得,上以留侯策即止。而周昌廷爭之彊,上問其說,昌為人吃,又盛怒,曰,臣口不能言,然臣期期知其不可。陛下雖欲廢太子,臣期期不奉詔。上欣然而笑。既罷,呂后側耳於東箱聽,見周昌,為跪謝曰,微君,太子幾廢。

是後戚姬子如意為趙王,年十歲,高祖憂即萬歲之後不全也。趙堯年少,為符璽御史。趙人方與公謂御史大夫周昌曰,君之史趙堯,年雖少,然奇才也,君必異之,是且代君之位。周昌笑曰,堯年少,刀筆吏耳,何能至是乎。居頃之,趙堯侍高祖。高祖獨心不樂,悲歌,群臣不知上之所以然。趙堯進請問曰,陛下所為不樂,非為趙王年少而戚夫人與呂后有卻邪。備萬歲之後而趙王不能自全乎。高祖曰,然。吾私憂之,不知所出。堯曰,陛下獨宜為趙王置貴彊相,及呂后、太子、群臣素所敬憚乃可。高祖曰,然。吾念之欲如是,而群臣誰可者。堯曰,御史大夫周昌,其人堅忍質直,且自呂后、太子及大臣皆素敬憚之。獨昌可。高祖曰,善。於是乃召周昌,謂曰,吾欲固煩公,公彊為我相趙王。周昌泣曰,臣初起從陛下,陛下獨柰何中道而棄之於諸侯乎。高祖曰,吾極知其左遷,然吾私憂趙王,念非公無可者。公不得已彊行。於是徙御史大夫周昌為趙相。

既行久之,高祖持御史大夫印弄之,曰,誰可以為御史大夫者。孰視趙堯,曰,無以易堯。遂拜趙堯為御史大夫。堯亦前有軍功食邑,及以御史大夫從擊陳豨有功,封為江邑侯。

高祖崩,呂太后使使召趙王,其相周昌令王稱疾不行。使者三反,周昌固為不遣趙王。於是高后患之,乃使使召周昌。周昌至,謁高后,高后怒而罵周昌曰,爾不知我之怨戚氏乎。而不遣趙王,何。昌既徵,高后使使召趙王,趙王果來。至長安月餘,飲藥而死。周昌因謝病不朝見,三歲而死。

後五歲,高后聞御史大夫江邑侯趙堯高祖時定趙王如意之畫,乃抵堯罪,以廣阿侯任敖為御史大夫。

任敖者,故沛獄吏。高祖嘗辟吏,吏系呂后,遇之不謹。任敖素善高祖,怒,擊傷主呂后吏。及高祖初起,敖以客從為御史,守豐二歲,高祖立為漢王,東擊項籍,敖遷為上黨守。陳豨反時,敖堅守,封為廣阿侯,食千八百戶。高后時為御史大夫。三歲免,以平陽侯曹窋為御史大夫。高后崩,與大臣共誅呂祿等。免,以淮南相張蒼為御史大夫。

蒼與絳侯等尊立代王為孝文皇帝。四年,丞相灌嬰卒,張蒼為丞相。

自漢興至孝文二十餘年,會天下初定,將相公卿皆軍吏。張蒼為計相時,緒正律歷。以高祖十月始至霸上,因故秦時本以十月為歲首,弗革。推五德之運,以為漢當水德之時,尚黑如故。吹律調樂,入之音聲,及以比定律令。若百工,天下作程品。至於為丞相,卒就之,故漢家言律歷者,本之張蒼。蒼本好書,無所不觀,無所不通,而尤善律歷。

張蒼德王陵。王陵者,安國侯也。及蒼貴,常父事王陵。陵死後,蒼為丞相,洗沐,常先朝陵夫人上食,然後敢歸家。

蒼為丞相十餘年,魯人公孫臣上書言漢土德時,其符有黃龍當見。詔下其議張蒼,張蒼以為非是,罷之。其後黃龍見成紀,於是文帝召公孫臣以為博士,草土德之歷制度,更元年。張丞相由此自絀,謝病稱老。蒼任人為中候,大為姦利,上以讓蒼,蒼遂病免。蒼為丞相十五歲而免。孝景前五年,蒼卒,謚為文侯。子康侯代,八年卒。子類代為侯,八年,坐臨諸侯喪後就位不敬,國除。

初,張蒼父長不滿五尺,及生蒼,蒼長八尺餘,為侯、丞相。蒼子復長。及孫類,長六尺餘,坐法失侯。蒼之免相後,老,口中無齒,食乳,女子為乳母。妻妾以百數,嘗孕者不復幸。蒼年百有餘歲而卒。

申屠丞相嘉者,梁人,以材官蹶張從高帝擊項籍,遷為隊率。從擊黥布軍,為都尉。孝惠時,為淮陽守。孝文帝元年,舉故吏士二千石從高皇帝者,悉以為關內侯,食邑二十四人,而申屠嘉食邑五百戶。張蒼已為丞相,嘉遷為御史大夫。張蒼免相,孝文帝欲用皇后弟竇廣國為丞相,曰,恐天下以吾私廣國。廣國賢有行,故欲相之,念久之不可,而高帝時大臣又皆多死,餘見無可者,乃以御史大夫嘉為丞相,因故邑封為故安侯。

嘉為人廉直,門不受私謁。是時太中大夫鄧通方隆愛幸,賞賜累巨萬。文帝嘗燕飲通家,其寵如是。是時丞相入朝,而通居上傍,有怠慢之禮。丞相奏事畢,因言曰,陛下愛幸臣,則富貴之,至於朝廷之禮,不可以不肅。上曰,君勿言,吾私之。罷朝坐府中,嘉為檄召鄧通詣丞相府,不來,且斬通。通恐,入言文帝。文帝曰,汝第往,吾今使人召若。通至丞相府,免冠,徒跣,頓首謝。嘉坐自如,故不為禮,責曰,夫朝廷者,高皇帝之朝廷也。通小臣,戲殿上,大不敬,當斬。吏今行斬之。通頓首,首盡出血,不解。文帝度丞相已困通,使使者持節召通,而謝丞相曰,此吾弄臣,君釋之。鄧通既至,為文帝泣曰,丞相幾殺臣。

嘉為丞相五歲,孝文帝崩,孝景帝即位。二年,晁錯為內史,貴幸用事,諸法令多所請變更,議以謫罰侵削諸侯。而丞相嘉自絀所言不用,疾錯。錯為內史,門東出,不便,更穿一門南出。南出者,太上皇廟堧垣。嘉聞之,欲因此以法錯擅穿宗廟垣為門,奏請誅錯。錯客有語錯,錯恐,夜入宮上謁,自歸景帝。至朝,丞相奏請誅內史錯。景帝曰,錯所穿非真廟垣,乃外堧垣,故他官居其中,且又我使為之,錯無罪。罷朝,嘉謂長史曰,吾悔不先斬錯,乃先請之,為錯所賣。至舍,因歐血而死。謚為節侯。子共侯蔑代,三年卒。子侯去病代,三十一年卒。子侯臾代,六歲,坐為九江太守受故官送有罪,國除。

自申屠嘉死之後,景帝時開封侯陶青、桃侯劉舍為丞相。及今上時,柏至侯許昌、平棘侯薛澤、武彊侯莊青翟、高陵侯趙周等為丞相。皆以列侯繼嗣,娖娖廉謹,為丞相備員而已,無所能發明功名有著於當世者。

太史公曰,張蒼文學律歷,為漢名相,而絀賈生、公孫臣等言正朔服色事而不遵,明用秦之顓頊歷,何哉。周昌,木彊人也。任敖以舊德用。申屠嘉可謂剛毅守節矣,然無術學,殆與蕭、曹、陳平異矣。

孝武時丞相多甚,不記,莫錄其行起居狀略,且紀征和以來。

有車丞相,長陵人也。卒而有韋丞相代。韋丞相賢者,魯人也。以讀書術為吏,至大鴻臚。有相工相之,當至丞相。有男四人,使相工相之,至第二子,其名玄成。相工曰,此子貴,當封。韋丞相言曰,我即為丞相,有長子,是安從得之。後竟為丞相,病死,而長子有罪論,不得嗣,而立玄成。玄成時佯狂,不肯立,竟立之,有讓國之名。後坐騎至廟,不敬,有詔奪爵一級,為關內侯,失列侯,得食其故國邑。韋丞相卒,有魏丞相代。

魏丞相相者,濟陰人也。以文吏至丞相。其人好武,皆令諸吏帶劍,帶劍前奏事。或有不帶劍者,當入奏事,至乃借劍而敢入奏事。其時京兆尹趙君,丞相奏以免罪,使人執魏丞相,欲求脫罪而不聽。復使人脅恐魏丞相,以夫人賊殺待婢事而私獨奏請驗之,發吏卒至丞相舍,捕奴婢笞擊問之,實不以兵刃殺也。而丞相司直繁君奏京兆尹趙君迫脅丞相,誣以夫人賊殺婢,發吏卒圍捕丞相舍,不道,又得擅屏騎士事,趙京兆坐要斬。又有使掾陳平等劾中尚書,疑以獨擅劫事而坐之,大不敬,長史以下皆坐死,或下蠶室。而魏丞相竟以丞相病死。子嗣。後坐騎至廟,不敬,有詔奪爵一級,為關內侯,失列侯,得食其故國邑。魏丞相卒,以御史大夫邴吉代。

邴丞相吉者,魯國人也。以讀書好法令至御史大夫。孝宣帝時,以有舊故,封為列侯,而因為丞相。明於事,有大智,後世稱之。以丞相病死。子顯嗣。後坐騎至廟,不敬,有詔奪爵一級,失列侯,得食故國邑。顯為吏至太仆,坐官秏亂,身及子男有姦贓,免為庶人。

邴丞相卒,黃丞相代。長安中有善相工田文者,與韋丞相、魏丞相、邴丞相微賤時會於客家,田文言曰,今此三君者,皆丞相也。其後三人竟更相代為丞相,何見之明也。

黃丞相霸者,淮陽人也。以讀書為吏,至潁川太守。治潁川,以禮義條教喻告化之。犯法者,風曉令自殺。化大行,名聲聞。孝宣帝下制曰,潁川太守霸,以宣布詔令治民,道不拾遺,男女異路,獄中無重囚。賜爵關內侯,黃金百斤。徵為京兆尹而至丞相,復以禮義為治。以丞相病死。子嗣,後為列侯。黃丞相卒,以御史大夫于定國代。于丞相已有廷尉傳,在張廷尉語中。于丞相去,御史大夫韋玄成代。

韋丞相玄成者,即前韋丞相子也。代父,後失列侯。其人少時好讀書,明於詩、論語。為吏至衛尉,徙為太子太傅。御史大夫薛君免,為御史大夫。于丞相乞骸骨免,而為丞相,因封故邑為扶陽侯。數年,病死。孝元帝親臨喪,賜賞甚厚。子嗣後。其治容容隨世俗浮沈,而見謂諂巧。而相工本謂之當為侯代父,而後失之,復自游宦而起,至丞相。父子俱為丞相,世閒美之,豈不命哉。相工其先知之。韋丞相卒,御史大夫匡衡代。

丞相匡衡者,東海人也。好讀書,從博士受詩。家貧,衡傭作以給食飲。才下,數射策不中,至九,乃中丙科。其經以不中科故明習。補平原文學卒史。數年,郡不尊敬。御史徵之,以補百石屬薦為郎,而補博士,拜為太子少傅,而事孝元帝。孝元好詩,而遷為光祿勳,居殿中為師,授教左右,而縣官坐其旁聽,甚善之,日以尊貴。御史大夫鄭弘坐事免,而匡君為御史大夫。歲餘,韋丞相死,匡君代為丞相,封樂安侯。以十年之閒,不出長安城門而至丞相,豈非遇時而命也哉。

太史公曰,深惟士之游宦所以至封侯者,微甚。然多至御史大夫即去者。諸為大夫而丞相次也,其心冀幸丞相物故也。或乃陰私相毀害,欲代之。然守之日久不得,或為之日少而得之,至於封侯,真命也夫。御史大夫鄭君守之數年不得,匡君居之未滿歲,而韋丞相死,即代之矣,豈可以智巧得哉。多有賢聖之才,困妯囡者眾甚也。