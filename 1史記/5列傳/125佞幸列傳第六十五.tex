\chapter{佞幸列傳第六十五}

諺曰力田不如逢年,善仕不如遇合,固無虛言。非獨女以色媚,而士宦亦有之。

昔以色幸者多矣。至漢興,高祖至暴抗也,然籍孺以佞幸,孝惠時有閎孺。此兩人非有材能,徒以婉佞貴幸,與上臥起,公卿皆因關說。故孝惠時郎侍中皆冠鵕鸃,貝帶,傅脂粉,化閎、籍之屬也。兩人徙家安陵。

孝文時中寵臣,士人則鄧通,宦者則趙同、北宮伯子。北宮伯子以愛人長者,而趙同以星氣幸,常為文帝參乘,鄧通無伎能。鄧通,蜀郡南安人也,以濯船為黃頭郎。孝文帝夢欲上天,不能,有一黃頭郎從後推之上天,顧見其衣裻帶后穿。覺而之漸臺,以夢中陰目求推者郎,即見鄧通,其衣后穿,夢中所見也。召問其名姓,姓鄧氏,名通,文帝說焉,尊幸之日異。通亦願謹,不好外交,雖賜洗沐,不欲出。於是文帝賞賜通巨萬以十數,官至上大夫。文帝時時如鄧通家遊戲。然鄧通無他能,不能有所薦士,獨自謹其身以媚上而已。上使善相者相通,曰當貧餓死。文帝曰,能富通者在我也。何謂貧乎。於是賜鄧通蜀嚴道銅山,得自鑄錢,鄧氏錢布天下。其富如此。

文帝嘗病癰,鄧通常為帝唶吮之。文帝不樂,從容問通曰,天下誰最愛我者乎。通曰,宜莫如太子。太子入問病,文帝使唶癰,唶癰而色難之。已而聞鄧通常為帝唶吮之,心慚,由此怨通矣。及文帝崩,景帝立,鄧通免,家居。居無何,人有告鄧通盜出徼外鑄錢。下吏驗問,頗有之,遂竟案,盡沒入鄧通家,尚負責數巨萬。長公主賜鄧通,吏輒隨沒入之,一簪不得著身。於是長公主乃令假衣食。竟不得名一錢,寄死人家。

孝景帝時,中無寵臣,然獨郎中令周文仁,仁寵最過庸,乃不甚篤。

今天子中寵臣,士人則韓王孫嫣,宦者則李延年。嫣者,弓高侯孽孫也。今上為膠東王時,嫣與上學書相愛。及上為太子,愈益親嫣。嫣善騎射,善佞。上即位,欲事伐匈奴,而嫣先習胡兵,以故益尊貴,官至上大夫,賞賜擬於鄧通。時嫣常與上臥起。江都王入朝,有詔得從入獵上林中。天子車駕蹕道未行,而先使嫣乘副車,從數十百騎,騖馳視獸。江都王望見,以為天子,辟從者,伏謁道傍。嫣驅不見。既過,江都王怒,為皇太后泣曰,請得歸國入宿衛,比韓嫣。太后由此嗛嫣。嫣侍上,出入永巷不禁,以姦聞皇太后。皇太后怒,使使賜嫣死。上為謝,終不能得,嫣遂死。而案道侯韓說,其弟也,亦佞幸。

李延年,中山人也。父母及身兄弟及女,皆故倡也。延年坐法腐,給事狗中。而平陽公主言延年女弟善舞,上見,心說之,及入永巷,而召貴延年。延年善歌,為變新聲,而上方興天地祠,欲造樂詩歌弦之。延年善承意,弦次初詩。其女弟亦幸,有子男。延年佩二千石印,號協聲律。與上臥起,甚貴幸,埒如韓嫣也。久之,寖與中人亂,出入驕恣。及其女弟李夫人卒后,愛弛,則禽誅延年昆弟也。

自是之后,內寵嬖臣大底外戚之家,然不足數也。衛青、霍去病亦以外戚貴幸,然頗用材能自進。

太史公曰,甚哉愛憎之時。彌子瑕之行,足以觀后人佞幸矣。雖百世可知也。