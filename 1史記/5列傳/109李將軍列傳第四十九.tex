\chapter{李將軍列傳第四十九}

李將軍廣者,隴西成紀人也。其先曰李信,秦時為將,逐得燕太子丹者也。故槐裏,徙成紀。廣家世世受射。孝文帝十四年,匈奴大入蕭關,而廣以良家子從軍擊胡,用善騎射,殺首虜多,為漢中郎。廣從弟李蔡亦為郎,皆為武騎常侍,秩八百石。嘗從行,有所衝陷折關及格猛獸,而文帝曰,惜乎,子不遇時。如令子當高帝時,萬戶侯豈足道哉。

及孝景初立,廣為隴西都尉,徙為騎郎將。吳楚軍時,廣為驍騎都尉,從太尉亞夫擊吳楚軍,取旗,顯功名昌邑下。以梁王授廣將軍印,還,賞不行。徙為上谷太守,匈奴日以合戰。典屬國公孫昆邪為上泣曰,李廣才氣,天下無雙,自負其能,數與虜敵戰,恐亡之。於是乃徙為上郡太守。後廣轉為邊郡太守,徙上郡。嘗為隴西、北地、鴈門、代郡、雲中太守,皆以力戰為名。

匈奴大入上郡,天子使中貴人從廣勒習兵擊匈奴。中貴人將騎數十縱,見匈奴三人,與戰。三人還射,傷中貴人,殺其騎且盡。中貴人走廣。廣曰,是必射雕者也。廣乃遂從百騎往馳三人。三人亡馬步行,行數十里。廣令其騎張左右翼,而廣身自射彼三人者,殺其二人,生得一人,果匈奴射雕者也。已縛之上馬,望匈奴有數千騎,見廣,以為誘騎,皆驚,上山陳。廣之百騎皆大恐,欲馳還走。廣曰,吾去大軍數十里,今如此以百騎走,匈奴追射我立盡。今我留,匈奴必以我為大軍之誘,必不敢擊我。廣令諸騎曰,前。前未到匈奴陳二里所,止,令曰,皆下馬解鞍。其騎曰,虜多且近,即有急,柰何。廣曰,彼虜以我為走,今皆解鞍以示不走,用堅其意。於是胡騎遂不敢擊。有白馬將出護其兵,李廣上馬與十餘騎奔射殺胡白馬將,而復還至其騎中,解鞍,令士皆縱馬臥。是時會暮,胡兵終怪之,不敢擊。夜半時,胡兵亦以為漢有伏軍於旁欲夜取之,胡皆引兵而去。平旦,李廣乃歸其大軍。大軍不知廣所之,故弗從。

居久之,孝景崩,武帝立,左右以為廣名將也,於是廣以上郡太守為未央衛尉,而程不識亦為長樂衛尉。程不識故與李廣俱以邊太守將軍屯。及出擊胡,而廣行無部伍行陳,就善水草屯,舍止,人人自便,不擊刀鬬以自衛,莫府省約文書籍事,然亦遠斥候,未嘗遇害。程不識正部曲行伍營陳,擊刀鬬,士吏治軍簿至明,軍不得休息,然亦未嘗遇害。不識曰,李廣軍極簡易,然虜卒犯之,無以禁也,而其士卒亦佚樂,咸樂為之死。我軍雖煩擾,然虜亦不得犯我。是時漢邊郡李廣、程不識皆為名將,然匈奴畏李廣之略,士卒亦多樂從李廣而苦程不識。程不識孝景時以數直諫為太中大夫。為人廉,謹於文法。

後漢以馬邑城誘單于,使大軍伏馬邑旁谷,而廣為驍騎將軍,領屬護軍將軍。是時單于覺之,去,漢軍皆無功。其後四歲,廣以衛尉為將軍,出鴈門擊匈奴。匈奴兵多,破敗廣軍,生得廣。單于素聞廣賢,令曰,得李廣必生致之。胡騎得廣,廣時傷病,置廣兩馬閒,絡而盛臥廣。行十餘里,廣詳死,睨其旁有一胡兒騎善馬,廣暫騰而上胡兒馬,因推墮兒,取其弓,鞭馬南馳數十里,復得其餘軍,因引而入塞。匈奴捕者騎數百追之,廣行取胡兒弓,射殺追騎,以故得脫。於是至漢,漢下廣吏。吏當廣所失亡多,為虜所生得,當斬,贖為庶人。

頃之,家居數歲。廣家與故潁陰侯孫屏野居藍田南山中射獵。嘗夜從一騎出,從人田間飲。還至霸陵亭,霸陵尉醉,呵止廣。廣騎曰,故李將軍。尉曰,今將軍尚不得夜行,何乃故也。止廣宿亭下。居無何,匈奴入殺遼西太守,敗韓將軍,後韓將軍徙右北平。於是天子乃召拜廣為右北平太守。廣即請霸陵尉與俱,至軍而斬之。

廣居右北平,匈奴聞之,號曰漢之飛將軍,避之數歲,不敢入右北平。

廣出獵,見草中石,以為虎而射之,中石沒鏃,視之石也。因復更射之,終不能復入石矣。廣所居郡聞有虎,嘗自射之。及居右北平射虎,虎騰傷廣,廣亦竟射殺之。

廣廉,得賞賜輒分其麾下,飲食與士共之。終廣之身,為二千石四十餘年,家無餘財,終不言家產事。廣為人長,猨臂,其善射亦天性也,雖其子孫他人學者,莫能及廣。廣訥口少言,與人居則畫地為軍陳,射闊狹以飲。專以射為戲,竟死。廣之將兵,乏絕之處,見水,士卒不盡飲,廣不近水,士卒不盡食,廣不嘗食。寬緩不苛,士以此愛樂為用。其射,見敵急,非在數十步之內,度不中不發,發即應弦而倒。用此,其將兵數困辱,其射猛獸亦為所傷云。

居頃之,石建卒,於是上召廣代建為郎中令。元朔六年,廣復為後將軍,從大將軍軍出定襄,擊匈奴。諸將多中首虜率,以功為侯者,而廣軍無功。後二歲,廣以郎中令將四千騎出右北平,博望侯張騫將萬騎與廣俱,異道。行可數百里,匈奴左賢王將四萬騎圍廣,廣軍士皆恐,廣乃使其子敢往馳之。敢獨與數十騎馳,直貫胡騎,出其左右而還,告廣曰,胡虜易與耳。軍士乃安。廣為圜陳外向,胡急擊之,矢下如雨。漢兵死者過半,漢矢且盡。廣乃令士持滿毋發,而廣身自以大黃射其裨將,殺數人,胡虜益解。會日暮,吏士皆無人色,而廣意氣自如,益治軍。軍中自是服其勇也。明日,復力戰,而博望侯軍亦至,匈奴軍乃解去。漢軍罷,弗能追。是時廣軍幾沒,罷歸。漢法,博望侯留遲後期,當死,贖為庶人。廣軍功自如,無賞。

初,廣之從弟李蔡與廣俱事孝文帝。景帝時,蔡積功勞至二千石。孝武帝時,至代相。以元朔五年為輕車將車,從大將軍擊右賢王,有功中率,封為樂安侯。元狩二年中,代公孫弘為丞相。蔡為人在下中,名聲出廣下甚遠,然廣不得爵邑,官不過九卿,而蔡為列侯,位至三公。諸廣之軍吏及士卒或取封侯。廣嘗與望氣王朔燕語,曰,自漢擊匈奴而廣未嘗不在其中,而諸部校尉以下,才能不及中人,然以擊胡軍功取侯者數十人,而廣不為後人,然無尺寸之功以得封邑者,何也。豈吾相不當侯邪。且固命也。朔曰,將軍自念,豈嘗有所恨乎。廣曰,吾嘗為隴西守,羌嘗反,吾誘而降,降者八百餘人,吾詐而同日殺之。至今大恨獨此耳。朔曰,禍莫大於殺已降,此乃將軍所以不得侯者也。

後二歲,大將軍、驃騎將軍大出擊匈奴,廣數自請行。天子以為老,弗許,良久乃許之,以為前將軍。是歲,元狩四年也。

廣既從大將軍青擊匈奴,既出塞,青捕虜知單于所居,乃自以精兵走之,而令廣并於右將軍軍,出東道。東道少回遠,而大軍行水草少,其勢不屯行。廣自請曰,臣部為前將軍,今大將軍乃徙令臣出東道,且臣結發而與匈奴戰,今乃一得當單于,臣願居前,先死單于。大將軍青亦陰受上誡,以為李廣老,數奇,毋令當單于,恐不得所欲。而是時公孫敖新失侯,為中將軍從大將軍,大將軍亦欲使敖與俱當單于,故徙前將軍廣。廣時知之,固自辭於大將軍。大將軍不聽,令長史封書與廣之莫府,曰,急詣部,如書。廣不謝大將軍而起行,意甚慍怒而就部,引兵與右將軍食其合軍出東道。軍亡導,或失道,後大將軍。大將軍與單于接戰,單于遁走,弗能得而還。南絕幕,遇前將軍、右將軍。廣已見大將軍,還入軍。大將軍使長史持糒醪遺廣,因問廣、食其失道狀,青欲上書報天子軍曲折。廣未對,大將軍使長史急責廣之幕府對簿。廣曰,諸校尉無罪,乃我自失道。吾今自上簿。

至莫府,廣謂其麾下曰,廣結發與匈奴大小七十餘戰,今幸從大將軍出接單于兵,而大將軍又徙廣部行回遠,而又迷失道,豈非天哉。且廣年六十餘矣,終不能復對刀筆之吏。遂引刀自剄。廣軍士大夫一軍皆哭。百姓聞之,知與不知,無老壯皆為垂涕。而右將軍獨下吏,當死,贖為庶人。

廣子三人,曰當戶、椒、敢,為郎。天子與韓嫣戲,嫣少不遜,當戶擊嫣,嫣走。於是天子以為勇。當戶早死,拜椒為代郡太守,皆先廣死。當戶有遺腹子名陵。廣死軍時,敢從驃騎將軍。廣死明年,李蔡以丞相坐侵孝景園壖地,當下吏治,蔡亦自殺,不對獄,國除。李敢以校尉從驃騎將軍擊胡左賢王,力戰,奪左賢王鼓旗,斬首多,賜爵關內侯,食邑二百戶,代廣為郎中令。頃之,怨大將軍青之恨其父,乃擊傷大將軍,大將軍匿諱之。居無何,敢從上雍,至甘泉宮獵。驃騎將軍去病與青有親,射殺敢。去病時方貴幸,上諱云鹿觸殺之。居歲餘,去病死。而敢有女為太子中人,愛幸,敢男禹有寵於太子,然好利,李氏陵遲衰微矣。

李陵既壯,選為建章監,監諸騎。善射,愛士卒。天子以為李氏世將,而使將八百騎。嘗深入匈奴二千餘里,過居延視地形,無所見虜而還。拜為騎都尉,將丹陽楚人五千人,教射酒泉、張掖以屯衛胡。

數歲,天漢二年秋,貳師將軍李廣利將三萬騎擊匈奴右賢王於祁連天山,而使陵將其射士步兵五千人出居延北可千餘里,欲以分匈奴兵,毋令專走貳師也。陵既至期還,而單于以兵八萬圍擊陵軍。陵軍五千人,兵矢既盡,士死者過半,而所殺傷匈奴亦萬餘人。且引且戰,連鬬八日,還未到居延百餘里,匈奴遮狹絕道,陵食乏而救兵不到,虜急擊招降陵。陵曰,無面目報陛下。遂降匈奴。其兵盡沒,餘亡散得歸漢者四百餘人。

單于既得陵,素聞其家聲,及戰又壯,乃以其女妻陵而貴之。漢聞,族陵母妻子。自是之後,李氏名敗,而隴西之士居門下者皆用為恥焉。

太史公曰,傳曰,其身正,不令而行,其身不正,雖令不從。其李將軍之謂也。余睹李將軍悛悛如鄙人,口不能道辭。及死之日,天下知與不知,皆為盡哀。彼其忠實心誠信於士大夫也。諺曰桃李不言,下自成蹊。此言雖小,可以諭大也。