\chapter{魏其武安侯列傳第四十七}

魏其侯竇嬰者,孝文后從兄子也。父世觀津人。喜賓客。孝文時,嬰為吳相,病免。孝景初即位,為詹事。

梁孝王者,孝景弟也,其母竇太后愛之。梁孝王朝,因昆弟燕飲。是時上未立太子,酒酣,從容言曰,千秋之後傳梁王。太后驩。竇嬰引卮酒進上,曰,天下者,高祖天下,父子相傳,此漢之約也,上何以得擅傳梁王。太后由此憎竇嬰。竇嬰亦薄其官,因病免。太后除竇嬰門籍,不得入朝請。

孝景三年,吳楚反,上察宗室諸竇毋如竇嬰賢,乃召嬰。嬰入見,固辭謝病不足任。太后亦慚。於是上曰,天下方有急,王孫寧可以讓邪。乃拜嬰為大將軍,賜金千斤。嬰乃言袁盎、欒布諸名將賢士在家者進之。所賜金,陳之廊廡下,軍吏過,輒令財取為用,金無入家者。竇嬰守滎陽,監齊趙兵。七國兵已盡破,封嬰為魏其侯。諸游士賓客爭歸魏其侯。孝景時每朝議大事,條侯、魏其侯,諸列侯莫敢與亢禮。

孝景四年,立栗太子,使魏其侯為太子傅。孝景七年,栗太子廢,魏其數爭不能得。魏其謝病,屏居藍田南山之下數月,諸賓客辯士說之,莫能來。梁人高遂乃說魏其曰,能富貴將軍者,上也,能親將軍者,太后也。今將軍傅太子,太子廢而不能爭,爭不能得,又弗能死。自引謝病,擁趙女,屏閒處而不朝。相提而論,是自明揚主上之過。有如兩宮螫將軍,則妻子毋類矣。魏其侯然之,乃遂起,朝請如故。

桃侯免相,竇太后數言魏其侯。孝景帝曰,太后豈以為臣有愛,不相魏其。魏其者,沾沾自喜耳,多易。難以為相,持重。遂不用,用建陵侯衛綰為丞相。

武安侯田蚡者,孝景后同母弟也,生長陵。魏其已為大將軍後,方盛,蚡為諸郎,未貴,往來侍酒魏其,跪起如子姓。及孝景晚節,蚡益貴幸,為太中大夫。蚡辯有口,學槃盂諸書,王太后賢之。孝景崩,即日太子立,稱制,所鎮撫多有田蚡賓客計筴,蚡弟田勝,皆以太后弟,孝景後三年封蚡為武安侯,勝為周陽侯。

武安侯新欲用事為相,卑下賓客,進名士家居者貴之,欲以傾魏其諸將相。建元元年,丞相綰病免,上議置丞相、太尉。籍福說武安侯曰,魏其貴久矣,天下士素歸之。今將軍初興,未如魏其,即上以將軍為丞相,必讓魏其。魏其為丞相,將軍必為太尉。太尉、丞相尊等耳,又有讓賢名。武安侯乃微言太后風上,於是乃以魏其侯為丞相,武安侯為太尉。籍福賀魏其侯,因弔曰,君侯資性喜善疾惡,方今善人譽君侯,故至丞相,然君侯且疾惡,惡人眾,亦且毀君侯。君侯能兼容,則幸久,不能,今以毀去矣。魏其不聽。

魏其、武安俱好儒術,推轂趙綰為御史大夫,王臧為郎中令。迎魯申公,欲設明堂,令列侯就國,除關,以禮為服制,以興太平。舉適諸竇宗室毋節行者,除其屬籍。時諸外家為列侯,列侯多尚公主,皆不欲就國,以故毀日至竇太后。太后好黃老之言,而魏其、武安、趙綰、王臧等務隆推儒術,貶道家言,是以竇太后滋不說魏其等。及建元二年,御史大夫趙綰請無奏事東宮。竇太后大怒,乃罷逐趙綰、王臧等,而免丞相、太尉,以柏至侯許昌為丞相,武彊侯莊青翟為御史大夫。魏其、武安由此以侯家居。

武安侯雖不任職,以王太后故,親幸,數言事多效,天下吏士趨勢利者,皆去魏其歸武安,武安日益橫。建元六年,竇太后崩,丞相昌、御史大夫青翟坐喪事不辦,免。以武安侯蚡為丞相,以大司農韓安國為御史大夫。天下士郡諸侯愈益附武安。

武安者,貌侵,生貴甚。又以為諸侯王多長,上初即位,富於春秋,蚡以肺腑為京師相,非痛折節以禮詘之,天下不肅。當是時,丞相入奏事,坐語移日,所言皆聽。薦人或起家至二千石,權移主上。上乃曰,君除吏已盡未。吾亦欲除吏。嘗請考工地益宅,上怒曰,君何不遂取武庫。是後乃退。嘗召客飲,坐其兄蓋侯南鄉,自坐東鄉,以為漢相尊,不可以兄故私橈。武安由此滋驕,治宅甲諸第。田園極膏腴,而市買郡縣器物相屬於道。前堂羅鐘鼓,立曲旃,後房婦女以百數。諸侯奉金玉狗馬玩好,不可勝數。

魏其失竇太后,益疏不用,無勢,諸客稍稍自引而怠傲,唯灌將軍獨不失故。魏其日默默不得志,而獨厚遇灌將軍。

灌將軍夫者,潁陰人也。夫父張孟,嘗為潁陰侯嬰舍人,得幸,因進之至二千石,故蒙灌氏姓為灌孟。吳楚反時,潁陰侯灌何為將軍,屬太尉,請灌孟為校尉。夫以千人與父俱。灌孟年老,潁陰侯彊請之,郁郁不得意,故戰常陷堅,遂死吳軍中。軍法,父子俱從軍,有死事,得與喪歸。灌夫不肯隨喪歸,奮曰,願取吳王若將軍頭,以報父之仇。於是灌夫被甲持戟,募軍中壯士所善願從者數十人。及出壁門,莫敢前。獨二人及從奴十數騎馳入吳軍,至吳將麾下,所殺傷數十人。不得前,復馳還,走入漢壁,皆亡其奴,獨與一騎歸。夫身中大創十餘,適有萬金良藥,故得無死。夫創少瘳,又復請將軍曰,吾益知吳壁中曲折,請復往。將軍壯義之,恐亡夫,乃言太尉,太尉乃固止之。吳已破,灌夫以此名聞天下。

潁陰侯言之上,上以夫為中郎將。數月,坐法去。後家居長安,長安中諸公莫弗稱之。孝景時,至代相。孝景崩,今上初即位,以為淮陽天下交,勁兵處,故徙夫為淮陽太守。建元元年,入為太仆。二年,夫與長樂衛尉竇甫飲,輕重不得,夫醉,搏甫。甫,竇太后昆弟也。上恐太后誅夫,徙為燕相。數歲,坐法去官,家居長安。

灌夫為人剛直使酒,不好面諛。貴戚諸有勢在己之右,不欲加禮,必陵之,諸士在己之左,愈貧賤,尤益敬,與鈞。稠人廣眾,薦寵下輩。士亦以此多之。

夫不喜文學,好任俠,已然諾。諸所與交通,無非豪桀大猾。家累數千萬,食客日數十百人。陂池田園,宗族賓客為權利,橫於潁川。潁川兒乃歌之曰,潁水清,灌氏寧,潁水濁,灌氏族。

灌夫家居雖富,然失勢,卿相侍中賓客益衰。及魏其侯失勢,亦欲倚灌夫引繩批根生平慕之後棄之者。灌夫亦倚魏其而通列侯宗室為名高。兩人相為引重,其游如父子然。相得驩甚,無厭,恨相知晚也。

灌夫有服,過丞相。丞相從容曰,吾欲與仲孺過魏其侯,會仲孺有服。灌夫曰,將軍乃肯幸臨況魏其侯,夫安敢以服為解。請語魏其侯帳具,將軍旦日蚤臨。武安許諾。灌夫具語魏其侯如所謂武安侯。魏其與其夫人益市牛酒,夜灑埽,早帳具至旦。平明,令門下候伺。至日中,丞相不來。魏其謂灌夫曰,丞相豈忘之哉。灌夫不懌,曰,夫以服請,宜往。乃駕,自往迎丞相。丞相特前戲許灌夫,殊無意往。及夫至門,丞相尚臥。於是夫入見,曰,將軍昨日幸許過魏其,魏其夫妻治具,自旦至今,未敢嘗食。武安鄂謝曰,吾昨日醉,忽忘與仲孺言。乃駕往,又徐行,灌夫愈益怒。及飲酒酣,夫起舞屬丞相,丞相不起,夫從坐上語侵之。魏其乃扶灌夫去,謝丞相。丞相卒飲至夜,極驩而去。

丞相嘗使籍福請魏其城南田。魏其大望曰,老僕雖棄,將軍雖貴,寧可以勢奪乎。不許。灌夫聞,怒,罵籍福。籍福惡兩人有郤,乃謾自好謝丞相曰,魏其老且死,易忍,且待之。已而武安聞魏其、灌夫實怒不予田,亦怒曰,魏其子嘗殺人,蚡活之。蚡事魏其無所不可,何愛數頃田。且灌夫何與也。吾不敢複求田。武安由此大怨灌夫、魏其。

元光四年春,丞相言灌夫家在潁川,橫甚,民苦之。請案。上曰,此丞相事,何請。灌夫亦持丞相陰事,為奸利,受淮南王金與語言。賓客居間,遂止,俱解。

夏,丞相取燕王女為夫人,有太后詔,召列侯宗室皆往賀。魏其侯過灌夫,欲與俱。夫謝曰,夫數以酒失得過丞相,丞相今者又與夫有郤。魏其曰,事已解。彊與俱。飲酒酣,武安起為壽,坐皆避席伏。已魏其侯為壽,獨故人避席耳,餘半膝席。灌夫不悅。起行酒,至武安,武安膝席曰,不能滿觴。夫怒,因嘻笑曰,將軍貴人也,屬之。時武安不肯。行酒次至臨汝侯,臨汝侯方與程不識耳語,又不避席。夫無所發怒,乃罵臨汝侯曰,生平毀程不識不直一錢,今日長者為壽,乃效女兒呫囁耳語。武安謂灌夫曰,程李俱東西宮衛尉,今眾辱程將軍,仲孺獨不為李將軍地乎。灌夫曰,今日斬頭陷匈,何知程李乎。坐乃起更衣,稍稍去。魏其侯去,麾灌夫出。武安遂怒曰,此吾驕灌夫罪。乃令騎留灌夫。灌夫欲出不得。籍福起為謝,案灌夫項令謝。夫愈怒,不肯謝。武安乃麾騎縛夫置傳舍,召長史曰,今日召宗室,有詔。劾灌夫罵坐不敬,系居室。遂按其前事,遣吏分曹逐捕諸灌氏支屬,皆得棄市罪。魏其侯大媿,為資使賓客請,莫能解。武安吏皆為耳目,諸灌氏皆亡匿,夫系,遂不得告言武安陰事。

魏其銳身為救灌夫。夫人諫魏其曰,灌將軍得罪丞相,與太后家忤,寧可救邪。魏其侯曰,侯自我得之,自我捐之,無所恨。且終不令灌仲孺獨死,嬰獨生。乃匿其家,竊出上書。立召入,具言灌夫醉飽事,不足誅。上然之,賜魏其食,曰,東朝廷辯之。

魏其之東朝,盛推灌夫之善,言其醉飽得過,乃丞相以他事誣罪之。武安又盛毀灌夫所為橫恣,罪逆不道。魏其度不可奈何,因言丞相短。武安曰,天下幸而安樂無事,蚡得為肺腑,所好音樂狗馬田宅。蚡所愛倡優巧匠之屬,不如魏其、灌夫日夜招聚天下豪桀壯士與論議,腹誹而心謗,不仰視天而俯畫地,辟倪兩宮間,幸天下有變,而欲有大功。臣乃不知魏其等所為。於是上問朝臣,兩人孰是。御史大夫韓安國曰,魏其言灌夫父死事,身荷戟馳入不測之吳軍,身被數十創,名冠三軍,此天下壯士,非有大惡,爭杯酒,不足引他過以誅也。魏其言是也。丞相亦言灌夫通奸猾,侵細民,家累巨萬,橫恣潁川,淩轢宗室,侵犯骨肉,此所謂枝大於本,脛大於股,不折必披,丞相言亦是。唯明主裁之。主爵都尉汲黯是魏其。內史鄭當時是魏其,後不敢堅對。餘皆莫敢對。上怒內史曰,公平生數言魏其、武安長短,今日廷論,局趣效轅下駒,吾並斬若屬矣。即罷起入,上食太后。太后亦已使人候伺,具以告太后。太后怒,不食,曰,今我在也,而人皆藉吾弟,令我百歲後,皆魚肉之矣。且帝甯能為石人邪。此特帝在,即錄錄,設百歲後,是屬寧有可信者乎。上謝曰,俱宗室外家,故廷辯之。不然,此一獄吏所決耳。是時郎中令石建為上別言兩人事。

武安已罷朝,出止車門,召韓御史大夫載,怒曰,與長孺共一老禿翁,何為首鼠兩端。韓禦史良久謂丞相曰,君何不自喜。夫魏其毀君,君當免冠解印綬歸,曰臣以肺腑幸得待罪,固非其任,魏其言皆是。如此,上必多君有讓,不廢君。魏其必內愧,杜門齰舌自殺。今人毀君,君亦毀人,譬如賈豎女子爭言,何其無大體也。武安謝罪曰,爭時急,不知出此。

於是上使禦史簿責魏其所言灌夫,頗不讎,欺謾。劾系都司空。孝景時,魏其常受遺詔,曰事有不便,以便宜論上。及系,灌夫罪至族,事日急,諸公莫敢複明言於上。魏其乃使昆弟子上書言之,幸得複召見。書奏上,而案尚書大行無遺詔。詔書獨藏魏其家,家丞封。乃劾魏其矯先帝詔,罪當棄市。五年十月,悉論灌夫及家屬。魏其良久乃聞,聞即恚,病痱,不食欲死。或聞上無意殺魏其,魏其複食,治病,議定不死矣。乃有蜚語為惡言聞上,故以十二月晦論棄市渭城。

其春,武安侯病,專呼服謝罪。使巫視鬼者視之,見魏其、灌夫共守,欲殺之。竟死。子恬嗣。元朔三年,武安侯坐衣襜褕入宮,不敬。

淮南王安謀反覺,治。王前朝,武安侯為太尉,時迎王至霸上,謂王曰,上未有太子,大王最賢,高祖孫,即宮車晏駕,非大王立當誰哉。淮南王大喜,厚遺金財物。上自魏其時不直武安,特為太后故耳。及聞淮南王金事,上曰,使武安侯在者,族矣。

太史公曰,魏其、武安皆以外戚重,灌夫用一時決筴而名顯。魏其之舉以吳楚,武安之貴在日月之際。然魏其誠不知時變,灌夫無術而不遜,兩人相翼,乃成禍亂。武安負貴而好權,杯酒責望,陷彼兩賢。嗚呼哀哉。遷怒及人,命亦不延。眾庶不載,竟被惡言。嗚呼哀哉。禍所從來矣。