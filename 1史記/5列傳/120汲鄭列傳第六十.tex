\chapter{汲鄭列傳第六十}

汲黯字長孺,濮陽人也。其先有寵於古之衛君。至黯七世,世為卿大夫。黯以父任,孝景時為太子洗馬,以莊見憚。孝景帝崩,太子即位,黯為謁者。東越相攻,上使黯往視之。不至,至吳而還,報曰,越人相攻,固其俗然,不足以辱天子之使。河內失火,延燒千餘家,上使黯往視之。還報曰,家人失火,屋比延燒,不足憂也。臣過河南,河南貧人傷水旱萬餘家,或父子相食,臣謹以便宜,持節發河南倉粟以振貧民。臣請歸節,伏矯制之罪。上賢而釋之,遷為滎陽令。黯恥為令,病歸田里。上聞,乃召拜為中大夫。以數切諫,不得久留內,遷為東海太守。黯學黃老之言,治官理民,好清靜,擇丞史而任之。其治,責大指而已,不苛小。黯多病,臥閨閤內不出。歲餘,東海大治。稱之。上聞,召以為主爵都尉,列於九卿。治務在無為而已,弘大體,不拘文法。

黯為人性倨,少禮,面折,不能容人之過。合己者善待之,不合己者不能忍見,士亦以此不附焉。然好學,游俠,任氣節,內行修絜,好直諫,數犯主之顏色,常慕傅柏、袁盎之為人也。善灌夫、鄭當時及宗正劉棄。亦以數直諫,不得久居位。

當是時,太后弟武安侯蚡為丞相,中二千石來拜謁,蚡不為禮。然黯見蚡未嘗拜,常揖之。天子方招文學儒者,上曰吾欲云云,黯對曰,陛下內多欲而外施仁義,柰何欲效唐虞之治乎。上默然,怒,變色而罷朝。公卿皆為黯懼。上退,謂左右曰,甚矣,汲黯之戇也。群臣或數黯,黯曰,天子置公卿輔弼之臣,寧令從諛承意,陷主於不義乎。且已在其位,縱愛身,柰辱朝廷何。

黯多病,病且滿三月,上常賜告者數,終不愈。最後病,莊助為請告。上曰,汲黯何如人哉。助曰,使黯任職居官,無以踰人。然至其輔少主,守城深堅,招之不來,麾之不去,雖自謂賁育亦不能奪之矣。上曰,然。古有社稷之臣,至如黯,近之矣。

大將軍青侍中,上踞廁而視之。丞相弘燕見,上或時不冠。至如黯見,上不冠不見也。上嘗坐武帳中,黯前奏事,上不冠,望見黯,避帳中,使人可其奏。其見敬禮如此。

張湯方以更定律令為廷尉,黯數質責湯於上前,曰,公為正卿,上不能褒先帝之功業,下不能抑天下之邪心,安國富民,使囹圄空虛,二者無一焉。非苦就行,放析就功,何乃取斑皇帝約束紛更之為。公以此無種矣。黯時與湯論議,湯辯常在文深小苛,黯伉厲守高不能屈,忿發罵曰,天下謂刀筆吏不可以為公卿,果然。必湯也,令天下重足而立,側目而視矣。

是時,漢方征匈奴,招懷四夷。黯務少事,乘上閒,常言與胡和親,無起兵。上方向儒術,尊公孫弘。及事益多,吏民巧弄。上分別文法,湯等數奏決讞以幸。而黯常毀儒,面觸弘等徒懷詐飾智以阿人主取容,而刀筆吏專深文巧詆,陷人於罪,使不得反其真,以勝為功。上愈益貴弘、湯,弘、湯深心疾黯,唯天子亦不說也,欲誅之以事。弘為丞相,乃言上曰,右內史界部中多貴人宗室,難治,非素重臣不能任,請徙黯為右內史。為右內史數歲,官事不廢。

大將軍青既益尊,姊為皇后,然黯與亢禮。人或說黯曰,自天子欲群臣下大將軍,大將軍尊重益貴,君不可以不拜。黯曰,夫以大將軍有揖客,反不重邪。大將軍聞,愈賢黯,數請問國家朝廷所疑,遇黯過於平生。

淮南王謀反,憚黯,曰,好直諫,守節死義,難惑以非。至如說丞相弘,如發蒙振落耳。

天子既數征匈奴有功,黯之言益不用。

始黯列為九卿,而公孫弘、張湯為小吏。及弘、湯稍益貴,與黯同位,黯又非毀弘、湯等。已而弘至丞相,封為侯,湯至御史大夫,故黯時丞相史皆與黯同列,或尊用過之。黯褊心,不能無少望,見上,前言曰,陛下用群臣如積薪耳,后來者居上。上默然。有閒黯罷,上曰,人果不可以無學,觀黯之言也日益甚。

居無何,匈奴渾邪王率眾來降,漢發車二萬乘。縣官無錢,從民貰馬。民或匿馬,馬不具。上怒,欲斬長安令。黯曰,長安令無罪,獨斬黯,民乃肯出馬。且匈奴畔其主而降漢,漢徐以縣次傳之,何至令天下騷動,罷獘中國而以事夷狄之人乎。上默然。及渾邪至,賈人與市者,坐當死者五百餘人。黯請閒,見高門,曰,夫匈奴攻當路塞,絕和親,中國興兵誅之,死傷者不可勝計,而費以巨萬百數。臣愚以為陛下得胡人,皆以為奴婢以賜從軍死事者家,所鹵獲,因予之,以謝天下之苦,塞百姓之心。今縱不能,渾邪率數萬之眾來降,虛府庫賞賜,發良民侍養,譬若奉驕子。愚民安知市買長安中物而文吏繩以為闌出財物于邊關乎。陛下縱不能得匈奴之資以謝天下,又以微文殺無知者五百餘人,是所謂庇其葉而傷其枝者也,臣竊為陛下不取也。上默然,不許,曰,吾久不聞汲黯之言,今又復妄發矣。後數月,黯坐小法,會赦免官。於是黯隱於田園。

居數年,會更五銖錢,民多盜鑄錢,楚地尤甚。上以為淮陽,楚地之郊,乃召拜黯為淮陽太守。黯伏謝不受印,詔數彊予,然後奉詔。詔召見黯,黯為上泣曰,臣自以為填溝壑,不復見陛下,不意陛下復收用之。臣常有狗馬病,力不能任郡事,臣願為中郎,出入禁闥,補過拾遺,臣之願也。上曰,君薄淮陽邪。吾今召君矣。顧淮陽吏民不相得,吾徒得君之重,臥而治之。黯既辭行,過大行李息,曰,黯棄居郡,不得與朝廷議也。然御史大夫張湯智足以拒諫,詐足以飾非,務巧佞之語,辯數之辭,非肯正為天下言,專阿主意。主意所不欲,因而毀之,主意所欲,因而譽之。好興事,舞文法,內懷詐以御主心,外挾賊吏以為威重。公列九卿,不早言之,公與之俱受其僇矣。息畏湯,終不敢言。黯居郡如故治,淮陽政清。後張湯果敗,上聞黯與息言,抵息罪。令黯以諸侯相秩居淮陽。七歲而卒。

卒後,上以黯故,官其弟汲仁至九卿,子汲偃至諸侯相。黯姑姊子司馬安亦少與黯為太子洗馬。安文深巧善宦,官四至九卿,以河南太守卒。昆弟以安故,同時至二千石者十人。濮陽段宏始事蓋侯信,信任宏,宏亦再至九卿。然衛人仕者皆嚴憚汲黯,出其下。

鄭當時者,字莊,陳人也。其先鄭君嘗為項籍將,籍死,已而屬漢。高祖令諸故項籍臣名籍,鄭君獨不奉詔。詔盡拜名籍者為大夫,而逐鄭君。鄭君死孝文時。

鄭莊以任俠自喜,脫張羽於緦聲聞梁楚之閒。孝景時,為太子舍人。每五日洗沐,常置驛馬安諸郊,存諸故人,請謝賓客,夜以繼日,至其明旦,常恐不遍。莊好黃老之言,其慕長者如恐不見。年少官薄,然其游知交皆其大父行,天下有名之士也。武帝立,莊稍遷為魯中尉、濟南太守、江都相,至九卿為右內史。以武安侯魏其時議,貶秩為詹事,遷為大農令。

莊為太史,誡門下,客至,無貴賤無留門者。執賓主之禮,以其貴下人。莊廉,又不治其產業,仰奉賜以給諸公。然其餽遺人,不過算器食。每朝,候上之閒,說未嘗不言天下之長者。其推轂士及官屬丞史,誠有味其言之也,常引以為賢於己。未嘗名吏,與官屬言,若恐傷之。聞人之善言,進之上,唯恐後。山東士諸公以此翕然稱鄭莊。

鄭莊使視決河,自請治行五日。上曰,吾聞鄭莊行,千里不齎糧,請治行者何也。然鄭莊在朝,常趨和承意,不敢甚引當否。及晚節,漢征匈奴,招四夷,天下費多,財用益匱。莊任人賓客為大農僦人,多逋負。司馬安為淮陽太守,發其事,莊以此陷罪,贖為庶人。頃之,守長史。上以為老,以莊為汝南太守。數歲,以官卒。

鄭莊、汲黯始列為九卿,廉,內行修絜。此兩人中廢,家貧,賓客益落。及居郡,卒後家無餘貲財。莊兄弟子孫以莊故,至二千石六七人焉。

太史公曰,夫以汲、鄭之賢,有勢則賓客十倍,無勢則否,況眾人乎。下邽翟公有言,始翟公為廷尉,賓客闐門,及廢,門外可設雀羅。翟公復為廷尉,賓客欲往,翟公乃人署其門曰,一死一生,乃知交情。一貧一富,乃知交態。一貴一賤,交情乃見。汲、鄭亦云,悲夫。