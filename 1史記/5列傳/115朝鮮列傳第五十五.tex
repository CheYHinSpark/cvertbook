\chapter{朝鮮列傳第五十五}

朝鮮王滿者,故燕人也。自始全燕時嘗略屬真番、朝鮮,為置吏,筑鄣塞。秦滅燕,屬遼東外徼。漢興,為其遠難守,復修遼東故塞,至浿水為界,屬燕。燕王盧綰反,入匈奴,滿亡命,聚黨千餘人,魋結蠻夷服而東走出塞,渡浿水,居秦故空地上下鄣,稍役屬真番、朝鮮蠻夷及故燕、齊亡命者王之,都王險。

會孝惠、高后時天下初定,遼東太守即約滿為外臣,保塞外蠻夷,無使盜邊,諸蠻夷君長欲入見天子,勿得禁止。以聞,上許之,以故滿得兵威財物侵降其旁小邑,真番、臨屯皆來服屬,方數千里。

傳子至孫右渠,所誘漢亡人滋多,又未嘗入見,真番旁眾國欲上書見天子,又擁閼不通。元封二年,漢使涉何譙諭右渠,終不肯奉詔。何去至界上,臨浿水,使御刺殺送何者朝鮮裨王長,即渡,馳入塞,遂歸報天子曰殺朝鮮將。上為其名美,即不詰,拜何為遼東東部都尉。朝鮮怨何,發兵襲攻殺何。

天子募罪人擊朝鮮。其秋,遣樓船將軍楊仆從齊浮渤海,兵五萬人,左將軍荀彘出遼東,討右渠。右渠發兵距險。左將軍卒正多率遼東兵先縱,敗散,多還走,坐法斬。樓船將軍將齊兵七千人先至王險。右渠城守,窺知樓船軍少,即出城擊樓船,樓船軍敗散走。將軍楊仆失其眾,遁山中十餘日,稍求收散卒,復聚。左將軍擊朝鮮浿水西軍,未能破自前。

天子為兩將未有利,乃使衛山因兵威往諭右渠。右渠見使者頓首謝,願降,恐兩將詐殺臣,今見信節,請服降。遣太子入謝,獻馬五千匹,及饋軍糧。人眾萬餘,持兵,方渡浿水,使者及左將軍疑其為變,謂太子已服降,宜命人毋持兵。太子亦疑使者左將軍詐殺之,遂不渡浿水,復引歸。山還報天子,天子誅山。

左將軍破浿水上軍,乃前,至城下,圍其西北。樓船亦往會,居城南。右渠遂堅守城,數月未能下。

左將軍素侍中,幸,將燕代卒,悍,乘勝,軍多驕。樓船將齊卒,入海,固已多敗亡,其先與右渠戰,因辱亡卒,卒皆恐,將心慚,其圍右渠,常持和節。左將軍急擊之,朝鮮大臣乃陰閒使人私約降樓船,往來言,尚未肯決。左將軍數與樓船期戰,樓船欲急就其約,不會,左將軍亦使人求閒卻降下朝鮮,朝鮮不肯,心附樓船,以故兩將不相能。左將軍心意樓船前有失軍罪,今與朝鮮私善而又不降,疑其有反計,未敢發。天子曰將率不能,前乃使衛山諭降右渠,右渠遣太子,山使不能剸決,與左將軍計相誤,卒沮約。今兩將圍城,又乖異,以故久不決。使濟南太守公孫遂往正之,有便宜得以從事。遂至,左將軍曰,朝鮮當下久矣,不下者有狀。言樓船數期不會,具以素所意告遂,曰,今如此不取,恐為大害,非獨樓船,又且與朝鮮共滅吾軍。遂亦以為然,而以節召樓船將軍入左將軍營計事,即命左將軍麾下執捕樓船將軍,并其軍,以報天子。天子誅遂。

左將軍已并兩軍,即急擊朝鮮。朝鮮相路人、相韓陰、尼谿相參、將軍王唊相與謀曰,始欲降樓船,樓船今執,獨左將軍并將,戰益急,恐不能與,王又不肯降。陰、唊、路人皆亡降漢。路人道死。元封三年夏,尼谿相參乃使人殺朝鮮王右渠來降。王險城未下,故右渠之大臣成巳又反,復攻吏。左將軍使右渠子長降、相路人之子最告諭其民,誅成巳,以故遂定朝鮮,為四郡。封參為澅清侯,陰為荻苴侯,唊為平州侯,長降為幾侯。最以父死頗有功,為溫陽侯。

左將軍徵至,坐爭功相嫉,乖計,棄市。樓船將軍亦坐兵至洌口,當待左將軍,擅先縱,失亡多,當誅,贖為庶人。

太史公曰,右渠負固,國以絕祀。涉何誣功,為兵發首。樓船將狹,及難離咎。悔失番禺,乃反見疑。荀彘爭勞,與遂皆誅。兩軍俱辱,將率莫侯矣。