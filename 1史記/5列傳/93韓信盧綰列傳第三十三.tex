\chapter{韓信盧綰列傳第三十三}

韓王信者,故韓襄王孽孫也,長八尺五寸。及項梁之立楚後懷王也,燕、齊、趙、魏皆已前王,唯韓無有後,故立韓諸公子橫陽君成為韓王,欲以撫定韓故地。項梁敗死定陶,成奔懷王。沛公引兵擊陽城,使張良以韓司徒降下韓故地,得信,以為韓將,將其兵從沛公入武關。

沛公立為漢王,韓信從入漢中,乃說漢王曰,項王王諸將近地,而王獨遠居此,此左遷也。士卒皆山東人,跂而望歸,及其鋒東鄉,可以爭天下。漢王還定三秦,乃許信為韓王,先拜信為韓太尉,將兵略韓地。

項籍之封諸王皆就國,韓王成以不從無功,不遣就國,更以為列侯。及聞漢遣韓信略韓地,乃令故項籍游吳時吳令鄭昌為韓王以距漢。漢二年,韓信略定韓十餘城。漢王至河南,韓信急擊韓王昌陽城。昌降,漢王乃立韓信為韓王,常將韓兵從。三年,漢王出滎陽,韓王信、周苛等守滎陽。及楚敗滎陽,信降楚,已而得亡,復歸漢,漢復立以為韓王,竟從擊破項籍,天下定。五年春,遂與剖符為韓王,王潁川。

明年春,上以韓信材武,所王北近鞏、洛,南迫宛、葉,東有淮陽,皆天下勁兵處,乃詔徙韓王信王太原以北,備御胡,都晉陽。信上書曰,國被邊,匈奴數入,晉陽去塞遠,請治馬邑。上許之,信乃徙治馬邑。秋,匈奴冒頓大圍信,信數使使胡求和解。漢發兵救之,疑信數閒使,有二心,使人責讓信。信恐誅,因與匈奴約共攻漢,反,以馬邑降胡,擊太原。

七年冬,上自往擊,破信軍銅鞮,斬其將王喜。信亡走匈奴。其與白土人曼丘臣、王黃等立趙苗裔趙利為王,復收信敗散兵,而與信及冒頓謀攻漢。匈奴仗左右賢王將萬餘騎與王黃等屯廣武以南,至晉陽,與漢兵戰,漢大破之,追至于離石,復破之。匈奴復聚兵樓煩西北,漢令車騎擊破匈奴。匈奴常敗走,漢乘勝追北,聞冒頓居代谷,高皇帝居晉陽,使人視冒頓,還報曰可擊。上遂至平城。上出白登,匈奴騎圍上,上乃使人厚遺閼氏。閼氏乃說冒頓曰,今得漢地,猶不能居,且兩主不相緱居七日,胡騎稍引去。時天大霧,漢使人往來,胡不覺。護軍中尉陳平言上曰,胡者全兵,請令彊弩傅兩矢外向,徐行出圍。入平城,漢救兵亦到,胡騎遂解去。漢亦罷兵歸。韓信為匈奴將兵往來擊邊。

漢十年,信令王黃等說誤陳豨。十一年春,故韓王信復與胡騎入居參合,距漢。漢使柴將軍擊之,遺信書曰,陛下寬仁,諸侯雖有畔亡,而復歸,輒復故位號,不誅也。大王所知。今王以敗亡走胡,非有大罪,急自歸。韓王信報曰,陛下擢仆起閭巷,南面稱孤,此仆之幸也。滎陽之事,仆不能死,囚於項籍,此一罪也。及寇攻馬邑,仆不能堅守,以城降之,此二罪也。今反為寇將兵,與將軍爭一旦之命,此三罪也。夫種、蠡無一罪,身死亡,今仆有三罪於陛下,而欲求活於世,此伍子胥所以僨於吳也。今仆亡匿山谷閒,旦暮乞貸蠻夷,仆之思歸,如痿人不忘起,盲者不忘視也,勢不可耳。遂戰。柴將軍屠參合,斬韓王信。

信之入匈奴,與太子俱,及至穨當城,生子,因名曰穨當。韓太子亦生子,命曰嬰。至孝文十四年,穨當及嬰率其眾降漢。漢封穨當為弓高侯,嬰為襄城侯。吳楚軍時,弓高侯功冠諸將。傳子至孫,孫無子,失侯。嬰孫以不敬失侯。穨當孽孫韓嫣,貴幸,名富顯於當世。其弟說,再封,數稱將軍,卒為案道侯。子代,歲餘坐法死。後歲餘,說孫曾拜為龍頟侯,續說後。

盧綰者,豐人也,與高祖同里。盧綰親與高祖太上皇相愛,及生男,高祖、盧綰同日生,里中持羊酒賀兩家。及高祖、盧綰壯,俱學書,又相愛也。里中嘉兩家親相愛,生子同日,壯又相愛,復賀兩家羊酒。高祖為布衣時,有吏事辟匿,盧綰常隨出入上下。及高祖初起沛,盧綰以客從,入漢中為將軍,常侍中。從東擊項籍,以太尉常從,出入臥內,衣被飲食賞賜,群臣莫敢望,雖蕭曹等,特以事見禮,至其親幸,莫及盧綰。綰封為長安侯。長安,故咸陽也。

漢五年冬,以破項籍,乃使盧綰別將,與劉賈擊臨江王共尉,破之。七月還,從擊燕王臧荼,臧荼降。高祖已定天下,諸侯非劉氏而王者七人。欲王盧綰,為群臣觖望。及虜臧荼,乃下詔諸將相列侯,擇群臣有功者以為燕王。群臣知上欲王盧綰,皆言曰,太尉長安侯盧綰常從平定天下,功最多,可王燕。詔許之。漢五年八月,乃立虜綰為燕王。諸侯王得幸莫如燕王。

漢十一年秋,陳豨反代地,高祖如邯鄲擊豨兵,燕王綰亦擊其東北。當是時,陳豨使王黃求救匈奴。燕王綰亦使其臣張勝於匈奴,言豨等軍破。張勝至胡,故燕王臧茶子衍出亡在胡,見張勝曰,公所以重於燕者,以習胡事也。燕所以久存者,以諸侯數反,兵連不決也。今公為燕欲急滅豨等,豨等已盡,次亦至燕,公等亦且為虜矣。公何不令燕且緩陳豨而與胡和。事寬,得長王燕,即有漢急,可以安國。張勝以為然,豨私令匈奴助豨等擊燕。燕王綰疑張勝與胡反,上書請族張勝。勝還,具道所以為者。燕王寤,乃詐論它人,脫勝家屬,使得為匈奴閒,而陰使范齊之陳豨所,欲令久亡,連兵勿決。

漢十二年,東擊黥布,豨常將兵居代,漢使樊噲擊斬豨。其裨將降,言燕王綰使范齊通計謀於豨所。高祖使使召盧綰,綰稱病。上又使辟陽侯審食其、御史大夫趙堯往迎燕王,因驗問左右。綰愈恐,閉匿,謂其幸臣曰,非劉氏而王,獨我與長沙耳。往年春,漢族淮陰,夏,誅彭越,皆呂后計。今上病,屬任呂后。呂后婦人,專欲以事誅異姓王者及大功臣。乃遂稱病不行。其左右皆亡匿。語頗泄,辟陽侯聞之,歸具報上,上益怒。又得匈奴降者,降者言張勝亡在匈奴,為燕使。於是上曰,盧綰果反矣。使樊噲擊燕。燕王綰悉將其宮人家屬騎數千居長城下,侯伺,幸上病愈,自入謝。四月,高祖崩,盧綰遂將其眾亡入匈奴,匈奴以為東胡盧王。綰為蠻夷所侵奪,常思復歸。居歲餘,死胡中。

高后時,盧綰妻子亡降漢,會高后病,不能見,舍燕邸,為欲置酒見之。高祖竟崩,不得見。盧綰妻亦病死。

孝景中六年,盧綰孫他之,以東胡王降,封為亞谷侯。

陳豨者,宛朐人也,不知始所以得從。及高祖七年冬,韓王信反,入匈奴,上至平城還,乃封豨為列侯,以趙相國將監趙、代邊兵,邊兵皆屬焉。

豨常告歸過趙,趙相周昌見豨賓客隨之者千餘乘,邯鄲官舍皆滿。豨所以待賓客布衣交,皆出客下。豨還之代,周昌乃求入見。見上,具言豨賓客盛甚,擅兵於外數歲,恐有變。上乃令人覆案豨客居代者財物諸不法事,多連引豨。豨恐,陰令客通使王黃、曼丘臣所。及高祖十年七月,太上皇崩,使人召豨,豨稱病甚。九月,遂與王黃等反,自立為代王,劫略趙、代。

上聞,乃赦趙、代吏人為豨所詿誤劫略者,皆赦之。上自往,至邯鄲,喜曰,豨不南據漳水,北守邯鄲,知其無能為也。趙相奏斬常山守、尉,曰,常山二十五城,豨反,亡其二十城。上問曰,守、尉反乎。對曰,不反。上曰,是力不足也。赦之,復以為常山守、尉。上問周昌曰,趙亦有壯士可令將者乎。對曰,有四人。四人謁,上謾罵曰,豎子能為將乎。四人慚伏。上封之各千戶,以為將。左右諫曰,從入蜀、漢,伐楚,功未遍行,今此何功而封。上曰,非若所知。陳豨反,邯鄲以北皆豨有,吾以羽檄徵天下兵,未有至者,今唯獨邯鄲中兵耳。吾胡愛四千戶封四人,不以慰趙子弟。皆曰,善。於是上曰,陳豨將誰。曰,王黃、曼丘臣,皆故賈人。上曰,吾知之矣。乃各以千金購黃、臣等。

十一年冬,漢兵擊斬陳豨將侯敞、王黃於曲逆下,破豨將張春於聊城,斬首萬餘。太尉勃入定太原、代地。十二月,上自擊東垣,東垣不下,卒罵上,東垣降,卒罵者斬之,不罵者黥之。更命東垣為真定。王黃、曼丘臣其麾下受購賞之,皆生得,以故陳豨軍遂敗。

上還至洛陽。上曰,代居常山北,趙乃從山南有之,遠。乃立子恒為代王,都中都,代、鴈門皆屬代。

高祖十二年冬,樊噲軍卒追斬豨於靈丘。

太史公曰,韓信、盧綰非素積德累善之世,徼一時權變,以詐力成功,遭漢初定,故得列地,南面稱孤。內見疑彊大,外倚蠻貊以為援,是以日疏自危,事窮智困,卒赴匈奴,豈不哀哉。陳豨,梁人,其少時數稱慕魏公子,及將軍守邊,招致賓客而下士,名聲過實。周昌疑之,疵瑕頗起,懼禍及身,邪人進說,遂陷無道。於戲悲夫。夫計之生孰成敗於人也深矣。