\chapter{太史公自序第七十}

昔在顓頊,命南正重以司天,北正黎以司地。唐虞之際,紹重黎之後,使復典之,至于夏商,故重黎氏世序天地。其在周,程伯休甫其後也。當周宣王時,失其守而為司馬氏。司馬氏世典周史。惠襄之閒,司馬氏去周適晉。晉中軍隨會奔秦,而司馬氏入少梁。

自司馬氏去周適晉,分散,或在衛,或在趙,或在秦。其在衛者,相中山。在趙者,以傳劍論顯,蒯聵其後也。在秦者名錯,與張儀爭論,於是惠王使錯將伐蜀,遂拔,因而守之。錯孫靳,事武安君白起。而少梁更名曰夏陽。靳與武安君阬趙長平軍,還而與之俱賜死杜郵,葬於華池。靳孫昌,昌為秦主鐵官,當始皇之時。蒯聵玄孫卬為武信君將而徇朝歌。諸侯之相王,王卬於殷。漢之伐楚,卬歸漢,以其地為河內郡。昌生無澤,無澤為漢市長。無澤生喜,喜為五大夫,卒,皆葬高門。喜生談,談為太史公。

太史公學天官於唐都,受易於楊何,習道論於黃子。太史公仕於建元元封之閒,愍學者之不達其意而師悖,乃論六家之要指曰,

易大傳,天下一致而百慮,同歸而殊涂。夫陰陽、儒、墨、名、法、道德,此務為治者也,直所從言之異路,有省不省耳。嘗竊觀陰陽之術,大祥而眾忌諱,使人拘而多所畏,然其序四時之大順,不可失也。儒者博而寡要,勞而少功,是以其事難盡從,然其序君臣父子之禮,列夫婦長幼之別,不可易也。墨者儉而難遵,是以其事不可遍循,然其彊本節用,不可廢也。法家嚴而少恩,然其正君臣上下之分,不可改矣。名家使人儉而善失真,然其正名實,不可不察也。道家使人精神專一,動合無形,贍足萬物。其為術也,因陰陽之大順,采儒墨之善,撮名法之要,與時遷移,應物變化,立俗施事,無所不宜,指約而易操,事少而功多。儒者則不然。以為人主天下之儀表也,主倡而臣和,主先而臣隨。如此則主勞而臣逸。至於大道之要,去健羨,絀聰明,釋此而任術。夫神大用則竭,形大勞則敝。形神騷動,欲與天地長久,非所聞也。

夫陰陽四時、八位、十二度、二十四節各有教令,順之者昌,逆之者不死則亡,未必然也,故曰使人拘而多畏。夫春生夏長,秋收冬藏,此天道之大經也,弗順則無以為天下綱紀,故曰四時之大順,不可失也。

夫儒者以六藝為法。六藝經傳以千萬數,累世不能通其學,當年不能究其禮,故曰博而寡要,勞而少功。若夫列君臣父子之禮,序夫婦長幼之別,雖百家弗能易也。

墨者亦尚堯舜道,言其德行曰,堂高三尺,土階三等,茅茨不翦,采椽不刮。食土簋,啜土刑,糲粱之食,藜霍之羹。夏日葛衣,冬日鹿裘。其送死,桐棺三寸,舉音不盡其哀。教喪禮,必以此為萬民之率。使天下法若此,則尊卑無別也。夫世異時移,事業不必同,故曰儉而難遵。要曰彊本節用,則人給家足之道也。此墨子之所長,雖百長弗能廢也。

法家不別親疏,不殊貴賤,一斷於法,則親親尊尊之恩絕矣。可以行一時之計,而不可長用也,故曰嚴而少恩。若尊主卑臣,明分職不得相踰越,雖百家弗能改也。

名家苛察繳繞,使人不得反其意,專決於名而失人情,故曰使人儉而善失真。若夫控名責實,參伍不失,此不可不察也。

道家無為,又曰無不為,其實易行,其辭難知。其術以虛無為本,以因循為用。無成埶,無常形,故能究萬物之情。不為物先,不為物後,故能為萬物主。有法無法,因時為業,有度無度,因物與合。故曰聖人不朽,時變是守。虛者道之常也,因者君之綱也。群臣并至,使各自明也。其實中其聲者謂之端,實不中其聲者謂之窾。窾言不聽,姦乃不生,賢不肖自分,白黑乃形。在所欲用耳,何事不成。乃合大道,混混冥冥。光燿天下,復反無名。凡人所生者神也,所託者形也。神大用則竭,形大勞則敝,形神離則死。死者不可復生,離者不可復反,故聖人重之。由是觀之,神者生之本也,形者生之具也。不先定其神形,而曰我有以治天下,何由哉。

太史公既掌天官,不治民。有子曰遷。

遷生龍門,耕牧河山之陽。年十歲則誦古文。二十而南游江、淮,上會稽,探禹穴,闚九疑,浮於沅、湘,北涉汶、泗,講業齊、魯之都,觀孔子之遺風,鄉射鄒、嶧,戹困鄱、薛、彭城,過梁、楚以歸。於是遷仕為郎中,奉使西征巴、蜀以南,南略邛、笮、昆明,還報命。

是歲天子始建漢家之封,而太史公留滯周南,不得與從事,故發憤且卒。而子遷適使反,見父於河洛之閒。太史公執遷手而泣曰,余先周室之太史也。自上世嘗顯功名於虞夏,典天官事。後世中衰,絕於予乎。汝復為太史,則續吾祖矣。今天子接千歲之統,封泰山,而余不得從行,是命也夫,命也夫。余死,汝必為太史,為太史,無忘吾所欲論著矣。且夫孝始於事親,中於事君,終於立身。揚名於後世,以顯父母,此孝之大者。夫天下稱誦周公,言其能論歌文武之德,宣周邵之風,達太王王季之思慮,爰及公劉,以尊后稷也。幽厲之後,王道缺,禮樂衰,孔子修舊起廢,論詩書,作春秋,則學者至今則之。自獲麟以來四百有餘歲,而諸侯相兼,史記放絕。今漢興,海內一統,明主賢君忠臣死義之士,余為太史而弗論載,廢天下之史文,余甚懼焉,汝其念哉。遷俯首流涕曰,小子不敏,請悉論先人所次舊聞,弗敢闕。

卒三歲而遷為太史令,紬史記石室金匱之書。五年而當太初元年,十一月甲子朔旦冬至,天歷始改,建於明堂,諸神受紀。

太史公曰,先人有言,自周公卒五百歲而有孔子。孔子卒後至於今五百歲,有能紹明世,正易傳,繼春秋,本詩書禮樂之際。意在斯乎。意在斯乎。小子何敢讓焉。

上大夫壺遂曰,昔孔子何為而作春秋哉。太史公曰,余聞董生曰,周道衰廢,孔子為魯司寇,諸侯害之,大夫壅之。孔子知言之不用,道之不行也,是非二百四十二年之中,以為天下儀表,貶天子,退諸侯,討大夫,以達王事而已矣。子曰,我欲載之空言,不如見之於行事之深切著明也。夫春秋,上明三王之道,下辨人事之紀,別嫌疑,明是非,定猶豫,善善惡惡,賢賢賤不肖,存亡國,繼絕世,補敝起廢,王道之大者也。易著天地陰陽四時五行,故長於變,禮經紀人倫,故長於行,書記先王之事,故長於政,詩記山川谿谷禽獸草木牝牡雌雄,故長於風,樂樂所以立,故長於和,春秋辯是非,故長於治人。是故禮以節人,樂以發和,書以道事,詩以達意,易以道化,春秋以道義。撥亂世反之正,莫近於春秋。春秋文成數萬,其指數千。萬物之散聚皆在春秋。春秋之中,弒君三十六,亡國五十二,諸侯奔走不得保其社稷者不可勝數。察其所以,皆失其本已。故易曰失之豪釐,差以千里。故曰臣弒君,子弒父,非一旦一夕之故也,其漸久矣。故有國者不可以不知春秋,前有讒而弗見,後有賊而不知。為人臣者不可以不知春秋,守經事而不知其宜,遭變事而不知其權。為人君父而不通於春秋之義者,必蒙首惡之名。為人臣子而不通於春秋之義者,必陷篡弒之誅,死罪之名。其實皆以為善,為之不知其義,被之空言而不敢辭。夫不通禮義之旨,至於君不君,臣不臣,父不父,子不子。夫君不君則犯,臣不臣則誅,父不父則無道,子不子則不孝。此四行者,天下之大過也。以天下之大過予之,則受而弗敢辭。故春秋者,禮義之大宗也。夫禮禁未然之前,法施已然之後,法之所為用者易見,而禮之所為禁者難知。

壺遂曰,孔子之時,上無明君,下不得任用,故作春秋,垂空文以斷禮義,當一王之法。今夫子上遇明天子,下得守職,萬事既具,咸各序其宜,夫子所論,欲以何明。

太史公曰,唯唯,否否,不然。余聞之先人曰,伏羲至純厚,作易八卦。堯舜之盛,尚書載之,禮樂作焉。湯武之隆,詩人歌之。春秋采善貶惡,推三代之德,褒周室,非獨刺譏而已也。漢興以來,至明天子,獲符瑞,封禪,改正朔,易服色,受命於穆清,澤流罔極,海外殊俗,重譯款塞,請來獻見者,不可勝道。臣下百官力誦聖德,猶不能宣盡其意。且士賢能而不用,有國者之恥,主上明聖而德不布聞,有司之過也。且余嘗掌其官,廢明聖盛德不載,滅功臣世家賢大夫之業不述,墮先人所言,罪莫大焉。余所謂述故事,整齊其世傳,非所謂作也,而君比之於春秋,謬矣。

於是論次其文。七年而太史公遭李陵之禍,幽於縲紲。乃喟然而嘆曰,是余之罪也夫。是余之罪也夫。身毀不用矣。退而深惟曰,夫詩書隱約者,欲遂其志之思也。昔西伯拘羑里,演周易,孔子戹陳蔡,作春秋,屈原放逐,著離騷,左丘失明,厥有國語,孫子臏腳,而論兵法,不韋遷蜀,世傳呂覽,韓非囚秦,說難、孤憤,詩三百篇,大抵賢聖發憤之所為作也。此人皆意有所郁結,不得通其道也,故述往事,思來者。於是卒述陶唐以來,至于麟止,自黃帝始。

維昔黃帝,法天則地,四聖遵序,各成法度,唐堯遜位,虞舜不台,厥美帝功,萬世載之。作五帝本紀第一。

維禹之功,九州攸同,光唐虞際,德流苗裔,夏桀淫驕,乃放鳴條。作夏本紀第二。

維契作商,爰及成湯,太甲居桐,德盛阿衡,武丁得說,乃稱高宗,帝辛湛湎,諸侯不享。作殷本紀第三。

維棄作稷,德盛西伯,武王牧野,實撫天下,幽厲昏亂,既喪酆鎬,陵遲至赧,洛邑不祀。作周本紀第四。

維秦之先,伯翳佐禹,穆公思義,悼豪之旅,以人為殉,詩歌黃鳥,昭襄業帝。作秦本紀第五。

始皇既立,并兼六國,銷鋒鑄鐻,維偃干革,尊號稱帝,矜武任力,二世受運,子嬰降虜。作始皇本紀第六。

秦失其道,豪桀并擾,項梁業之,子羽接之,殺慶救趙,諸侯立之,誅嬰背懷,天下非之。作項羽本紀第七。

子羽暴虐,漢行功德,憤發蜀漢,還定三秦,誅籍業帝,天下惟寧,改制易俗。作高祖本紀第八。

惠之早霣,諸呂不台,崇彊祿、產,諸侯謀之,殺隱幽友,大臣洞疑,遂及宗禍。作呂太后本紀第九。

漢既初興,繼嗣不明,迎王踐祚,天下歸心,蠲除肉刑,開通關梁,廣恩博施,厥稱太宗。作孝文本紀第十。

諸侯驕恣,吳首為亂,京師行誅,七國伏辜,天下翕然,大安殷富。作孝景本紀第十一。

漢興五世,隆在建元,外攘夷狄,內修法度,封禪,改正朔,易服色。作今上本紀第十二。

維三代尚矣,年紀不可考,蓋取之譜牒舊聞,本于茲,於是略推,作三代世表第一。

幽厲之後,周室衰微,諸侯專政,春秋有所不紀,而譜牒經略,五霸更盛衰,欲睹周世相先後之意,作十二諸侯年表第二。

春秋之後,陪臣秉政,彊國相王,以至于秦,卒并諸夏,滅封地,擅其號。作六國年表第三。

秦既暴虐,楚人發難,項氏遂亂,漢乃扶義征伐,八年之閒,天下三嬗,事繁變眾,故詳著秦楚之際月表第四。

漢興已來,至于太初百年,諸侯廢立分削,譜紀不明,有司靡踵,彊弱之原云以世。作漢興已來諸侯年表第五。

維高祖元功,輔臣股肱,剖符而爵,澤流苗裔,忘其昭穆,或殺身隕國。作高祖功臣侯者年表第六。

惠景之閒,維申功臣宗屬爵邑,作惠景閒侯者年表第七。

北討彊胡,南誅勁越,征伐夷蠻,武功爰列。作建元以來侯者年表第八。

諸侯既彊,七國為從,子弟眾多,無爵封邑,推恩行義,其埶銷弱,德歸京師。作王子侯者年表第九。

國有賢相良將,民之師表也。維見漢興以來將相名臣年表,賢者記其治,不賢者彰其事。作漢興以來將相名臣年表第十。

維三代之禮,所損益各殊務,然要以近性情,通王道,故禮因人質為之節文,略協古今之變。作禮書第一。

樂者,所以移風易俗也。自雅頌聲興,則已好鄭衛之音,鄭衛之音所從來久矣。人情之所感,遠俗則懷。比樂書以述來古,作樂書第二。

非兵不彊,非德不昌,黃帝、湯、武以興,桀、紂、二世以崩,可不慎歟。司馬法所從來尚矣,太公、孫、吳、王子能紹而明之,切近世,極人變。作律書第三。

律居陰而治陽,歷居陽而治陰,律歷更相治,閒不容翲忽。五家之文怫異,維太初之元論。作歷書第四。

星氣之書,多雜禨祥,不經,推其文,考其應,不殊。比集論其行事,驗于軌度以次,作天官書第五。

受命而王,封禪之符罕用,用則萬靈罔不禋祀。追本諸神名山大川禮,作封禪書第六。

維禹浚川,九州攸寧,爰及宣防,決瀆通溝。作河渠書第七。

維幣之行,以通農商,其極則玩巧,并兼茲殖,爭於機利,去本趨末。作平準書以觀事變,第八。

太伯避歷,江蠻是適,文武攸興,古公王跡。闔廬弒僚,賓服荊楚,夫差克齊,子胥鴟夷,信嚭親越,吳國既滅。嘉伯之讓,作吳世家第一。

申、呂肖矣,尚父側微,卒歸西伯,文武是師,功冠群公,繆權于幽,番番黃髪,爰饗營丘。不背柯盟,桓公以昌,九合諸侯,霸功顯彰。田闞爭寵,姜姓解亡。嘉父之謀,作齊太公世家第二。

依之違之,周公綏之,憤發文德,天下和之,輔翼成王,諸侯宗周。隱桓之際,是獨何哉。三桓爭彊,魯乃不昌。嘉旦金縢,作周公世家第三。

武王克紂,天下未協而崩。成王既幼,管蔡疑之,淮夷叛之,於是召公率德,安集王室,以寧東土。燕噲之禪,乃成禍亂。嘉甘棠之詩,作燕世家第四。

管蔡相武庚,將寧舊商,及旦攝政,二叔不饗,殺鮮放度,周公為盟,大任十子,周以宗彊。嘉仲悔過,作管蔡世家第五。

王后不絕,舜禹是說,維德休明,苗裔蒙烈。百世享祀,爰周陳杞,楚實滅之。齊田既起,舜何人哉。作陳杞世家第六。

收殷餘民,叔封始邑,申以商亂,酒材是告,及朔之生,衛頃不寧,南子惡蒯聵,子父易名。周德卑微,戰國既彊,衛以小弱,角獨後亡。喜彼康誥,作衛世家第七。

嗟箕子乎。嗟箕子乎。正言不用,乃反為奴。武庚既死,周封微子。襄公傷於泓,君子孰稱。景公謙德,熒惑退行。剔成暴虐,宋乃滅亡。喜微子問太師,作宋世家第八。

武王既崩,叔虞邑唐。君子譏名,卒滅武公。驪姬之愛,亂者五世,重耳不得意,乃能成霸。六卿專權,晉國以秏。嘉文公錫珪鬯,作晉世家第九。

重黎業之,吳回接之,殷之季世,粥子牒之。周用熊繹,熊渠是續。莊王之賢,乃復國陳,既赦鄭伯,班師華元。懷王客死,蘭咎屈原,好諛信讒,楚并於秦。嘉莊王之義,作楚世家第十。

少康之子,實賓南海,文身斷發,黿鱓與處,既守封禺,奉禹之祀。句踐困彼,乃用種、蠡。嘉句踐夷蠻能修其德,滅彊吳以尊周室,作越王句踐世家第十一。

桓公之東,太史是庸。及侵周禾,王人是議。祭仲要盟,鄭久不昌。子產之仁,紹世稱賢。三晉侵伐,鄭納於韓。嘉厲公納惠王,作鄭世家第十二。

維驥騄耳,乃章造父。趙夙事獻,衰續厥緒。佐文尊王,卒為晉輔。襄子困辱,乃禽智伯。主父生縛,餓死探爵。王遷辟淫,良將是斥。嘉鞅討周亂,作趙世家第十三。

畢萬爵魏,卜人知之。及絳戮干,戎翟和之。文侯慕義,子夏師之。惠王自矜,齊秦攻之。既疑信陵,諸侯罷之。卒亡大梁,王假廝之。嘉武佐晉文申霸道,作魏世家第十四。

韓厥陰德,趙武攸興。紹絕立廢,晉人宗之。昭侯顯列,申子庸之。疑非不信,秦人襲之。嘉厥輔晉匡周天子之賦,作韓世家第十五。

完子避難,適齊為援,陰施五世,齊人歌之。成子得政,田和為侯。王建動心,乃遷于共。嘉威、宣能撥濁世而獨宗周,作田敬仲完世家第十六。

周室既衰,諸侯恣行。仲尼悼禮廢樂崩,追修經術,以達王道,匡亂世反之於正,見其文辭,為天下制儀法,垂六藝之統紀於後世。作孔子世家第十七。

桀、紂失其道而湯、武作,周失其道而春秋作。秦失其政,而陳涉發跡,諸侯作難,風起雲蒸,卒亡秦族。天下之端,自涉發難。作陳涉世家第十八。

成皋之臺,薄氏始基。詘意適代,厥崇諸竇。栗姬偩貴,王氏乃遂。陳後太驕,卒尊子夫。嘉夫德若斯,作外戚世家十九。

漢既譎謀,禽信於陳,越荊剽輕,乃封弟交為楚王,爰都彭城,以彊淮泗,為漢宗藩。戊溺於邪,禮復紹之。嘉游輔祖,作楚元王世家二十。

維祖師旅,劉賈是與,為布所襲,喪其荊、吳。營陵激呂,乃王瑯邪,怵午信齊,往而不歸,遂西入關,遭立孝文,獲復王燕。天下未集,賈、澤以族,為漢藩輔。作荊燕世家第二十一。

天下已平,親屬既寡,悼惠先壯,實鎮東土。哀王擅興,發怒諸呂,駟鈞暴戾,京師弗許。厲之內淫,禍成主父。嘉肥股肱,作齊悼惠王世家第二十二。

楚人圍我滎陽,相守三年,蕭何填撫山西,推計踵兵,給糧食不絕,使百姓愛漢,不樂為楚。作蕭相國世家第二十三。

與信定魏,破趙拔齊,遂弱楚人。續何相國,不變不革,黎庶攸寧。嘉參不伐功矜能,作曹相國世家第二十四。

運籌帷幄之中,制勝於無形,子房計謀其事,無知名,無勇功,圖難於易,為大於細。作留侯世家第二十五。

六奇既用,諸侯賓從於漢,呂氏之事,平為本謀,終安宗廟,定社稷。作陳丞相世家第二十六。

諸呂為從,謀弱京師,而勃反經合於權,吳楚之兵,亞夫駐於昌邑,以戹齊趙,而出委以梁。作絳侯世家第二十七。

七國叛逆,蕃屏京師,唯梁為捍,偩愛矜功,幾獲于禍。嘉其能距吳楚,作梁孝王世家第二十八。

五宗既王,親屬洽和,諸侯大小為藩,爰得其宜,僭擬之事稍衰貶矣。作五宗世家第二十九。

三子之王,文辭可觀。作三王世家第三十。

末世爭利,維彼奔義,讓國餓死,天下稱之。作伯夷列傳第一。

晏子儉矣,夷吾則奢,齊桓以霸,景公以治。作管晏列傳第二。

李耳無為自化,清凈自正,韓非揣事情,循埶理。作老子韓非列傳第三。

自古王者而有司馬法,穰苴能申明之。作司馬穰苴列傳第四。

非信廉仁勇不能傳兵論劍,與道同符,內可以治身,外可以應變,君子比德焉。作孫子吳起列傳第五。

維建遇讒,爰及子奢,尚既匡父,伍員奔吳。作伍子胥列傳第六。

孔氏述文,弟子興業,咸為師傅,崇仁厲義。作仲尼弟子列傳第七。

鞅去衛適秦,能明其術,彊霸孝公,後世遵其法。作商君列傳第八。

天下患衡秦毋饜,而蘇子能存諸侯,約從以抑貪彊。作蘇秦列傳第九。

六國既從親,而張儀能明其說,復散解諸侯。作張儀列傳第十。

秦所以東攘雄諸侯,樗裏、甘茂之策。作樗裏甘茂列傳第十一。

苞河山,圍大梁,使諸侯斂手而事秦者,魏冉之功。作穰侯列傳第十二。

南拔鄢郢,北摧長平,遂圍邯鄲,武安為率,破荊滅趙,王翦之計。作白起王翦列傳第十三。

獵儒墨之遺文,明禮義之統紀,絕惠王利端,列往世興衰。作孟子荀卿列傳第十四。

好客喜士,士歸于薛,為齊捍楚魏。作孟嘗君列傳第十五。

爭馮亭以權,如楚以救邯鄲之圍,使其君復稱於諸侯。作平原君虞卿列傳第十六。

能以富貴下貧賤,賢能詘於不肖,唯信陵君為能行之。作魏公子列傳第十七。

以身徇君,遂脫彊秦,使馳說之士南鄉走楚者,黃歇之義。作春申君列傳第十八

能忍於魏齊,而信威於彊秦,推賢讓位,二子有之。作范睢蔡澤列傳第十九。

率行其謀,連五國兵,為弱燕報彊齊之讎,雪其先君之恥。作樂毅列傳第二十。

能信意彊秦,而屈體廉子,用徇其君,俱重於諸侯。作廉頗藺相如列傳第二十一。

湣王既失臨淄而奔莒,唯田單用即墨破走騎劫,遂存齊社稷。作田單列傳第二十二。

能設詭說解患於圍城,輕爵祿,樂肆志。作魯仲連鄒陽列傳第二十三。

作辭以諷諫,連類以爭義,離騷有之。作屈原賈生列傳第二十四。

結子楚親,使諸侯之士斐然爭入事秦。作呂不韋列傳第二十五。

曹子匕首,魯獲其田,齊明其信,豫讓義不為二心。作刺客列傳第二十六。

能明其畫,因時推秦,遂得意於海內,斯為謀首。作李斯列傳第二十七。

為秦開地益眾,北靡匈奴,據河為塞,因山為固,建榆中。作蒙恬列傳第二十八。

填趙塞常山以廣河內,弱楚權,明漢王之信於天下。作張耳陳餘列傳第二十九。

收西河、上黨之兵,從至彭城,越之侵掠梁地以苦項羽。作魏豹彭越列傳第三十。

以淮南叛楚歸漢,漢用得大司馬殷,卒破子羽于垓下。作黥布列傳第三十一。

楚人迫我京索,而信拔魏趙,定燕齊,使漢三分天下有其二,以滅項籍。作淮陰侯列傳第三十二。

楚漢相距鞏洛,而韓信為填潁川,盧綰絕籍糧餉。作韓信盧綰列傳第三十三。

諸侯畔項王,唯齊連子羽城陽,漢得以閒遂入彭城。作田儋列傳第三十四。

攻城野戰,獲功歸報,噲、商有力焉,非獨鞭策,又與之脫難。作樊酈列傳第三十五。

漢既初定,文理未明,蒼為主計,整齊度量,序律歷。作張丞相列傳第三十六。

結言通使,約懷諸侯,諸侯咸親,歸漢為藩輔。作酈生陸賈列傳第三十七。

欲詳知秦楚之事,維周∴從高祖,平定諸侯。作傅靳蒯成列傳第三十八。

徙彊族,都關中,和約匈奴,明朝廷禮,次宗廟儀法。作劉敬叔孫通列傳第三十九。

能摧剛作柔,卒為列臣,欒公不劫於埶而倍死。作季布欒布列傳第四十。

敢犯顏色以達主義,不顧其身,為國家樹長畫。作袁盎晁錯列傳第四十一。

守法不失大理,言古賢人,增主之明。作張釋之馮唐列傳第四十二。

敦厚慈孝,訥於言,敏於行,務在鞠躬,君子長者。作萬石張叔列傳第四十三。

守節切直,義足以言廉,行足以厲賢,任重權不可以非理撓。作田叔列傳第四十四。

扁鵲言醫,為方者宗,守數精明,後世循序,弗能易也,而倉公可謂近之矣。作扁鵲倉公列傳第四十五。

維仲之省,厥濞王吳,遭漢初定,以填撫江淮之閒。作吳王濞列傳第四十六。

吳楚為亂,宗屬唯嬰賢而喜士,士鄉之,率師抗山東滎陽。作魏其武安列傳第四十七。

智足以應近世之變,寬足用得人。作韓長孺列傳第四十八。

勇於當敵,仁愛士卒,號令不煩,師徒鄉之。作李將軍列傳第四十九。

自三代以來,匈奴常為中國患害,欲知彊弱之時,設備征討,作匈奴列傳第五十。

直曲塞,廣河南,破祁連,通西國,靡北胡。作衛將軍驃騎列傳第五十一。

大臣宗室以侈靡相高,唯弘用節衣食為百吏先。作平津侯列傳第五十二。

漢既平中國,而佗能集楊越以保南藩,納貢職。作南越列傳第五十三。

吳之叛逆,甌人斬濞,葆守封禺為臣。作東越列傳第五十四。

燕丹散亂遼閒,滿收其亡民,厥聚海東,以集真藩,葆塞為外臣。作朝鮮列傳第五十五。

唐蒙使略通夜郎,而邛笮之君請為內臣受吏。作西南夷列傳第五十六。

子虛之事,大人賦說,靡麗多誇,然其指風諫,歸於無為。作司馬相如列傳第五十七。

黥布叛逆,子長國之,以填江淮之南,安剽楚庶民。作淮南衡山列傳第五十八。

奉法循理之吏,不伐功矜能,百姓無稱,亦無過行。作循吏列傳第五十九。

正衣冠立於朝廷,而群臣莫敢言浮說,長孺矜焉,好薦人,稱長者,壯有溉。作汲鄭列傳第六十。

自孔子卒,京師莫崇庠序,唯建元元狩之閒,文辭粲如也。作儒林列傳第六十一。

民倍本多巧,姦軌弄法,善人不能化,唯一切嚴削為能齊之。作酷吏列傳第六十二。

漢既通使大夏,而西極遠蠻,引領內鄉,欲觀中國。作大宛列傳第六十三。

救人於緦振人不贍,仁者有乎,不既信,不倍言,義者有取焉。作游俠列傳第六十四。

夫事人君能說主耳目,和主顏色,而獲親近,非獨色愛,能亦各有所長。作佞幸列傳第六十五。

不流世俗,不爭埶利,上下無所凝滯,人莫之害,以道之用。作滑稽列傳第六十六。

齊、楚、秦、趙為日者,各有俗所用。欲循觀其大旨,作日者列傳第六十七。

三王不同龜,四夷各異卜,然各以決吉凶。略闚其要,作龜策列傳第六十八。

布衣匹夫之人,不害於政,不妨百姓,取與以時而息財富,智者有采焉。作貨殖列傳第六十九。

維我漢繼五帝末流,接三代絕業。周道廢,秦撥去古文,焚滅詩書,故明堂石室金匱玉版圖籍散亂。於是漢興,蕭何次律令,韓信申軍法,張蒼為章程,叔孫通定禮儀,則文學彬彬稍進,詩書往往閒出矣。自曹參薦蓋公言黃老,而賈生、晁錯明申、商,公孫弘以儒顯,百年之閒,天下遺文古事靡不畢集太史公。太史公仍父子相續纂其職。曰,於戲。余維先人嘗掌斯事,顯於唐虞,至于周,復典之,故司馬氏世主天官。至於余乎,欽念哉。欽念哉。罔羅天下放失舊聞,王跡所興,原始察終,見盛觀衰,論考之行事,略推三代,錄秦漢,上記軒轅,下至于茲,著十二本紀,既科條之矣。并時異世,年差不明,作十表。禮樂損益,律歷改易,兵權山川鬼神,天人之際,承敝通變,作八書。二十八宿環北辰,三十輻共一轂,運行無窮,輔拂股肱之臣配焉,忠信行道,以奉主上,作三十世家。扶義俶儻,不令己失時,立功名於天下,作七十列傳。凡百三十篇,五十二萬六千五百字,為太史公書。序略,以拾遺補闕,成一家之言,厥協六經異傳,整齊百家雜語,藏之名山,副在京師,俟後世聖人君子。第七十。

太史公曰,余述歷黃帝以來至太初而訖,百三十篇。