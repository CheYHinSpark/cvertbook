\chapter{李斯列傳第二十七}

李斯者,楚上蔡人也。年少時,為郡小吏,見吏舍廁中鼠食不絜,近人犬,數驚恐之。斯入倉,觀倉中鼠,食積粟,居大廡之下,不見人犬之憂。於是李斯乃嘆曰,人之賢不肖譬如鼠矣,在所自處耳。

乃從荀卿學帝王之術。學已成,度楚王不足事,而六國皆弱,無可為建功者,欲西入秦。辭於荀卿曰,斯聞得時無怠,今萬乘方爭時,游者主事。今秦王欲吞天下,稱帝而治,此布衣馳騖之時而游說者之秋也。處卑賤之位而計不為者,此禽鹿視肉,人面而能閒行者耳。故詬莫大於卑賤,而悲莫甚於窮困。久處卑賤之位,困苦之地,非世而惡利,自託於無為,此非士之情也。故斯將西說秦王矣。

至秦,會莊襄王卒,李斯乃求為秦相文信侯呂不韋舍人,不韋賢之,任以為郎。李斯因以得說,說秦王曰,胥人者,去其幾也。成大功者,在因瑕釁而遂忍之。昔者秦穆公之霸,終不東并六國者,何也。諸侯尚眾,周德未衰,故五伯迭興,更尊周室。自秦孝公以來,周室卑微,諸侯相兼,關東為六國,秦之乘勝役諸侯,蓋六世矣。今諸侯服秦,譬若郡縣。夫以秦之彊,大王之賢,由灶上騷除,足以滅諸侯,成帝業,為天下一統,此萬世之一時也。今怠而不急就,諸侯復彊,相聚約從,雖有黃帝之賢,不能并也。秦王乃拜斯為長史,聽其計,陰遣謀士齎持金玉以游說諸侯。諸侯名士可下以財者,厚遺結之,不肯者,利劍刺之。離其君臣之計,秦王乃使其良將隨其後。秦王拜斯為客卿。

會韓人鄭國來閒秦,以作注溉渠,已而覺。秦宗室大臣皆言秦王曰,諸侯人來事秦者,大抵為其主游閒於秦耳,請一切逐客。李斯議亦在逐中。斯乃上書曰,

臣聞吏議逐客,竊以為過矣。昔繆公求士,西取由余於戎,東得百里奚於宛,迎蹇叔於宋,來丕豹、公孫支於晉。此五子者,不產於秦,而繆公用之,并國二十,遂霸西戎。孝公用商鞅之法,移風易俗,民以殷盛,國以富彊,百姓樂用,諸侯親服,獲楚、魏之師,舉地千里,至今治彊。惠王用張儀之計,拔三川之地,西并巴、蜀,北收上郡,南取漢中,包九夷,制鄢、郢,東據成皋之險,割膏腴之壤,遂散六國之從,使之西面事秦,功施到今。昭王得范雎,廢穰侯,逐華陽,彊公室,杜私門,蠶食諸侯,使秦成帝業。此四君者,皆以客之功。由此觀之,客何負於秦哉。向使四君卻客而不內,疏士而不用,是使國無富利之實而秦無彊大之名也。

今陛下致昆山之玉,有隨、和之寶,垂明月之珠,服太阿之劍,乘纖離之馬,建翠鳳之旗,樹靈鼉之鼓。此數寶者,秦不生一焉,而陛下說之,何也。必秦國之所生然後可,則是夜光之璧不飾朝廷,犀象之器不為玩好,鄭、衛之女不充後宮,而駿良駃騠不實外廄,江南金錫不為用,西蜀丹青不為采。所以飾後宮充下陳娛心意說耳目者,必出於秦然後可,則是宛珠之簪,傅璣之珥,阿縞之衣,錦繡之飾不進於前,而隨俗雅化佳冶窈窕趙女不立於側也。夫擊甕叩缶彈箏搏髀,而歌呼嗚嗚快耳者,真秦之聲也,鄭、衛、桑閒、昭、虞、武、象者,異國之樂也。今棄擊甕叩缶而就鄭衛,退彈箏而取昭虞,若是者何也。快意當前,適觀而已矣。今取人則不然。不問可否,不論曲直,非秦者去,為客者逐。然則是所重者在乎色樂珠玉,而所輕者在乎人民也。此非所以跨海內制諸侯之術也。

臣聞地廣者粟多,國大者人眾,兵彊則士勇。是以太山不讓土壤,故能成其大,河海不擇細流,故能就其深,王者不卻眾庶,故能明其德。是以地無四方,民無異國,四時充美,鬼神降福,此五帝、三王之所以無敵也。今乃棄黔首以資敵國,卻賓客以業諸侯,使天下之士退而不敢西向,裹足不入秦,此所謂藉寇兵而齎盜糧者也。

夫物不產於秦,可寶者多,士不產於秦,而願忠者眾。今逐客以資敵國,損民以益讎,內自虛而外樹怨於諸侯,求國無危,不可得也。

秦王乃除逐客之令,復李斯官,卒用其計謀。官至廷尉。二十餘年,竟并天下,尊主為皇帝,以斯為丞相。夷郡縣城,銷其兵刃,示不復用。使秦無尺土之封,不立子弟為王,功臣為諸侯者,使後無戰攻之患。

始皇三十四年,置酒咸陽宮,博士仆射周青臣等頌始皇威德。齊人淳于越進諫曰,臣聞之,殷周之王千餘歲,封子弟功臣自為支輔。今陛下有海內,而子弟為匹夫,卒有田常、六卿之患,臣無輔弼,何以相救哉。事不師古而能長久者,非所聞也。今青臣等又面諛以重陛下過,非忠臣也。始皇下其議丞相。丞相謬其說,絀其辭,乃上書曰,古者天下散亂,莫能相一,是以諸侯并作,語皆道古以害今,飾虛言以亂實,人善其所私學,以非上所建立。今陛下并有天下,別白黑而定一尊,而私學乃相與非法教之制,聞令下,即各以其私學議之,入則心非,出則巷議,非主以為名,異趣以為高,率群下以造謗。如此不禁,則主勢降乎上,黨與成乎下。禁之便。臣請諸有文學詩書百家語者,蠲除去之。令到滿三十日弗去,黥為城旦。所不去者,醫藥卜筮種樹之書。若有欲學者,以吏為師。始皇可其議,收去詩書百家之語以愚百姓,使天下無以古非今。明法度,定律令,皆以始皇起。同文書。治離宮別館,周遍天下。明年,又巡狩,外攘四夷,斯皆有力焉。

斯長男由為三川守,諸男皆尚秦公主,女悉嫁秦諸公子。三川守李由告歸咸陽,李斯置酒於家,百官長皆前為壽,門廷車騎以千數。李斯喟然而嘆曰,嗟乎。吾聞之荀卿曰物禁大盛。夫斯乃上蔡布衣,閭巷之黔首,上不知其駑下,遂擢至此。當今人臣之位無居臣上者,可謂富貴極矣。物極則衰,吾未知所稅駕也。

始皇三十七年十月,行出游會稽,并海上,北抵瑯邪。丞相斯、中車府令趙高兼行符璽令事,皆從。始皇有二十餘子,長子扶蘇以數直諫上,上使監兵上郡,蒙恬為將。少子胡亥愛,請從,上許之。餘子莫從。

其年七月,始皇帝至沙丘,病甚,令趙高為書賜公子扶蘇曰,以兵屬蒙恬,與喪會咸陽而葬。書已封,未授使者,始皇崩。書及璽皆在趙高所,獨子胡亥、丞相李斯、趙高及幸宦者五六人知始皇崩,餘群臣皆莫知也。李斯以為上在外崩,無真太子,故祕之。置始皇居輼輬車中,百官奏事上食如故,宦者輒從輼輬車中可諸奏事。

趙高因留所賜扶蘇璽書,而謂公子胡亥曰,上崩,無詔封王諸子而獨賜長子書。長子至,即立為皇帝,而子無尺寸之地,為之柰何。胡亥曰,固也。吾聞之,明君知臣,明父知子。父捐命,不封諸子,何可言者。趙高曰,不然。方今天下之權,存亡在子與高及丞相耳,願子圖之。且夫臣人與見臣於人,制人與見制於人,豈可同日道哉。胡亥曰,廢兄而立弟,是不義也,不奉父詔而畏死,是不孝也,能薄而材谫,彊因人之功,是不能也,三者逆德,天下不服,身殆傾危,社稷不血食。高曰,臣聞湯、武殺其主,天下稱義焉,不為不忠。衛君殺其父,而衛國載其德,孔子著之,不為不孝。夫大行不小謹,盛德不辭讓,鄉曲各有宜而百官不同功。故顧小而忘大,後必有害,狐疑猶豫,後必有悔。斷而敢行,鬼神避之,後有成功。願子遂之。胡亥喟然嘆曰,今大行未發,喪禮未終,豈宜以此事干丞相哉。趙高曰,時乎時乎,閒不及謀。贏糧躍馬,唯恐後時。

胡亥既然高之言,高曰,不與丞相謀,恐事不能成,臣請為子與丞相謀之。高乃謂丞相斯曰,上崩,賜長子書,與喪會咸陽而立為嗣。書未行,今上崩,未有知者也。所賜長子書及符璽皆在胡亥所,定太子在君侯與高之口耳。事將何如。斯曰,安得亡國之言。此非人臣所當議也。高曰,君侯自料能孰與蒙恬。功高孰與蒙恬。謀遠不失孰與蒙恬。無怨於天下孰與蒙恬。長子舊而信之孰與蒙恬。斯曰,此五者皆不及蒙恬,而君責之何深也。高曰,高固內官之廝役也,幸得以刀筆之文進入秦宮,管事二十餘年,未嘗見秦免罷丞相功臣有封及二世者也,卒皆以誅亡。皇帝二十餘子,皆君之所知。長子剛毅而武勇,信人而奮士,即位必用蒙恬為丞相,君侯終不懷通侯之印歸於鄉里,明矣。高受詔教習胡亥,使學以法事數年矣,未嘗見過失。慈仁篤厚,輕財重士,辯於心而詘於口,盡禮敬士,秦之諸子未有及此者,可以為嗣。君計而定之。斯曰,君其反位。斯奉主之詔,聽天之命,何慮之可定也。高曰,安可危也,危可安也。安危不定,何以貴聖。斯曰,斯,上蔡閭巷布衣也,上幸擢為丞相,封為通侯,子孫皆至尊位重祿者,故將以存亡安危屬臣也。豈可負哉。夫忠臣不避死而庶幾,孝子不勤勞而見危,人臣各守其職而已矣。君其勿復言,將令斯得罪。高曰,蓋聞聖人遷徙無常,就變而從時,見末而知本,觀指而睹歸。物固有之,安得常法哉。方今天下之權命懸於胡亥,高能得志焉。且夫從外制中謂之惑,從下制上謂之賊。故秋霜降者草花落,水搖動者萬物作,此必然之效也。君何見之晚。斯曰,吾聞晉易太子,三世不安,齊桓兄弟爭位,身死為戮,紂殺親戚,不聽諫者,國為丘墟,遂危社稷,三者逆天,宗廟不血食。斯其猶人哉,安足為謀。高曰,上下合同,可以長久,中外若一,事無表裏。君聽臣之計,即長有封侯,世世稱孤,必有喬松之壽,孔、墨之智。今釋此而不從,禍及子孫,足以為寒心。善者因禍為福,君何處焉。斯乃仰天而嘆,垂淚太息曰,嗟乎。獨遭亂世,既以不能死,安託命哉。於是斯乃聽高。高乃報胡亥曰,臣請奉太子之明命以報丞相,丞相斯敢不奉令。

於是乃相與謀,詐為受始皇詔丞相,立子胡亥為太子。更為書賜長子扶蘇曰,朕巡天下,禱祠名山諸神以延壽命。今扶蘇與將軍蒙恬將師數十萬以屯邊,十有餘年矣,不能進而前,士卒多秏,無尺寸之功,乃反數上書直言誹謗我所為,以不得罷歸為太子,日夜怨望。扶蘇為人子不孝,其賜劍以自裁。將軍恬與扶蘇居外,不匡正,宜知其謀。為人臣不忠,其賜死,以兵屬裨將王離。封其書以皇帝璽,遣胡亥客奉書賜扶蘇於上郡。

使者至,發書,扶蘇泣,入內舍,欲自殺。蒙恬止扶蘇曰,陛下居外,未立太子,使臣將三十萬眾守邊,公子為監,此天下重任也。今一使者來,即自殺,安知其非詐。請復請,復請而後死,未暮也。使者數趣之。扶蘇為人仁,謂蒙恬曰,父而賜子死,尚安復請。即自殺。蒙恬不肯死,使者即以屬吏,系於陽周。

使者還報,胡亥、斯、高大喜。至咸陽,發喪,太子立為二世皇帝。以趙高為郎中令,常侍中用事。

二世燕居,乃召高與謀事,謂曰,夫人生居世閒也,譬猶騁六驥過決隙也。吾既已臨天下矣,欲悉耳目之所好,窮心志之所樂,以安宗廟而樂萬姓,長有天下,終吾年壽,其道可乎。高曰,此賢主之所能行也,而昏亂主之所禁也。臣請言之,不敢避斧鉞之誅,願陛下少留意焉。夫沙丘之謀,諸公子及大臣皆疑焉,而諸公子盡帝兄,大臣又先帝之所置也。今陛下初立,此其屬意怏怏皆不服,恐為變。且蒙恬已死,蒙毅將兵居外,臣戰戰栗栗,唯恐不終。且陛下安得為此樂乎。二世曰,為之柰何。趙高曰,嚴法而刻刑,令有罪者相坐誅,至收族,滅大臣而遠骨肉,貧者富之,賤者貴之。盡除去先帝之故臣,更置陛下之所親信者近之。此則陰德歸陛下,害除而姦謀塞,群臣莫不被潤澤,蒙厚德,陛下則高枕肆志寵樂矣。計莫出於此。二世然高之言,乃更為法律。於是群臣諸公子有罪,輒下高,令鞠治之。殺大臣蒙毅等,公子十二人僇死咸陽市,十公主僇死於杜,財物入於縣官,相連坐者不可勝數。

公子高欲奔,恐收族,乃上書曰,先帝無恙時,臣入則賜食,出則乘輿。御府之衣,臣得賜之,中廄之寶馬,臣得賜之。臣當從死而不能,為人子不孝,為人臣不忠。不忠者無名以立於世,臣請從死,願葬酈山之足。唯上幸哀憐之。書上,胡亥大說,召趙高而示之,曰,此可謂急乎。趙高曰,人臣當憂死而不暇,何變之得謀。胡亥可其書,賜錢十萬以葬。

法令誅罰日益刻深,群臣人人自危,欲畔者眾。又作阿房之宮,治直道、馳道,賦斂愈重,戍傜無已。於是楚戍卒陳勝、吳廣等乃作亂,起於山東,傑俊相立,自置為侯王,叛秦,兵至鴻門而卻。李斯數欲請閒諫,二世不許。而二世責問李斯曰,吾有私議而有所聞於韓子也,曰堯之有天下也,堂高三尺,采椽不斲,茅茨不翦,雖逆旅之宿不勤於此矣。冬日鹿裘,夏日葛衣,粢糲之食,藜藿之羹,飯土匭,啜土鉶,雖監門之養不觳於此矣。禹鑿龍門,通大夏,疏九河,曲九防,決渟水致之海,而股無胈,脛無毛,手足胼胝,面目黎黑,遂以死于外,葬於會稽,臣虜之勞不烈於此矣。然則夫所貴於有天下者,豈欲苦形勞神,身處逆旅之宿,口食監門之養,手持臣虜之作哉。此不肖人之所勉也,非賢者之所務也。彼賢人之有天下也,專用天下適己而已矣,此所貴於有天下也。夫所謂賢人者,必能安天下而治萬民,今身且不能利,將惡能治天下哉。笔吾願賜志廣欲,長享天下而無害,為之柰何。李斯子由為三川守,群盜吳廣等西略地,過去弗能禁。章邯以破逐廣等兵,使者覆案三川相屬,誚讓斯居三公位,如何令盜如此。李斯恐懼,重爵祿,不知所出,乃阿二世意,欲求容,以書對曰,

夫賢主者,必且能全道而行督責之術者也。督責之,則臣不敢不竭能以徇其主矣。此臣主之分定,上下之義明,則天下賢不肖莫敢不盡力竭任以徇其君矣。是故主獨制於天下而無所制也。能窮樂之極矣,賢明之主也,可不察焉。

故申子曰有天下而不恣睢,命之曰以天下為桎梏者,無他焉,不能督責,而顧以其身勞於天下之民,若堯、禹然,故謂之桎梏也。夫不能修申、韓之明術,行督責之道,專以天下自適也,而徒務苦形勞神,以身徇百姓,則是黔首之役,非畜天下者也,何足貴哉。夫以人徇己,則己貴而人賤,以己徇人,則己賤而人貴。故徇人者賤,而人所徇者貴,自古及今,未有不然者也。凡古之所為尊賢者,為其貴也,而所為惡不肖者,為其賤也。而堯、禹以身徇天下者也,因隨而尊之,則亦失所為尊賢之心矣,夫可謂大繆矣。謂之為桎梏,不亦宜乎。不能督責之過也。

故韓子曰,慈母有敗子而嚴家無格虜者,何也。則能罰之加焉必也。故商君之法,刑棄灰於道者。夫棄灰,薄罪也,而被刑,重罰也。彼唯明主為能深督輕罪。夫罪輕且督深,而況有重罪乎。故民不敢犯也。是故韓子曰布帛尋常,庸人不釋,鑠金百溢,盜跖不搏者,非庸人之心重,尋常之利深,而盜跖之欲淺也,又不以盜跖之行,為輕百鎰之重也。搏必隨手刑,則盜跖不搏百鎰,而罰不必行也,則庸人不釋尋常。是故城高五丈,而樓季不輕犯也,泰山之高百仞,而跛羊牧其上。夫樓季也而難五丈之限,豈跛羊也而易百仞之高哉。峭塹之勢異也。明主聖王之所以能久處尊位,長執重勢,而獨擅天下之利者,非有異道也,能獨斷而審督責,必深罰,故天下不敢犯也。今不務所以不犯,而事慈母之所以敗子也,則亦不察於聖人之論矣。夫不能行聖人之術,則舍為天下役何事哉。可不哀邪。

且夫儉節仁義之人立於朝,則荒肆之樂輟矣,諫說論理之臣閒於側,則流漫之志詘矣,烈士死節之行顯於世,則淫康之虞廢矣。故明主能外此三者,而獨操主術以制聽從之臣,而修其明法,故身尊而勢重也。凡賢主者,必將能拂世磨俗,而廢其所惡,立其所欲,故生則有尊重之勢,死則有賢明之謚也。是以明君獨斷,故權不在臣也。然後能滅仁義之涂,掩馳說之口,困烈士之行,塞聰揜明,內獨視聽,故外不可傾以仁義烈士之行,而內不可奪以諫說忿爭之辯。故能犖然獨行恣睢之心而莫之敢逆。若此然後可謂能明申、韓之術,而修商君之法。法修術明而天下亂者,未之聞也。故曰王道約而易操也。唯明主為能行之。若此則謂督責之誠,則臣無邪,臣無邪則天下安,天下安則主嚴尊,主嚴尊則督責必,督責必則所求得,所求得則國家富,國家富則君樂豐。故督責之術設,則所欲無不得矣。群臣百姓救過不給,何變之敢圖。若此則帝道備,而可謂能明君臣之術矣。雖申、韓復生,不能加也。

書奏,二世悅。於是行督責益嚴,稅民深者為明吏。二世曰,若此則可謂能督責矣。刑者相半於道,而死人日成積於市。殺人眾者為忠臣。二世曰,若此則可謂能督責矣。

初,趙高為郎中令,所殺及報私怨眾多,恐大臣入朝奏事毀惡之,乃說二世曰,天子所以貴者,但以聞聲,群臣莫得見其面,故號曰朕。且陛下富於春秋,未必盡通諸事,今坐朝廷,譴舉有不當者,則見短於大臣,非所以示神明於天下也。且陛下深拱禁中,與臣及侍中習法者待事,事來有以揆之。如此則大臣不敢奏疑事,天下稱聖主矣。二世用其計,乃不坐朝廷見大臣,居禁中。趙高常侍中用事,事皆決於趙高。

高聞李斯以為言,乃見丞相曰,關東群盜多,今上急益發繇治阿房宮,聚狗馬無用之物。臣欲諫,為位賤。此真君侯之事,君何不諫。李斯曰,固也,吾欲言之久矣。今時上不坐朝廷,上居深宮,吾有所言者,不可傳也,欲見無閒。趙高謂曰,君誠能諫,請為君候上閒語君。於是趙高待二世方燕樂,婦女居前,使人告丞相,上方閒,可奏事。丞相至宮門上謁,如此者三。二世怒曰,吾常多閒日,丞相不來。吾方燕私,丞相輒來請事。丞相豈少我哉。且固我哉。趙高因曰,如此殆矣。夫沙丘之謀,丞相與焉。今陛下已立為帝,而丞相貴不益,此其意亦望裂地而王矣。且陛下不問臣,臣不敢言。丞相長男李由為三川守,楚盜陳勝等皆丞相傍縣之子,以故楚盜公行,過三川,城守不肯擊。高聞其文書相往來,未得其審,故未敢以聞。且丞相居外,權重於陛下。二世以為然。欲案丞相,恐其不審,乃使人案驗三川守與盜通狀。李斯聞之。

是時二世在甘泉,方作觳抵優俳之觀。李斯不得見,因上書言趙高之短曰,臣聞之,臣疑其君,無不危國,妾疑其夫,無不危家。今有大臣於陛下擅利擅害,與陛下無異,此甚不便。昔者司城子罕相宋,身行刑罰,以威行之,朞年遂劫其君。田常為簡公臣,爵列無敵於國,私家之富與公家均,布惠施德,下得百姓,上得群臣,陰取齊國,殺宰予於庭,即弒簡公於朝,遂有齊國。此天下所明知也。今高有邪佚之志,危反之行,如子罕相宋也,私家之富,若田氏之於齊也。兼行田常、子罕之逆道而劫陛下之威信,其志若韓玘為韓安相也。陛下不圖,臣恐其為變也。二世曰,何哉。夫高,故宦人也,然不為安肆志,不以危易心,絜行修善,自使至此,以忠得進,以信守位,朕實賢之,而君疑之,何也。且朕少失先人,無所識知,不習治民,而君又老,恐與天下絕矣。朕非屬趙君,當誰任哉。且趙君為人精廉彊力,下知人情,上能適朕,君其勿疑。李斯曰,不然。夫高,故賤人也,無識於理,貪欲無厭,求利不止,列勢次主,求欲無窮,臣故曰殆。二世已前信趙高,恐李斯殺之,乃私告趙高。高曰,丞相所患者獨高,高已死,丞相即欲為田常所為。於是二世曰,其以李斯屬郎中令。

趙高案治李斯。李斯拘執束縛,居囹圄中,仰天而嘆曰,嗟乎,悲夫。不道之君,何可為計哉。昔者桀殺關龍逢,紂殺王子比干,吳王夫差殺伍子胥。此三臣者,豈不忠哉,然而不免於死,身死而所忠者非也。今吾智不及三子,而二世之無道過於桀、紂、夫差,吾以忠死,宜矣。且二世之治豈不亂哉。日者夷其兄弟而自立也,殺忠臣而貴賤人,作為阿房之宮,賦斂天下。吾非不諫也,而不吾聽也。凡古聖王,飲食有節,車器有數,宮室有度,出令造事,加費而無益於民利者禁,故能長久治安。今行逆於昆弟,不顧其咎,侵殺忠臣,不思其殃,大為宮室,厚賦天下,不愛其費,三者已行,天下不聽。今反者已有天下之半矣,而心尚未寤也,而以趙高為佐,吾必見寇至咸陽,麋鹿游於朝也。

於是二世乃使高案丞相獄,治罪,責斯與子由謀反狀,皆收捕宗族賓客。趙高治斯,榜掠千餘,不勝痛,自誣服。斯所以不死者,自負其辯,有功,實無反心,幸得上書自陳,幸二世之寤而赦之。李斯乃從獄中上書曰,臣為丞相治民,三十餘年矣。逮秦地之陜隘。先王之時秦地不過千里,兵數十萬。臣盡薄材,謹奉法令,陰行謀臣,資之金玉,使游說諸侯,陰修甲兵,飾政教,官鬬士,尊功臣,盛其爵祿,故終以脅韓弱魏,破燕、趙,夷齊、楚,卒兼六國,虜其王,立秦為天子。罪一矣。地非不廣,又北逐胡、貉,南定百越,以見秦之彊。罪二矣。尊大臣,盛其爵位,以固其親。罪三矣。立社稷,修宗廟,以明主之賢。罪四矣。更剋畫,平斗斛度量文章,布之天下,以樹秦之名。罪五矣。治馳道,興游觀,以見主之得意。罪六矣。緩刑罰,薄賦斂,以遂主得眾之心,萬民戴主,死而不忘。罪七矣。若斯之為臣者,罪足以死固久矣。上幸盡其能力,乃得至今,願陛下察之。書上,趙高使吏棄去不奏,曰,囚安得上書。

趙高使其客十餘輩詐為御史、謁者、侍中,更往覆訊斯。斯更以其實對,輒使人復榜之。後二世使人驗斯,斯以為如前,終不敢更言,辭服。奏當上,二世喜曰,微趙君,幾為丞相所賣。及二世所使案三川之守至,則項梁已擊殺之。使者來,會丞相下吏,趙高皆妄為反辭。

二世二年七月,具斯五刑,論腰斬咸陽市。斯出獄,與其中子俱執,顧謂其中子曰,吾欲與若復牽黃犬俱出上蔡東門逐狡兔,豈可得乎。遂父子相哭,而夷三族。

李斯已死,二世拜趙高為中丞相,事無大小輒決於高。高自知權重,乃獻鹿,謂之馬。二世問左右,此乃鹿也。左右皆曰馬也。二世驚,自以為惑,乃召太卜,令卦之,太卜曰,陛下春秋郊祀,奉宗廟鬼神,齋戒不明,故至于此。可依盛德而明齋戒。於是乃入上林齋戒。日游弋獵,有行人入上林中,二世自射殺之。趙高教其女婿咸陽令閻樂劾不知何人賊殺人移上林。高乃諫二世曰,天子無故賊殺不辜人,此上帝之禁也,鬼神不享,天且降殃,當遠避宮以禳之。二世乃出居望夷之宮。

留三日,趙高詐詔衛士,令士皆素服持兵內鄉,入告二世曰,山東群盜兵大至。二世上觀而見之,恐懼,高既因劫令自殺。引璽而佩之,左右百官莫從,上殿,殿欲壞者三。高自知天弗與,群臣弗許,乃召始皇弟,授之璽。

子嬰既位,患之,乃稱疾不聽事,與宦者韓談及其子謀殺高。高上謁,請病,因召入,令韓談刺殺之,夷其三族。

子嬰立三月,沛公兵從武關入,至咸陽,群臣百官皆畔,不適。子嬰與妻子自系其頸以組,降軹道旁。沛公因以屬吏。項王至而斬之。遂以亡天下。

太史公曰,李斯以閭閻歷諸侯,入事秦,因以瑕釁,以輔始皇,卒成帝業,斯為三公,可謂尊用矣。斯知六藝之歸,不務明政以補主上之缺,持爵祿之重,阿順茍合,嚴威酷刑,聽高邪說,廢適立庶。諸侯已畔,斯乃欲諫爭,不亦末乎。人皆以斯極忠而被五刑死,察其本,乃與俗議之異。不然,斯之功且與周、召列矣。