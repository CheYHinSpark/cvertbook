\chapter{孟子荀卿列傳第十四}

太史公曰,余讀孟子書,至梁惠王問何以利吾國,未嘗不廢書而嘆也。曰,嗟乎,利誠亂之始也。夫子罕言利者,常防其原也。故曰放於利而行,多怨。自天子至於庶人,好利之獘何以異哉。

孟軻,騶人也。受業子思之門人。道既通,游事齊宣王,宣王不能用。適梁,梁惠王不果所言,則見以為迂遠而闊於事情。當是之時,秦用商君,富國彊兵,楚、魏用吳起,戰勝弱敵,齊威王、宣王用孫子、田忌之徒,而諸侯東面朝齊。天下方務於合從連衡,以攻伐為賢,而孟軻乃述唐、虞、三代之德,是以所如者不合。退而與萬章之徒序詩書,述仲尼之意,作孟子七篇。其後有騶子之屬。

齊有三騶子。其前騶忌,以鼓琴干威王,因及國政,封為成侯而受相印,先孟子。

其次騶衍,後孟子。騶衍睹有國者益淫侈,不能尚德,若大雅整之於身,施及黎庶矣。乃深觀陰陽消息而作怪迂之變,終始、大聖之篇十餘萬言。其語閎大不經,必先驗小物,推而大之,至於無垠。先序今以上至黃帝,學者所共術,大并世盛衰,因載其禨祥度制,推而遠之,至天地未生,窈冥不可考而原也。先列中國名山大川,通谷禽獸,水土所殖,物類所珍,因而推之,及海外人之所不能睹。稱引天地剖判以來,五德轉移,治各有宜,而符應若茲。以為儒者所謂中國者,於天下乃八十一分居其一分耳。中國名曰赤縣神州。赤縣神州內自有九州,禹之序九州是也,不得為州數。中國外如赤縣神州者九,乃所謂九州也。於是有裨海環之,人民禽獸莫能相通者,如一區中者,乃為一州。如此者九,乃有大瀛海環其外,天地之際焉。其術皆此類也。然要其歸,必止乎仁義節儉,君臣上下六親之施,始也濫耳。王公大人初見其術,懼然顧化,其後不能行之。

是以騶子重於齊。適梁,惠王郊迎,執賓主之禮。適趙,平原君側行撇席。如燕,昭王擁彗先驅,請列弟子之座而受業,筑碣石宮,身親往師之。作主運。其游諸侯見尊禮如此,豈與仲尼菜色陳蔡,孟軻困於齊梁同乎哉。笔武王以仁義伐紂而王,伯夷餓不食周粟,衛靈公問陳,而孔子不答,梁惠王謀欲攻趙,孟軻稱大王去邠。此豈有意阿世俗茍合而已哉。持方枘欲內圜鑿,其能入乎。或曰,伊尹負鼎而勉湯以王,百里奚飯牛車下而繆公用霸,作先合,然後引之大道。騶衍其言雖不軌,儻亦有牛鼎之意乎。

自騶衍與齊之稷下先生,如淳于髡、慎到、環淵、接子、田駢、騶奭之徒,各著書言治亂之事,以干世主,豈可勝道哉。

淳于髡,齊人也。博聞彊記,學無所主。其諫說,慕晏嬰之為人也,然而承意觀色為務。客有見髡於梁惠王,惠王屏左右,獨坐而再見之,終無言也。惠王怪之,以讓客曰,子之稱淳于先生,管、晏不及,及見寡人,寡人未有得也。豈寡人不足為言邪。何故哉。客以謂髡。髡曰,固也。吾前見王,王志在驅逐,後復見王,王志在音聲,吾是以默然。客具以報王,王大駭,曰,嗟乎,淳于先生誠聖人也。前淳于先生之來,人有獻善馬者,寡人未及視,會先生至。後先生之來,人有獻謳者,未及試,亦會先生來。寡人雖屏人,然私心在彼,有之。後淳于髡見,壹語連三日三夜無倦。惠王欲以卿相位待之,髡因謝去。於是送以安車駕駟,束帛加璧,黃金百鎰。終身不仕。

慎到,趙人。田駢、接子,齊人。環淵,楚人。皆學黃老道德之術,因發明序其指意。故慎到著十二論,環淵著上下篇,而田駢、接子皆有所論焉。

騶奭者,齊諸騶子,亦頗采騶衍之術以紀文。

於是齊王嘉之,自如淳于髡以下,皆命曰列大夫,為開第康莊之衢,高門大屋,尊寵之。覽天下諸侯賓客,言齊能致天下賢士也。

荀卿,趙人。年五十始來游學於齊。騶衍之術迂大而閎辯,奭也文具難施,淳于髡久與處,時有得善言。故齊人頌曰,談天衍,雕龍奭,炙轂過髡。田駢之屬皆已死齊襄王時,而荀卿最為老師。齊尚修列大夫之缺,而荀卿三為祭酒焉。齊人或讒荀卿,荀卿乃適楚,而春申君以為蘭陵令。春申君死而荀卿廢,因家蘭陵。李斯嘗為弟子,已而相秦。荀卿嫉濁世之政,亡國亂君相屬,不遂大道而營於巫祝,信禨祥,鄙儒小拘,如莊周等又猾稽亂俗,於是推儒、墨、道德之行事興壞,序列著數萬言而卒。因葬蘭陵。

而趙亦有公孫龍為堅白同異之辯,劇子之言,魏有李悝,盡地力之教,楚有尸子、長盧,阿之吁子焉。自如孟子至于吁子,世多有其書,故不論其傳云。

蓋墨翟,宋之大夫,善守御,為節用。或曰并孔子時,或曰在其後。