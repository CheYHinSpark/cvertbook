\chapter{司馬穰苴列傳第四}

司馬穰苴者,田完之苗裔也。齊景公時,晉伐阿、甄,而燕侵河上,齊師敗績。景公患之。晏嬰乃薦田穰苴曰,穰苴雖田氏庶孽,然其人文能附眾,武能威敵,願君試之。景公召穰苴,與語兵事,大說之,以為將軍,將兵捍燕晉之師。穰苴曰,臣素卑賤,君擢之閭伍之中,加之大夫之上,士卒未附,百姓不信,人微權輕,願得君之寵臣,國之所尊,以監軍,乃可。於是景公許之,使莊賈往。穰苴既辭,與莊賈約曰,旦日日中會於軍門。穰苴先馳至軍,立表下漏待賈。賈素驕貴,以為將己之軍而己為監,不甚急,親戚左右送之,留飲。日中而賈不至。穰苴則仆表決漏,入,行軍勒兵,申明約束。約束既定,夕時,莊賈乃至。穰苴曰,何後期為。賈謝曰,不佞大夫親戚送之,故留。穰苴曰,將受命之日則忘其家,臨軍約束則忘其親,援枹鼓之急則忘其身。今敵國深侵,邦內騷動,士卒暴露於境,君寢不安席,食不甘味,百姓之命皆懸於君,何謂相送乎。召軍正問曰,軍法期而後至者雲何。對曰,當斬。莊賈懼,使人馳報景公,請救。既往,未及反,於是遂斬莊賈以徇三軍。三軍之士皆振慄。久之,景公遣使者持節赦賈,馳入軍中。穰苴曰,將在軍,君令有所不受。問軍正曰,馳三軍法何。正曰,當斬。使者大懼。穰苴曰,君之使不可殺之。乃斬其仆,車之左駙,馬之左驂,以徇三軍。遣使者還報,然後行。士卒次舍井灶飲食問疾醫藥,身自拊循之。悉取將軍之資糧享士卒,身與士卒平分糧食。最比其羸弱者,三日而後勒兵。病者皆求行,爭奮出為之赴戰。晉師聞之,為罷去。燕師聞之,度水而解。於是追擊之,遂取所亡封內故境而引兵歸。未至國,釋兵旅,解約束,誓盟而後入邑。景公與諸大夫郊迎,勞師成禮,然後反歸寢。既見穰苴,尊為大司馬。田氏日以益尊於齊。

已而大夫鮑氏、高、國之屬害之,譖於景公。景公退穰苴,苴發疾而死。田乞、田豹之徒由此怨高、國等。其後及田常殺簡公,盡滅高子、國子之族。至常曾孫和,因自立為齊威王,用兵行威,大放穰苴之法,而諸侯朝齊。

齊威王使大夫追論古者司馬兵法而附穰苴於其中,因號曰司馬穰苴兵法。

太史公曰,余讀司馬兵法,閎廓深遠,雖三代征伐,未能竟其義,如其文也,亦少褒矣。若夫穰苴,區區為小國行師,何暇及司馬兵法之揖讓乎。世既多司馬兵法,以故不論,著穰苴之列傳焉。