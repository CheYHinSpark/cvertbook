\chapter{季布欒布列傳第四十}

季布者,楚人也。為氣任俠,有名於楚。項籍使將兵,數窘漢王。及項羽滅,高祖購求布千金,敢有舍匿,罪及三族。季布匿濮陽周氏。周氏曰,漢購將軍急,跡且至臣家,將軍能聽臣,臣敢獻計,即不能,願先自剄。季布許之。乃髡鉗季布,衣褐衣,置廣柳車中,并與其家僮數十人,之魯朱家所賣之。朱家心知是季布,乃買而置之田。誡其子曰,田事聽此奴,必與同食。朱家乃乘軺車之洛陽,見汝陰侯滕公。滕公留朱家飲數日。因謂滕公曰,季布何大罪,而上求之急也。滕公曰,布數為項羽窘上,上怨之,故必欲得之。朱家曰,君視季布何如人也。曰,賢者也。朱家曰,臣各為其主用,季布為項籍用,職耳。項氏臣可盡誅邪。今上始得天下,獨以己之私怨求一人,何示天下之不廣也。且以季布之賢而漢求之急如此,此不北走胡即南走越耳。夫忌壯士以資敵國,此伍子胥所以鞭荊平王之墓也。君何不從容為上言邪。汝陰侯滕公心知朱家大俠,意季布匿其所,乃許曰,諾。待閒,果言如朱家指。上乃赦季布。當是時,諸公皆多季布能摧剛為柔,朱家亦以此名聞當世。季布召見,謝,上拜為郎中。

孝惠時,為中郎將。單于嘗為書嫚呂后,不遜,呂后大怒,召諸將議之。上將軍樊噲曰,臣願得十萬眾,橫行匈奴中。諸將皆阿呂后意,曰然。季布曰,樊噲可斬也。夫高帝將兵四十餘萬眾,困於平城,今噲柰何以十萬眾橫行匈奴中,面欺。且秦以事於胡,陳勝等起。于今創痍未瘳,噲又面諛,欲搖動天下。是時殿上皆恐,太后罷朝,遂不復議擊匈奴事。

季布為河東守,孝文時,人有言其賢者,孝文召,欲以為御史大夫。復有言其勇,使酒難近。至,留邸一月,見罷。季布因進曰,臣無功竊寵,待罪河東。陛下無故召臣,此人必有以臣欺陛下者,今臣至,無所受事,罷去,此人必有以毀臣者。夫陛下以一人之譽而召臣,一人之毀而去臣,臣恐天下有識聞之有以闚陛下也。上默然慚,良久曰,河東吾股肱郡,故特召君耳。布辭之官。

楚人曹丘生,辯士,數招權顧金錢。事貴人趙同等,與竇長君善。季布聞之,寄書諫竇長君曰,吾聞曹丘生非長者,勿與通。及曹丘生歸,欲得書請季布。竇長君曰,季將軍不說足下,足下無往。固請書,遂行。使人先發書,季布果大怒,待曹丘。曹丘至,即揖季布曰,楚人諺曰得黃金百,不如得季布一諾,足下何以得此聲於梁楚閒哉。且仆楚人,足下亦楚人也。仆游揚足下之名於天下,顧不重邪。何足下距仆之深也。季布乃大說,引入,留數月,為上客,厚送之。季布名所以益聞者,曹丘揚之也。

季布弟季心,氣蓋關中,遇人恭謹,為任俠,方數千里,士皆爭為之死。嘗殺人,亡之吳,從袁絲匿。長事袁絲,弟畜灌夫、籍福之屬。嘗為中司馬,中尉郅都不敢不加禮。少年多時時竊籍其名以行。當是時,季心以勇,布以諾,著聞關中。

季布母弟丁公,為楚將。丁公為項羽逐窘高祖彭城西,短兵接,高祖急,顧丁公曰,兩賢豈相戹哉。於是丁公引兵而還,漢王遂解去。及項王滅,丁公謁見高祖。高祖以丁公徇軍中,曰,丁公為項王臣不忠,使項王失天下者,乃丁公也。遂斬丁公,曰,使後世為人臣者無效丁公。

欒布者,梁人也。始梁王彭越為家人時,嘗與布游。窮困,賃傭於齊,為酒人保。數歲,彭越去之巨野中為盜,而布為人所略賣,為奴於燕。為其家主報仇,燕將臧荼舉以為都尉。臧荼後為燕王,以布為將。及臧荼反,漢擊燕,虜布。梁王彭越聞之,乃言上,請贖布以為梁大夫。

使於齊,未還,漢召彭越,責以謀反,夷三族。已而梟彭越頭於雒陽下,詔曰,有敢收視者,輒捕之。布從齊還,奏事彭越頭下,祠而哭之。吏捕布以聞。上召布,罵曰,若與彭越反邪。吾禁人勿收,若獨祠而哭之,與越反明矣。趣亨之。方提趣湯,布顧曰,願一言而死。上曰,何言。布曰,方上之困於彭城,敗滎陽、成皋閒,項王所以不能遂西,徒以彭王居梁地,與漢合從苦楚也。當是之時,彭王一顧,與楚則漢破,與漢而楚破。且垓下之會,微彭王,項氏不亡。天下已定,彭王剖符受封,亦欲傳之萬世。今陛下一徵兵於梁,彭王病不行,而陛下疑以為反,反形未見,以苛小案誅滅之,臣恐功臣人人自危也。今彭王已死,臣生不如死,請就亨。於是上乃釋布罪,拜為都尉。

孝文時,為燕相,至將軍。布乃稱曰,窮困不能辱身下志,非人也,富貴不能快意,非賢也。於是嘗有德者厚報之,有怨者必以法滅之。吳楚反時,以軍功封俞侯,復為燕相。燕齊之閒皆為欒布立社,號曰欒公社。

景帝中五年薨。子賁嗣,為太常,犧牲不如令,國除。

太史公曰,以項羽之氣,而季布以勇顯於楚,身屨軍搴旗者數矣,可謂壯士。然至被刑戮,為人奴而不死,何其下也。彼必自負其材,故受辱而不羞,欲有所用其未足也,故終為漢名將。賢者誠重其死。夫婢妾賤人感慨而自殺者,非能勇也,其計畫無復之耳。欒布哭彭越,趣湯如歸者,彼誠知所處,不自重其死。雖往古烈士,何以加哉。