\chapter{蘇秦列傳第九}

蘇秦者,東周雒陽人也。東事師於齊,而習之於鬼谷先生。

出游數歲,大困而歸。兄弟嫂妹妻妾竊皆笑之,曰,周人之俗,治產業,力工商,逐什二以為務。今子釋本而事口舌,困,不亦宜乎。蘇秦聞之而慚,自傷,乃閉室不出,出其書遍觀之。曰,夫士業已屈首受書,而不能以取尊榮,雖多亦奚以為。於是得周書陰符,伏而讀之。期年,以出揣摩,曰,此可以說當世之君矣。求說周顯王。顯王左右素習知蘇秦,皆少之。弗信。

乃西至秦。秦孝公卒。說惠王曰,秦四塞之國,被山帶渭,東有關河,西有漢中,南有巴蜀,北有代馬,此天府也。以秦士民之眾,兵法之教,可以吞天下,稱帝而治。秦王曰,毛羽未成,不可以高蜚,文理未明,不可以并兼。方誅商鞅,疾辯士,弗用。

乃東之趙。趙肅侯令其弟成為相,號奉陽君。奉陽君弗說之。

去游燕,歲餘而後得見。說燕文侯曰,燕東有朝鮮、遼東,北有林胡、樓煩,西有雲中、九原,南有♀、易水,地方二千餘里,帶甲數十萬,車六百乘,騎六千匹,粟支數年。南有碣石、鴈門之饒,北有棗栗之利,民雖不佃作而足於棗栗矣。此所謂天府者也。

夫安樂無事,不見覆軍殺將,無過燕者。大王知其所以然乎。夫燕之所以不犯寇被甲兵者,以趙之為蔽其南也。秦趙五戰,秦再勝而趙三勝。秦趙相斃,而王以全燕制其後,此燕之所以不犯寇也。且夫秦之攻燕也,踰雲中、九原,過代、上谷,彌地數千里,雖得燕城,秦計固不能守也。秦之不能害燕亦明矣。今趙之攻燕也,發號出令,不至十日而數十萬之軍軍於東垣矣。渡嘑沱,涉易水,不至四五日而距國都矣。故曰秦之攻燕也,戰於千里之外,趙之攻燕也,戰於百里之內。夫不憂百里之患而重千里之外,計無過於此者。是故願大王與趙從親,天下為一,則燕國必無患矣。

文侯曰,子言則可,然吾國小,西迫彊趙,南近齊,齊、趙彊國也。子必欲合從以安燕,寡人請以國從。

於是資蘇秦車馬金帛以至趙。而奉陽君已死,即因說趙肅侯曰,天下卿相人臣及布衣之士,皆高賢君之行義,皆願奉教陳忠於前之日久矣。雖然,奉陽君而君不任事,是以賓客游士莫敢自盡於前者。今奉陽君捐館舍,君乃今復與士民相親也,臣故敢進其愚慮。

竊為君計者,莫若安民無事,且無庸有事於民也。安民之本,在於擇交,擇交而得則民安,擇交而不得則民終身不安。請言外患,齊秦為兩敵而民不得安,倚秦攻齊而民不得安,倚齊攻秦而民不得安。故夫謀人之主,伐人之國,常苦出辭斷絕人之交也。願君慎勿出於口。請別白黑所以異,陰陽而已矣。君誠能聽臣,燕必致旃裘狗馬之地,齊必致魚鹽之海,楚必致橘柚之園,韓、魏、中山皆可使致湯沐之奉,而貴戚父兄皆可以受封侯。夫割地包利,五伯之所以覆軍禽將而求也,封侯貴戚,湯武之所以放弒而爭也。今君高拱而兩有之,此臣之所以為君願也。

今大王與秦,則秦必弱韓、魏,與齊,則齊必弱楚、魏。魏弱則割河外,韓弱則效宜陽,宜陽效則上郡絕,河外割則道不通,楚弱則無援。此三策者,不可不孰計也。

夫秦下軹道,則南陽危,劫韓包周,則趙氏自操兵,據衛取卷,則齊必入朝秦。秦欲已得乎山東,則必舉兵而向趙矣。秦甲渡河踰漳,據番吾,則兵必戰於邯鄲之下矣。此臣之所為君患也。

當今之時,山東之建國莫彊於趙。趙地方二千餘里,帶甲數十萬,車千乘,騎萬匹,粟支數年。西有常山,南有河漳,東有清河,北有燕國。燕固弱國,不足畏也。秦之所害於天下者莫如趙,然而秦不敢舉兵伐趙者,何也。畏韓、魏之議其後也。然則韓、魏,趙之南蔽也。秦之攻韓、魏也,無有名山大川之限,稍蠶食之,傅國都而止。韓、魏不能支秦,必入臣於秦。秦無韓、魏之規,則禍必中於趙矣。此臣之所為君患也。

臣聞堯無三夫之分,舜無咫尺之地,以有天下,禹無百人之聚,以王諸侯,湯武之士不過三千,車不過三百乘,卒不過三萬,立為天子,誠得其道也。是故明主外料其敵之彊弱,內度其士卒賢不肖,不待兩軍相當而勝敗存亡之機固已形於胸中矣,豈揜於眾人之言而以冥冥決事哉。

臣竊以天下之地圖案之,諸侯之地五倍於秦,料度諸侯之卒十倍於秦,六國為一,并力西鄉而攻秦,秦必破矣。今西面而事之,見臣於秦。夫破人之與破於人也,臣人之與臣於人也,豈可同日而論哉。

夫衡人者,皆欲割諸侯之地以予秦。秦成,則高臺榭,美宮室,聽竽瑟之音,前有樓闕軒轅,後有長姣美人,國被秦患而不與其憂。是故夫衡人日夜務以秦權恐愒諸侯以求割地,故願大王孰計之也。

臣聞明主絕疑去讒,屏流言之跡,塞朋黨之門,故尊主廣地彊兵之計臣得陳忠於前矣。故竊為大王計,莫如一韓、魏、齊、楚、燕、趙以從親,以畔秦。令天下之將相會於洹水之上,通質,刳白馬而盟。要約曰,秦攻楚,齊、魏各出銳師以佐之,韓絕其糧道,趙涉河漳,燕守常山之北。秦攻韓魏,則楚絕其後,齊出銳師而佐之,趙涉河漳,燕守雲中。秦攻齊,則楚絕其後,韓守城皋,魏塞其道,趙涉河漳、博關,燕出銳師以佐之。秦攻燕,則趙守常山,楚軍武關,齊涉勃海,韓、魏皆出銳師以佐之。秦攻趙,則韓軍宜陽,楚軍武關,魏軍河外,齊涉清河,燕出銳師以佐之。諸侯有不如約者,以五國之兵共伐之。六國從親以賓秦,則秦甲必不敢出於函谷以害山東矣。如此,則霸王之業成矣。

趙王曰,寡人年少,立國日淺,未嘗得聞社稷之長計也。今上客有意存天下,安諸侯寡人敬以國從。乃飾車百乘,黃金千溢,白璧百雙,錦繡千純,以約諸侯。

是時周天子致文武之胙於秦惠王。惠王使犀首攻魏,禽將龍賈,取魏之雕陰,且欲東兵。蘇秦恐秦兵之至趙也,乃激怒張儀,入之于秦。

於是說韓宣王曰,韓北有鞏、成皋之固,西有宜陽、商阪之塞,東有宛、穰、洧水,南有陘山,地方九百餘里,帶甲數十萬,天下之彊弓勁弩皆從韓出。谿子、少府時力、距來者,皆射六百步之外。韓卒超足而射,百發不暇止,遠者括蔽洞胸,近者鏑弇心。韓卒之劍戟皆出於冥山、棠谿、墨陽、合賻、鄧師、宛馮、龍淵、太阿,皆陸斷牛馬,水截鵠鴈,當敵則斬堅甲鐵幕,革抉跋芮,無不畢具。以韓卒之勇,被堅甲,蹠勁弩,帶利劍,一人當百,不足言也。夫以韓之勁與大王之賢,乃西面事秦,交臂而服,羞社稷而為天下笑,無大於此者矣。是故願大王孰計之。

大王事秦,秦必求宜陽、成皋。今茲效之,明年又復求割地。與則無地以給之,不與則棄前功而受後禍。且大王之地有盡而秦之求無已,以有盡之地而逆無已之求,此所謂市怨結禍者也,不戰而地已削矣。臣聞鄙諺曰,寧為雞口,無為牛後。今西面交臂而臣事秦,何異於牛後乎。夫以大王之賢,挾彊韓之兵,而有牛後之名,臣竊為大王羞之。

於是韓王勃然作色,攘臂瞋目,按劍仰天太息曰,寡人雖不肖,必不能事秦。今主君詔以趙王之教,敬奉社稷以從。

又說魏襄王曰,大王之地,南有鴻溝、陳、汝南、許、郾、昆陽、召陵、舞陽、新都、新郪,東有淮、潁、煑棗、無胥,西有長城之界,北有河外、卷、衍、酸棗,地方千里。地名雖小,然而田舍廬廡之數,曾無所芻牧。人民之眾,車馬之多,日夜行不絕,輷輷殷殷,若有三軍之眾。臣竊量大王之國不下楚。然衡人怵王交彊虎狼之秦以侵天下,卒有秦患,不顧其禍。夫挾彊秦之勢以內劫其主,罪無過此者。魏,天下之彊國也,王,天下之賢王也。今乃有意西面而事秦,稱東藩,筑帝宮,受冠帶,祠春秋,臣竊為大王恥之。

臣聞越王句踐戰敝卒三千人,禽夫差於干遂,武王卒三千人,革車三百乘,制紂於牧野,豈其士卒眾哉,誠能奮其威也。今竊聞大王之卒,武士二十萬,蒼頭二十萬,奮擊二十萬,廝徒十萬,車六百乘,騎五千匹。此其過越王句踐、武王遠矣,今乃聽於群臣之說而欲臣事秦。夫事秦必割地以效實,故兵未用而國已虧矣。凡群臣之言事秦者,皆姦人,非忠臣也。夫為人臣,割其主之地以求外交,偷取一時之功而不顧其後,破公家而成私門,外挾彊秦之勢以內劫其主,以求割地,願大王孰察之。

周書曰,綿綿不絕,蔓蔓柰何。豪氂不伐,將用斧柯。前慮不定,後有大患,將柰之何。大王誠能聽臣,六國從親,專心并力壹意,則必無彊秦之患。故敝邑趙王使臣效愚計,奉明約,在大王之詔詔之。

魏王曰,寡人不肖,未嘗得聞明教。今主君以趙王之詔詔之,敬以國從。

因東說齊宣王曰,齊南有泰山,東有瑯邪,西有清河,北有勃海,北所謂四塞之國也。齊地方二千餘里,帶甲數十萬,粟如丘山。三軍之良,五家之兵,進如鋒矢,戰如雷霆,解如風雨。即有軍役,未嘗倍泰山,絕清河,涉勃海也。臨菑之中七萬戶,臣竊度之,不下戶三男子,三七二十一萬,不待發於遠縣,而臨菑之卒固已二十一萬矣。臨菑甚富而實,其民無不吹竽鼓瑟,彈琴擊筑,鬬雞走狗,六博蹋鞠者。臨菑之涂,車轂擊,人肩摩,連衽成帷,舉袂成幕,揮汗成雨,家殷人足,志高氣揚。夫以大王之賢與齊之彊,天下莫能當。今乃西面而事秦,臣竊為大王羞之。

且夫韓、魏之所以重畏秦者,為與秦接境壤界也。兵出而相當,不出十日而戰勝存亡之機決矣。韓、魏戰而勝秦,則兵半折,四境不守,戰而不勝,則國已危亡隨其後。是故韓、魏之所以重與秦戰,而輕為之臣也。今秦之攻齊則不然。倍韓、魏之地,過衛陽晉之道,徑乎亢父之險,車不得方軌,騎不得比行,百人守險,千人不敢過也。秦雖欲深入,則狼顧,恐韓、魏之議其後也。是故恫疑虛猲,驕矜而不敢進,則秦之不能害齊亦明矣。

夫不深料秦之無柰齊何,而欲西面而事之,是群臣之計過也。今無臣事秦之名而有彊國之實,臣是故願大王少留意計之。

齊王曰,寡人不敏,僻遠守海,窮道東境之國也,未嘗得聞餘教。今足下以趙王詔詔之,敬以國從。

乃西南說楚威王曰,楚,天下之彊國也,王,天下之賢王也。西有黔中、巫郡,東有夏州、海陽,南有洞庭、蒼梧,北有陘塞、郇陽,地方五千餘里,帶甲百萬,車千乘,騎萬匹,粟支十年。此霸王之資也。夫以楚之彊與王之賢,天下莫能當也。今乃欲西面而事秦,則諸侯莫不西面而朝於章臺之下矣。

秦之所害莫如楚,楚彊則秦弱,秦彊則楚弱,其勢不兩立。故為大王計,莫如從親以孤秦。大王不從親,秦必起兩軍,一軍出武關,一軍下黔中,則鄢郢動矣。

臣聞治之其未亂也,為之其未有也。患至而後憂之,則無及已。故願大王蚤孰計之。

大王誠能聽臣,臣請令山東之國奉四時之獻,以承大王之明詔,委社稷,奉宗廟,練士厲兵,在大王之所用之。大王誠能用臣之愚計,則韓、魏、齊、燕、趙、衛之妙音美人必充後宮,燕、代橐駝良馬必實外廄。故從合則楚王,衡成則秦帝。今釋霸王之業,而有事人之名,臣竊為大王不取也。

夫秦,虎狼之國也,有吞天下之心。秦,天下之仇讎也。衡人皆欲割諸侯之地以事秦,此所謂養仇而奉讎者也。夫為人臣,割其主之地以外交彊虎狼之秦,以侵天下,卒有秦患,不顧其禍。夫外挾彊秦之威以內劫其主,以求割地,大逆不忠,無過此者。故從親則諸侯割地以事楚,衡合則楚割地以事秦,此兩策者相去遠矣,二者大王何居焉。故敝邑趙王使臣效愚計,奉明約,在大王詔之。

楚王曰,寡人之國西與秦接境,秦有舉巴蜀并漢中之心。秦,虎狼之國,不可親也。而韓、魏迫於秦患,不可與深謀,與深謀恐反人以入於秦,故謀未發而國已危矣。寡人自料以楚當秦,不見勝也,內與群臣謀,不足恃也。寡人臥不安席,食不甘味,心搖搖然如縣旌而無所終薄。今主君欲一天下,收諸侯,存危國,寡人謹奉社稷以從。

於是六國從合而并力焉。蘇秦為從約長,并相六國。

北報趙王,乃行過雒陽,車騎輜重,諸侯各發使送之甚眾,疑於王者。周顯王聞之恐懼,除道,使人郊勞。蘇秦之昆弟妻嫂側目不敢仰視,俯伏侍取食。蘇秦笑謂其嫂曰,何前倨而後恭也。嫂委蛇蒲服,以面掩地而謝曰,見季子位高金多也。蘇秦喟然嘆曰,此一人之身,富貴則親戚畏懼之,貧賤則輕易之,況眾人乎。且使我有雒陽負郭田二頃,吾豈能佩六國相印乎。於是散千金以賜宗族朋友。初,蘇秦之燕,貸人百錢為資,乃得富貴,以百金償之。遍報諸所嘗見德者。其從者有一人獨未得報,乃前自言。蘇秦曰,我非忘子。子之與我至燕,再三欲去我易水之上,方是時,我困,故望子深,是以後子。子今亦得矣。

蘇秦既約六國從親,歸趙,趙肅侯封為武安君,乃投從約書於秦。秦兵不敢闚函谷關十五年。

其後秦使犀首欺齊、魏,與共伐趙,欲敗從約。齊、魏伐趙,趙王讓蘇秦。蘇秦恐,請使燕,必報齊。蘇秦去趙而從約皆解。

秦惠王以其女為燕太子婦。是歲,文侯卒,太子立,是為燕易王。易王初立,齊宣王因燕喪伐燕,取十城。易王謂蘇秦曰,往日先生至燕,而先王資先生見趙,遂約六國從。今齊先伐趙,次至燕,以先生之故為天下笑,先生能為燕得侵地乎。蘇秦大慚,曰,請為王取之。

蘇秦見齊王,再拜,俯而慶,仰而弔。齊王曰,是何慶弔相隨之速也。蘇秦曰,臣聞饑人所以饑而不食烏喙者,為其愈充腹而與餓死同患也。今燕雖弱小,即秦王之少婿也。大王利其十城而長與彊秦為仇。今使弱燕為鴈行而彊秦敝其後,以招天下之精兵,是食烏喙之類也。齊王愀然變色曰,然則柰何。蘇秦曰,臣聞古之善制事者,轉禍為福,因敗為功。大王誠能聽臣計,即歸燕之十城。燕無故而得十城,必喜,秦王知以己之故而歸燕之十城,亦必喜。此所謂棄仇讎而得石交者也。夫燕、秦俱事齊,則大王號令天下,莫敢不聽。是王以虛辭附秦,以十城取天下。此霸王之業也。王曰,善。於是乃歸燕之十城。

人有毀蘇秦者曰,左右賣國反覆之臣也,將作亂。蘇秦恐得罪歸,而燕王不復官也。蘇秦見燕王曰,臣,東周之鄙人也,無有分寸之功,而王親拜之於廟而禮之於廷。今臣為王卻齊之兵而攻得十城,宜以益親。今來而王不官臣者,人必有以不信傷臣於王者。臣之不信,王之福也。臣聞忠信者,所以自為也,進取者,所以為人也。且臣之說齊王,曾非欺之也。臣棄老母於東周,固去自為而行進取也。今有孝如曾參,廉如伯夷,信如尾生。得此三人者以事大王,何若。王曰,足矣。蘇秦曰,孝如曾參,義不離其親一宿於外,王又安能使之步行千里而事弱燕之危王哉。廉如伯夷,義不為孤竹君之嗣,不肯為武王臣,不受封侯而餓死首陽山下。有廉如此,王又安能使之步行千里而行進取於齊哉。信如尾生,與女子期於梁下,女子不來,水至不去,抱柱而死。有信如此,王又安能使之步行千里卻齊之彊兵哉。臣所謂以忠信得罪於上者也。燕王曰,若不忠信耳,豈有以忠信而得罪者乎。蘇秦曰,不然。臣聞客有遠為吏而其妻私於人者,其夫將來,其私者憂之,妻曰勿憂,吾已作藥酒待之矣。居三日,其夫果至,妻使妾舉藥酒進之。妾欲言酒之有藥,則恐其逐主母也,欲勿言乎,則恐其殺主父也。於是乎詳僵而棄酒。主父大怒,笞之五十。故妾一僵而覆酒,上存主父,下存主母,然而不免於笞,惡在乎忠信之無罪也。夫臣之過,不幸而類是乎。燕王曰,先生復就故官。益厚遇之。

易王母,文侯夫人也,與蘇秦私通。燕王知之,而事之加厚。蘇秦恐誅,乃說燕王曰,臣居燕不能使燕重,而在齊則燕必重。燕王曰,唯先生之所為。於是蘇秦詳為得罪於燕而亡走齊,齊宣王以為客卿。

齊宣王卒,湣王即位,說湣王厚葬以明孝,高宮室大苑囿以明得意,欲破敝齊而為燕。燕易王卒,燕噲立為王。其後齊大夫多與蘇秦爭寵者,而使人刺蘇秦,不死,殊而走。齊王使人求賊,不得。蘇秦且死,乃謂齊王曰,臣即死,車裂臣以徇於市,曰蘇秦為燕作亂於齊,如此則臣之賊必得矣。於是如其言,而殺蘇秦者果自出,齊王因而誅之。燕聞之曰,甚矣,齊之為蘇生報仇也。

蘇秦既死,其事大泄。齊後聞之,乃恨怒燕。燕甚恐。蘇秦之弟曰代,代弟蘇厲,見兄遂,亦皆學。及蘇秦死,代乃求見燕王,欲襲故事。曰,臣,東周之鄙人也。竊聞大王義甚高,鄙人不敏,釋鉏耨而干大王。至於邯鄲,所見者絀於所聞於東周,臣竊負其志。及至燕廷,觀王之群臣下吏,王,天下之明王也。燕王曰,子所謂明王者何如也。對曰,臣聞明王務聞其過,不欲聞其善,臣請謁王之過。夫齊、趙者,燕之仇讎也,楚、魏者,燕之援國也。今王奉仇讎以伐援國,非所以利燕也。王自慮之,此則計過,無以聞者,非忠臣也。王曰,夫齊者固寡人之讎,所欲伐也,直患國敝力不足也。子能以燕伐齊,則寡人舉國委子。對曰,凡天下戰國七,燕處弱焉。獨戰則不能,有所附則無不重。南附楚,楚重,西附秦,秦重,中附韓、魏,韓、魏重。且茍所附之國重,此必使王重矣。今夫齊,長主而自用也。南攻楚五年,畜聚竭,西困秦三年,士卒罷敝,北與燕人戰,覆三軍,得二將。然而以其餘兵南面舉五千乘之大宋,而包十二諸侯。此其君欲得,其民力竭,惡足取乎。且臣聞之,數戰則民勞,久師則兵敝矣。燕王曰,吾聞齊有清濟、濁河可以為固,長城、鉅防足以為塞,誠有之乎。對曰,天時不與,雖有清濟、濁河,惡足以為固。民力罷敝,雖有長城、鉅防,惡足以為塞。且異日濟西不師,所以備趙也,河北不師,所以備燕也。今濟西河北盡已役矣,封內敝矣。夫驕君必好利,而亡國之臣必貪於財。王誠能無羞從子母弟以為質,寶珠玉帛以事左右,彼將有德燕而輕亡宋,則齊可亡已。燕王曰,吾終以子受命於天矣。燕乃使一子質於齊。而蘇厲因燕質子而求見齊王。齊王怨蘇秦,欲囚蘇厲。燕質子為謝,已遂委質為齊臣。

燕相子之與蘇代婚,而欲得燕權,乃使蘇代侍質子於齊。齊使代報燕,燕王噲問曰,齊王其霸乎。曰,不能。曰,何也。曰,不信其臣。於是燕王專任子之,已而讓位,燕大亂。齊伐燕,殺王噲、子之。燕立昭王,而蘇代、蘇厲遂不敢入燕,皆終歸齊,齊善待之。

蘇代過魏,魏為燕執代。齊使人謂魏王曰,齊請以宋地封涇陽君,秦必不受。秦非不利有齊而得宋地也,不信齊王與蘇子也。今齊魏不和如此其甚,則齊不欺秦。秦信齊,齊秦合,涇陽君有宋地,非魏之利也。故王不如東蘇子,秦必疑齊而不信蘇子矣。齊秦不合,天下無變,伐齊之形成矣。於是出蘇代。代之宋,宋善待之。

齊伐宋,宋急,蘇代乃遺燕昭王書曰,

夫列在萬乘而寄質於齊,名卑而權輕,奉萬乘助齊伐宋,民勞而實費,夫破宋,殘楚淮北,肥大齊,讎彊而國害,此三者皆國之大敗也。然且王行之者,將以取信於齊也。齊加不信於王,而忌燕愈甚,是王之計過矣。夫以宋加之淮北,彊萬乘之國也,而齊并之,是益一齊也。北夷方七百里,加之以魯、衛,彊萬乘之國也,而齊并之,是益二齊也。夫一齊之彊,燕猶狼顧而不能支,今以三齊臨燕,其禍必大矣。

雖然,智者舉事,因禍為福,轉敗為功。齊紫,敗素也,而賈十倍,越王句踐棲於會稽,復殘彊吳而霸天下,此皆因禍為福,轉敗為功者也。

今王若欲因禍為福,轉敗為功,則莫若挑霸齊而尊之,使使盟於周室,焚秦符,曰其大上計,破秦,其次,必長賓之。秦挾賓以待破,秦王必患之。秦五世伐諸侯,今為齊下,秦王之志茍得窮齊,不憚以國為功。然則王何不使辯士以此言說秦王曰,燕、趙破宋肥齊,尊之為之下者,燕、趙非利之也。燕、趙不利而勢為之者,以不信秦王也。然則王何不使可信者接收燕、趙,令涇陽君、高陵君先於燕、趙。秦有變,因以為質,則燕、趙信秦。秦為西帝,燕為北帝,趙為中帝,立三帝以令於天下。韓、魏不聽則秦伐之,齊不聽則燕、趙伐之,天下孰敢不聽。天下服聽,因驅韓、魏以伐齊,曰必反宋地,歸楚淮北。反宋地,歸楚淮北,燕、趙之所利也,并立三帝,燕、趙之所願也。夫實得所利,尊得所願,燕、趙棄齊如脫屣矣。今不收燕、趙,齊霸必成。諸侯贊齊而王不從,是國伐也,諸侯贊齊而王從之,是名卑也。今收燕、趙,國安而名尊,不收燕、趙,國危而名卑。夫去尊安而取危卑,智者不為也。秦王聞若說,必若刺心然。則王何不使辯士以此若言說秦。秦必取,齊必伐矣。

夫取秦,厚交也,伐齊,正利也。尊厚交,務正利,聖王之事也。

燕昭王善其書,曰,先人嘗有德蘇氏,子之之亂而蘇氏去燕。燕欲報仇於齊,非蘇氏莫可。乃召蘇代,復善待之,與謀伐齊。竟破齊,湣王出走。

久之,秦召燕王,燕王欲往,蘇代約燕王曰,楚得枳而國亡,齊得宋而國亡,齊、楚不得以有枳、宋而事秦者,何也。則有功者,秦之深讎也。秦取天下,非行義也,暴也。秦之行暴,正告天下。

告楚曰,蜀地之甲,乘船浮於汶,乘夏水而下江,五日而至郢。漢中之甲,乘船出於巴,乘夏水而下漢,四日而至五渚。寡人積甲宛東下隨,智者不及謀,勇土不及怒,寡人如射隼矣。王乃欲待天下之攻函谷,不亦遠乎。楚王為是故,十七年事秦。

秦正告韓曰,我起乎少曲,一日而斷大行。我起乎宜陽而觸平陽,二日而莫不盡繇。我離兩周而觸鄭,五日而國舉。韓氏以為然,故事秦。

秦正告魏曰,我舉安邑,塞女戟,韓氏太原卷。我下軹,道南陽,封冀,包兩周。乘夏水,浮輕舟,錟弩在前,錟戈在後,決滎口,魏無大梁,決白馬之口,魏無外黃、濟陽,決宿胥之口,魏無虛、頓丘。陸攻則擊河內,水攻則滅大梁。魏氏以為然,故事秦。

秦欲攻安邑,恐齊救之,則以宋委於齊。曰,宋王無道,為木人以象寡人,射其面。寡人地絕兵遠,不能攻也。王茍能破宋有之,寡人如自得之。已得安邑,塞女戟,因以破宋為齊罪。

秦欲攻韓,恐天下救之,則以齊委於天下。曰,齊王四與寡人約,四欺寡人,必率天下以攻寡人者三。有齊無秦,有秦無齊,必伐之,必亡之。已得宜陽、少曲,致藺、離石,因以破齊為天下罪。

秦欲攻魏重楚,則以南陽委於楚。曰,寡人固與韓且絕矣。殘均陵,塞鄳阸,茍利於楚,寡人如自有之。魏棄與國而合於秦,因以塞鄳阸為楚罪。

兵困於林中,重燕、趙,以膠東委於燕,以濟西委於趙。已得講於魏,至公子延,因犀首屬行而攻趙。

兵傷於譙石,而遇敗於陽馬,而重魏,則以葉、蔡委於魏。已得講於趙,則劫魏,魏不為割。困則使太后弟穰侯為和,嬴則兼欺舅與母。

適燕者曰以膠東,適趙者曰以濟西,適魏者曰以葉、蔡,適楚者曰以塞鄳阸,適齊者曰以宋,此必令言如循環,用兵如刺蜚,母不能制,舅不能約。

龍賈之戰,岸門之戰,封陵之戰,高商之戰,趙莊之戰,秦之所殺三晉之民數百萬,今其生者皆死秦之孤也。西河之外,上雒之地,三川晉國之禍,三晉之半,秦禍如此其大也。而燕、趙之秦者,皆以爭事秦說其主,此臣之所大患也。

燕昭王不行。蘇代復重於燕。

燕使約諸侯從親如蘇秦時,或從或不,而天下由此宗蘇氏之從約。代、厲皆以壽死,名顯諸侯。

太史公曰,蘇秦兄弟三人,皆游說諸侯以顯名,其術長於權變。而蘇秦被反閒以死,天下共笑之,諱學其術。然世言蘇秦多異,異時事有類之者皆附之蘇秦。夫蘇秦起閭閻,連六國從親,此其智有過人者。吾故列其行事,次其時序,毋令獨蒙惡聲焉。