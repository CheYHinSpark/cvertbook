\chapter{黥布列傳第三十一}

黥布者,六人也,姓英氏。秦時為布衣。少年,有客相之曰,當刑而王。及壯,坐法黥。布欣然笑曰,人相我當刑而王,幾是乎。人有聞者,共俳笑之。布已論輸麗山,麗山之徒數十萬人,布皆與其徒長豪桀交通,乃率其曹偶,亡之江中為群盜。

陳勝之起也,布乃見番君,與其眾叛秦,聚兵數千人。番君以其女妻之。章邯之滅陳勝,破呂臣軍,布乃引兵北擊秦左右校,破之清波,引兵而東。聞項梁定江東會稽,涉江而西。陳嬰以項氏世為楚將,乃以兵屬項梁,渡淮南,英布、蒲將軍亦以兵屬項梁。

項梁涉淮而西,擊景駒、秦嘉等,布常冠軍。項梁至薛,聞陳王定死,迺立楚懷王。項梁號為武信君,英布為當陽君。項梁敗死定陶,懷王徙都彭城,諸將英布亦皆保聚彭城。當是時,秦急圍趙,趙數使人請救。懷王使宋義為上將,范曾為末將,項籍為次將,英在、蒲將軍皆為將軍,悉屬宋義,北救趙。及項籍殺宋義於河上,懷王因立籍為上將軍,諸將皆屬項籍。項籍使布先渡河擊秦,布數有利,籍迺悉引兵涉河從之,遂破秦軍,降章邯等。楚兵常勝,功冠諸侯。諸侯兵皆以服屬楚者,以布數以少敗眾也。

項籍之引兵西至新安,又使布等夜擊阬章邯秦卒二十餘萬人。至關,不得入,又使布等先從閒道破關下軍,遂得入,至咸陽。布常為軍鋒。項王封諸將,立布為九江王,都六。

漢元年四月,諸侯皆罷戲下,各就國。項氏立懷王為義帝,徙都長沙,乃陰令九江王布等行擊之。其八月,布使將擊義帝,追殺之郴縣。

漢二年,齊王田榮畔楚,項王往擊齊,徵兵九江,九江王布稱病不往,遣將將數千人行。漢之敗楚彭城,布又稱病不佐楚。項王由此怨布,數使使者誚讓召布,布愈恐,不敢往。項王方北憂齊、趙,西患漢,所與者獨九江王,又多布材,欲親用之,以故未擊。

漢三年,漢王擊楚,大戰彭城,不利,出梁地,至虞,謂左右曰,如彼等者,無足與計天下事。謁者隨何進曰,不審陛下所謂。漢王曰,孰能為我使淮南,令之發兵倍楚,留項王於齊數月,我之取天下可以百全。隨何曰,臣請使之。乃與二十人俱,使淮南。至,因太宰主之,三日不得見。隨何因說太宰曰,王之不見何,必以楚為彊,以漢為弱,此臣之所以為使。使何得見,言之而是邪,是大王所欲聞也,言之而非邪,使何等二十人伏斧質淮南市,以明王倍漢而與楚也。太宰乃言之王,王見之。隨何曰,漢王使臣敬進書大王御者,竊怪大王與楚何親也。淮南王曰,寡人北鄉而臣事之。隨何曰,大王與項王俱列為諸侯,北鄉而臣事之,必以楚為彊,可以託國也。項王伐齊,身負板筑,以為士卒先,大王宜悉淮南之眾,身自將之,為楚軍前鋒,今乃發四千人以助楚。夫北面而臣事人者,固若是乎。夫漢王戰於彭城,項王未出齊也,大王宜騷淮南之兵渡淮,日夜會戰彭城下,大王撫萬人之眾,無一人渡淮者,垂拱而觀其孰勝。夫託國於人者,固若是乎。大王提空名以鄉楚,而欲厚自託,臣竊為大王不取也。然而大王不背楚者,以漢為弱也。夫楚兵雖彊,天下負之以不義之名,以其背盟約而殺義帝也。然而楚王恃戰勝自彊,漢王收諸侯,還守成皋、滎陽,下蜀、漢之粟,深溝壁壘,分卒守徼乘塞,楚人還兵,閒以梁地,深入敵國八九百里,欲戰則不得,攻城則力不能,老弱轉糧千里之外,楚兵至滎陽、成皋,漢堅守而不動,進則不得攻,退則不得解。故曰楚兵不足恃也。使楚勝漢,則諸侯自危懼而相救。夫楚之彊,適足以致天下之兵耳。故楚不如漢,其勢易見也。今大王不與萬全之漢而自託於危亡之楚,臣竊為大王惑之。臣非以淮南之兵足以亡楚也。夫大王發兵而倍楚,項王必留,留數月,漢之取天下可以萬全。臣請與大王提劍而歸漢,漢王必裂地而封大王,又況淮南,淮南必大王有也。故漢王敬使使臣進愚計,願大王之留意也。淮南王曰,請奉命。陰許畔楚與漢,未敢泄也。

楚使者在,方急責英布發兵,舍傳舍。隨何直入,坐楚使者上坐,曰,九江王已歸漢,楚何以得發兵。布愕然。楚使者起。何因說布曰,事已搆,可遂殺楚使者,無使歸,而疾走漢并力。布曰,如使者教,因起兵而擊之耳。於是殺使者,因起兵而攻楚。楚使項聲、龍且攻淮南,項王留而攻下邑。數月,龍且擊淮南,破布軍。布欲引兵走漢,恐楚王殺之,故閒行與何俱歸漢。

淮南王至,上方踞床洗,召布入見,布大怒,悔來,欲自殺。出就舍,帳御飲食從官如漢王居,布又大喜過望。於是乃使人入九江。楚已使項伯收九江兵,盡殺布妻子。布使者頗得故人幸臣,將眾數千人歸漢。漢益分布兵而與俱北,收兵至成皋。四年七月,立布為淮南王,與擊項籍。

漢五年,布使人入九江,得數縣。六年,布與劉賈入九江,誘大司馬周殷,周殷反楚,遂舉九江兵與漢擊楚,破之垓下。

項籍死,天下定,上置酒。上折隨何之功,謂何為腐儒,為天下安用腐儒。隨何跪曰,夫陛下引兵攻彭城,楚王未去齊也,陛下發步卒五萬人,騎五千,能以取淮南乎。上曰,不能。隨何曰,陛下使何與二十人使淮南,至,如陛下之意,是何之功賢於步卒五萬人騎五千也。然而陛下謂何腐儒,為天下安用腐儒,何也。上曰,吾方圖子之功。乃以隨何為護軍中尉。布遂剖符為淮南王,都六,九江、廬江、衡山、豫章郡皆屬布。

七年,朝陳。八年,朝雒陽。九年,朝長安。

十一年,高后誅淮陰侯,布因心恐。夏,漢誅梁王彭越,醢之,盛其醢遍賜諸侯。至淮南,淮南王方獵,見醢,因大恐,陰令人部聚兵,候伺旁郡警急。

布所幸姬疾,請就醫,醫家與中大夫賁赫對門,姬數如醫家,賁赫自以為侍中,乃厚餽遺,從姬飲醫家。姬侍王,從容語次,譽赫長者也。王怒曰,汝安從知之。具說狀。王疑其與亂。赫恐,稱病。王愈怒,欲捕赫。赫言變事,乘傳詣長安。布使人追,不及。赫至,上變,言布謀反有端,可先未發誅也。上讀其書,語蕭相國。相國曰,布不宜有此,恐仇怨妄誣之。請擊赫,使人微驗淮南王。淮南王布見赫以罪亡,上變,固已疑其言國陰事,漢使又來,頗有所驗,遂族赫家,發兵反。反書聞,上乃赦賁赫,以為將軍。

上召諸將問曰,布反,為之柰何。皆曰,發兵擊之,阬豎子耳。何能為乎。汝陰侯滕公召故楚令尹問之。令尹曰,是故當反。滕公曰,上裂地而王之,疏爵而貴之,南面而立萬乘之主,其反何也。令尹曰,往年殺彭越,前年殺韓信,此三人者,同功一體之人也。自疑禍及身,故反耳。滕公言之上曰,臣客故楚令尹薛公者,其人有籌筴之計,可問。上乃召見問薛公。薛公對曰,布反不足怪也。使布出於上計,山東非漢之有也,出於中計,勝敗之數未可知也,出於下計,陛下安枕而臥矣。上曰,何謂上計。令尹對曰,東取吳,西取楚,并齊取魯,傳檄燕、趙,固守其所,山東非漢之有也。何謂中計。東取吳,西取楚,并韓取魏,據敖庾之粟,塞成皋之口,勝敗之數未可知也。何謂下計。東取吳,西取下蔡,歸重於越,身歸長沙,陛下安枕而臥,漢無事矣。上曰,是計將安出。令尹對曰,出下計。上曰,何謂廢上中計而出下計。令尹曰,布故麗山之徒也,自致萬乘之主,此皆為身,不顧後為百姓萬世慮者也,故曰出下計。上曰,善。封薛公千戶。乃立皇子長為淮南王。上遂發兵自將東擊布。

布之初反,謂其將曰,上老矣,厭兵,必不能來。使諸將,諸將獨患淮陰、彭越,今皆已死,餘不足畏也。故遂反。果如薛公籌之,東擊荊,荊王劉賈走死富陵。盡劫其兵,渡淮擊楚。楚發兵與戰徐、僮閒,為三軍,欲以相救為奇。或說楚將曰,布善用兵,民素畏之。且兵法,諸侯戰其地為散地。今別為三,彼敗吾一軍,餘皆走,安能相救。不聽。布果破其一軍,其二軍散走。

遂西,與上兵遇蘄西,會甀。布兵精甚,上乃壁庸城,望布軍置陳如項籍軍,上惡之。與布相望見,遙謂布曰,何苦而反。布曰,欲為帝耳。上怒罵之,遂大戰。布軍敗走,渡淮,數止戰,不利,與百餘人走江南。布故與番君婚,以故長沙哀王使人紿布,偽與亡,誘走越,故信而隨之番陽。番陽人殺布茲鄉民田舍,遂滅黥布。

立皇子長為淮南王,封賁赫為期思侯,諸將率多以功封者。

太史公曰,英布者,其先豈春秋所見楚滅英、六,皋陶之後哉。身被刑法,何其拔興之暴也。項氏之所阬殺人以千萬數,而布常為首虐。功冠諸侯,用此得王,亦不免於身為世大僇。禍之興自愛姬殖,妒媢生患,竟以滅國。