\chapter{魏公子列傳第十七}

魏公子無忌者,魏昭王子少子而魏安釐王異母弟也。昭王薨,安釐王即位,封公子為信陵君。是時范睢亡魏相秦,以怨魏齊故,秦兵圍大梁,破魏華陽下軍,走芒卯。魏王及公子患之。

公子為人仁而下士,士無賢不肖皆謙而禮交之,不敢以其富貴驕士。士以此方數千里爭往歸之,致食客三千人。當是時,諸侯以公子賢,多客,不敢加兵謀魏十餘年。

公子與魏王博,而北境傳舉烽,言趙寇至,且入界。魏王釋博,欲召大臣謀。公子止王曰,趙王田獵耳,非為寇也。復博如故。王恐,心不在博。居頃,復從北方來傳言曰,趙王獵耳,非為寇也。魏王大驚,曰,公子何以知之。公子曰,臣之客有能深得趙王陰事者,趙王所為,客輒以報臣,臣以此知之。是後魏王畏公子之賢能,不敢任公子以國政。

魏有隱士曰侯嬴,年七十,家貧,為大梁夷門監者。公子聞之,往請,欲厚遺之。不肯受,曰,臣修身絜行數十年,終不以監門困故而受公子財。公子於是乃置酒大會賓客。坐定,公子從車騎,虛左,自迎夷門侯生。侯生攝敝衣冠,直上載公子上坐,不讓,欲以觀公子。公子執轡愈恭。侯生又謂公子曰,臣有客在市屠中,願枉車騎過之。公子引車入市,侯生下見其客朱亥,俾倪故久立,與其客語,微察公子。公子顏色愈和。當是時,魏將相宗室賓客滿堂,待公子舉酒。市人皆觀公子執轡。從騎皆竊罵侯生。侯生視公子色終不變,乃謝客就車。至家,公子引侯生坐上坐,遍贊賓客,賓客皆驚。酒酣,公子起,為壽侯生前。侯生因謂公子曰,今日嬴之為公子亦足矣。嬴乃夷門抱關者也,而公子親枉車騎,自迎嬴於眾人廣坐之中,不宜有所過,今公子故過之。然嬴欲就公子之名,故久立公子車騎市中,過客以觀公子,公子愈恭。市人皆以嬴為小人,而以公子為長者能下士也。於是罷酒,侯生遂為上客。

侯生謂公子曰,臣所過屠者朱亥,此子賢者,世莫能知,故隱屠閒耳。公子往數請之,朱亥故不復謝,公子怪之。

魏安釐王二十年,秦昭王已破趙長平軍,又進兵圍邯鄲。公子姊為趙惠文王弟平原君夫人,數遺魏王及公子書,請救於魏。魏王使將軍晉鄙將十萬眾救趙。秦王使使者告魏王曰,吾攻趙旦暮且下,而諸侯敢救者,已拔趙,必移兵先擊之。魏王恐,使人止晉鄙,留軍壁鄴,名為救趙,實持兩端以觀望。平原君使者冠蓋相屬於魏,讓魏公子曰,勝所以自附為婚姻者,以公子之高義,為能急人之困。今邯鄲旦暮降秦而魏救不至,安在公子能急人之困也。且公子縱輕勝,棄之降秦,獨不憐公子姊邪。公子患之,數請魏王,及賓客辯士說王萬端。魏王畏秦,終不聽公子。公子自度終不能得之於王,計不獨生而令趙亡,乃請賓客,約車騎百餘乘,欲以客往赴秦軍,與趙俱死。

行過夷門,見侯生,具告所以欲死秦軍狀。辭決而行,侯生曰,公子勉之矣,老臣不能從。公子行數里,心不快,曰,吾所以待侯生者備矣,天下莫不聞,今吾且死而侯生曾無一言半辭送我,我豈有所失哉。復引車還,問侯生。侯生笑曰,臣固知公子之還也。曰,公子喜士,名聞天下。今有難,無他端而欲赴秦軍,譬若以肉投餒虎,何功之有哉。尚安事客。然公子遇臣厚,公子往而臣不送,以是知公子恨之復返也。公子再拜,因問。侯生乃屏人閒語,曰,嬴聞晉鄙之兵符常在王臥內,而如姬最幸,出入王臥內,力能竊之。嬴聞如姬父為人所殺,如姬資之三年,自王以下欲求報其父仇,莫能得。如姬為公子泣,公子使客斬其仇頭,敬進如姬。如姬之欲為公子死,無所辭,顧未有路耳。公子誠一開口請如姬,如姬必許諾,則得虎符奪晉鄙軍,北救趙而西卻秦,此五霸之伐也。公子從其計,請如姬。如姬果盜晉鄙兵符與公子。

公子行,侯生曰,將在外,主令有所不受,以便國家。公子即合符,而晉鄙不授公子兵而復請之,事必危矣。臣客屠者朱亥可與俱,此人力士。晉鄙聽,大善,不聽,可使擊之。於是公子泣。侯生曰,公子畏死邪。何泣也。公子曰,晉鄙嚄唶宿將,往恐不聽,必當殺之,是以泣耳,豈畏死哉。於是公子請朱亥。朱亥笑曰,臣乃市井鼓刀屠者,而公子親數存之,所以不報謝者,以為小禮無所用。今公子有急,此乃臣效命之秋也。遂與公子俱。公子過謝侯生。侯生曰,臣宜從,老不能。請數公子行日,以至晉鄙軍之日,北鄉自剄,以送公子。公子遂行。

至鄴,矯魏王令代晉鄙。晉鄙合符,疑之,舉手視公子曰,今吾擁十萬之眾,屯於境上,國之重任,今單車來代之,何如哉。欲無聽。朱亥袖四十斤鐵椎,椎殺晉鄙,公子遂將晉鄙軍。勒兵下令軍中曰,父子俱在軍中,父歸,兄弟俱在軍中,兄歸,獨子無兄弟,歸養。得選兵八萬人,進兵擊秦軍。秦軍解去,遂救邯鄲,存趙。趙王及平原君自迎公子於界,平原君負ゆ矢為公子先引。趙王再拜曰,自古賢人未有及公子者也。當此之時,平原君不敢自比於人。公子與侯生決,至軍,侯生果北鄉自剄。

魏王怒公子之盜其兵符,矯殺晉鄙,公子亦自知也。已卻秦存趙,使將將其軍歸魏,而公子獨與客留趙。趙孝成王德公子之矯奪晉鄙兵而存趙,乃與平原君計,以五城封公子。公子聞之,意驕矜而有自功之色。客有說公子曰,物有不可忘,或有不可不忘。夫人有德於公子,公子不可忘也,公子有德於人,願公子忘之也。且矯魏王令,奪晉鄙兵以救趙,於趙則有功矣,於魏則未為忠臣也。公子乃自驕而功之,竊為公子不取也。於是公子立自責,似若無所容者。趙王埽除自迎,執主人之禮,引公子就西階。公子側行辭讓,從東階上。自言罪過,以負於魏,無功於趙。趙王侍酒至暮,口不忍獻五城,以公子退讓也。公子竟留趙。趙王以鄗為公子湯沐邑,魏亦復以信陵奉公子。公子留趙。

公子聞趙有處士毛公藏於博徒,薛公藏於賣漿家,公子欲見兩人,兩人自匿不肯見公子。公子聞所在,乃閒步往從此兩人游,甚歡。平原君聞之,謂其夫人曰,始吾聞夫人弟公子天下無雙,今吾聞之,乃妄從博徒賣漿者游,公子妄人耳。夫人以告公子。公子乃謝夫人去,曰,始吾聞平原君賢,故負魏王而救趙,以稱平原君。平原君之游,徒豪舉耳,不求士也。無忌自在大梁時,常聞此兩人賢,至趙,恐不得見。以無忌從之游,尚恐其不我欲也,今平原君乃以為羞,其不足從游。乃裝為去。夫人具以語平原君。平原君乃免冠謝,固留公子。平原君門下聞之,半去平原君歸公子,天下士復往歸公子,公子傾平原君客。

公子留趙十年不歸。秦聞公子在趙,日夜出兵東伐魏。魏王患之,使使往請公子。公子恐其怒之,乃誡門下,有敢為魏王使通者,死。賓客皆背魏之趙,莫敢勸公子歸。毛公、薛公兩人往見公子曰,公子所以重於趙,名聞諸侯者,徒以有魏也。今秦攻魏,魏急而公子不恤,使秦破大梁而夷先王之宗廟,公子當何面目立天下乎。語未及卒,公子立變色,告車趣駕歸救魏。

魏王見公子,相與泣,而以上將軍印授公子,公子遂將。魏安釐王三十年,公子使使遍告諸侯。諸侯聞公子將,各遣將將兵救魏。公子率五國之兵破秦軍於河外,走蒙驁。遂乘勝逐秦軍至函谷關,抑秦兵,秦兵不敢出。當是時,公子威振天下,諸侯之客進兵法,公子皆名之,故世俗稱魏公子兵法。

秦王患之,乃行金萬斤於魏,求晉鄙客,令毀公子於魏王曰,公子亡在外十年矣,今為魏將,諸侯將皆屬,諸侯徒聞魏公子,不聞魏王。公子亦欲因此時定南面而王,諸侯畏公子之威,方欲共立之。秦數使反閒,偽賀公子得立為魏王未也。魏王日聞其毀,不能不信,後果使人代公子將。公子自知再以毀廢,乃謝病不朝,與賓客為長夜飲,飲醇酒,多近婦女。日夜為樂飲者四歲,竟病酒而卒。其歲,魏安釐王亦薨。

秦聞公子死,使蒙驁攻魏,拔二十城,初置東郡。其後秦稍蠶食魏,十八歲而虜魏王,屠大梁。

高祖始微少時,數聞公子賢。及即天子位,每過大梁,常祠公子。高祖十二年,從擊黥布還,為公子置守冢五家,世世歲以四時奉祠公子。

太史公曰,吾過大梁之墟,求問其所謂夷門。夷門者,城之東門也。天下諸公子亦有喜士者矣,然信陵君之接巖穴隱者,不恥下交,有以也。名冠諸侯,不虛耳。高祖每過之而令民奉祠不絕也。