\chapter{范雎蔡澤列傳第十九}

范雎者,魏人也,字叔。游說諸侯,欲事魏王,家貧無以自資,乃先事魏中大夫須賈。

須賈為魏昭王使於齊,范雎從。留數月,未得報。齊襄王聞睢辯口,乃使人賜睢金十斤及牛酒,睢辭謝不敢受。須賈知之,大怒,以為睢持魏國陰事告齊,故得此饋,令睢受其牛酒,還其金。既歸,心怒睢,以告魏相。魏相,魏之諸公子,曰魏齊。魏齊大怒,使舍人笞擊睢,折脅摺齒。睢詳死,即卷以簀,置廁中。賓客飲者醉,更溺睢,故僇辱以懲後,令無妄言者。睢從簀中謂守者曰,公能出我,我必厚謝公。守者乃請出棄簀中死人。魏齊醉,曰,可矣。范雎得出。後魏齊悔,復召求之。魏人鄭安平聞之,乃遂操范雎亡,伏匿,更名姓曰張祿。

當此時,秦昭王使謁者王稽於魏。鄭安平詐為卒,侍王稽。王稽問,魏有賢人可與俱西游者乎。鄭安平曰,臣里中有張祿先生,欲見君,言天下事。其人有仇,不敢晝見。王稽曰,夜與俱來。鄭安平夜與張祿見王稽。語未究,王稽知范雎賢,謂曰,先生待我於三亭之南。與私約而去。

王稽辭魏去,過載范雎入秦。至湖,望見車騎從西來。范雎曰,彼來者為誰。王稽曰,秦相穰侯東行縣邑。范雎曰,吾聞穰侯專秦權,惡內諸侯客,此恐辱我,我寧且匿車中。有頃,穰侯果至,勞王稽,因立車而語曰,關東有何變。曰,無有。又謂王稽曰,謁君得無與諸侯客子俱來乎。無益,徒亂人國耳。王稽曰,不敢。即別去。范雎曰,吾聞穰侯智士也,其見事遲,鄉者疑車中有人,忘索之。於是范雎下車走,曰,此必悔之。行十餘里,果使騎還索車中,無客,乃已。王稽遂與范雎入咸陽。

已報使,因言曰,魏有張祿先生,天下辯士也。曰秦王之國危於累卵,得臣則安。然不可以書傳也。臣故載來。秦王弗信,使舍食草具。待命歲餘。

當是時,昭王已立三十六年。南拔楚之鄢郢,楚懷王幽死於秦。秦東破齊。湣王嘗稱帝,後去之。數困三晉。厭天下辯士,無所信。

穰侯,華陽君,昭王母宣太后之弟也,而涇陽君、高陵君皆昭王同母弟也。穰侯相,三人者更將,有封邑,以太后故,私家富重於王室。及穰侯為秦將,且欲越韓、魏而伐齊綱壽,欲以廣其陶封。范雎乃上書曰,

臣聞明主立政,有功者不得不賞,有能者不得不官,勞大者其祿厚,功多者其爵尊,能治眾者其官大。故無能者不敢當職焉,有能者亦不得蔽隱。使以臣之言為可,願行而益利其道,以臣之言為不可,久留臣無為也。語曰,庸主賞所愛而罰所惡,明主則不然,賞必加於有功,而刑必斷於有罪。今臣之胸不足以當椹質,而要不足以待斧鉞,豈敢以疑事嘗試於王哉。雖以臣為賤人而輕辱,獨不重任臣者之無反復於王邪。

且臣聞周有砥砨,宋有結綠,梁有縣藜,楚有和樸,此四寶者,土之所生,良工之所失也,而為天下名器。然則聖王之所棄者,獨不足以厚國家乎。

臣聞善厚家者取之於國,善厚國者取之於諸侯。天下有明主則諸侯不得擅厚者,何也。為其割榮也。良醫知病人之死生,而聖主明於成敗之事,利則行之,害則捨之,疑則少嘗之,雖舜禹復生,弗能改已。語之至者,臣不敢載之於書,其淺者又不足聽也。意者臣愚而不概於王心邪。亡其言臣者賤而不可用乎。自非然者,臣願得少賜游觀之閒,望見顏色。一語無效,請伏斧質。

於是秦昭王大說,乃謝王稽,使以傳車召范雎。

於是范雎乃得見於離宮,詳為不知永巷而入其中。王來而宦者怒,逐之,曰,王至。范雎繆為曰,秦安得王。秦獨有太后、穰侯耳。欲以感怒昭王。昭王至,聞其與宦者爭言,遂延迎,謝曰,寡人宜以身受命久矣,會義渠之事急,寡人旦暮自請太后,今義渠之事已,寡人乃得受命。竊閔然不敏,敬執賓主之禮。范雎辭讓。是日觀范雎之見者,群臣莫不灑然變色易容者。

秦王屏左右,宮中虛無人。秦王跽而請曰,先生何以幸教寡人。范雎曰,唯唯。有閒,秦王復跽而請曰,先生何以幸教寡人。范雎曰,唯唯。若是者三。秦王跽曰,先生卒不幸教寡人邪。范雎曰,非敢然也。臣聞昔者呂尚之遇文王也,身為漁父而釣於渭濱耳。若是者,交疏也。已說而立為太師,載與俱歸者,其言深也。故文王遂收功於呂尚而卒王天下。鄉使文王疏呂尚而不與深言,是周無天子之德,而文武無與成其王業也。今臣羈旅之臣也,交疏於王,而所願陳者皆匡君之事,處人骨肉之閒,願效愚忠而未知王之心也。此所以王三問而不敢對者也。臣非有畏而不敢言也。臣知今日言之於前而明日伏誅於後,然臣不敢避也。大王信行臣之言,死不足以為臣患,亡不足以為臣憂,漆身為厲被髪為狂不足以為臣恥。且以五帝之聖焉而死,三王之仁焉而死,五伯之賢焉而死,烏獲、任鄙之力焉而死,成荊、孟賁、王慶忌、夏育之勇焉而死。死者,人之所必不免也。處必然之勢,可以少有補於秦,此臣之所大願也,臣又何患哉。伍子胥橐載而出昭關,夜行晝伏,至於陵水,無以糊其口,厀行蒲伏,稽首肉袒,鼓腹吹篪,乞食於吳市,卒興吳國,闔閭為伯。使臣得盡謀如伍子胥,加之以幽囚,終身不復見,是臣之說行也,臣又何憂。箕子、接輿漆身為厲,被髪為狂,無益於主。假使臣得同行於箕子,可以有補於所賢之主,是臣之大榮也,臣有何恥。臣之所恐者,獨恐臣死之後,天下見臣之盡忠而身死,因以是杜口裹足,莫肯鄉秦耳。足下上畏太后之嚴,下惑於姦臣之態,居深宮之中,不離阿保之手,終身迷惑,無與昭姦。大者宗廟滅覆,小者身以孤危,此臣之所恐耳。若夫窮辱之事,死亡之患,臣不敢畏也。臣死而秦治,是臣死賢於生。秦王跽曰,先生是何言也。夫秦國辟遠,寡人愚不肖,先生乃幸辱至於此,是天以寡人慁先生而存先王之宗廟也。寡人得受命於先生,是天所以幸先王,而不棄其孤也。先生柰何而言若是。事無小大,上及太后,下至大臣,願先生悉以教寡人,無疑寡人也。范雎拜,秦王亦拜。

范雎曰,大王之國,四塞以為固,北有甘泉、谷口,南帶涇、渭,右隴、蜀,左關、阪,奮擊百萬,戰車千乘,利則出攻,不利則入守,此王者之地也。民怯於私鬬而勇於公戰,此王者之民也。王并此二者而有之。夫以秦卒之勇,車騎之眾,以治諸侯,譬若施韓盧而搏蹇兔也,霸王之業可致也,而群臣莫當其位。至今閉關十五年,不敢窺兵於山東者,是穰侯為秦謀不忠,而大王之計有所失也。秦王跽曰,寡人願聞失計。

然左右多竊聽者,范雎恐,未敢言內,先言外事,以觀秦王之俯仰。因進曰,夫穰侯越韓、魏而攻齊綱壽,非計也。少出師則不足以傷齊,多出師則害於秦。臣意王之計,欲少出師而悉韓、魏之兵也,則不義矣。今見與國之不親也,越人之國而攻,可乎。其於計疏矣。且昔齊湣王南攻楚,破軍殺將,再辟地千里,而齊尺寸之地無得焉者,豈不欲得地哉,形勢不能有也。諸侯見齊之罷獘,君臣之不和也,興兵而伐齊,大破之。士辱兵頓,皆咎其王,曰,誰為此計者乎。王曰,文子為之。大臣作亂,文子出走。攻齊所以大破者,以其伐楚而肥韓、魏也。此所謂借賊兵而齎盜糧者也。王不如遠交而近攻,得寸則王之寸也,得尺亦王之尺也。今釋此而遠攻,不亦繆乎。且昔者中山之國地方五百里,趙獨吞之,功成名立而利附焉,天下莫之能害也。今夫韓、魏,中國之處而天下之樞也,王其欲霸,必親中國以為天下樞,以威楚、趙。楚彊則附趙,趙彊則附楚,楚、趙皆附,齊必懼矣。齊懼,必卑辭重幣以事秦。齊附而韓、魏因可虜也。昭王曰,吾欲親魏久矣,而魏多變之國也,寡人不能親。請問親魏柰何。對曰,王卑詞重幣以事之,不可,則割地而賂之,不可,因舉兵而伐之。王曰,寡人敬聞命矣。乃拜范雎為客卿,謀兵事。卒聽范雎謀,使五大夫綰伐魏,拔懷。後二歲,拔邢丘。

客卿范雎復說昭王曰,秦韓之地形,相錯如繡。秦之有韓也,譬如木之有蠹也,人之有心腹之病也。天下無變則已,天下有變,其為秦患者孰大於韓乎。王不如收韓。昭王曰,吾固欲收韓,韓不聽,為之柰何。對曰,韓安得無聽乎。王下兵而攻滎陽,則鞏、成皋之道不通,北斷太行之道,則上黨之師不下。王一興兵而攻滎陽,則其國斷而為三。夫韓見必亡,安得不聽乎。若韓聽,而霸事因可慮矣。王曰,善。且欲發使於韓。

范雎日益親,復說用數年矣,因請閒說曰,臣居山東時,聞齊之有田文,不聞其有王也,聞秦之有太后、穰侯、華陽、高陵、涇陽,不聞其有王也。夫擅國之謂王,能利害之謂王,制殺生之威之謂王。今太后擅行不顧,穰侯出使不報,華陽、涇陽等擊斷無諱,高陵進退不請。四貴備而國不危者,未之有也。為此四貴者下,乃所謂無王也。然則權安得不傾,令安得從王出乎。臣聞善治國者,乃內固其威而外重其權。穰侯使者操王之重,決制於諸侯,剖符於天下,政適伐國,莫敢不聽。戰勝攻取則利歸於陶,國獘御於諸侯,戰敗則結怨於百姓,而禍歸於社稷。詩曰木實繁者披其枝,披其枝者傷其心,大其都者危其國,尊其臣者卑其主。崔杼、淖齒管齊,射王股,擢王筋,縣之於廟梁,宿昔而死。李兌管趙,囚主父於沙丘,百日而餓死。今臣聞秦太后、穰侯用事,高陵、華陽、涇陽佐之,卒無秦王,此亦淖齒、李兌之類也。且夫三代所以亡國者,君專授政,縱酒馳騁弋獵,不聽政事。其所授者,妒賢嫉能,御下蔽上,以成其私,不為主計,而主不覺悟,故失其國。今自有秩以上至諸大吏,下及王左右,無非相國之人者。見王獨立於朝,臣竊為王恐,萬世之後,有秦國者非王子孫也。昭王聞之大懼,曰,善。於是廢太后,逐穰侯、高陵、華陽、涇陽君於關外。秦王乃拜范雎為相。收穰侯之印,使歸陶,因使縣官給車牛以徙,千乘有餘。到關,關閱其寶器,寶器珍怪多於王室。

秦封范雎以應,號為應侯。當是時,秦昭王四十一年也。

范雎既相秦,秦號曰張祿,而魏不知,以為范雎已死久矣。魏聞秦且東伐韓、魏,魏使須賈於秦。范雎聞之,為微行,敝衣閒步之邸,見須賈。須賈見之而驚曰,范叔固無恙乎。范雎曰,然。須賈笑曰,范叔有說於秦邪。曰,不也。睢前日得過於魏相,故亡逃至此,安敢說乎。須賈曰,今叔何事。范雎曰臣為人庸賃。須賈意哀之,留與坐飲食,曰,范叔一寒如此哉。乃取其一綈袍以賜之。須賈因問曰,秦相張君,公知之乎。吾聞幸於王,天下之事皆決於相君。今吾事之去留在張君。孺子豈有客習於相君者哉。范雎曰,主人翁習知之。唯睢亦得謁,睢請為見君於張君。須賈曰,吾馬病,車軸折,非大車駟馬,吾固不出。范雎曰,願為君借大車駟馬於主人翁。

范雎歸取大車駟馬,為須賈御之,入秦相府。府中望見,有識者皆避匿。須賈怪之。至相舍門,謂須賈曰,待我,我為君先入通於相君。須賈待門下,持車良久,問門下曰,范叔不出,何也。門下曰,無范叔。須賈曰,鄉者與我載而入者。門下曰,乃吾相張君也。須賈大驚,自知見賣,乃肉袒厀行,因門下人謝罪。於是范雎盛帷帳,待者甚眾,見之。須賈頓首言死罪,曰,賈不意君能自致於青雲之上,賈不敢復讀天下之書,不敢復與天下之事。賈有湯鑊之罪,請自屏於胡貉之地,唯君死生之。范雎曰,汝罪有幾。曰,擢賈之發以續賈之罪,尚未足。范雎曰,汝罪有三耳。昔者楚昭王時而申包胥為楚卻吳軍,楚王封之以荊五千戶,包胥辭不受,為丘墓之寄於荊也。今睢之先人丘墓亦在魏,公前以睢為有外心於齊而惡睢於魏齊,公之罪一也。當魏齊辱我於廁中,公不止,罪二也。更醉而溺我,公其何忍乎。罪三矣。然公之所以得無死者,以綈袍戀戀,有故人之意,故釋公。乃謝罷。入言之昭王,罷歸須賈。

須賈辭於范雎,范雎大供具,盡請諸侯使,與坐堂上,食飲甚設。而坐須賈於堂下,置莝豆其前,令兩黥徒夾而馬食之。數曰,為我告魏王,急持魏齊頭來。不然者,我且屠大梁。須賈歸,以告魏齊。魏齊恐,亡走趙。匿平原君所。

范雎既相,王稽謂范雎曰,事有不可知者三,有不柰何者亦三。宮車一日晏駕,是事之不可知者一也。君卒然捐館舍,是事之不可知者二也。使臣卒然填溝壑,是事之不可知者三也。宮車一日晏駕,君雖恨於臣,無可柰何。君卒然捐館舍,君雖恨於臣,亦無可柰何。使臣卒然填溝壑,君雖恨於臣,亦無可柰何。范雎不懌,乃入言於王曰,非王稽之忠,莫能內臣於函谷關,非大王之賢聖,莫能貴臣。今臣官至於相,爵在列侯,王稽之官尚止於謁者,非其內臣之意也。昭王召王稽,拜為河東守,三歲不上計。又任鄭安平,昭王以為將軍。范雎於是散家財物,盡以報所嘗困戹者。一飯之德必償,睚眥之怨必報。

范雎相秦二年,秦昭王之四十二年,東伐韓少曲、高平,拔之。

秦昭王聞魏齊在平原君所,欲為范雎必報其仇,乃詳為好書遺平原君曰,寡人聞君之高義,願與君為布衣之友,君幸過寡人,寡人願與君為十日之飲。平原君畏秦,且以為然,而入秦見昭王。昭王與平原君飲數日,昭王謂平原君曰,昔周文王得呂尚以為太公,齊桓公得管夷吾以為仲父,今范君亦寡人之叔父也。范君之仇在君之家,願使人歸取其頭來,不然,吾不出君於關。平原君曰,貴而為交者,為賤也,富而為交者,為貧也。夫魏齊者,勝之友也,在,固不出也,今又不在臣所。昭王乃遺趙王書曰,王之弟在秦,范君之仇魏齊在平原君之家。王使人疾持其頭來,不然,吾舉兵而伐趙,又不出王之弟於關。趙孝成王乃發卒圍平原君家,急,魏齊夜亡出,見趙相虞卿。虞卿度趙王終不可說,乃解其相印,與魏齊亡,閒行,念諸侯莫可以急抵者,乃復走大梁,欲因信陵君以走楚。信陵君聞之,畏秦,猶豫未肯見,曰,虞卿何如人也。時侯嬴在旁,曰,人固未易知,知人亦未易也。夫虞卿躡屩檐簦,一見趙王,賜白璧一雙,黃金百鎰,再見,拜為上卿,三見,卒受相印,封萬戶侯。當此之時,天下爭知之。夫魏齊窮困過虞卿,虞卿不敢重爵祿之尊,解相印,捐萬戶侯而閒行。急士之窮而歸公子,公子曰何如人。人固不易知,知人亦未易也。信陵君大慚,駕如野迎之。魏齊聞信陵君之初難見之,怒而自剄。趙王聞之,卒取其頭予秦。秦昭王乃出平原君歸趙。

昭王四十三年,秦攻韓汾陘,拔之,因城河上廣武。

後五年,昭王用應侯謀,縱反閒賣趙,趙以其故,令馬服子代廉頗將。秦大破趙於長平,遂圍邯鄲。已而與武安君白起有隙,言而殺之。任鄭安平,使擊趙。鄭安平為趙所圍,急,以兵二萬人降趙。應侯席槁請罪。秦之法,任人而所任不善者,各以其罪罪之。於是應侯罪當收三族。秦昭王恐傷應侯之意,乃下令國中,有敢言鄭安平事者,以其罪罪之。而加賜相國應侯食物日益厚,以順適其意。後二歲,王稽為河東守,與諸侯通,坐法誅。而應侯日益以不懌。

昭王臨朝嘆息,應侯進曰,臣聞主憂臣辱,主辱臣死。今大王中朝而憂,臣敢請其罪。昭王曰,吾聞楚之鐵劍利而倡優拙。夫鐵劍利則士勇,倡優拙則思慮遠。夫以遠思慮而御勇士,吾恐楚之圖秦也。夫物不素具,不可以應卒,今武安君既死,而鄭安平等畔,內無良將而外多敵國,吾是以憂。欲以激勵應侯。應侯懼,不知所出。蔡澤聞之,往入秦也。

蔡澤者,燕人也。游學干諸侯小大甚眾,不遇。而從唐舉相,曰,吾聞先生相李兌,曰百日之內持國秉,有之乎。曰,有之。曰,若臣者何如。唐舉孰視而笑曰,先生曷鼻,巨肩,魋顏,蹙齃,膝攣。吾聞聖人不相,殆先生乎。蔡澤知唐舉戲之,乃曰,富貴吾所自有,吾所不知者壽也,願聞之。唐舉曰,先生之壽,從今以往者四十三歲。蔡澤笑謝而去,謂其御者曰,吾持粱刺齒肥,躍馬疾驅,懷黃金之印,結紫綬於要,揖讓人主之前,食肉富貴,四十三年足矣。去之趙,見逐。之韓、魏,遇奪釜鬲於涂。聞應侯任鄭安平、王稽皆負重罪於秦,應侯內慚,蔡澤乃西入秦。

將見昭王,使人宣言以感怒應侯曰,燕客蔡澤,天下雄俊弘辯智士也。彼一見秦王,秦王必困君而奪君之位。應侯聞,曰,五帝三代之事,百家之說,吾既知之,眾口之辯,吾皆摧之,是惡能困我而奪我位乎。使人召蔡澤。蔡澤入,則揖應。應侯固不快,及見之,又倨,應侯因讓之曰,子嘗宣言欲代我相秦,寧有之乎。對曰,然。應侯曰,請聞其說。蔡澤曰,吁,君何見之晚也。夫四時之序,成功者去。夫人生百體堅彊,手足便利,耳目聰明而心聖智,豈非士之願與。應侯曰,然。蔡澤曰,質仁秉義,行道施德,得志於天下,天下懷樂敬愛而尊慕之,皆願以為君王,豈不辯智之期與。應侯曰,然。蔡澤復曰,富貴顯榮,成理萬物,使各得其所,性命壽長,終其天年而不夭傷,天下繼其統,守其業,傳之無窮,名實純粹,澤流千里,世世稱之而無絕,與天地終始,豈道德之符而聖人所謂吉祥善事者與。應侯曰,然。

蔡澤曰,若夫秦之商君,楚之吳起,越之大夫種,其卒然亦可願與。應侯知蔡澤之欲困己以說,復謬曰,何為不可。夫公孫鞅之事孝公也,極身無貳慮,盡公而不顧私,設刀鋸以禁姦邪,信賞罰以致治,披腹心,示情素,蒙怨咎,欺舊友,奪魏公子卬,安秦社稷,利百姓,卒為秦禽將破敵,攘地千里。吳起之事悼王也,使私不得害公,讒不得蔽忠,言不取茍合,行不取茍容,不為危易行,行義不辟難,然為霸主彊國,不辭禍凶。大夫種之事越王也,主雖困辱,悉忠而不解,主雖絕亡,盡能而弗離,成功而弗矜,貴富而不驕怠。若此三子者,固義之至也,忠之節也。是故君子以義死難,視死如歸,生而辱不如死而榮。士固有殺身以成名,雖義之所在,雖死無所恨。何為不可哉。

蔡澤曰,主聖臣賢,天下之盛福也,君明臣直,國之福也,父慈子孝,夫信妻貞,家之福也。故比干忠而不能存殷,子胥智而不能完吳,申生孝而晉國亂。是皆有忠臣孝子,而國家滅亂者,何也。無明君賢父以聽之,故天下以其君父為僇辱而憐其臣子。今商君、吳起、大夫種之為人臣,是也,其君,非也。故世稱三子致功而不見德,豈慕不遇世死乎。夫待死而後可以立忠成名,是微子不足仁,孔子不足聖,管仲不足大也。夫人之立功,豈不期於成全邪。身與名俱全者,上也。名可法而身死者,其次也。名在僇辱而身全者,下也。於是應侯稱善。

蔡澤少得閒,因曰,夫商君、吳起、大夫種,其為人臣盡忠致功則可願矣,閎夭事文王,周公輔成王也,豈不亦忠聖乎。以君臣論之,商君、吳起、大夫種其可願孰與閎夭、周公哉。應侯曰,商君、吳起、大夫種弗若也。蔡澤曰,然則君之主慈仁任忠,惇厚舊故,其賢智與有道之士為膠漆,義不倍功臣,孰與秦孝公、楚悼王、越王乎。應侯曰,未知何如也。蔡澤曰,今主親忠臣,不過秦孝公、楚悼王、越王,君之設智,能為主安危修政,治亂彊兵,批患折難,廣地殖穀,富國足家,彊主,尊社稷,顯宗廟,天下莫敢欺犯其主,主之威蓋震海內,功彰萬里之外,聲名光輝傳於千世,君孰與商君、吳起、大夫種。應侯曰,不若。蔡澤曰,今主之親忠臣不忘舊故不若孝公、悼王、句踐,而君之功績愛信親幸又不若商君、吳起、大夫種,然而君之祿位貴盛,私家之富過於三子,而身不退者,恐患之甚於三子,竊為君危之。語曰日中則移,月滿則虧。物盛則衰,天地之常數也。進退盈縮,與時變化,聖人之常道也。故國有道則仕,國無道則隱。聖人曰飛龍在天,利見大人。不義而富且貴,於我如浮雲。今君之怨已讎而德已報,意欲至矣,而無變計,竊為君不取也。且夫翠、鵠、犀、象,其處勢非不遠死也,而所以死者,惑於餌也。蘇秦、智伯之智,非不足以辟辱遠死也,而所以死者,惑於貪利不止也。是以聖人制禮節欲,取於民有度,使之以時,用之有止,故志不溢,行不驕,常與道俱而不失,故天下承而不絕。昔者齊桓公九合諸侯,一匡天下,至於葵丘之會,有驕矜之志,畔者九國。吳王夫差兵無敵於天下,勇彊以輕諸侯,陵齊晉,故遂以殺身亡國。夏育、太史噭叱呼駭三軍,然而身死於庸夫。此皆乘至盛而不返道理,不居卑退處儉約之患也。夫商君為秦孝公明法令,禁姦本,尊爵必賞,有罪必罰,平權衡,正度量,調輕重,決裂阡陌,以靜生民之業而一其俗,勸民耕農利土,一室無二事,力田稸積,習戰陳之事,是以兵動而地廣,兵休而國富,故秦無敵於天下,立威諸侯,成秦國之業。功已成矣,而遂以車裂。楚地方數千里,持戟百萬,白起率數萬之師以與楚戰,一戰舉鄢郢以燒夷陵,再戰南并蜀漢。又越韓、魏而攻彊趙,北阬馬服,誅屠四十餘萬之眾,盡之于長平之下,流血成川,沸聲若雷,遂入圍邯鄲,使秦有帝業。楚、趙天下之彊國而秦之仇敵也,自是之後,楚、趙皆懾伏不敢攻秦者,白起之勢也。身所服者七十餘城,功已成矣,而遂賜劍死於杜郵。吳起為楚悼王立法,卑減大臣之威重,罷無能,廢無用,損不急之官,塞私門之請,一楚國之俗,禁游客之民,精耕戰之士,南收楊越,北并陳、蔡,破橫散從,使馳說之士無所開其口,禁朋黨以勵百姓,定楚國之政,兵震天下,威服諸侯。功已成矣,而卒枝解。大夫種為越王深謀遠計,免會稽之危,以亡為存,因辱為榮,墾草入邑,辟地殖穀,率四方之士,專上下之力,輔句踐之賢,報夫差之讎,卒擒勁吳。令越成霸。功已彰而信矣,句踐終負而殺之。此四子者,功成不去,禍至於此。此所謂信而不能詘,往而不能返者也。范蠡知之,超然辟世,長為陶朱公。君獨不觀夫博者乎。或欲大投,或欲分功,此皆君之所明知也。今君相秦,計不下席,謀不出廊廟,坐制諸侯,利施三川,以實宜陽,決羊腸之險,塞太行之道,又斬范、中行之涂,六國不得合從,棧道千里,通於蜀漢,使天下皆畏秦,秦之欲得矣,君之功極矣,此亦秦之分功之時也。如是而不退,則商君、白公、吳起、大夫種是也。吾聞之,鑒於水者見面之容,鑒於人者知吉與凶。書曰成功之下,不可久處。四子之禍,君何居焉。君何不以此時歸相印,讓賢者而授之,退而巖居川觀,必有伯夷之廉,長為應侯。世世稱孤,而有許由、延陵季子之讓,喬松之壽,孰與以禍終哉。即君何居焉。忍不能自離,疑不能自決,必有四子之禍矣。易曰亢龍有悔,此言上而不能下,信而不能詘,往而不能自返者也。願君孰計之。應侯曰,善。吾聞欲而不知足,失其所以欲,有而不知止,失其所以有。先生幸教,睢敬受命。於是乃延入坐,為上客。

後數日,入朝,言於秦昭王曰,客新有從山東來者曰蔡澤,其人辯士,明於三王之事,五伯之業,世俗之變,足以寄秦國之政。臣之見人甚眾,莫及,臣不如也。臣敢以聞。秦昭王召見,與語,大說之,拜為客卿。應侯因謝病請歸相印。昭王彊起應侯,應侯遂稱病甐。范雎免相,昭王新說蔡澤計畫,遂拜為秦相,東收周室。

蔡澤相秦數月,人或惡之,懼誅,乃謝病歸相印,號為綱成君。居秦十餘年,事昭王、孝文王、莊襄王。卒事始皇帝,為秦使於燕,三年而燕使太子丹入質於秦。

太史公曰,韓子稱長袖善舞,多錢善賈,信哉是言也。范雎、蔡澤世所謂一切辯士,然游說諸侯至白首無所遇者,非計策之拙,所為說力少也。及二人羈旅入秦,繼踵取卿相,垂功於天下者,固彊弱之勢異也。然士亦有偶合,賢者多如此二子,不得盡意,豈可勝道哉。然二子不困緦惡能激乎。