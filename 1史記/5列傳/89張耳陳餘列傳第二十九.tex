\chapter{張耳陳餘列傳第二十九}

張耳者,大梁人也。其少時,及魏公子毋忌為客。張耳嘗亡命游外黃。外黃富人女甚美,嫁庸奴,亡其夫,去抵父客。父客素知張耳,乃謂女曰,必欲求賢夫,從張耳。女聽,乃卒為請決,嫁之張耳。張耳是時脫身游,女家厚奉給張耳,張耳以故致千里客。乃宦魏為外黃令。名由此益賢。陳餘者,亦大梁人也,好儒術,數游趙苦陘。富人公乘氏以其女妻之,亦知陳餘非庸人也。餘年少,父事張耳,兩人相與為刎頸交。

秦之滅大梁也,張耳家外黃。高祖為布衣時,嘗數從張耳游,客數月。秦滅魏數歲,已聞此兩人魏之名士也,購求有得張耳千金,陳餘五百金。張耳、陳餘乃變名姓,俱之陳,為里監門以自食。兩人相對。里吏嘗有過笞陳餘,陳餘欲起,張耳躡之,使受笞。吏去,張耳乃引陳餘之桑下而數之曰,始吾與公言何如。今見小辱而欲死一吏乎。陳餘然之。秦詔書購求兩人,兩人亦反用門者以令里中。

陳涉起蘄,至入陳,兵數萬。張耳、陳餘上謁陳涉。涉及左右生平數聞張耳、陳餘賢,未嘗見,見即大喜。

陳中豪傑父老乃說陳涉曰,將軍身被堅執銳,率士卒以誅暴秦,復立楚社稷,存亡繼絕,功德宜為王。且夫監臨天下諸將,不為王不可,願將軍立為楚王也。陳涉問此兩人,兩人對曰,夫秦為無道,破人國家,滅人社稷,絕人後世,罷百姓之力,盡百姓之財。將軍瞋目張膽,出萬死不顧一生之計,為天下除殘也。今始至陳而王之,示天下私。願將軍毋王,急引兵而西,遣人立六國後,自為樹黨,為秦益敵也。敵多則力分,與眾則兵彊。如此野無交兵,縣無守城,誅暴秦,據咸陽以令諸侯。諸侯亡而得立,以德服之,如此則帝業成矣。今獨王陳,恐天下解也。陳涉不聽,遂立為王。

陳餘乃復說陳王曰,大王舉梁、楚而西,務在入關,未及收河北也。臣嘗游趙,知其豪桀及地形,願請奇兵北略趙地。於是陳王以故所善陳人武臣為將軍,邵騷為護軍,以張耳、陳餘為左右校尉,予卒三千人,北略趙地。

武臣等從白馬渡河,至諸縣,說其豪桀曰,秦為亂政虐刑以殘賊天下,數十年矣。北有長城之役,南有五嶺之戍,外內騷動,百姓罷敝,頭會箕斂,以供軍費,財匱力盡,民不聊生。重之以苛法峻刑,使天下父子不相安。陳王奮臂為天下倡始,王楚之地,方二千里,莫不響應,家自為怒,人自為鬬,各報其怨而攻其讎,縣殺其令丞,郡殺其守尉。今已張大楚,王陳,使吳廣、周文將卒百萬西擊秦。於此時而不成封侯之業者,非人豪也。諸君試相與計之。夫天下同心而苦秦久矣。因天下之力而攻無道之君,報父兄之怨而成割地有土之業,此士之一時也。豪桀皆然其言。乃行收兵,得數萬人,號武臣為武信君。下趙十城,餘皆城守,莫肯下。

乃引兵東北擊范陽。范陽人蒯通說范陽令曰,竊聞公之將死,故弔。雖然,賀公得通而生。范陽令曰,何以弔之。對曰,秦法重,足下為范陽令十年矣,殺人之父,孤人之子,斷人之足,黥人之首,不可勝數。然而慈父孝子莫敢倳刃公之腹中者,畏秦法耳。今天下大亂,秦法不施,然則慈父孝子且倳刃公之腹中以成其名,此臣之所以弔公也。今諸侯畔秦矣,武信君兵且至,而君堅守范陽,少年皆爭殺君,下武信君。君急遣臣見武信君,可轉禍為福,在今矣。

范陽令乃使蒯通見武信君曰,足下必將戰勝然後略地,攻得然後下城,臣竊以為過矣。誠聽臣之計,可不攻而降城,不戰而略地,傳檄而千里定,可乎。武信君曰,何謂也。蒯通曰,今范陽令宜整頓其士卒以守戰者也,怯而畏死,貪而重富貴,故欲先天下降,畏君以為秦所置吏,誅殺如前十城也。然今范陽少年亦方殺其令,自以城距君。君何不齎臣侯印,拜范陽令,范陽令則以城下君,少年亦不敢殺其令。令范陽令乘朱輪華轂,使驅馳燕、趙郊。燕、趙郊見之,皆曰此范陽令,先下者也,即喜矣,燕、趙城可毋戰而降也。此臣之所謂傳檄而千里定者也。武信君從其計,因使蒯通賜范陽令侯印。趙地聞之,不戰以城下者三十餘城。

至邯鄲,張耳、陳餘聞周章軍入關,至戲卻,又聞諸將為陳王徇地,多以讒毀得罪誅,怨陳王不其筴不以為將而以為校尉。乃說武臣曰,陳王起蘄,至陳而王,非必立六國後。將軍今以三千人下趙數十城,獨介居河北,不王無以填之。且陳王聽讒,還報,恐不脫於禍。又不如立其兄弟,不,即立趙後。將軍毋失時,時閒不容息。武臣乃聽之,遂立為趙王。以陳餘為大將軍,張耳為右丞相,邵騷為左丞相。

使人報陳王,陳王大怒,欲盡族武臣等家,而發兵擊趙。陳王相國房君諫曰,秦未亡而誅武臣等家,此又生一秦也。不如因而賀之,使急引兵西擊秦。陳王然之,從其計,徙系武臣等家宮中,封張耳子敖為成都君。

陳王使使者賀趙,令趣發兵西入關。張耳、陳餘說武臣曰,王王趙,非楚意,特以計賀王。楚已滅秦,必加兵於趙。願王毋西兵,北徇燕、代,南收河內以自廣。趙南據大河,北有燕、代,楚雖勝秦,必不敢制趙。趙王以為然,因不西兵,而使韓廣略燕,李良略常山,張黡略上黨。

韓廣至燕,燕人因立廣為燕王。趙王乃與張耳、陳餘北略地燕界。趙王閒出,為燕軍所得。燕將囚之,欲與分趙地半,乃歸王。使者往,燕輒殺之以求地。張耳、陳餘患之。有廝養卒謝其舍中曰,吾為公說燕,與趙王載歸。舍中皆笑曰,使者往十餘輩,輒死,若何以能得王。乃走燕壁。燕將見之,問燕將曰,知臣何欲。燕將曰,若欲得趙王耳。曰,君知張耳、陳餘何如人也。燕將曰,賢人也。曰,知其志何欲。曰,欲得其王耳。趙養卒乃笑曰,君未知此兩人所欲也。夫武臣、張耳、陳餘杖馬箠下趙數十城,此亦各欲南面而王,豈欲為卿相終己邪。夫臣與主豈可同日而道哉,顧其勢初定,未敢參分而王,且以少長先立武臣為王,以持趙心。今趙地已服,此兩人亦欲分趙而王,時未可耳。今君乃囚趙王。此兩人名為求趙王,實欲燕殺之,此兩人分趙自立。夫以一趙尚易燕,況以兩賢王左提右挈,而責殺王之罪,滅燕易矣。燕將以為然,乃歸趙王,養卒為御而歸。

李良已定常山,還報,趙王復使良略太原。至石邑,秦兵塞井陘,未能前。秦將詐稱二世使人遺李良書,不封,曰,良嘗事我得顯幸。良誠能反趙為秦,赦良罪,貴良。良得書,疑不信。乃還之邯鄲,益請兵。未至,道逢趙王姊出飲,從百餘騎。李良望見,以為王,伏謁道旁。王姊醉,不知其將,使騎謝李良。李良素貴,起,慚其從官。從官有一人曰,天下畔秦,能者先立。且趙王素出將軍下,今女兒乃不為將軍下車,請追殺之。李良已得秦書,固欲反趙,未決,因此怒,遣人追殺王姊道中,乃遂將其兵襲邯鄲。邯鄲不知,竟殺武臣、邵騷。趙人多為張耳、陳餘耳目者,以故得脫出。收其兵,得數萬人。客有說張耳曰,兩君羈旅,而欲附趙,難,獨立趙後,扶以義,可就功。乃求得趙歇,立為趙王,居信都。李良進兵擊陳餘,陳餘敗李良,李良走歸章邯。

章邯引兵至邯鄲,皆徙其民河內,夷其城郭。張耳與趙王歇走入鉅鹿城,王離圍之。陳餘北收常山兵,得數萬人,軍鉅鹿北。章邯軍鉅鹿南棘原,筑甬道屬河,餉王離。王離兵食多,急攻鉅鹿。鉅鹿城中食盡兵少,張耳數使人召前陳餘,陳餘自度兵少,不敵秦,不敢前。數月,張耳大怒,怨陳餘,使張黶、陳澤往讓陳餘曰,始吾與公為刎頸交,今王與耳旦暮且死,而公擁兵數萬,不肯相救,安在其相為死。茍必信,胡不赴秦軍俱死。且有十一二相全。陳餘曰,吾度前終不能救趙,徒盡亡軍。且餘所以不俱死,欲為趙王、張君報秦。今必俱死,如以肉委餓虎,何益。張黶、陳澤曰,事已急,要以俱死立信,安知後慮。陳餘曰,吾死顧以為無益。必如公言。乃使五千人令張黶、陳澤先嘗秦軍,至皆沒。

當是時,燕、齊、楚聞趙急,皆來救。張敖亦北收代兵,得萬餘人,來,皆壁餘旁,未敢擊秦。項羽兵數絕章邯甬道,王離軍乏食,項羽悉引兵渡河,遂破章邯。章邯引兵解,諸侯軍乃敢擊圍鉅鹿秦軍,遂虜王離。涉閒自殺。卒存鉅鹿者,楚力也。

於是趙王歇、張耳乃得出鉅鹿,謝諸侯。張耳與陳餘相見,責讓陳餘以不肯救趙,及問張黶、陳澤所在。陳餘怒曰,張黶、陳澤以必死責臣,臣使將五千人先嘗秦軍,皆沒不出。張耳不信,以為殺之,數問陳餘。陳餘怒曰,不意君之望臣深也。豈以臣為重去將哉。乃脫解印綬,推予張耳。張耳亦愕不受。陳餘起如廁。客有說張耳曰,臣聞天與不取,反受其咎。今陳將軍與君印,君不受,反天不祥。急取之。張耳乃佩其印,收其麾下。而陳餘還,亦望張耳不讓,遂趨出。張耳遂收其兵。陳餘獨與麾下所善數百人之河上澤中漁獵。由此陳餘、張耳遂有卻。

趙王歇復居信都。張耳從項羽諸侯入關。漢元年二月,項羽立諸侯王,張耳雅游,人多為之言,項羽亦素數聞張耳賢,乃分趙立張耳為常山王,治信都。信都更名襄國。

陳餘客多說項羽曰,陳餘、張耳一體有功於趙。項羽以陳餘不從入關,聞其在南皮,即以南皮旁三縣以封之,而徙趙王歇王代。

張耳之國,陳餘愈益怒,曰,張耳與餘功等也,今張耳王,餘獨侯,此項羽不平。及齊王田榮畔楚,陳餘乃使夏說說田榮曰,項羽為天下宰不平,盡王諸將善地,徙故王王惡地,今趙王乃居代。願王假臣兵,請以南皮為捍蔽。田榮欲樹黨於趙以反楚,乃遣兵從陳餘。陳餘因悉三縣兵襲常山王張耳。張耳敗走,念諸侯無可歸者,曰,漢王與我有舊故,而項羽又彊,立我,我欲之楚。甘公曰,漢王之入關,五星聚東井。東井者,秦分也。先至必霸。楚雖彊,後必屬漢。故耳走漢。漢王亦還定三秦,方圍章邯廢丘。張耳謁漢王,漢王厚遇之。

陳餘已敗張耳,皆復收趙地,迎趙王於代,復為趙王。趙王德陳餘,立以為代王。陳餘為趙王弱,國初定,不之國,留傅趙王,而使夏說以相國守代。

漢二年,東擊楚,使使告趙,欲與俱。陳餘曰,漢殺張耳乃從。於是漢王求人類張耳者斬之,持其頭遺陳餘。陳餘乃遣兵助漢。漢之敗於彭城西,陳餘亦復覺張耳不死,即背漢。

漢三年,韓信已定魏地,遣張耳與韓信擊破趙井陘,斬陳餘泜水上,追殺趙王歇襄國。漢立張耳為趙王。漢五年,張耳薨,謚為景王。子敖嗣立為趙王。高祖長女魯元公主為趙王敖后。

漢七年,高祖從平城過趙,趙王朝夕袒韛蔽,自上食,禮甚卑,有子婿禮。高祖箕踞詈,甚慢易之。趙相貫高、趙午等年六十餘,故張耳客也。生平為氣,乃怒曰,吾王孱王也。說王曰,夫天下豪桀并起,能者先立。今王事高祖甚恭,而高祖無禮,請為王殺之。張敖齧其指出血,曰,君何言之誤。且先人亡國,賴高祖得復國,德流子孫,秋豪皆高祖力也。願君無復出口。貫高、趙午等十餘人皆相謂曰,乃吾等非也。吾王長者,不倍德。且吾等義不辱,今怨高祖辱我王,故欲殺之,何乃汙王為乎。令事成歸王,事敗獨身坐耳。

漢八年,上從東垣還,過趙,貫高等乃壁人柏人,要之置廁。上過欲宿,心動,問曰,縣名為何。曰,柏人。柏人者,迫於人也。不宿而去。

漢九年,貫高怨家知其謀,乃上變告之。於是上皆并逮捕趙王、貫高等。十餘人皆爭自剄,貫高獨怒罵曰,誰令公為之。今王實無謀,而并捕王,公等皆死,誰白王不反者。乃轞車膠致,與王詣長安。治張敖之罪。上乃詔趙群臣賓客有敢從王皆族。貫高與客孟舒等十餘人,皆自髡鉗,為王家奴,從來。貫高至,對獄,曰,獨吾屬為之,王實不知。吏治榜笞數千,刺剟,身無可擊者,終不復言。呂后數言張王以魯元公主故,不宜有此。上怒曰,使張敖據天下,豈少而女乎。不聽。廷尉以貫高事辭聞,上曰,壯士。誰知者,以私問之。中大夫泄公曰,臣之邑子,素知之。此固趙國立名義不侵為然諾者也。上使泄公持節問之箯輿前。仰視曰,泄公邪。泄公勞苦如生平驩,與語,問張王果有計謀不。高曰,人情寧不各愛其父母妻子乎。今吾三族皆以論死,豈以王易吾親哉。顧為王實不反,獨吾等為之。具道本指所以為者王不知狀。於是泄公入,具以報,上乃赦趙王。

上賢貫高為人能立然諾,使泄公具告之,曰,張王已出。因赦貫高。貫高喜曰,吾王審出乎。泄公曰,然。泄公曰,上多足下,故赦足下。貫高曰,所以不死一身無餘者,白張王不反也。今王已出,吾責已塞,死不恨矣。且人臣有篡殺之名,何面目復事上哉。縱上不殺我,我不愧於心乎。乃仰絕骯,遂死。當此之時,名聞天下。

張敖已出,以尚魯元公主故,封為宣平侯。於是上賢張王諸客,以鉗奴從張王入關,無不為諸侯相、郡守者。及孝惠、高后、文帝、孝景時,張王客子孫皆得為二千石。

張敖,高后六年薨。子偃為魯元王。以母呂后女故,呂后封為魯元王。元王弱,兄弟少,乃封張敖他姬子二人,壽為樂昌侯,侈為信都侯。高后崩,諸呂無道,大臣誅之,而廢魯元王及樂昌侯、信諸侯。孝文帝即位,復封故魯元王偃為南宮侯,續張氏。

太史公曰,張耳、陳餘,世傳所稱賢者,其賓客廝役,莫非天下俊桀,所居國無不取卿相者。然張耳、陳餘始居約時,相然信以死,豈顧問哉。及據國爭權,卒相滅亡,何鄉者相慕用之誠,後相倍之戾也。豈非以勢利交哉。名譽雖高,賓客雖盛,所由殆與大伯、延陵季子異矣。