\chapter{屈原賈生列傳第二十四}

屈原者,名平,楚之同姓也。為楚懷王左徒。博聞彊志,明於治亂,嫻于辭令。入則與王圖議國事,以出號令,出則接遇賓客,應對諸侯。王甚任之。

上官大夫與之同列,爭寵而心害其能。懷王使屈原造為憲令,屈平屬草槁未定。上官大夫見而欲奪之,屈平不與,因讒之曰,王使屈平為令,眾莫不知,每一令出,平伐其功,以為非我莫能為也。王怒而疏屈平。

屈平疾王聽之不聰也,讒諂之蔽明也,邪曲之害公也,方正之不容也,故憂愁幽思而作離騷。離騷者,猶離憂也。夫天者,人之始也,父母者,人之本也。人窮則反本,故勞苦倦極,未嘗不呼天也,疾痛慘怛,未嘗不呼父母也。屈平正道直行,竭忠盡智以事其君,讒人閒之,可謂窮矣。信而見疑,忠而被謗,能無怨乎。屈平之作離騷,蓋自怨生也。國風好色而不淫,小雅怨誹而不亂。若離騷者,可謂兼之矣。上稱帝嚳,下道齊桓,中述湯武,以刺世事。明道德之廣崇,治亂之條貫,靡不畢見。其文約,其辭微,其志絜,其行廉,其稱文小而其指極大,舉類邇而見義遠。其志絜,故其稱物芳。其行廉,故死而不容自疏。濯淖汙泥之中,蟬蛻於濁穢,以浮游塵埃之外,不獲世之滋垢,皭然泥而不滓者也。推此志也,雖與日月爭光可也。

屈平既絀,其後秦欲伐齊,齊與楚從親,惠王患之,乃令張儀詳去秦,厚幣委質事楚,曰,秦甚憎齊,齊與楚從親,楚誠能絕齊,秦願獻商、於之地六百里。楚懷王貪而信張儀,遂絕齊,使使如秦受地。張儀詐之曰,儀與王約六里,不聞六百里。楚使怒去,歸告懷王。懷王怒,大興師伐秦。秦發兵擊之,大破楚師於丹、淅,斬首八萬,虜楚將屈丐,遂取楚之漢中地。懷王乃悉發國中兵以深入擊秦,戰於藍田。魏聞之,襲楚至鄧。楚兵懼,自秦歸。而齊竟怒不救楚,楚大困。

明年,秦割漢中地與楚以和。楚王曰,不願得地,願得張儀而甘心焉。張儀聞,乃曰,以一儀而當漢中地,臣請往如楚。如楚,又因厚幣用事者臣靳尚,而設詭辯於懷王之寵姬鄭袖。懷王竟聽鄭袖,復釋去張儀。是時屈平既疏,不復在位,使於齊,顧反,諫懷王曰,何不殺張儀。懷王悔,追張儀不及。

其後諸侯共擊楚,大破之,殺其將唐眛。

時秦昭王與楚婚,欲與懷王會。懷王欲行,屈平曰,秦虎狼之國,不可信,不如毋行。懷王稚子子蘭勸王行,柰何絕秦歡。懷王卒行。入武關,秦伏兵絕其後,因留懷王,以求割地。懷王怒,不聽。亡走趙,趙不內。復之秦,竟死於秦而歸葬。

長子頃襄王立,以其弟子蘭為令尹。楚人既咎子蘭以勸懷王入秦而不反也。

屈平既嫉之,雖放流,睠顧楚國,系心懷王,不忘欲反,冀幸君之一悟,俗之一改也。其存君興國而欲反覆之,一篇之中三致志焉。然終無可柰何,故不可以反,卒以此見懷王之終不悟也。人君無愚智賢不肖,莫不欲求忠以自為,舉賢以自佐,然亡國破家相隨屬,而聖君治國累世而不見者,其所謂忠者不忠,而所謂賢者不賢也。懷王以不知忠臣之分,故內惑於鄭袖,外欺於張儀,疏屈平而信上官大夫、令尹子蘭。兵挫地削,亡其六郡,身客死於秦,為天下笑。此不知人之禍也。易曰,井泄不食,為我心惻,可以汲。王明,并受其福。王之不明,豈足福哉。

令尹子蘭聞之大怒,卒使上官大夫短屈原於頃襄王,頃襄王怒而遷之。

屈原至於江濱,被髪行吟澤畔。顏色憔悴,形容枯槁。漁父見而問之曰,子非三閭大夫歟。何故而至此。屈原曰,舉世混濁而我獨清,眾人皆醉而我獨醒,是以見放。漁父曰,夫聖人者,不凝滯於物而能與世推移。舉世混濁,何不隨其流而揚其波。眾人皆醉,何不餔其糟而啜其醨。何故懷瑾握瑜而自令見放為。屈原曰,吾聞之,新沐者必彈冠,新浴者必振衣,人又誰能以身之察察,受物之汶汶者乎。寧赴常流而葬乎江魚腹中耳,又安能以皓皓之白而蒙世俗之溫蠖乎。

乃作懷沙之賦。其辭曰,

陶陶孟夏兮,草木莽莽。傷懷永哀兮,汩徂南土。眴兮窈窈,孔靜幽墨。冤結紆軫兮,離愍之長鞠,撫情效志兮,俛詘以自抑。

刓方以為圜兮,常度未替,易初本由兮,君子所鄙。章畫職墨兮,前度未改,內直質重兮,大人所盛。巧匠不斲兮,孰察其揆正。玄文幽處兮,矇謂之不章,離婁微睇兮,瞽以為無明。變白而為黑兮,倒上以為下。鳳皇在笯兮,雞雉翔舞。同糅玉石兮,一而相量。夫黨人之鄙妒兮,羌不知吾所臧。

任重載盛兮,陷滯而不濟,懷瑾握瑜兮,窮不得余所示。邑犬群吠兮,吠所怪也,誹駿疑桀兮,固庸態也。文質疏內兮,眾不知吾之異采,材樸委積兮,莫知余之所有。重仁襲義兮,謹厚以為豐,重華不可牾兮,孰知余之從容。迸固有不并兮,豈知其故也。湯禹久遠兮,邈不可慕也。懲違改忿兮,抑心而自彊,離湣而不遷兮,願志之有象。進路北次兮,日昧昧其將暮,含憂虞哀兮,限之以大故。

亂曰,浩浩沅、湘兮,分流汨兮。修路幽拂兮,道遠忽兮。曾唫恒悲兮,永嘆慨兮。世既莫吾知兮,人心不可謂兮。懷情抱質兮,獨無匹兮。伯樂既歿兮,驥將焉程兮。人生稟命兮,各有所錯兮。定心廣志,餘何畏懼兮。曾傷爰哀,永嘆喟兮。世溷不吾知,心不可謂兮。知死不可讓兮,願勿愛兮。明以告君子兮,吾將以為類兮。

於是懷石遂自沈汨羅以死。

屈原既死之後,楚有宋玉、唐勒、景差之徒者,皆好辭而以賦見稱,然皆祖屈原之從容辭令,終莫敢直諫。其後楚日以削,數十年竟為秦所滅。

自屈原沈汨羅後百有餘年,漢有賈生,為長沙王太傅,過湘水,投書以弔屈原。

賈生名誼,雒陽人也。年十八,以能誦詩屬書聞於郡中。吳廷尉為河南守,聞其秀才,召置門下,甚幸愛。孝文皇帝初立,聞河南守吳公治平為天下第一,故與李斯同邑而常學事焉,乃徵為廷尉。廷尉乃言賈生年少,頗通諸子百家之書。文帝召以為博士。

是時賈生年二十餘,最為少。每詔令議下,諸老先生不能言,賈生盡為之對,人人各如其意所欲出。諸生於是乃以為能,不及也。孝文帝說之,超遷,一歲中至太中大夫。

賈生以為漢興至孝文二十餘年,天下和洽,而固當改正朔,易服色,法制度,定官名,興禮樂,乃悉草具其事儀法,色尚黃,數用五,為官名,悉更秦之法。孝文帝初即位,謙讓未遑也。諸律令所更定,及列侯悉就國,其說皆自賈生發之。於是天子議以為賈生任公卿之位。絳、灌、東陽侯、馮敬之屬盡害之,乃短賈生曰,雒陽之人,年少初學,專欲擅權,紛亂諸事。於是天子後亦疏之,不用其議,乃以賈生為長沙王太傅。

賈生既辭往行,聞長沙卑溼,自以壽不得長,又以適去,意不自得。及渡湘水,為賦以弔屈原。其辭曰,

共承嘉惠兮,俟罪長沙。側聞屈原兮,自沈汨羅。造託湘流兮,敬弔先生。遭世罔極兮,乃隕厥身。嗚呼哀哉,逢時不祥。鸞鳳伏竄兮,鴟梟翺翔,闒茸尊顯兮,讒諛得志,賢聖逆曳兮,方正倒植。世謂伯夷貪兮,謂盜跖廉,莫邪為頓兮,鉛刀為铦。于嗟嚜嚜兮,生之無故。斡棄周鼎兮寶康瓠,騰駕罷牛兮驂蹇驢,驥垂兩耳兮服鹽車。章甫薦屨兮,漸不可久,嗟苦先生兮,獨離此咎。

訊曰,已矣,國其莫我知,獨堙郁兮其誰語。鳳漂漂其高遰兮,夫固自縮而遠去。襲九淵之神龍兮,沕深潛以自珍。彌融爚以隱處兮,夫豈從螘與蛭螾。所貴聖人之神德兮,遠濁世而自藏。使騏驥可得系羈兮,豈云異夫犬羊。般紛紛其離此尤兮,亦夫子之辜也。瞝九州而相君兮,何必懷此都也。鳳皇翔于千仞之上兮,覽德惪而下之,見細德之險徵兮,搖增翮逝而去之。彼尋常之汙瀆兮,豈能容吞舟之魚。橫江湖之鱣鱏兮,固將制於蟻螻。

賈生為長沙王太傅三年,有鸮飛入賈生舍,止于坐隅。楚人命鸮曰服。賈生既以適居長沙,長沙卑溼,自以為壽不得長,傷悼之,乃為賦以自廣。其辭曰,

單閼之歲兮,四月孟夏,庚子日施兮,服集予舍,止于坐隅,貌甚閒暇。異物來集兮,私怪其故,發書占之兮,筴言其度。曰野鳥入處兮,主人將去。請問于服兮,予去何之。吉乎告我,凶言其菑。淹數之度兮,語予其期。服乃嘆息,舉首奮翼,口不能言,請對以意。

萬物變化兮,固無休息。斡流而遷兮,或推而還。形氣轉續兮,變化而嬗。沕穆無窮兮,胡可勝言。禍兮福所倚,福兮禍所伏,憂喜聚門兮,吉凶同域。彼吳彊大兮,夫差以敗,越棲會稽兮,句踐霸世。斯游遂成兮,卒被五刑,傅說胥靡兮,乃相武丁。夫禍之與福兮,何異糾纆。命不可說兮,孰知其極。水激則旱兮,矢激則遠。萬物回薄兮,振蕩相轉。雲蒸雨降兮,錯繆相紛。大專槃物兮,坱軋無垠。天不可與慮兮,道不可與謀。遲數有命兮,惡識其時。

且夫天地為鑪兮,造化為工,陰陽為炭兮,萬物為銅。合散消息兮,安有常則,千變萬化兮,未始有極。忽然為人兮,何足控摶,化為異物兮,又何足患。小知自私兮,賤彼貴我,通人大觀兮,物無不可。貪夫徇財兮,烈士徇名,夸者死權兮,品庶馮生。述迫之徒兮,或趨西東,大人不曲兮,億變齊同。拘士系俗兮,攌如囚拘,至人遺物兮,獨與道俱。眾人或或兮,好惡積意,真人淡漠兮,獨與道息。釋知遺形兮,超然自喪,寥廓忽荒兮,與道翺翔。乘流則逝兮,得坻則止,縱軀委命兮,不私與己。其生若浮兮,其死若休,澹乎若深淵之靜,氾乎若不系之舟。不以生故自寶兮,養空而浮,德人無累兮,知命不憂。細故遰葪兮,何足以疑。

後歲餘,賈生徵見。孝文帝方受釐,坐宣室。上因感鬼神事,而問鬼神之本。賈生因具道所以然之狀。至夜半,文帝前席。既罷,曰,吾久不見賈生,自以為過之,今不及也。居頃之,拜賈生為梁懷王太傅。梁懷王,文帝之少子,愛,而好書,故令賈生傅之。

文帝復封淮南厲王子四人皆為列侯。賈生諫,以為患之興自此起矣。賈生數上疏,言諸侯或連數郡,非古之制,可稍削之。文帝不聽。

居數年,懷王騎,墮馬而死,無後。賈生自傷為傅無狀,哭泣歲餘,亦死。賈生之死時年三十三矣。及孝文崩,孝武皇帝立,舉賈生之孫二人至郡守,而賈嘉最好學,世其家,與余通書。至孝昭時,列為九卿。

太史公曰,余讀離騷、天問、招魂、哀郢,悲其志。適長沙,觀屈原所自沈淵,未嘗不垂涕,想見其為人。及見賈生弔之,又怪屈原以彼其材,游諸侯,何國不容,而自令若是。讀服烏賦,同死生,輕去就,又爽然自失矣。