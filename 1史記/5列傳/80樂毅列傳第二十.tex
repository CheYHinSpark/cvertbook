\chapter{樂毅列傳第二十}

樂毅者,其先祖曰樂羊。樂羊為魏文侯將,伐取中山,魏文侯封樂羊以靈壽。樂羊死,葬於靈壽,其後子孫因家焉。中山復國,至趙武靈王時復滅中山,而樂氏後有樂毅。

樂毅賢,好兵,趙人舉之。及武靈王有沙丘之亂,乃去趙適魏。聞燕昭王以子之之亂而齊大敗燕,燕昭王怨齊,未嘗一日而忘報齊也。燕國小,辟遠,力不能制,於是屈身下士,先禮郭隗以招賢者。樂毅於是為魏昭王使於燕,燕王以客禮待之。樂毅辭讓,遂委質為臣,燕昭王以為亞卿,久之。

當是時,齊湣王彊,南敗楚相唐眛於重丘,西摧三晉於觀津,遂與三晉擊秦,助趙滅中山,破宋,廣地千餘里。與秦昭王爭重為帝,已而復歸之。諸侯皆欲背秦而服於齊。湣王自矜,百姓弗堪。於是燕昭王問伐齊之事。樂毅對曰,齊,霸國之餘業也,地大人眾,未易獨攻也。王必欲伐之,莫如與趙及楚、魏。於是使樂毅約趙惠文王,別使連楚、魏,令趙嚪說秦以伐齊之利。諸侯害齊湣王之驕暴,皆爭合從與燕伐齊。樂毅還報,燕昭王悉起兵,使樂毅為上將軍,趙惠文王以相國印授樂毅。樂毅於是并護趙、楚、韓、魏、燕之兵以伐齊,破之濟西。諸侯兵罷歸,而燕軍樂毅獨追,至于臨菑。齊湣王之敗濟西,亡走,保於莒。樂毅獨留徇齊,齊皆城守。樂毅攻入臨菑,盡取齊寶財物祭器輸之燕。燕昭王大說,親至濟上勞軍,行賞饗士,封樂毅於昌國,號為昌國君。於是燕昭王收齊鹵獲以歸,而使樂毅復以兵平齊城之不下者。

樂毅留徇齊五歲,下齊七十餘城,皆為郡縣以屬燕,唯獨莒、即墨未服。會燕昭王死,子立為燕惠王。惠王自為太子時嘗不快於樂毅,及即位,齊之田單聞之,乃縱反閒於燕,曰,齊城不下者兩城耳。然所以不早拔者,聞樂毅與燕新王有隙,欲連兵且留齊,南面而王齊。齊之所患,唯恐他將之來。於是燕惠王固已疑樂毅,得齊反閒,乃使騎劫代將,而召樂毅。樂毅知燕惠王之不善代之,畏誅,遂西降趙。趙封樂毅於觀津,號曰望諸君。尊寵樂毅以警動於燕、齊。

齊田單後與騎劫戰,果設詐誑燕軍,遂破騎劫於即墨下,而轉戰逐燕,北至河上,盡復得齊城,而迎襄王於莒,入于臨菑。

燕惠王後悔使騎劫代樂毅,以故破軍亡將失齊,又怨樂毅之降趙,恐趙用樂毅而乘燕之獘以伐燕。燕惠王乃使人讓樂毅,且謝之曰,先王舉國而委將軍,將軍為燕破齊,報先王之讎,天下莫不震動,寡人豈敢一日而忘將軍之功哉。會先王棄群臣,寡人新即位,左右誤寡人。寡人之使騎劫代將軍,為將軍久暴露於外,故召將軍且休,計事。將軍過聽,以與寡人有隙,遂捐燕歸趙。將軍自為計則可矣,而亦何以報先王之所以遇將軍之意乎。樂毅報遺燕惠王書曰,

臣不佞,不能奉承王命,以順左右之心,恐傷先王之明,有害足下之義,故遁逃走趙。今足下使人數之以罪,臣恐侍御者不察先王之所以畜幸臣之理,又不白臣之所以事先王之心,故敢以書對。

臣聞賢聖之君不以祿私親,其功多者賞之,其能當者處之。故察能而授官者,成功之君也,論行而結交者,立名之士也。臣竊觀先王之舉也,見有高世主之心,故假節於魏,以身得察於燕。先王過舉,廁之賓客之中,立之群臣之上,不謀父兄,以為亞卿。臣竊不自知,自以為奉令承教,可幸無罪,故受令而不辭。

先王命之曰,我有積怨深怒於齊,不量輕弱,而欲以齊為事。臣曰,夫齊,霸國之餘業而最勝之遺事也。練於兵甲,習於戰攻。王若欲伐之,必與天下圖之。與天下圖之,莫若結於趙。且又淮北、宋地,楚魏之所欲也,趙若許而約四國攻之,齊可大破也。先王以為然,具符節南使臣於趙。顧反命,起兵擊齊。以天之道,先王之靈,河北之地隨先王而舉之濟上。濟上之軍受命擊齊,大敗齊人。輕卒銳兵,長驅至國。齊王遁而走莒,僅以身免,珠玉財寶車甲珍器盡收入于燕。齊器設於寧臺,大呂陳於元英,故鼎反乎磿室,薊丘之植植於汶篁,自五伯已來,功未有及先王者也。先王以為慊於志,故裂地而封之,使得比小國諸侯。臣竊不自知,自以為奉命承教,可幸無罪,是以受命不辭。

臣聞賢聖之君,功立而不廢,故著於春秋,蚤知之士,名成而不毀,故稱於後世。若先王之報怨雪恥,夷萬乘之彊國,收八百歲之蓄積,及至棄群臣之日,餘教未衰,執政任事之臣,修法令,慎庶孽,施及乎萌隸,皆可以教後世。

臣聞之,善作者不必善成,善始者不必善終。昔伍子胥說聽於闔閭,而吳王遠跡至郢,夫差弗是也,賜之鴟夷而浮之江。吳王不寤先論之可以立功,故沈子胥而不悔,子胥不蚤見主之不同量,是以至於入江而不化。

夫免身立功,以明先王之跡,臣之上計也。離毀辱之誹謗,墮先王之名,臣之所大恐也。臨不測之罪,以幸為利,義之所不敢出也。

臣聞古之君子,交絕不出惡聲,忠臣去國,不絜其名。臣雖不佞,數奉教於君子矣。恐侍御者之親左右之說,不察疏遠之行,故敢獻書以聞,唯君王之留意焉。

於是燕王復以樂毅子樂閒為昌國君,而樂毅往來復通燕,燕、趙以為客卿。樂毅卒於趙。

樂閒居燕三十餘年,燕王喜用其相栗腹之計,欲攻趙,而問昌國君樂閒。樂閒曰,趙,四戰之國也,其民習兵,伐之不可。燕王不聽,遂伐趙。趙使廉頗擊之,大破栗腹之軍於鄗,禽栗腹、樂乘。樂乘者,樂閒之宗也。於是樂閒奔趙,趙遂圍燕。燕重割地以與趙和,趙乃解而去。

燕王恨不用樂閒,樂閒既在趙,乃遺樂閒書曰,紂之時,箕子不用,犯諫不怠,以冀其聽,商容不達,身祇辱焉,以冀其變。及民志不入,獄囚自出,然後二子退隱。故紂負桀暴之累,二子不失忠聖之名。何者。其憂患之盡矣。今寡人雖愚,不若紂之暴也,燕民雖亂,不若殷民之甚也。室有語,不相盡,以告鄰里。二者,寡人不為君取也。

樂閒、樂乘怨燕不聽其計,二人卒留趙。趙封樂乘為武襄君。

其明年,樂乘、廉頗為趙圍燕,燕重禮以和,乃解。後五歲,趙孝成王卒。襄王使樂乘代廉頗。廉頗攻樂乘,樂乘走,廉頗亡入魏。其後十六年而秦滅趙。

其後二十餘年,高帝過趙,問,樂毅有後世乎。對曰,有樂叔。高帝封之樂卿,號曰華成君。華成君,樂毅之孫也。而樂氏之族有樂瑕公、樂臣公,趙且為秦所滅,亡之齊高密。樂臣公善修黃帝、老子之言,顯聞於齊,稱賢師。

太史公曰,始齊之蒯通及主父偃讀樂毅之報燕王書,未嘗不廢書而泣也。樂臣公學黃帝、老子,其本師號曰河上丈人,不知其所出。河上丈人教安期生,安期生教毛翕公,毛翕公教樂瑕公,樂瑕公教樂臣公,樂臣公教蓋公。蓋公教於齊高密、膠西,為曹相國師。