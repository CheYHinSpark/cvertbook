\chapter{孫子吳起列傳第五}

孫子武者,齊人也。以兵法見於吳王闔廬。闔廬曰,子之十三篇,吾盡觀之矣,可以小試勒兵乎。對曰,可。闔廬曰,可試以婦人乎。曰,可。於是許之,出宮中美女,得百八十人。孫子分為二隊,以王之寵姬二人各為隊長,皆令持戟。令之曰,汝知而心與左右手背乎。婦人曰,知之。孫子曰,前,則視心,左,視左手,右,視右手,後,即視背。婦人曰,諾。約束既布,乃設鈇鉞,即三令五申之。於是鼓之右,婦人大笑。孫子曰,約束不明,申令不熟,將之罪也。復三令五申而鼓之左,婦人復大笑。孫子曰,約束不明,申令不熟,將之罪也,既已明而不如法者,吏士之罪也。乃欲斬左右隊長。吳王從臺上觀,見且斬愛姬,大駭。趣使使下令曰,寡人已知將軍能用兵矣。寡人非此二姬,食不甘味,願勿斬也。孫子曰,臣既已受命為將,將在軍,君命有所不受。遂斬隊長二人以徇。用其次為隊長,於是復鼓之。婦人左右前後跪起皆中規矩繩墨,無敢出聲。於是孫子使使報王曰,兵既整齊,王可試下觀之,唯王所欲用之,雖赴水火猶可也。吳王曰,將軍罷休就舍,寡人不願下觀。孫子曰,王徒好其言,不能用其實。於是闔廬知孫子能用兵,卒以為將。西破彊楚,入郢,北威齊晉,顯名諸侯,孫子與有力焉。

孫武既死,後百餘歲有孫臏。臏生阿鄄之閒,臏亦孫武之後世子孫也。孫臏嘗與龐涓俱學兵法。龐涓既事魏,得為惠王將軍,而自以為能不及孫臏,乃陰使召孫臏。臏至,龐涓恐其賢於己,疾之,則以法刑斷其兩足而黥之,欲隱勿見。

齊使者如梁,孫臏以刑徒陰見,說齊使。齊使以為奇,竊載與之齊。齊將田忌善而客待之。忌數與齊諸公子馳逐重射。孫子見其馬足不甚相遠,馬有上、中、下輩。於是孫子謂田忌曰,君弟重射,臣能令君勝。田忌信然之,與王及諸公子逐射千金。及臨質,孫子曰,今以君之下駟與彼上駟,取君上駟與彼中駟,取君中駟與彼下駟。既馳三輩畢,而田忌一不勝而再勝,卒得王千金。於是忌進孫子於威王。威王問兵法,遂以為師。

其後魏伐趙,趙急,請救於齊。齊威王欲將孫臏,臏辭謝曰,刑餘之人不可。於是乃以田忌為將,而孫子為師,居輜車中,坐為計謀。田忌欲引兵之趙,孫子曰,夫解雜亂紛糾者不控棬,救鬬者不搏撠,批亢擣虛,形格勢禁,則自為解耳。今梁趙相攻,輕兵銳卒必竭於外,老弱罷於內。君不若引兵疾走大梁,據其街路,衝其方虛,彼必釋趙而自救。是我一舉解趙之圍而收獘於魏也。田忌從之,魏果去邯鄲,與齊戰於桂陵,大破梁軍。

後十三歲,魏與趙攻韓,韓告急於齊。齊使田忌將而往,直走大梁。魏將龐涓聞之,去韓而歸,齊軍既已過而西矣。孫子謂田忌曰,彼三晉之兵素悍勇而輕齊,齊號為怯,善戰者因其勢而利導之。兵法,百里而趣利者蹶上將,五十里而趣利者軍半至。使齊軍入魏地為十萬灶,明日為五萬灶,又明日為三萬灶。龐涓行三日,大喜,曰,我固知齊軍怯,入吾地三日,士卒亡者過半矣。乃棄其步軍,與其輕銳倍日并行逐之。孫子度其行,暮當至馬陵。馬陵道陜,而旁多阻隘,可伏兵,乃斫大樹白而書之曰龐涓死于此樹之下。於是令齊軍善射者萬弩,夾道而伏,期曰暮見火舉而俱發。龐涓果夜至斫木下,見白書,乃鉆火燭之。讀其書未畢,齊軍萬弩俱發,魏軍大亂相失。龐涓自知智窮兵敗,乃自剄,曰,遂成豎子之名。齊因乘勝盡破其軍,虜魏太子申以歸。孫臏以此名顯天下,世傳其兵法。

吳起者,衛人也,好用兵。嘗學於曾子,事魯君。齊人攻魯,魯欲將吳起,吳起取齊女為妻,而魯疑之。吳起於是欲就名,遂殺其妻,以明不與齊也。魯卒以為將。將而攻齊,大破之。

魯人或惡吳起曰,起之為人,猜忍人也。其少時,家累千金,游仕不遂,遂破其家,鄉黨笑之,吳起殺其謗己者三十餘人,而東出衛郭門。與其母訣,齧臂而盟曰,起不為卿相,不復入衛。遂事曾子。居頃之,其母死,起終不歸。曾子薄之,而與起絕。起乃之魯,學兵法以事魯君。魯君疑之,起殺妻以求將。夫魯小國,而有戰勝之名,則諸侯圖魯矣。且魯衛兄弟之國也,而君用起,則是棄衛。魯君疑之,謝吳起。

吳起於是聞魏文侯賢,欲事之。文侯問李克曰,吳起何如人哉。李克曰,起貪而好色,然用兵司馬穰苴不能過也。於是魏文候以為將,擊秦,拔五城。

起之為將,與士卒最下者同衣食。臥不設席,行不騎乘,親裹贏糧,與士卒分勞苦。卒有病疽者,起為吮之。卒母聞而哭之。人曰,子卒也,而將軍自吮其疽,何哭為。母曰,非然也。往年吳公吮其父,其父戰不旋踵,遂死於敵。吳公今又吮其子,妾不知其死所矣。是以哭之。

文侯以吳起善用兵,廉平,盡能得士心,乃以為西河守,以拒秦、韓。

魏文侯既卒,起事其子武侯。武侯浮西河而下,中流,顧而謂吳起曰,美哉乎山河之固,此魏國之寶也。起對曰,在德不在險。昔三苗氏左洞庭,右彭蠡,德義不修,禹滅之。夏桀之居,左河濟,右泰華,伊闕在其南,羊腸在其北,修政不仁,湯放之。殷紂之國,左孟門,右太行,常山在其北,大河經其南,修政不德,武王殺之。由此觀之,在德不在險。若君不修德,舟中之人盡為敵國也。武侯曰,善。

吳起為西河守,甚有聲名。魏置相,相田文。吳起不悅,謂田文曰,請與子論功,可乎。田文曰,可。起曰,將三軍,使士卒樂死,敵國不敢謀,子孰與起。文曰,不如子。起曰,治百官,親萬民,實府庫,子孰與起。文曰,不如子。起曰,守西河而秦兵不敢東鄉,韓趙賓從,子孰與起。文曰,不如子。起曰,此三者,子皆出吾下,而位加吾上,何也。文曰,主少國疑,大臣未附,百姓不信,方是之時,屬之於子乎。屬之於我乎。起默然良久,曰,屬之子矣。文曰,此乃吾所以居子之上也。吳起乃自知弗如田文。

田文既死,公叔為相,尚魏公主,而害吳起。公叔之仆曰,起易去也。公叔曰,柰何。其仆曰,吳起為人節廉而自喜名也。君因先與武侯言曰,夫吳起賢人也,而侯之國小,又與彊秦壤界,臣竊恐起之無留心也。武侯即曰,柰何。君因謂武侯曰,試延以公主,起有留心則必受之。無留心則必辭矣。以此卜之。君因召吳起而與歸,即令公主怒而輕君。吳起見公主之賤君也,則必辭。於是吳起見公主之賤魏相,果辭魏武侯。武侯疑之而弗信也。吳起懼得罪,遂去,即之楚。

楚悼王素聞起賢,至則相楚。明法審令,捐不急之官,廢公族疏遠者,以撫養戰鬬之士。要在彊兵,破馳說之言從橫者。於是南平百越,北并陳蔡,卻三晉,西伐秦。諸侯患楚之彊。故楚之貴戚盡欲害吳起。及悼王死,宗室大臣作亂而攻吳起,吳起走之王尸而伏之。擊起之徒因射刺吳起,并中悼王。悼王既葬,太子立,乃使令尹盡誅射吳起而并中王尸者。坐射起而夷宗死者七十餘家。

太史公曰,世俗所稱師旅,皆道孫子十三篇,吳起兵法,世多有,故弗論,論其行事所施設者。語曰,能行之者未必能言,能言之者未必能行。孫子籌策龐涓明矣,然不能蚤救患於被刑。吳起說武侯以形勢不如德,然行之於楚,以刻暴少恩亡其軀。悲夫。