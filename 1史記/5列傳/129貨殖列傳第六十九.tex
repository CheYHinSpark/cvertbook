\chapter{貨殖列傳第六十九}

老子曰,至治之極,鄰國相望,雞狗之聲相聞,民各甘其食,美其服,安其俗,樂其業,至老死,不相往來。必用此為務,輓近世涂民耳目,則幾無行矣。

太史公曰,夫神農以前,吾不知已。至若詩書所述虞夏以來,耳目欲極聲色之好,口欲窮芻豢之味,身安逸樂,而心誇矜輓能之榮使。俗之漸民久矣,雖戶說以眇論,終不能化。故善者因之,其次利道之,其次教誨之,其次整齊之,最下者與之爭。

夫山西饒材、竹、穀、纑、旄、玉石,山東多魚、鹽、漆、絲、聲色,江南出枏、梓、薑、桂、金、錫、連、丹沙、犀、瑁、珠璣、齒革,龍門、碣石北多馬、牛、羊、旃裘、筋角,銅、鐵則千里往往山出棋置,此其大較也。皆中國人民所喜好謠俗被服飲食奉生送死之具也。故待農而食之,虞而出之,工而成之,商而通之。此寧有政教發徵期會哉。人各任其能,竭其力,以得所欲。故物賤之徵貴,貴之徵賤,各勸其業,樂其事,若水之趨下,日夜無休時,不召而自來,不求而民出之。豈非道之所符,而自然之驗邪。

周書曰,農不出則乏其食,工不出則乏其事,商不出則三寶絕,虞不出則財匱少。財匱少而山澤不辟矣。此四者,民所衣食之原也。原大則饒,原小則鮮。上則富國,下則富家。貧富之道,莫之奪予,而巧者有餘,拙者不足。故太公望封於營丘,地澙鹵,人民寡,於是太公勸其女功,極技巧,通魚鹽,則人物歸之,繦至而輻湊。故齊冠帶衣履天下,海岱之閒斂袂而往朝焉。其後齊中衰,管子修之,設輕重九府,則桓公以霸,九合諸侯,一匡天下,而管氏亦有三歸,位在陪臣,富於列國之君。是以齊富彊至於威、宣也。

故曰,倉廩實而知禮節,衣食足而知榮辱。禮生於有而廢於無。故君子富,好行其德,小人富,以適其力。淵深而魚生之,山深而獸往之,人富而仁義附焉。富者得埶益彰,失埶則客無所之,以而不樂。夷狄益甚。諺曰,千金之子,不死於市。此非空言也。故曰,天下熙熙,皆為利來,天下壤壤,皆為利往。夫千乘之王,萬家之侯,百室之君,尚猶患貧,而況匹夫編戶之民乎。

昔者越王句踐困於會稽之上,乃用范蠡、計然。計然曰,知斗則修備,時用則知物,二者形則萬貨之情可得而觀已。故歲在金,穰,水,毀,木,饑,火,旱。旱則資舟,水則資車,物之理也。六歲穰,六歲旱,十二歲一大饑。夫糶,二十病農,九十病末。末病則財不出,農病則草不辟矣。上不過八十,下不減三十,則農末俱利,平糶齊物,關市不乏,治國之道也。積著之理,務完物,無息幣。以物相貿易,腐敗而食之貨勿留,無敢居貴。論其有餘不足,則知貴賤。貴上極則反賤,賤下極則反貴。貴出如糞土,賤取如珠玉。財幣欲其行如流水。修之十年,國富,厚賂戰士,士赴矢石,如渴得飲,遂報彊吳,觀兵中國,稱號五霸。

范蠡既雪會稽之恥,乃喟然而嘆曰,計然之策七,越用其五而得意。既已施於國,吾欲用之家。乃乘扁舟浮於江湖,變名易姓,適齊為鴟夷子皮,之陶為朱公。朱公以為陶天下之中,諸侯四通,貨物所交易也。乃治產積居。與時逐而不責於人。故善治生者,能擇人而任時。十九年之中三致千金,再分散與貧交疏昆弟。此所謂富好行其德者也。後年衰老而聽子孫,子孫修業而息之,遂至巨萬。故言富者皆稱陶朱公。

子贛既學於仲尼,退而仕於衛,廢著鬻財於曹、魯之閒,七十子之徒,賜最為饒益。原憲不厭糟糠,匿於窮巷。子貢結駟連騎,束帛之幣以聘享諸侯,所至,國君無不分庭與之抗禮。夫使孔子名布揚於天下者,子貢先後之也。此所謂得埶而益彰者乎。

白圭,周人也。當魏文侯時,李克務盡地力,而白圭樂觀時變,故人棄我取,人取我與。夫歲孰取穀,予之絲漆,繭出取帛絮,予之食。太陰在卯,穰,明歲衰惡。至午,旱,明歲美。至酉,穰,明歲衰惡。至子,大旱,明歲美,有水。至卯,積著率歲倍。欲長錢,取下穀,長石斗,取上種。能薄飲食,忍嗜欲,節衣服,與用事僮仆同苦樂,趨時若猛獸摯鳥之發。故曰,吾治生產,猶伊尹、呂尚之謀,孫吳用兵,商鞅行法是也。是故其智不足與權變,勇不足以決斷,仁不能以取予,彊不能有所守,雖欲學吾術,終不告之矣。蓋天下言治生祖白圭。白圭其有所試矣,能試有所長,非茍而已也。

猗頓用盬鹽起。而邯鄲郭縱以鐵冶成業,與王者埒富。

烏氏倮牧,及眾,斥賣,求奇繒物,閒獻遺戎王。戎王什倍其償,與之畜,畜至用谷量馬牛。秦始皇帝令倮比封君,以時與列臣朝請。而巴蜀寡婦清,其先得丹穴,而擅其利數世,家亦不訾。清,寡婦也,能守其業,用財自衛,不見侵犯。秦皇帝以為貞婦而客之,為筑女懷清臺。夫倮鄙人牧長,清窮鄉寡婦,禮抗萬乘,名顯天下,豈非以富邪。

漢興,海內為一,開關梁,弛山澤之禁,是以富商大賈周流天下,交易之物莫不通,得其所欲,而徙豪傑諸侯彊族於京師。

關中自汧、雍以東至河、華,膏壤沃野千里,自虞夏之貢以為上田,而公劉適邠,大王、王季在岐,文王作豐,武王治鎬,故其民猶有先王之遺風,好稼穡,殖五穀,地重,重為邪。及秦文、德、繆居雍,隙隴蜀之貨物而多賈。獻公徙櫟邑,櫟邑北卻戎翟,東通三晉,亦多大賈。孝、昭治咸陽,因以漢都,長安諸陵,四方輻湊并至而會,地小人眾,故其民益玩巧而事末也。南則巴蜀。巴蜀亦沃野,地饒炧、薑、丹沙、石、銅、鐵、竹、木之器。南御滇僰,僰僮。西近邛笮,笮馬、旄牛。然四塞,棧道千里,無所不通,唯褒斜綰轂其口,以所多易所鮮。天水、隴西、北地、上郡與關中同俗,然西有羌中之利,北有戎翟之畜,畜牧為天下饒。然地亦窮險,唯京師要其道。故關中之地,於天下三分之一,而人眾不過什三,然量其富,什居其六。

昔唐人都河東,殷人都河內,周人都河南。夫三河在天下之中,若鼎足,王者所更居也,建國各數百千歲,土地小狹,民人眾,都國諸侯所聚會,故其俗纖儉習事。楊、平陽陳西賈秦、翟,北賈種、代。種、代,石北也,地邊胡,數被寇。人民矜懻忮,好氣,任俠為姦,不事農商。然迫近北夷,師旅亟往,中國委輸時有奇羨。其民羯羠不均,自全晉之時固已患其僄悍,而武靈王益厲之,其謠俗猶有趙之風也。故楊、平陽陳掾其閒,得所欲。溫、軹西賈上黨,北賈趙、中山。中山地薄人眾,猶有沙丘紂淫地餘民,民俗懁急,仰機利而食。丈夫相聚游戲,悲歌慨,起則相隨椎剽,休則掘冢作巧姦冶,多美物,為倡優。女子則鼓鳴瑟,跕屣,游媚貴富,入後宮,遍諸侯。

然邯鄲亦漳、河之閒一都會也。北通燕、涿,南有鄭、衛。鄭、衛俗與趙相類,然近梁、魯,微重而矜節。濮上之邑徙野王,野王好氣任俠,衛之風也。

夫燕亦勃、碣之閒一都會也。南通齊、趙,東北邊胡。上谷至遼東,地踔遠,人民希,數被寇,大與趙、代俗相類,而民雕捍少慮,有魚鹽棗栗之饒。北鄰烏桓、夫餘,東綰穢貉、朝鮮、真番之利。

洛陽東賈齊、魯,南賈梁、楚。故泰山之陽則魯,其陰則齊。

齊帶山海,膏壤千里,宜桑麻,人民多文綵布帛魚鹽。臨菑亦海岱之閒一都會也。其俗寬緩闊達,而足智,好議論,地重,難動搖,怯於眾鬬,勇於持刺,故多劫人者,大國之風也。其中具五民。

而鄒、魯濱洙、泗,猶有周公遺風,俗好儒,備於禮,故其民齪齪。頗有桑麻之業,無林澤之饒。地小人眾,儉嗇,畏罪遠邪。及其衰,好賈趨利,甚於周人。

夫自鴻溝以東,芒、碭以北,屬巨野,此梁、宋也。陶、睢陽亦一都會也。昔堯作於成陽,舜漁於雷澤,湯止于亳。其俗猶有先王遺風,重厚多君子,好稼穡,雖無山川之饒,能惡衣食,致其蓄藏。

越、楚則有三俗。夫自淮北沛、陳、汝南、南郡,此西楚也。其俗剽輕,易發怒,地薄,寡於積聚。江陵故郢都,西通巫、巴,東有雲夢之饒。陳在楚夏之交,通魚鹽之貨,其民多賈。徐、僮、取慮,則清刻,矜己諾。

彭城以東,東海、吳、廣陵,此東楚也。其俗類徐、僮。朐、繒以北,俗則齊。浙江南則越。夫吳自闔廬、春申、王濞三人招致天下之喜游子弟,東有海鹽之饒,章山之銅,三江、五湖之利,亦江東一都會也。

衡山、九江、江南、豫章、長沙,是南楚也,其俗大類西楚。郢之後徙壽春,亦一都會也。而合肥受南北潮,皮革、鮑、木輸會也。與閩中、干越雜俗,故南楚好辭,巧說少信。江南卑溼,丈夫早夭。多竹木。豫章出黃金,長沙出連、錫,然堇堇物之所有,取之不足以更費。九疑、蒼梧以南至儋耳者,與江南大同俗,而楊越多焉。番禺亦其一都會也,珠璣、犀、瑁、果、布之湊。

潁川、南陽,夏人之居也。夏人政尚忠樸,猶有先王之遺風。潁川敦願。秦末世,遷不軌之民於南陽。南陽西通武關、鄖關,東南受漢、江、淮。宛亦一都會也。俗雜好事,業多賈。其任俠,交通潁川,故至今謂之夏人。

夫天下物所鮮所多,人民謠俗,山東食海鹽,山西食鹽鹵,領南、沙北固往往出鹽,大體如此矣。

總之,楚越之地,地廣人希,飯稻羹魚,或火耕而水耨,果隋蠃蛤,不待賈而足,地埶饒食,無饑饉之患,以故呰窳偷生,無積聚而多貧。是故江淮以南,無凍餓之人,亦無千金之家。沂、泗水以北,宜五穀桑麻六畜,地小人眾,數被水旱之害,民好畜藏,故秦、夏、梁、魯好農而重民。三河、宛、陳亦然,加以商賈。齊、趙設智巧,仰機利。燕、代田畜而事蠶。

由此觀之,賢人深謀於廊廟,論議朝廷,守信死節隱居巖穴之士設為名高者安歸乎。歸於富厚也。是以廉吏久,久更富,廉賈歸富。富者,人之情性,所不學而俱欲者也。故壯士在軍,攻城先登,陷陣卻敵,斬將搴旗,前蒙矢石,不避湯火之難者,為重賞使也。其在閭巷少年,攻剽椎埋,劫人作姦,掘冢鑄幣,任俠并兼,借交報仇,篡逐幽隱,不避法禁,走死地如騖者,其實皆為財用耳。今夫趙女鄭姬,設形容,揳鳴琴,揄長袂,躡利屣,目挑心招,出不遠千里,不擇老少者,奔富厚也。游閒公子,飾冠劍,連車騎,亦為富貴容也。弋射漁獵,犯晨夜,冒霜雪,馳阬谷,不避猛獸之害,為得味也。博戲馳逐,鬬雞走狗,作色相矜,必爭勝者,重失負也。醫方諸食技術之人,焦神極能,為重糈也。吏士舞文弄法,刻章偽書,不避刀鋸之誅者,沒於賂遺也。農工商賈畜長,固求富益貨也。此有知盡能索耳,終不餘力而讓財矣。

諺曰,百里不販樵,千里不販糴。居之一歲,種之以穀,十歲,樹之以木,百歲,來之以德。德者,人物之謂也。今有無秩祿之奉,爵邑之入,而樂與之比者。命曰素封。封者食租稅,歲率戶二百。千戶之君則二十萬,朝覲聘享出其中。庶民農工商賈,率亦歲萬息二千,百萬之家則二十萬,而更傜租賦出其中。衣食之欲,恣所好美矣。故曰陸地牧馬二百蹄,牛蹄角千,千足羊,澤中千足彘,水居千石魚陂,山居千章之材。安邑千樹棗,燕、秦千樹栗,蜀、漢、江陵千樹橘,淮北、常山已南,河濟之閒千樹萩,陳、夏千畝漆,齊、魯千畝桑麻,渭川千畝竹,及名國萬家之城,帶郭千畝畝鐘之田,若千畝卮茜,千畦薑韭,此其人皆與千戶侯等。然是富給之資也,不窺市井,不行異邑,坐而待收,身有處士之義而取給焉。若至家貧親老,妻子軟弱,歲時無以祭祀進醵,飲食被服不足以自通,如此不慚恥,則無所比矣。是以無財作力,少有鬬智,既饒爭時,此其大經也。今治生不待危身取給,則賢人勉焉。是故本富為上,末富次之,姦富最下。無巖處奇士之行,而長貧賤,好語仁義,亦足羞也。

凡編戶之民,富相什則卑下之,伯則畏憚之,千則役,萬則仆,物之理也。夫用貧求富,農不如工,工不如商,刺繡文不如倚市門,此言末業,貧者之資也。通邑大都,酤一歲千釀,醯醬千瓨,漿千甔,屠牛羊彘千皮,販穀糶千鐘,薪槁千車,船長千丈,木千章,竹竿萬,其軺車百乘,牛車千兩,木器髤者千枚,銅器千鈞,素木鐵器若炧茜千石,馬蹄蹾千,牛千足,羊彘千雙,僮手指千,筋角丹沙千斤,其帛絮細布千鈞,文采千匹,榻布皮革千石,漆千斗,糱麹鹽豉千荅,鮐鮆千斤,鯫千石,鮑千鈞,棗栗千石者三之,狐鼦裘千皮,羔羊裘千石,旃席千具,佗果菜千鐘,子貸金錢千貫,節駔會,貪賈三之,廉賈五之,此亦比千乘之家,其大率也。佗雜業不中什二,則非吾財也。

請略道當世千里之中,賢人所以富者,令後世得以觀擇焉。

蜀卓氏之先,趙人也,用鐵冶富。秦破趙,遷卓氏。卓氏見虜略,獨夫妻推輦,行詣遷處。諸遷虜少有餘財,爭與吏,求近處,處葭萌。唯卓氏曰,此地狹薄。吾聞汶山之下,沃野,下有蹲鴟,至死不饑。民工於市,易賈。乃求遠遷。致之臨邛,大喜,即鐵山鼓鑄,運籌策,傾滇蜀之民,富至僮千人。田池射獵之樂,擬於人君。

程鄭,山東遷虜也,亦冶鑄,賈椎髻之民,富埒卓氏,俱居臨邛。

宛孔氏之先,梁人也,用鐵冶為業。秦伐魏,遷孔氏南陽。大鼓鑄,規陂池,連車騎,游諸侯,因通商賈之利,有游閒公子之賜與名。然其贏得過當,愈於纖嗇,家致富數千金,故南陽行賈盡法孔氏之雍容。

魯人俗儉嗇,而曹邴氏尤甚,以鐵冶起,富至巨萬。然家自父兄子孫約,俛有拾,仰有取,貰貸行賈遍郡國。鄒、魯以其故多去文學而趨利者,以曹邴氏也。

齊俗賤奴虜,而刀閒獨愛貴之。桀黠奴,人之所患也,唯刀閒收取,使之逐漁鹽商賈之利,或連車騎,交守相,然愈益任之。終得其力,起富數千萬。故曰寧爵毋刀,言其能使豪奴自饒而盡其力。

周人既纖,而師史尤甚,轉轂以百數,賈郡國,無所不至。洛陽街居在齊秦楚趙之中,貧人學事富家,相矜以久賈,數過邑不入門,設任此等,故師史能致七千萬。

宣曲任氏之先,為督道倉吏。秦之敗也,豪傑皆爭取金玉,而任氏獨窖倉粟。楚漢相距滎陽也,民不得耕種,米石至萬,而豪傑金玉盡歸任氏,任氏以此起富。富人爭奢侈,而任氏折節為儉,力田畜。田畜人爭取賤賈,任氏獨取貴善。富者數世。然任公家約,非田畜所出弗衣食,公事不畢則身不得飲酒食肉。以此為閭里率,故富而主上重之。

塞之斥也,唯橋姚已致馬千匹,牛倍之,羊萬頭,粟以萬鐘計。吳楚七國兵起時,長安中列侯封君行從軍旅,齎貸子錢,子錢家以為侯邑國在關東,關東成敗未決,莫肯與。唯無鹽氏出捐千金貸,其息什之。三月,吳楚平,一歲之中,則無鹽氏之息什倍,用此富埒關中。

關中富商大賈,大抵盡諸田,田嗇、田蘭。韋家栗氏,安陵、杜杜氏,亦巨萬。

此其章章尤異者也。皆非有爵邑奉祿弄法犯姦而富,盡椎埋去就,與時俯仰,獲其贏利,以末致財,用本守之,以武一切,用文持之,變化有概,故足術也。若至力農畜,工虞商賈,為權利以成富,大者傾郡,中者傾縣,下者傾鄉里者,不可勝數。

夫纖嗇筋力,治生之正道也,而富者必用奇勝。田農,掘業,而秦揚以蓋一州。掘冢,姦事也,而田叔以起。博戲,惡業也,而桓發用富。行賈,丈夫賤行也,而雍樂成以饒。販脂,辱處也,而雍伯千金。賣漿,小業也,而張氏千萬。灑削,薄技也,而郅氏鼎食。胃脯,簡微耳,濁氏連騎。馬醫,淺方,張裏擊鐘。此皆誠壹之所致。由是觀之,富無經業,則貨無常主,能者輻湊,不肖者瓦解。千金之家比一都之君,巨萬者乃與王者同樂。豈所謂素封者邪。非也。