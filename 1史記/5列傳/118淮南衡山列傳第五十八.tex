\chapter{淮南衡山列傳第五十八}

淮南厲王長者,高祖少子也,其母故趙王張敖美人。高祖八年,從東垣過趙,趙王獻之美人。厲王母得幸焉,有身。趙王敖弗敢內宮,為筑外宮而捨之。及貫高等謀反柏人事發覺,并逮治王,盡收捕王母兄弟美人,系之河內。厲王母亦系,告吏曰,得幸上,有身。吏以聞上,上方怒趙王,未理厲王母。厲王母弟趙兼因辟陽侯言呂后,呂后妒,弗肯白,辟陽侯不彊爭。及厲王母已生厲王,恚,即自殺。吏奉厲王詣上,上悔,令呂后母之,而葬厲王母真定。真定,厲王母之家在焉,父世縣也。

高祖十一年七月,淮南王黥布反,立子長為淮南王,王黥布故地,凡四郡。上自將兵擊滅布,厲王遂即位。厲王蚤失母,常附呂后,孝惠、呂后時以故得幸無患害,而常心怨辟陽侯,弗敢發。及孝文帝初即位,淮南王自以為最親,驕蹇,數不奉法。上以親故,常寬赦之。三年,入朝。甚橫。從上入苑囿獵,與上同車,常謂上大兄。厲王有材力,力能扛鼎,乃往請辟陽侯。辟陽侯出見之,即自袖鐵椎椎辟陽侯,令從者魏敬剄之。厲王乃馳走闕下,肉袒謝曰,臣母不當坐趙事,其時辟陽侯力能得之呂后,弗爭,罪一也。趙王如意子母無罪,呂后殺之,辟陽侯弗爭,罪二也。呂后王諸呂,欲以危劉氏,辟陽侯弗爭,罪三也。臣謹為天下誅賊臣辟陽侯,報母之仇,謹伏闕下請罪。孝文傷其志,為親故,弗治,赦厲王。當是時,薄太后及太子諸大臣皆憚厲王,厲王以此歸國益驕恣,不用漢法,出入稱警蹕,稱制,自為法令,擬於天子。

六年,令男子但等七十人與棘蒲侯柴武太子奇謀,以輂車四十乘反谷口,令人使閩越、匈奴。事覺,治之,使使召淮南王。淮南王至長安。

丞相臣張倉、典客臣馮敬、行御史大夫事宗正臣逸、廷尉臣賀、備盜賊中尉臣福昧死言,淮南王長廢先帝法,不聽天子詔,居處無度,為黃屋蓋乘輿,出入擬於天子,擅為法令,不用漢法。及所置吏,以其郎中春為丞相,聚收漢諸侯人及有罪亡者,匿與居,為治家室,賜其財物爵祿田宅,爵或至關內侯,奉以二千石,所不當得,欲以有為。大夫但、士五開章等七十人與棘蒲侯太子奇謀反,欲以危宗廟社稷。使開章陰告長,與謀使閩越及匈奴發其兵。開章之淮南見長,長數與坐語飲食,為家室娶婦,以二千石俸奉之。開章使人告但,已言之王。春使使報但等。吏覺知,使長安尉奇等往捕開章。長匿不予,與故中尉蕑忌謀,殺以閉口。為棺槨衣衾,葬之肥陵邑,謾吏曰不知安在。又詳聚土,樹表其上,曰開章死,埋此下。及長身自賊殺無罪者一人,令吏論殺無罪者六人,為亡命棄市罪詐捕命者以除罪,擅罪人,罪人無告劾,系治城旦舂以上十四人,赦免罪人,死罪十八人,城旦舂以下五十八人,賜人爵關內侯以下九十四人。前日長病,陛下憂苦之,使使者賜書、棗脯。長不欲受賜,不肯見拜使者。南海民處廬江界中者反,淮南吏卒擊之。陛下以淮南民貧苦,遣使者賜長帛五千匹,以賜吏卒勞苦者。長不欲受賜,謾言曰無勞苦者。南海民王織上書獻璧皇帝,忌擅燔其書,不以聞。吏請召治忌,長不遣,謾言曰忌病。春又請長,願入見,長怒曰女欲離我自附漢。長當棄市,臣請論如法。

制曰,朕不忍致法於王,其與列侯二千石議。

臣倉、臣敬、臣逸、臣福、臣賀昧死言,臣謹與列侯吏二千石臣嬰等四十三人議,皆曰長不奉法度,不聽天子詔,乃陰聚徒黨及謀反者,厚養亡命,欲以有為。臣等議論如法。

制曰,朕不忍致法於王,其赦長死罪,廢勿王。

臣倉等昧死言,長有大死罪,陛下不忍致法,幸赦,廢勿王。臣請處蜀郡嚴道邛郵,遣其子母從居,縣為筑蓋家室,皆廩食給薪菜鹽豉炊食器席蓐。臣等昧死請,請布告天下。

制曰,計食長給肉日五斤,酒二斗。令故美人才人得幸者十人從居。他可。

盡誅所與謀者。於是乃遣淮南王,載以輜車,令縣以次傳。是時袁盎諫上曰,上素驕淮南王,弗為置嚴傅相,以故至此。且淮南王為人剛,今暴摧折之。臣恐卒逢霧露病死。陛下為有殺弟之名,柰何。上曰,吾特苦之耳,今復之。縣傳淮南王者皆不敢發車封。淮南王乃謂侍者曰,誰謂乃公勇者。吾安能勇。吾以驕故不聞吾過至此。人生一世閒,安能邑邑如此。乃不食死。至雍,雍令發封,以死聞。上哭甚悲,謂袁盎曰,吾不聽公言,卒亡淮南王。盎曰,不可柰何,願陛下自寬。上曰,為之柰何。盎曰,獨斬丞相、御史以謝天下乃可。上即令丞相、御史逮考諸縣傳送淮南王不發封餽侍者,皆棄市。乃以列侯葬淮南王於雍,守冢三十戶。

孝文八年,上憐淮南王,淮南王有子四人,皆七八歲,乃封子安為阜陵侯,子勃為安陽侯,子賜為陽周侯,子良為東成侯。

孝文十二年,民有作歌歌淮南厲王曰,一尺布,尚可縫,一斗粟,尚可舂。兄弟二人不能相容。上聞之,乃嘆曰,堯舜放逐骨肉,周公殺管蔡,天下稱聖。何者。不以私害公。天下豈以我為貪淮南王地邪。乃徙城陽王王淮南故地,而追尊謚淮南王為厲王,置園復如諸侯儀。

孝文十六年,徙淮南王喜復故城陽。上憐淮南厲王廢法不軌,自使失國蚤死,乃立其三子,阜陵侯安為淮南王,安陽侯勃為衡山王,陽周侯賜為廬江王,皆復得厲王時地,參分之。東城侯良前薨,無後也。

孝景三年,吳楚七國反,吳使者至淮南,淮南王欲發兵應之。其相曰,大王必欲發兵應吳,臣願為將。王乃屬相兵。淮南相已將兵,因城守,不聽王而為漢,漢亦使曲城侯將兵救淮南,淮南以故得完。吳使者至廬江,廬江王弗應,而往來使越。吳使者至衡山,衡山王堅守無二心。孝景四年,吳楚已破,衡山王朝,上以為貞信,乃勞苦之曰,南方卑溼。徙衡山王王濟北,所以褒之。及薨,遂賜謚為貞王。廬江王邊越,數使使相交,故徙為衡山王,王江北。淮南王如故。

淮南王安為人好讀書鼓琴,不喜弋獵狗馬馳騁,亦欲以行陰德拊循百姓,流譽天下。時時怨望厲王死,時欲畔逆,未有因也。及建元二年,淮南王入朝。素善武安侯,武安侯時為太尉,乃逆王霸上,與王語曰,方今上無太子,大王親高皇帝孫,行仁義,天下莫不聞。即宮車一日晏駕,非大王當誰立者。淮南王大喜,厚遺武安侯金財物。陰結賓客,拊循百姓,為畔逆事。建元六年,彗星見,淮南王心怪之。或說王曰,先吳軍起時,彗星出長數尺,然尚流血千里。今彗星長竟天,天下兵當大起。王心以為上無太子,天下有變,諸侯并爭,愈益治器械攻戰具,積金錢賂遺郡國諸侯游士奇材。諸辨士為方略者,妄作妖言,諂諛王,王喜,多賜金錢,而謀反滋甚。

淮南王有女陵,慧,有口辯。王愛陵,常多予金錢,為中诇長安,約結上左右。元朔三年,上賜淮南王几杖,不朝。淮南王王后荼,王愛幸之。王后生太子遷,遷取王皇太后外孫修成君女為妃。王謀為反具,畏太子妃知而內泄事,乃與太子謀,令詐弗愛,三月不同席。王乃詳為怒太子,閉太子使與妃同內三月,太子終不近妃。妃求去,王乃上書謝歸去之。王后荼、太子遷及女陵得愛幸王,擅國權,侵奪民田宅,妄致系人。

元朔五年,太子學用劍,自以為人莫及,聞郎中雷被巧,乃召與戲。被一再辭讓,誤中太子。太子怒,被恐。此時有欲從軍者輒詣京師,被即願奮擊匈奴。太子遷數惡被於王,王使郎中令斥免,欲以禁后,被遂亡至長安,上書自明。詔下其事廷尉、河南。河南治,逮淮南太子,王、王后計欲無遣太子,遂發兵反,計猶豫,十餘日未定。會有詔,即訊太子。當是時,淮南相怒壽春丞留太子逮不遣,劾不敬。王以請相,相弗聽。王使人上書告相,事下廷尉治。蹤跡連王,王使人候伺漢公卿,公卿請逮捕治王。王恐事發,太子遷謀曰,漢使即逮王,王令人衣衛士衣,持戟居庭中,王旁有非是,則刺殺之,臣亦使人刺殺淮南中尉,乃舉兵,未晚。是時上不許公卿請,而遣漢中尉宏即訊驗王。王聞漢使來,即如太子謀計。漢中尉至,王視其顏色和,訊王以斥雷被事耳,王自度無何,不發。中尉還,以聞。公卿治者曰,淮南王安擁閼奮擊匈奴者雷被等,廢格明詔,當棄市。詔弗許。公卿請廢勿王,詔弗許。公卿請削五縣,詔削二縣。使中尉宏赦淮南王罪,罰以削地。中尉入淮南界,宣言赦王。王初聞漢公卿請誅之,未知得削地,聞漢使來,恐其捕之,乃與太子謀刺之如前計。及中尉至,即賀王,王以故不發。其後自傷曰,吾行仁義見削,甚恥之。然淮南王削地之后,其為反謀益甚。諸使道從長安來,為妄妖言,言上無男,漢不治,即喜,即言漢廷治,有男,王怒,以為妄言,非也。

王日夜與伍被、左吳等案輿地圖,部署兵所從入。王曰,上無太子,宮車即晏駕,廷臣必徵膠東王,不即常山王,諸侯并爭,吾可以無備乎。且吾高祖孫,親行仁義,陛下遇我厚,吾能忍之,萬世之後,吾寧能北面臣事豎子乎。

王坐東宮,召伍被與謀,曰,將軍上。被悵然曰,上寬赦大王,王復安得此亡國之語乎。臣聞子胥諫吳王,吳王不用,乃曰臣今見麋鹿游姑蘇之臺也。今臣亦見宮中生荊棘,露霑衣也。王怒,系伍被父母,囚之三月。復召曰,將軍許寡人乎。被曰,不,直來為大王畫耳。臣聞聰者聽於無聲,明者見於未形,故聖人萬舉萬全。昔文王一動而功顯于千世,列為三代,此所謂因天心以動作者也,故海內不期而隨。此千歲之可見者。夫百年之秦,近世之吳楚,亦足以喻國家之存亡矣。臣不敢避子胥之誅,願大王毋為吳王之聽。昔秦絕聖人之道,殺術士,燔詩書,棄禮義,尚詐力,任刑罰,轉負海之粟致之西河。當是之時,男子疾耕不足於糟糠,女子紡績不足於蓋形。遣蒙恬筑長城,東西數千里,暴兵露師常數十萬,死者不可勝數,僵尸千里,流血頃畝,百姓力竭,欲為亂者十家而五。又使徐福入海求神異物,還為偽辭曰,臣見海中大神,言曰,汝西皇之使邪。臣答曰,然。汝何求。曰,願請延年益壽藥。神曰,汝秦王之禮薄,得觀而不得取。即從臣東南至蓬萊山,見芝成宮闕,有使者銅色而龍形,光上照天。於是臣再拜問曰,宜何資以獻。海神曰,以令名男子若振女與百工之事,即得之矣。秦皇帝大說,遣振男女三千人,資之五穀種種百工而行。徐福得平原廣澤,止王不來。於是百姓悲痛相思,欲為亂者十家而六。又使尉佗踰五嶺攻百越。尉佗知中國勞極,止王不來,使人上書,求女無夫家者三萬人,以為士卒衣補。秦皇帝可其萬五千人。於是百姓離心瓦解,欲為亂者十家而七。客謂高皇帝曰,時可矣。高皇帝曰,待之,聖人當起東南閒。不一年,陳勝吳廣發矣。高皇始於豐沛,一倡天下不期而響應者不可勝數也。此所謂蹈瑕候閒,因秦之亡而動者也。百姓願之,若旱之望雨,故起於行陳之中而立為天子,功高三王,德傳無窮。今大王見高皇帝得天下之易也,獨不觀近世之吳楚乎。夫吳王賜號為劉氏祭酒,復不朝,王四郡之眾,地方數千里,內鑄消銅以為錢,東煑海水以為鹽,上取江陵木以為船,一船之載當中國數十兩車,國富民眾。行珠玉金帛賂諸侯宗室大臣,獨竇氏不與。計定謀成,舉兵而西。破於大梁,敗於狐父,奔走而東,至於丹徒,越人禽之,身死絕祀,為天下笑。夫以吳越之眾不能成功者何。誠逆天道而不知時也。方今大王之兵眾不能十分吳楚之一,天下安寧有萬倍於秦之時,願大王從臣之計。大王不從臣之計,今見大王事必不成而語先泄也。臣聞微子過故國而悲,於是作麥秀之歌,是痛紂之不用王子比干也。故孟子曰紂貴為天子,死曾不若匹夫。是紂先自絕於天下久矣,非死之日而天下去之。今臣亦竊悲大王棄千乘之君,必且賜絕命之書,為群臣先,死於東宮也。於是王氣怨結而不揚,涕滿匡而橫流,即起,歷階而去。

王有孽子不害,最長,王弗愛,王、王后、太子皆不以為子兄數。不害有子建,材高有氣,常怨望太子不省其父,又怨時諸侯皆得分子弟為侯,而淮南獨二子,一為太子,建父獨不得為侯。建陰結交,欲告敗太子,以其父代之。太子知之,數捕系而榜笞建。建具知太子之謀欲殺漢中尉,即使所善壽春莊芷以元朔六年上書於天子曰,毒藥苦於口利於病,忠言逆於耳利於行。今淮南王孫建,材能高,淮南王王后荼、荼子太子遷常疾害建。建父不害無罪,擅數捕系,欲殺之。今建在,可徵問,具知淮南陰事。書聞,上以其事下廷尉,廷尉下河南治。是時故辟陽侯孫審卿善丞相公孫弘,怨淮南厲王殺其大父,乃深購淮南事於弘,弘乃疑淮南有畔逆計謀,深窮治其獄。河南治建,辭引淮南太子及黨與。淮南王患之,欲發,問伍被曰,漢廷治亂。伍被曰,天下治。王意不說,謂伍被曰,公何以言天下治也。被曰,被竊觀朝廷之政,君臣之義,父子之親,夫婦之別,長幼之序,皆得其理,上之舉錯遵古之道,風俗紀綱未有所缺也。重裝富賈,周流天下,道無不通,故交易之道行。南越賓服,羌僰入獻,東甌入降,廣長榆,開朔方,匈奴折翅傷翼,失援不振。雖未及古太平之時,然猶為治也。王怒,被謝死罪。王又謂被曰,山東即有兵,漢必使大將軍將而制山東,公以為大將軍何如人也。被曰,被所善者黃義,從大將軍擊匈奴,還,告被曰,大將軍遇士大夫有禮,於士卒有恩,眾皆樂為之用。騎上下山若蜚,材干絕人。被以為材能如此,數將習兵,未易當也。及謁者曹梁使長安來,言大將軍號令明,當敵勇敢,常為士卒先。休舍,穿井未通,須士卒盡得水,乃敢飲。軍罷,卒盡已度河,乃度。皇太后所賜金帛,盡以賜軍吏。雖古名將弗過也。王默然。

淮南王見建已徵治,恐國陰事且覺,欲發,被又以為難,乃復問被曰,公以為吳興兵是邪非也。被曰,以為非也。吳王至富貴也,舉事不當,身死丹徒,頭足異處,子孫無遺類。臣聞吳王悔之甚。願王孰慮之,無為吳王之所悔。王曰,男子之所死者一言耳。且吳何知反,漢將一日過成皋者四十餘人。今我令樓緩先要成皋之口,周被下潁川兵塞轘轅、伊闕之道,陳定發南陽兵守武關。河南太守獨有雒陽耳,何足憂。然此北尚有臨晉關、河東、上黨與河內、趙國。人言曰絕成皋之口,天下不通。據三川之險,招山東之兵,舉事如此,公以為何如。被曰,臣見其禍,未見其福也。王曰,左吳、趙賢、朱驕如皆以為有福,什事九成,公獨以為有禍無福,何也。被曰,大王之群臣近幸素能使眾者,皆前系詔獄,餘無可用者。王曰,陳勝、吳廣無立錐之地,千人之聚,起於大澤,奮臂大呼而天下響應,西至於戲而兵百二十萬。今吾國雖小,然而勝兵者可得十餘萬,非直適戍之眾,釠鑿棘矜也,公何以言有禍無福。被曰,往者秦為無道,殘賊天下。興萬乘之駕,作阿房之宮,收太半之賦,發閭左之戍,父不寧子,兄不便弟,政苛刑峻,天下熬然若焦,民皆引領而望,傾耳而聽,悲號仰天,叩心而怨上,故陳勝大呼,天下響應。當今陛下臨制天下,一齊海內,汎愛蒸庶,布德施惠。口雖未言,聲疾雷霆,令雖未出,化馳如神,心有所懷,威動萬里,下之應上,猶影響也。而大將軍材能不特章邯、楊熊也。大王以陳勝、吳廣諭之,被以為過矣。王曰,茍如公言,不可徼幸邪。被曰,被有愚計。王曰,柰何。被曰,當今諸侯無異心,百姓無怨氣。朔方之郡田地廣,水草美,民徙者不足以實其地。臣之愚計,可偽為丞相御史請書,徙郡國豪桀任俠及有耐罪以上,赦令除其罪,產五十萬以上者,皆徙其家屬朔方之郡,益發甲卒,急其會日。又偽為左右都司空上林中都官詔獄書,逮諸侯太子幸臣。如此則民怨,諸侯懼,即使辯武隨而說之,儻可徼幸什得一乎。王曰,此可也。雖然,吾以為不至若此。於是王乃令官奴入宮,作皇帝璽,丞相、御史、大將軍、軍吏、中二千石、都官令、丞印,及旁近郡太守、都尉印,漢使節法冠,欲如伍被計。使人偽得罪而西,事大將軍、丞相,一日發兵,使人即刺殺大將軍青,而說丞相下之,如發蒙耳。

王欲發國中兵,恐其相、二千石不聽。王乃與伍被謀,先殺相、二千石,偽失火宮中,相、二千石救火,至即殺之。計未決,又欲令人衣求盜衣,持羽檄,從東方來,呼曰南越兵入界,欲因以發兵。乃使人至廬江、會稽為求盜,未發。王問伍被曰,吾舉兵西鄉,諸侯必有應我者,即無應,柰何。被曰,南收衡山以擊廬江,有尋陽之船,守下雉之城,結九江之浦,絕豫章之口,彊弩臨江而守,以禁南郡之下,東收江都、會稽,南通勁越,屈彊江淮閒,猶可得延歲月之壽。王曰,善,無以易此。急則走越耳。

於是廷尉以王孫建辭連淮南王太子遷聞。上遣廷尉監因拜淮南中尉,逮捕太子。至淮南,淮南王聞,與太子謀召相、二千石,欲殺而發兵。召相,相至,內史以出為解。中尉曰,臣受詔使,不得見王。王念獨殺相而內史中尉不來,無益也,即罷相。王猶豫,計未決。太子念所坐者謀刺漢中尉,所與謀者已死,以為口絕,乃謂王曰,群臣可用者皆前系,今無足與舉事者。王以非時發,恐無功,臣願會逮。王亦偷欲休,即許太子。太子即自剄,不殊。伍被自詣吏,因告與淮南王謀反,反蹤跡具如此。

吏因捕太子、王后,圍王宮,盡求捕王所與謀反賓客在國中者,索得反具以聞。上下公卿治,所連引與淮南王謀反列侯二千石豪傑數千人,皆以罪輕重受誅。衡山王賜,淮南王弟也,當坐收,有司請逮捕衡山王。天子曰,諸侯各以其國為本,不當相坐。與諸侯王列侯會肄丞相諸侯議。趙王彭祖、列侯臣讓等四十三人議,皆曰,淮南王安甚大逆無道,謀反明白,當伏誅。膠西王臣端議曰,淮南王安廢法行邪,懷詐偽心,以亂天下,熒惑百姓,倍畔宗廟,妄作妖言。春秋曰臣無將,將而誅。安罪重於將,謀反形已定。臣端所見其書節印圖及他逆無道事驗明白,甚大逆無道,當伏其法。而論國吏二百石以上及比者,宗室近幸臣不在法中者,不能相教,當皆免官削爵為士伍,毋得宦為吏。其非吏,他贖死金二斤八兩。以章臣安之罪,使天下明知臣子之道,毋敢復有邪僻倍畔之意。丞相弘、廷尉湯等以聞,天子使宗正以符節治王。未至,淮南王安自剄殺。王后荼、太子遷諸所與謀反者皆族。天子以伍被雅辭多引漢之美,欲勿誅。廷尉湯曰,被首為王畫反謀,被罪無赦。遂誅被。國除為九江郡。

衡山王賜,王后乘舒生子三人,長男爽為太子,次男孝,次女無采。又姬徐來生子男女四人,美人厥姬生子二人。衡山王、淮南王兄弟相責望禮節,閒不相能。衡山王聞淮南王作為畔逆反具,亦心結賓客以應之,恐為所并。

元光六年,衡山王入朝,其謁者衛慶有方術,欲上書事天子,王怒,故劾慶死罪,彊榜服之。衡山內史以為非是,卻其獄。王使人上書告內史,內史治,言王不直。王又數侵奪人田,壞人冢以為田。有司請逮治衡山王。天子不許,為置吏二百石以上。衡山王以此恚,與奚慈、張廣昌謀,求能為兵法候星氣者,日夜從容王密謀反事。

王后乘舒死,立徐來為王后。厥姬俱幸。兩人相妒,厥姬乃惡王后徐來於太子曰,徐來使婢蠱道殺太子母。太子心怨徐來。徐來兄至衡山,太子與飲,以刃刺傷王后兄。王后怨怒,數毀惡太子於王。太子女弟無采,嫁棄歸,與奴姦,又與客姦。太子數讓無采,無采怒,不與太子通。王后聞之,即善遇無采。無采及中兄孝少失母,附王后,王后以計愛之,與共毀太子,王以故數擊笞太子。元朔四年中,人有賊傷王后假母者,王疑太子使人傷之,笞太子。後王病,太子時稱病不侍。孝王后、無采惡太子,太子實不病,自言病,有喜色。王大怒,欲廢太子,立其弟孝。王后知王決廢太子,又欲并廢孝。王后有侍者,善舞,王幸之,王后欲令侍者與孝亂以汙之,欲并廢兄弟而立其子廣代太子。太子爽知之,念后數惡己無已時,欲與亂以止其口。王后飲,太子前為壽,因據王后股,求與王后臥。王后怒,以告王。王乃召,欲縛而笞之。太子知王常欲廢己立其弟孝,乃謂王曰,孝與王御者姦,無采與奴姦,王彊食,請上書。即倍王去。王使人止之,莫能禁,乃自駕追捕太子。太子妄惡言,王械系太子宮中。孝日益親幸。王奇孝材能,乃佩之王印,號曰將軍,令居外宅,多給金錢,招致賓客。賓客來者,微知淮南、衡山有逆計,日夜從容勸之。王乃使孝客江都人救赫、陳喜作輣車鏃矢,刻天子璽,將相軍吏印。王日夜求壯士如周丘等,數稱引吳楚反時計畫,以約束。衡山王非敢效淮南王求即天子位,畏淮南起并其國,以為淮南已西,發兵定江淮之閒而有之,望如是。

元朔五年秋,衡山王當朝,六年過淮南,淮南王乃昆弟語,除前卻,約束反具。衡山王即上書謝病,上賜書不朝。

元朔六年中,衡山王使人上書請廢太子爽,立孝為太子。爽聞,即使所善白嬴之長安上書,言孝作輣車鏃矢,與王御者姦,欲以敗孝。白嬴至長安,未及上書,吏捕嬴,以淮南事系。王聞爽使白嬴上書,恐言國陰事,即上書反告太子爽所為不道棄市罪事。事下沛郡治。元狩元年冬,有司公卿下沛郡求捕所與淮南謀反者未得,得陳喜於衡山王子孝家。吏劾孝首匿喜。孝以為陳喜雅數與王計謀反,恐其發之,聞律先自告除其罪,又疑太子使白嬴上書發其事,即先自告,告所與謀反者救赫、陳喜等。廷尉治驗,公卿請逮捕衡山王治之。天子曰,勿捕。遣中尉安、大行息即問王,王具以情實對。吏皆圍王宮而守之。中尉大行還,以聞,公卿請遣宗正、大行與沛郡雜治王。王聞,即自剄殺。孝先自告反,除其罪,坐與王御婢姦,棄市。王后徐來亦坐蠱殺前王后乘舒,及太子爽坐王告不孝,皆棄市。諸與衡山王謀反者皆族。國除為衡山郡。

太史公曰,詩之所謂戎狄是膺,荊舒是懲,信哉是言也。淮南、衡山親為骨肉,疆土千里,列為諸侯,不務遵蕃臣職以承輔天子,而專挾邪僻之計,謀為畔逆,仍父子再亡國,各不終其身,為天下笑。此非獨王過也,亦其俗薄,臣下漸靡使然也。夫荊楚彊勇輕悍,好作亂,乃自古記之矣。