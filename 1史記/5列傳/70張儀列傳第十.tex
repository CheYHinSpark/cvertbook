\chapter{張儀列傳第十}

張儀者,魏人也。始嘗與蘇秦俱事鬼谷先生,學術,蘇秦自以不及張儀。

張儀已學游說諸侯。嘗從楚相飲,已而楚相亡璧,門下意張儀,曰,儀貧無行,必此盜相君之璧。共執張儀,掠笞數百,不服,醳之。其妻曰,嘻。子毋讀書游說,安得此辱乎。張儀謂其妻曰,視吾舌尚在不。其妻笑曰,舌在也。儀曰,足矣。

蘇秦已說趙王而得相約從親,然恐秦之攻諸侯,敗約後負,念莫可使用於秦者,乃使人微感張儀曰,子始與蘇秦善,今秦已當路,子何不往游,以求通子之願。張儀於是之趙,上謁求見蘇秦。蘇秦乃誡門下人不為通,又使不得去者數日。已而見之,坐之堂下,賜仆妾之食。因而數讓之曰,以子之材能,乃自令困辱至此。吾寧不能言而富貴子,子不足收也。謝去之。張儀之來也,自以為故人,求益,反見辱,怒,念諸侯莫可事,獨秦能苦趙,乃遂入秦。

蘇秦已而告其舍人曰,張儀,天下賢士,吾殆弗如也。今吾幸先用,而能用秦柄者,獨張儀可耳。然貧,無因以進。吾恐其樂小利而不遂,故召辱之,以激其意。子為我陰奉之。乃言趙王,發金幣車馬,使人微隨張儀,與同宿舍,稍稍近就之,奉以車馬金錢,所欲用,為取給,而弗告。張儀遂得以見秦惠王。惠王以為客卿,與謀伐諸侯。

蘇秦之舍人乃辭去。張儀曰,賴子得顯,方且報德,何故去也。舍人曰,臣非知君,知君乃蘇君。蘇君憂秦伐趙敗從約,以為非君莫能得秦柄,故感怒君,使臣陰奉給君資,盡蘇君之計謀。今君已用,請歸報。張儀曰,嗟乎,此在吾術中而不悟,吾不及蘇君明矣。吾又新用,安能謀趙乎。為吾謝蘇君,蘇君之時,儀何敢言。且蘇君在,儀寧渠能乎。張儀既相秦,為文檄告楚相曰,始吾從若飲,我不盜而璧,若笞我。若善守汝國,我顧且盜而城。

苴蜀相攻擊,各來告急於秦。秦惠王欲發兵以伐蜀,以為道險狹難至,而韓又來侵秦,秦惠王欲先伐韓,後伐蜀,恐不利,欲先伐蜀,恐韓襲秦之敝。猶豫未能決。司馬錯與張儀爭論於惠王之前,司馬錯欲伐蜀,張儀曰,不如伐韓。王曰,請聞其說。

儀曰,親魏善楚,下兵三川,塞什谷之口,當屯留之道,魏絕南陽,楚臨南鄭,秦攻新城、宜陽,以臨二周之郊,誅周王之罪,侵楚、魏之地。周自知不能救,九鼎寶器必出。據九鼎,案圖籍,挾天子以令於天下,天下莫敢不聽,此王業也。今夫蜀,西僻之國而戎翟之倫也,敝兵勞眾不足以成名,得其地不足以為利。臣聞爭名者於朝,爭利者於市。今三川、周室,天下之朝市也,而王不爭焉,顧爭於戎翟,去王業遠矣。

司馬錯曰,不然。臣聞之,欲富國者務廣其地,欲彊兵者務富其民,欲王者務博其德,三資者備而王隨之矣。今王地小民貧,故臣願先從事於易。夫蜀,西僻之國也,而戎翟之長也,有桀紂之亂。以秦攻之,譬如使豺狼逐群羊。得其地足以廣國,取其財足以富民繕兵,不傷眾而彼已服焉。拔一國而天下不以為暴,利盡西海而天下不以為貪,是我一舉而名實附也,而又有禁暴止亂之名。今攻韓,劫天子,惡名也,而未必利也,又有不義之名,而攻天下所不欲,危矣。臣請謁其故,周,天下之宗室也,齊,韓之與國也。周自知失九鼎,韓自知亡三川,將二國并力合謀,以因乎齊、趙而求解乎楚、魏,以鼎與楚,以地與魏,王弗能止也。此臣之所謂危也。不如伐蜀完。

惠王曰,善,寡人請聽子。卒起兵伐蜀,十月,取之,遂定蜀,貶蜀王更號為侯,而使陳莊相蜀。蜀既屬秦,秦以益彊,富厚,輕諸侯。

秦惠王十年,使公子華與張儀圍蒲陽,降之。儀因言秦復與魏,而使公子繇質於魏。儀因說魏王曰,秦王之遇魏甚厚,魏不可以無禮。魏因入上郡、少梁,謝秦惠王。惠王乃以張儀為相,更名少梁曰夏陽。

儀相秦四歲,立惠王為王。居一歲,為秦將,取陜。筑上郡塞。

其後二年,使與齊、楚之相會齧桑。東還而免相,相魏以為秦,欲令魏先事秦而諸侯效之。魏王不肯聽儀。秦王怒,伐取魏之曲沃、平周,復陰厚張儀益甚。張儀慚,無以歸報。留魏四歲而魏襄王卒,哀王立。張儀復說哀王,哀王不聽。於是張儀陰令秦伐魏。魏與秦戰,敗。

明年,齊又來敗魏於觀津。秦復欲攻魏,先敗韓申差軍,斬首八萬,諸侯震恐。而張儀復說魏王曰,魏地方不至千里,卒不過三十萬。地四平,諸侯四通輻湊,無名山大川之限。從鄭至梁二百餘里,車馳人走,不待力而至。梁南與楚境,西與韓境,北與趙境,東與齊境,卒戍四方,守亭鄣者不下十萬。梁之地勢,固戰場也。梁南與楚而不與齊,則齊攻其東,東與齊而不與趙,則趙攻其北,不合於韓,則韓攻其西,不親於楚,則楚攻其南,此所謂四分五裂之道也。

且夫諸侯之為從者,將以安社稷尊主彊兵顯名也。今從者一天下,約為昆弟,刑白馬以盟洹水之上,以相堅也。而親昆弟同父母,尚有爭錢財,而欲恃詐偽反覆蘇秦之餘謀,其不可成亦明矣。

大王不事秦,秦下兵攻河外,據卷、衍、燕、酸棗,劫衛取陽晉,則趙不南,趙不南而梁不北,梁不北則從道絕,從道絕則大王之國欲毋危不可得也。秦折韓而攻梁,韓怯於秦,秦韓為一,梁之亡可立而須也。此臣之所為大王患也。

為大王計,莫如事秦。事秦則楚、韓必不敢動,無楚、韓之患,則大王高枕而臥,國必無憂矣。

且夫秦之所欲弱者莫如楚,而能弱楚者莫如梁。楚雖有富大之名而實空虛,其卒雖多,然而輕走易北,不能堅戰。悉梁之兵南面而伐楚,勝之必矣。割楚而益梁,虧楚而適秦,嫁禍安國,此善事也。大王不聽臣,秦下甲士而東伐,雖欲事秦,不可得矣。

且夫從人多奮辭而少可信,說一諸侯而成封侯,是故天下之游談士莫不日夜搤腕瞋目切齒以言從之便,以說人主。人主賢其辯而牽其說,豈得無眩哉。

臣聞之,積羽沈舟,群輕折軸,眾口鑠金,積毀銷骨,故願大王審定計議,且賜骸骨辟魏。

哀王於是乃倍從約而因儀請成於秦。張儀歸,復相秦。三歲而魏復背秦為從。秦攻魏,取曲沃。明年,魏復事秦。

秦欲伐齊,齊楚從親,於是張儀往相楚。楚懷王聞張儀來,虛上舍而自館之。曰,此僻陋之國,子何以教之。儀說楚王曰,大王誠能聽臣,閉關絕約於齊,臣請獻商於之地六百里,使秦女得為大王箕帚之妾,秦楚娶婦嫁女,長為兄弟之國。此北弱齊而西益秦也,計無便此者。楚王大說而許之。群臣皆賀,陳軫獨弔之。楚王怒曰,寡人不興師發兵得六百里地,群臣皆賀,子獨弔,何也。陳軫對曰,不然,以臣觀之,商於之地不可得而齊秦合,齊秦合則患必至矣。楚王曰,有說乎。陳軫對曰,夫秦之所以重楚者,以其有齊也。今閉關絕約於齊,則楚孤。秦奚貪夫孤國,而與之商於之地六百里。張儀至秦,必負王,是北絕齊交,西生患於秦也,而兩國之兵必俱至。善為王計者,不若陰合而陽絕於齊,使人隨張儀。茍與吾地,絕齊未晚也,不與吾地,陰合謀計也。楚王曰,願陳子閉口毋復言,以待寡人得地。乃以相印授張儀,厚賂之。於是遂閉關絕約於齊,使一將軍隨張儀。

張儀至秦,詳失綏墮車,不朝三月。楚王聞之,曰,儀以寡人絕齊未甚邪。乃使勇士至宋,借宋之符,北罵齊王。齊王大怒,折節而下秦。秦齊之交合,張儀乃朝,謂楚使者曰,臣有奉邑六里,願以獻大王左右。楚使者曰,臣受令於王,以商於之地六百里,不聞六里。還報楚王,楚王大怒,發兵而攻秦。陳軫曰,軫可發口言乎。攻之不如割地反以賂秦,與之并兵而攻齊,是我出地於秦,取償於齊也,王國尚可存。楚王不聽,卒發兵而使將軍屈丐擊秦。秦齊共攻楚,斬首八萬,殺屈丐,遂取丹陽、漢中之地。楚又復益發兵而襲秦,至藍田,大戰,楚大敗,於是楚割兩城以與秦平。

秦要楚欲得黔中地,欲以武關外易之。楚王曰,不願易地,願得張儀而獻黔中地。秦王欲遣之,口弗忍言。張儀乃請行。惠王曰,彼楚王怒子之負以商於之地,是且甘心於子。張儀曰,秦彊楚弱,臣善靳尚,尚得事楚夫人鄭袖,袖所言皆從。且臣奉王之節使楚,楚何敢加誅。假令誅臣而為秦得黔中之地,臣之上願。遂使楚。楚懷王至則囚張儀,將殺之。靳尚謂鄭袖曰,子亦知子之賤於王乎。鄭袖曰,何也。靳尚曰,秦王甚愛張儀而不欲出之,今將以上庸之地六縣賂楚,美人聘楚,以宮中善歌謳者為媵。楚王重地尊秦,秦女必貴而夫人斥矣。不若為言而出之。於是鄭袖日夜言懷王曰,人臣各為其主用。今地未入秦,秦使張儀來,至重王。王未有禮而殺張儀,秦必大怒攻楚。妾請子母俱遷江南,毋為秦所魚肉也。懷王後悔,赦張儀,厚禮之如故。

張儀既出,未去,聞蘇秦死,乃說楚王曰,秦地半天下,兵敵四國,被險帶河,四塞以為固。虎賁之士百餘萬,車千乘,騎萬匹,積粟如丘山。法令既明,士卒安難樂死,主明以嚴,將智以武,雖無出甲,席卷常山之險,必折天下之脊,天下有後服者先亡。且夫為從者,無以異於驅群羊而攻猛虎,虎之與羊不格明矣。今王不與猛虎而與群羊,臣竊以為大王之計過也。

凡天下彊國,非秦而楚,非楚而秦,兩國交爭,其勢不兩立。大王不與秦,秦下甲據宜陽,韓之上地不通。下河東,取成皋,韓必入臣,梁則從風而動。秦攻楚之西,韓、梁攻其北,社稷安得毋危。

且夫從者聚群弱而攻至彊,不料敵而輕戰,國貧而數舉兵,危亡之術也。臣聞之,兵不如者勿與挑戰,粟不如者勿與持久。夫從人飾辯虛辭,高主之節,言其利不言其害,卒有秦禍,無及為已。是故願大王之孰計之。

秦西有巴蜀,大船積粟,起於汶山,浮江已下,至楚三千餘里。舫船載卒,一舫載五十人與三月之食,下水而浮,一日行三百餘里,里數雖多,然而不費牛馬之力,不至十日而距扜關。扜關驚,則從境以東盡城守矣,黔中、巫郡非王之有。秦舉甲出武關,南面而伐,則北地絕。秦兵之攻楚也,危難在三月之內,而楚待諸侯之救,在半歲之外,此其勢不相及也。夫恃弱國之救,忘彊秦之禍,此臣所以為大王患也。

大王嘗與吳人戰,五戰而三勝,陣卒盡矣,偏守新城,存民苦矣。臣聞功大者易危,而民敝者怨上。夫守易危之功而逆彊秦之心,臣竊為大王危之。

且夫秦之所以不出兵函谷十五年以攻齊、趙者,陰謀有合天下之心。楚嘗與秦構難,戰於漢中,楚人不勝,列侯執珪死者七十餘人,遂亡漢中。楚王大怒,興兵襲秦,戰於藍田。此所謂兩虎相搏者也。夫秦楚相敝而韓魏以全制其後,計無危於此者矣。願大王孰計之。

秦下甲攻衛陽晉,必大關天下之匈。大王悉起兵以攻宋,不至數月而宋可舉,舉宋而東指,則泗上十二諸侯盡王之有也。

凡天下而以信約從親相堅者蘇秦,封武安君,相燕,即陰與燕王謀伐破齊而分其地,乃詳有罪出走入齊,齊王因受而相之,居二年而覺,齊王大怒,車裂蘇秦於市。夫以一詐偽之蘇秦,而欲經營天下,混一諸侯,其不可成亦明矣。

今秦與楚接境壤界,固形親之國也。大王誠能聽臣,臣請使秦太子入質於楚,楚太子入質於秦,請以秦女為大王箕帚之妾,效萬室之都以為湯沐之邑,長為昆弟之國,終身無相攻伐。臣以為計無便於此者。

於是楚王已得張儀而重出黔中地與秦,欲許之。屈原曰,前大王見欺於張儀,張儀至,臣以為大王烹之,今縱弗忍殺之,又聽其邪說,不可。懷王曰,許儀而得黔中,美利也。後而倍之,不可。故卒許張儀,與秦親。

張儀去楚,因遂之韓,說韓王曰,韓地險惡山居,五穀所生,非菽而麥,民之食大抵飯菽飯藿羹。一歲不收,民不饜糟糠。地不過九百里,無二歲之食。料大王之卒,悉之不過三十萬,而廝徒負養在其中矣。除守徼亭鄣塞,見卒不過二十萬而已矣。秦帶甲百餘萬,車千乘,騎萬匹,虎賁之士跿跔科頭貫頤奮戟者,至不可勝計。秦馬之良,戎兵之眾,探前趹後蹄閒三尋騰者,不可勝數。山東之士被甲蒙胄以會戰,秦人捐甲徒裼以趨敵,左挈人頭,右挾生虜。夫秦卒與山東之卒,猶孟賁之與怯夫,以重力相壓,猶烏獲之與嬰兒。夫戰孟賁、烏獲之士以攻不服之弱國,無異垂千鈞之重於鳥卵之上,必無幸矣。

夫群臣諸侯不料地之寡,而聽從人之甘言好辭,比周以相飾也,皆奮曰聽吾計可以彊霸天下。夫不顧社稷之長利而聽須臾之說,詿誤人主,無過此者。

大王不事秦,秦下甲據宜陽,斷韓之上地,東取成皋、滎陽,則鴻臺之宮、桑林之苑非王之有也。夫塞成皋,絕上地,則王之國分矣。先事秦則安,不事秦則危。夫造禍而求其福報,計淺而怨深,逆秦而順楚,雖欲毋亡,不可得也。

故為大王計,莫如為秦。秦之所欲莫如弱楚,而能弱楚者如韓。非以韓能彊於楚也,其地勢然也。今王西面而事秦以攻楚,秦王必喜。夫攻楚以利其地,轉禍而說秦,計無便於此者。

韓王聽儀計。張儀歸報,秦惠王封儀五邑,號曰武信君。使張儀東說齊湣王曰,天下彊國無過齊者,大臣父兄殷眾富樂。然而為大王計者,皆為一時之說,不顧百世之利。從人說大王者,必曰齊西有彊趙,南有韓與梁。齊,負海之國也,地廣民眾,兵彊士勇,雖有百秦,將無柰齊何。大王賢其說而不計其實。夫從人朋黨比周,莫不以從為可。臣聞之,齊與魯三戰而魯三勝,國以危亡隨其後,雖有戰勝之名,而有亡國之實。是何也。齊大而魯小也。今秦之與齊也,猶齊之與魯也。秦趙戰於河漳之上,再戰而趙再勝秦,戰於番吾之下,再戰又勝秦。四戰之後,趙之亡卒數十萬,邯鄲僅存,雖有戰勝之名而國已破矣。是何也。秦彊而趙弱。

今秦楚嫁女娶婦,為昆弟之國。韓獻宜陽,梁效河外,趙入朝澠池,割河閒以事秦。大王不事秦,秦驅韓梁攻齊之南地,悉趙兵渡清河,指博關,臨菑、即墨非王之有也。國一日見攻,雖欲事秦,不可得也。是故願大王孰計之也。

齊王曰,齊僻陋,隱居東海之上,未嘗聞社稷之長利也。乃許張儀。

張儀去,西說趙王曰,敝邑秦王使使臣效愚計於大王。大王收率天下以賓秦,秦兵不敢出函谷關十五年。大王之威行於山東,敝邑恐懼懾伏,繕甲厲兵,飾車騎,習馳射,力田積粟,守四封之內,愁居懾處,不敢動搖,唯大王有意督過之也。

今以大王之力,舉巴蜀,并漢中,包兩周,遷九鼎,守白馬之津。秦雖僻遠,然而心忿含怒之日久矣。今秦有敝甲凋兵,軍於澠池,願渡河踰漳,據番吾,會邯鄲之下,願以甲子合戰,以正殷紂之事,敬使使臣先聞左右。

凡大王之所信為從者恃蘇秦。蘇秦熒惑諸侯,以是為非,以非為是,欲反齊國,而自令車裂於市。夫天下之不可一亦明矣。今楚與秦為昆弟之國,而韓梁稱為東藩之臣,齊獻魚鹽之地,此斷趙之右臂也。夫斷右臂而與人鬬,失其黨而孤居,求欲毋危,豈可得乎。

今秦發三將軍,其一軍塞午道,告齊使興師渡清河,軍於邯鄲之東,一軍軍成皋,驅韓梁軍於河外,一軍軍於澠池。約四國為一以攻趙,趙破,必四分其地。是故不敢匿意隱情,先以聞於左右。臣竊為大王計,莫如與秦王遇於澠池,面相見而口相結,請案兵無攻。願大王之定計。

趙王曰,先王之時,奉陽君專權擅勢,蔽欺先王,獨擅綰事,寡人居屬師傅,不與國謀計。先王棄群臣,寡人年幼,奉祀之日新,心固竊疑焉,以為一從不事秦,非國之長利也。乃且願變心易慮,割地謝前過以事秦。方將約車趨行,適聞使者之明詔。趙王許張儀,張儀乃去。

北之燕,說燕昭王曰,大王之所親莫如趙。昔趙襄子嘗以其姊為代王妻,欲并代,約與代王遇於句注之塞。乃令工人作為金斗,長其尾,令可以擊人。與代王飲,陰告廚人曰,即酒酣樂,進熱啜,反鬬以擊之。於是酒酣樂,進熱啜,廚人進斟,因反鬬以擊代王,殺之,王腦涂地。其姊聞之,因摩笄以自刺,故至今有摩笄之山。代王之亡,天下莫不聞。

夫趙王之很戾無親,大王之所明見,且以趙王為可親乎。趙興兵攻燕,再圍燕都而劫大王,大王割十城以謝。今趙王已入朝澠池,效河閒以事秦。今大王不事秦,秦下甲雲中、九原,驅趙而攻燕,則易水、長城非大王之有也。

且今時趙之於秦猶郡縣也,不敢妄舉師以攻伐。今王事秦,秦王必喜,趙不敢妄動,是西有彊秦之援,而南無齊趙之患,是故願大王孰計之。

燕王曰,寡人蠻夷僻處,雖大男子裁如嬰兒,言不足以采正計。今上客幸教之,請西面而事秦,獻恒山之尾五城。燕王聽儀。儀歸報,未至咸陽而秦惠王卒,武王立。武王自為太子時不說張儀,及即位,群臣多讒張儀曰,無信,左右賣國以取容。秦必復用之,恐為天下笑。諸侯聞張儀有卻武王,皆畔衡,復合從。

秦武王元年,群臣日夜惡張儀未已,而齊讓又至。張儀懼誅,乃因謂秦武王曰,儀有愚計,願效之。王曰,柰何。對曰,為秦社稷計者,東方有大變,然後王可以多割得地也。今聞齊王甚憎儀,儀之所在,必興師伐之。故儀願乞其不肖之身之梁,齊必興師而伐梁。梁齊之兵連於城下而不能相去,王以其閒伐韓,入三川,出兵函谷而毋伐,以臨周,祭器必出。挾天子,按圖籍,此王業也。秦王以為然,乃具革車三十乘,入儀之梁。齊果興師伐之。梁哀王恐。張儀曰,王勿患也,請令罷齊兵。乃使其舍人馮喜之楚,借使之齊,謂齊王曰,王甚憎張儀,雖然,亦厚矣王之託儀於秦也。齊王曰,寡人憎儀,儀之所在,必興師伐之,何以託儀。對曰,是乃王之託儀也。夫儀之出也,固與秦王約曰,為王計者,東方有大變,然後王可以多割得地。今齊王甚憎儀,儀之所在,必興師伐之。故儀願乞其不肖之身之梁,齊必興師伐之。齊梁之兵連於城下而不能相去,王以其閒伐韓,入三川,出兵函谷而無伐,以臨周,祭器必出。挾天子,案圖籍,此王業也。秦王以為然,故具革車三十乘而入之梁也。今儀入梁,王果伐之,是王內罷國而外伐與國,廣鄰敵以內自臨,而信儀於秦王也。此臣之所謂託儀也。齊王曰,善。乃使解兵。

張儀相魏一歲,卒於魏也。

陳軫者,游說之士。與張儀俱事秦惠王,皆貴重,爭寵。張儀惡陳軫於秦王曰,軫重幣輕使秦楚之閒,將為國交也。今楚不加善於秦而善軫者,軫自為厚而為王薄也。且軫欲去秦而之楚,王胡不聽乎。王謂陳軫曰,吾聞子欲去秦之楚,有之乎。軫曰,然。王曰,儀之言果信矣。軫曰,非獨儀知之也,行道之士盡知之矣。昔子胥忠於其君而天下爭以為臣,曾參孝於其親而天下願以為子。故賣仆妾不出閭巷而售者,良仆妾也,出婦嫁於鄉曲者,良婦也。今軫不忠其君,楚亦何以軫為忠乎。忠且見棄,軫不之楚何歸乎。王以其言為然,遂善待之。

居秦期年,秦惠王終相張儀,而陳軫奔楚。楚未之重也,而使陳軫使於秦。過梁,欲見犀首。犀首謝弗見。軫曰,吾為事來,公不見軫,軫將行,不得待異日。犀首見之。陳軫曰,公何好飲也。犀首曰,無事也。曰,吾請令公厭事可乎。曰,柰何。曰,田需約諸侯從親,楚王疑之,未信也。公謂於王曰,臣與燕、趙之王有故,數使人來,曰,無事何不相見,願謁行於王。王雖許公,公請毋多車,以車三十乘,可陳之於庭,明言之燕、趙。燕、趙客聞之,馳車告其王,使人迎犀首。楚王聞之大怒,曰,田需與寡人約,而犀首之燕、趙,是欺我也。怒而不聽其事。齊聞犀首之北,使人以事委焉。犀首遂行,三國相事皆斷於犀首。軫遂至秦。

韓魏相攻,期年不解。秦惠王欲救之,問於左右。左右或曰救之便,或曰勿救便,惠王未能為之決。陳軫適至秦,惠王曰,子去寡人之楚,亦思寡人不。陳軫對曰,王聞夫越人莊舄乎。王曰,不聞。曰,越人莊舄仕楚執珪,有頃而病。楚王曰,舄故越之鄙細人也,今仕楚執珪,貴富矣,亦思越不。中謝對曰,凡人之思故,在其病也。彼思越則越聲,不思越則楚聲。使人往聽之,猶尚越聲也。今臣雖棄逐之楚,豈能無秦聲哉。惠王曰,善。今韓魏相攻,期年不解,或謂寡人救之便,或曰勿救便,寡人不能決,願子為子主計之餘,為寡人計之。陳軫對曰,亦嘗有以夫卞莊子刺虎聞於王者乎。莊子欲刺虎,館豎子止之,曰,兩虎方且食牛,食甘必爭,爭則必鬬,鬬則大者傷,小者死,從傷而刺之,一舉必有雙虎之名。卞莊子以為然,立須之。有頃,兩虎果鬬,大者傷,小者死。莊子從傷者而刺之,一舉果有雙虎之功。今韓魏相攻,期年不解,是必大國傷,小國亡,從傷而伐之,一舉必有兩實。此猶莊子刺虎之類也。臣主與王何異也。惠王曰,善。卒弗救。大國果傷,小國亡,秦興兵而伐,大剋之。此陳軫之計也。

犀首者,魏之陰晉人也,名衍,姓公孫氏。與張儀不善。

張儀為秦之魏,魏王相張儀。犀首弗利,故令人謂韓公叔曰,張儀已合秦魏矣,其言曰魏攻南陽,秦攻三川。魏王所以貴張子者,欲得韓地也。且韓之南陽已舉矣,子何不少委焉以為衍功,則秦魏之交可錯矣。然則魏必圖秦而棄儀,收韓而相衍。公叔以為便,因委之犀首以為功。果相魏。張儀去。

義渠君朝於魏。犀首聞張儀復相秦,害之。犀首乃謂義渠君曰,道遠不得復過,請謁事情。曰,中國無事,秦得燒掇焚杅君之國,有事,秦將輕使重幣事君之國。其後五國伐秦。會陳軫謂秦王曰,義渠君者,蠻夷之賢君也,不如賂之以撫其志。秦王曰,善。乃以文繡千純,婦女百人遺義渠君。義渠君致群臣而謀曰,此公孫衍所謂邪。乃起兵襲秦,大敗秦人李伯之下。

張儀已卒之後,犀首入相秦。嘗佩五國之相印,為約長。

太史公曰,三晉多權變之士,夫言從衡彊秦者大抵皆三晉之人也。夫張儀之行事甚於蘇秦,然世惡蘇秦者,以其先死,而儀振暴其短以扶其說,成其衡道。要之,此兩人真傾危之士哉。