\chapter{蒙恬列傳第二十八}

蒙恬者,其先齊人也。恬大父蒙驁,自齊事秦昭王,官至上卿。秦莊襄王元年,蒙驁為秦將,伐韓,取成皋、滎陽,作置三川郡。二年,蒙驁攻趙,取三十七城。始皇三年,蒙驁攻韓,取十三城。五年,蒙驁攻魏,取二十城,作置東郡。始皇七年,蒙驁卒。驁子曰武,武子曰恬。恬嘗書獄典文學。始皇二十三年,蒙武為秦裨將軍,與王翦攻楚,大破之,殺項燕。二十四年,蒙武攻楚,虜楚王。蒙恬弟毅。

始皇二十六年,蒙恬因家世得為秦將,攻齊,大破之,拜為內史。秦已并天下,乃使蒙恬將三十萬眾北逐戎狄,收河南。筑長城,因地形,用制險塞,起臨洮,至遼東,延袤萬餘里。於是渡河,據陽山,逶蛇而北。暴師於外十餘年,居上郡。是時蒙恬威振匈奴。始皇甚尊寵蒙氏,信任賢之。而親近蒙毅,位至上卿,出則參乘,入則御前。恬任外事而毅常為內謀,名為忠信,故雖諸將相莫敢與之爭焉。

趙高者,諸趙疏遠屬也。趙高昆弟數人,皆生隱宮,其母被刑僇,世世卑賤。秦王聞高彊力,通於獄法,舉以為中車府令。高既私事公子胡亥,喻之決獄。高有大罪,秦王令蒙毅法治之。毅不敢阿法,當高罪死,除其宦籍。帝以高之敦於事也,赦之,復其官爵。

始皇欲游天下,道九原,直抵甘泉,乃使蒙恬通道,自九原抵甘泉,塹山堙谷,千八百里。道未就。

始皇三十七年冬,行出游會稽,并海上,北走瑯邪。道病,使蒙毅還禱山川,未反。

始皇至沙丘崩,祕之,群臣莫知。是時丞相李斯、公子胡亥、中車府令趙高常從。高雅得幸於胡亥,欲立之,又怨蒙毅法治之而不為己也。因有賊心,乃與丞相李斯、公子胡亥陰謀,立胡亥為太子。太子已立,遣使者以罪賜公子扶蘇、蒙恬死。扶蘇已死,蒙恬疑而復請之。使者以蒙恬屬吏,更置。胡亥以李斯舍人為護軍。使者還報,胡亥已聞扶蘇死,即欲釋蒙恬。趙高恐蒙氏復貴而用事,怨之。

毅還至,趙高因為胡亥忠計,欲以滅蒙氏,乃言曰,臣聞先帝欲舉賢立太子久矣,而毅諫曰不可。若知賢而俞弗立,則是不忠而惑主也。以臣愚意,不若誅之。胡亥聽而系蒙毅於代。前已囚蒙恬於陽周。喪至咸陽,已葬,太子立為二世皇帝,而趙高親近,日夜毀惡蒙氏,求其罪過,舉劾之。

子嬰進諫曰,臣聞故趙王遷殺其良臣李牧而用顏聚,燕王喜陰用荊軻之謀而倍秦之約,齊王建殺其故世忠臣而用后勝之議。此三君者,皆各以變古者失其國而殃及其身。今蒙氏,秦之大臣謀士也,而主欲一旦棄去之,臣竊以為不可。臣聞輕慮者不可以治國,獨智者不可以存君。誅殺忠臣而立無節行之人,是內使群臣不相信而外使鬬士之意離也,臣竊以為不可。

胡亥不聽。而遣御史曲宮乘傳之代,令蒙毅曰,先主欲立太子而卿難之。今丞相以卿為不忠,罪及其宗。朕不忍,乃賜卿死,亦甚幸矣。卿其圖之。毅對曰,以臣不能得先主之意,則臣少宦,順幸沒世。可謂知意矣。以臣不知太子之能,則太子獨從,周旋天下,去諸公子絕遠,臣無所疑矣。夫先主之舉用太子,數年之積也,臣乃何言之敢諫,何慮之敢謀。非敢飾辭以避死也,為羞累先主之名,願大夫為慮焉,使臣得死情實。且夫順成全者,道之所貴也,刑殺者,道之所卒也。昔者秦穆公殺三良而死,罪百里奚而非其罪也,故立號曰繆。昭襄王殺武安君白起。楚平王殺伍奢。吳王夫差殺伍子胥。此四君者,皆為大失,而天下非之,以其君為不明,以是籍於諸侯。故曰用道治者不殺無罪,而罰不加於無辜。唯大夫留心。使者知胡亥之意,不聽蒙毅之言,遂殺之。

二世又遣使者之陽周,令蒙恬曰,君之過多矣,而卿弟毅有大罪,法及內史。恬曰,自吾先人,及至子孫,積功信於秦三世矣。今臣將兵三十餘萬,身雖囚系,其勢足以倍畔,然自知必死而守義者,不敢辱先人之教,以不忘先主也。昔周成王初立,未離襁緥周公旦負王以朝,卒定天下。及成王有病甚殆,公旦自揃其爪以沈於河,曰,王未有識,是旦執事。有罪殃,旦受其不祥。乃書而藏之記府,可謂信矣。及王能治國,有賊臣言,周公旦欲為亂久矣,王若不備,必有大事。王乃大怒,周公旦走而奔於楚。成王觀於記府,得周公旦沈書,乃流涕曰,孰謂周公旦欲為亂乎。殺言之者而反周公旦。故周書曰必參而伍之。今恬之宗,世無二心,而事卒如此,是必孽臣逆亂,內陵之道也。夫成王失而復振則卒昌,桀殺關龍逢,紂殺王子比干而不悔,身死則國亡。臣故曰過可振而諫可覺也。察於參伍,上聖之法也。凡臣之言,非以求免於咎也,將以諫而死,願陛下為萬民思從道也。使者曰,臣受詔行法於將軍,不敢以將軍言聞於上也。蒙恬喟然太息曰,我何罪於天,無過而死乎。良久,徐曰,恬罪固當死矣。起臨洮屬之遼東,城塹萬餘里,此其中不能無絕地脈哉。此乃恬之罪也。乃吞藥自殺。

太史公曰,吾適北邊,自直道歸,行觀蒙恬所為秦筑長城亭障,塹山堙谷,通直道,固輕百姓力矣。夫秦之初滅諸侯,天下之心未定,痍傷者未瘳,而恬為名將,不以此時彊諫,振百姓之急,養老存孤,務修眾庶之和,而阿意興功,此其兄弟遇誅,不亦宜乎。何乃罪地脈哉。