\chapter{田叔列傳第四十四}

田叔者,趙陘城人也。其先,齊田氏苗裔也。叔喜劍,學黃老術於樂巨公所。叔為人刻廉自喜,喜游諸公。趙人舉之趙相趙午,午言之趙王張敖所,趙王以為郎中。數歲,切直廉平,趙王賢之,未及遷。

會陳豨反代,漢七年,高祖往誅之,過趙,趙王張敖自持案進食,禮恭甚,高祖箕踞罵之。是時趙相趙午等數十人皆怒,謂張王曰,王事上禮備矣,今遇王如是,臣等請為亂。趙王齧指出血,曰,先人失國,微陛下,臣等當蟲出。公等柰何言若是。毋復出口矣。於是貫高等曰,王長者,不倍德。卒私相與謀弒上。會事發覺,漢下詔捕趙王及群臣反者。於是趙午等皆自殺,唯貫高就系。是時漢下詔書,趙有敢隨王者罪三族。唯孟舒、田叔等十餘人赭衣自髡鉗,稱王家奴,隨趙王敖至長安。貫高事明白,趙王敖得出,廢為宣平侯,乃進言田叔等十餘人。上盡召見,與語,漢廷臣毋能出其右者,上說,盡拜為郡守、諸侯相。叔為漢中守十餘年,會高后崩,諸呂作亂,大臣誅之,立孝文帝。

孝文帝既立,召田叔問之曰,公知天下長者乎。對曰,臣何足以知之。上曰,公,長者也,宜知之。叔頓首曰,故雲中守孟舒,長者也。是時孟舒坐虜大入塞盜劫,雲中尤甚,免。上曰,先帝置孟舒雲中十餘年矣,虜曾一人,孟舒不能堅守,毋故士卒戰死者數百人。長者固殺人乎。公何以言孟舒為長者也。叔叩頭對曰,是乃孟舒所以為長者也。夫貫高等謀反,上下明詔,趙有敢隨張王,罪三族。然孟舒自髡鉗,隨張王敖之所在,欲以身死之,豈自知為雲中守哉。漢與楚相距,士卒罷敝。匈奴冒頓新服北夷,來為邊害,孟舒知士卒罷敝,不忍出言,士爭臨城死敵,如子為父,弟為兄,以故死者數百人。孟舒豈故驅戰之哉。是乃孟舒所以為長者也。於是上曰,賢哉孟舒。復召孟舒以為雲中守。

後數歲,叔坐法失官。梁孝王使人殺故吳相袁盎,景帝召田叔案梁,具得其事,還報。景帝曰,梁有之乎。叔對曰,死罪。有之。上曰,其事安在。田叔曰,上毋以梁事為也。上曰,何也。曰,今梁王不伏誅,是漢法不行也,如其伏法,而太后食不甘味,臥不安席,此憂在陛下也。景帝大賢之,以為魯相。

魯相初到,民自言相,訟王取其財物百餘人。田叔取其渠率二十人,各笞五十,餘各搏二十,怒之曰,王非若主邪。何自敢言若主。魯王聞之大慚,發中府錢,使相償之。相曰,王自奪之,使相償之,是王為惡而相為善也。相毋與償之。於是王乃盡償之。

魯王好獵,相常從入苑中,王輒休相就館舍,相出,常暴坐待王苑外。王數使人請相休,終不休,曰,我王暴露苑中,我獨何為就舍。魯王以故不大出游。

數年,叔以官卒,魯以百金祠,少子仁不受也,曰,不以百金傷先人名。

仁以壯健為衛將軍舍人,數從擊匈奴。衛將軍進言仁,仁為郎中。數歲,為二千石丞相長史,失官。其後使刺舉三河。上東巡,仁奏事有辭,上說,拜為京輔都尉。月餘,上遷拜為司直。數歲,坐太子事。時左相自將兵,令司直田仁主閉守城門,坐縱太子,下吏誅死。仁發兵,長陵令車千秋上變仁,仁族死。陘城今在中山國。

太史公曰,孔子稱曰居是國必聞其政,田叔之謂乎。義不忘賢,明主之美以救過。仁與余善,余故并論之。

褚先生曰,臣為郎時,聞之曰田仁故與任安相善。任安,滎陽人也。少孤貧困,為人將車之長安,留,求事為小吏,未有因緣也,因占著名數。武功,扶風西界小邑也,谷口蜀道近山。安以為武功小邑,無豪,易高也,安留,代人為求盜亭父。後為亭長。邑中人民俱出獵,任安常為人分麋鹿雉兔,部署老小當壯劇易處,眾人皆喜,曰,無傷也,任少卿分別平,有智略。明日復合會,會者數百人。任少卿曰,某子甲何為不來乎。諸人皆怪其見之疾也。其後除為三老,舉為親民,出為三百石長,治民。坐上行出游共帳不辦,斥免。

乃為衛將軍舍人,與田仁會,俱為舍人,居門下,同心相愛。此二人家貧,無錢用以事將軍家監,家監使養惡齧馬。兩人同床臥,仁竊言曰,不知人哉家監也。任安曰,將軍尚不知人,何乃家監也。衛將軍從此兩人過平陽主,主家令兩人與騎奴同席而食,此二子拔刀列斷席別坐。主家皆怪而惡之,莫敢呵。

其後有詔募擇衛將軍舍人以為郎,將軍取舍人中富給者,令具鞌馬絳衣玉具劍,欲入奏之。會賢大夫少府趙禹來過衛將軍,將軍呼所舉舍人以示趙禹。趙禹以次問之,十餘人無一人習事有智略者。趙禹曰,吾聞之,將門之下必有將類。傳曰不知其君視其所使,不知其子視其所友。今有詔舉將軍舍人者,欲以觀將軍而能得賢者文武之士也。今徒取盎人子上之,又無智略,如木偶人衣之綺繡耳,將柰之何。於是趙禹悉召衛將軍舍人百餘人,以次問之,得田仁、任安,曰,獨此兩人可耳,餘無可用者。衛將軍見此兩人貧,意不平。趙禹去,謂兩人曰,各自具樾象絳衣。兩人對曰,家貧無用具也。將軍怒曰,今兩君家自為貧,何為出此言。鞅鞅如有移德於我者,何也。將軍不得已,上籍以聞。有詔召見衛將軍舍人,此二人前見,詔問能略相推第也。田仁對曰,提桴鼓立軍門,使士大夫樂死戰鬬,仁不及任安。任安對曰,夫決嫌疑,定是非,辯治官,使百姓無怨心,安不及仁也。武帝大笑曰,善。使任安護北軍,使田仁護邊田穀於河上。此兩人立名天下。

其後用任安為益州刺史,以田仁為丞相長史。

田仁上書言,天下郡太守多為姦利,三河尤甚,臣請先刺舉三河。三河太守皆內倚中貴人,與三公有親屬,無所畏憚,宜先正三河以警天下姦吏。是時河南、河內太守皆御史大夫杜父兄子弟也,河東太守石丞相子孫也。是時石氏九人為二千石,方盛貴。田仁數上書言之。杜大夫及石氏使人謝,謂田少卿曰,吾非敢有語言也,願少卿無相誣汙也。仁已刺三河,三河太守皆下吏誅死。仁還奏事,武帝說,以仁為能不畏彊御,拜仁為丞相司直,威振天下。

其後逢太子有兵事,丞相自將兵,使司直主城門。司直以為太子骨肉之親,父子之閒不甚欲近,去之諸陵過。是時武帝在甘泉,使御史大夫暴君下責丞相何為縱太子,丞相對言使司直部守城門而開太子。上書以聞,請捕系司直。司直下吏,誅死。

是時任安為北軍使者護軍,太子立車北軍南門外,召任安,與節令發兵。安拜受節,入,閉門不出。武帝聞之,以為任安為詳邪,不傅事,何也。任安笞辱北軍錢官小吏,小吏上書言之,以為受太子節,言幸與我其鮮好者。書上聞,武帝曰,是老吏也,見兵事起,欲坐觀成敗,見勝者欲合從之,有兩心。安有當死之罪甚眾,吾常活之,今懷詐,有不忠之心。下安吏,誅死。夫月滿則虧,物盛則衰,天地之常也。知進而不知退,久乘富貴,禍積為祟。故范蠡之去越,辭不受官位,名傳後世,萬歲不忘,豈可及哉。後進者慎戒之。