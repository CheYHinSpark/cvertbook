\chapter{衛將軍驃騎列傳第五十一}

大將軍衛青者,平陽人也。其父鄭季,為吏,給事平陽侯家,與侯妾衛媼通,生青。青同母兄衛長子,而姊衛子夫自平陽公主家得幸天子,故冒姓為衛氏。字仲卿。長子更字長君。長君母號為衛媼。媼長女衛孺,次女少兒,次女即子夫。後子夫男弟步、廣皆冒衛氏。

青為侯家人,少時歸其父,其父使牧羊。先母之子皆奴畜之,不以為兄弟數。青嘗從入至甘泉居室,有一鉗徒相青曰,貴人也,官至封侯。青笑曰,人奴之生,得毋笞罵即足矣,安得封侯事乎。

青壯,為侯家騎,從平陽主。建元二年春,青姊子夫得入宮幸上。皇后,堂邑大長公主女也,無子,妒。大長公主聞衛子夫幸,有身,妒之,乃使人捕青。青時給事建章,未知名。大長公主執囚青,欲殺之。其友騎郎公孫敖與壯士往篡取之,以故得不死。上聞,乃召青為建章監,侍中,及同母昆弟貴,賞賜數日閒累千金。孺為太仆公孫賀妻。少兒故與陳掌通,上召貴掌。公孫敖由此益貴。子夫為夫人。青為大中大夫。

元光五年,青為車騎將軍,擊匈奴,出上谷,太仆公孫賀為輕車將軍,出雲中,大中大夫公孫敖為騎將軍,出代郡,衛尉李廣為驍騎將軍,出雁門,軍各萬騎。青至蘢城,斬首虜數百。騎將軍敖亡七千騎,衛尉李廣為虜所得,得脫歸,皆當斬,贖為庶人。賀亦無功。

元朔元年春,衛夫人有男,立為皇后。其秋,青為車騎將軍,出雁門,三萬騎擊匈奴,斬首虜數千人。明年,匈奴入殺遼西太守,虜略漁陽二千餘人,敗韓將軍軍。漢令將軍李息擊之,出代,令車騎將軍青出雲中以西至高闕。遂略河南地,至于隴西,捕首虜數千,畜數十萬,走白羊、樓煩王。遂以河南地為朔方郡。以三千八百戶封青為長平侯。青校尉蘇建有功,以千一百戶封建為平陵侯。使建筑朔方城。青校尉張次公有功,封為岸頭侯。天子曰,匈奴逆天理,亂人倫,暴長虐老,以盜竊為務,行詐諸蠻夷,造謀藉兵,數為邊害,故興師遣將,以征厥罪。詩不云乎,薄伐玁狁,至于太原,出車彭彭,城彼朔方。今車騎將軍青度西河至高闕,獲首虜二千三百級,車輜畜產畢收為鹵,已封為列侯,遂西定河南地,按榆谿舊塞,絕梓領,梁北河,討蒲泥,破符離,斬輕銳之卒,捕伏聽者三千七十一級,執訊獲丑,驅馬牛羊百有餘萬,全甲兵而還,益封青三千戶。其明年,匈奴入殺代郡太守友,入略鴈門千餘人。其明年,匈奴大入代、定襄、上郡,殺略漢數千人。

其明年,元朔之五年春,漢令車騎將軍青將三萬騎,出高闕,衛尉蘇建為游擊將軍,左內史李沮為彊弩將軍,太仆公孫賀為騎將軍,代相李蔡為輕車將軍,皆領屬車騎將軍,俱出朔方,大行李息、岸頭侯張次公為將軍,出右北平,咸擊匈奴。匈奴右賢王當衛青等兵,以為漢兵不能至此,飲醉。漢兵夜至,圍右賢王,右賢王驚,夜逃,獨與其愛妾一人壯騎數百馳,潰圍北去。漢輕騎校尉郭成等逐數百里,不及,得右賢裨王十餘人,眾男女萬五千餘人,畜數千百萬,於是引兵而還。至塞,天子使使者持大將軍印,即軍中拜車騎將軍青為大將軍,諸將皆以兵屬大將軍,大將軍立號而歸。天子曰,大將軍青躬率戎士,師大捷,獲匈奴王十有餘人,益封青六千戶。而封青子伉為宜春侯,青子不疑為陰安侯,青子登為發干侯。青固謝曰,臣幸得待罪行閒,賴陛下神靈,軍大捷,皆諸校尉力戰之功也。陛下幸已益封臣青。臣青子在繦緥中,未有勤勞,上幸列地封為三侯,非臣待罪行閒所以勸士力戰之意也。伉等三人何敢受封。天子曰,我非忘諸校尉功也,今固且圖之。乃詔御史曰,護軍都尉公孫敖三從大將軍擊匈奴,常護軍,傅校獲王,以千五百戶封敖為合騎侯。都尉韓說從大將軍出窳渾,至匈奴右賢王庭,為麾下搏戰獲王,以千三百戶封說為龍頟侯。騎將軍公孫賀從大將軍獲王,以千三百戶封賀為南窌侯。輕車將軍李蔡再從大將軍獲王,以千六百戶封蔡為樂安侯。校尉李朔,校尉趙不虞,校尉公孫戎奴,各三從大將軍獲王,以千三百戶封朔為涉軹侯,以千三百戶封不虞為隨成侯,以千三百戶封戎奴為從平侯。將軍李沮、李息及校尉豆如意有功,賜爵關內侯,食邑各三百戶。其秋,匈奴入代,殺都尉朱英。

其明年春,大將軍青出定襄,合騎侯敖為中將軍,太仆賀為左將軍,翕侯趙信為前將軍,衛尉蘇建為右將軍,郎中令李廣為後將軍,右內史李沮為彊弩將軍,咸屬大將軍,斬首數千級而還。月餘,悉復出定襄擊匈奴,斬首虜萬餘人。右將軍建、前將軍信并軍三千餘騎,獨逢單于兵,與戰一日餘,漢兵且盡。前將軍故胡人,降為翕侯,見急,匈奴誘之,遂將其餘騎可八百,奔降單于。右將軍蘇建盡亡其軍,獨以身得亡去,自歸大將軍。大將軍問其罪正閎、長史安、議郎周霸等,建當云何。霸曰,自大將軍出,未嘗斬裨將。今建棄軍,可斬以明將軍之威。閎、安曰,不然。兵法小敵之堅,大敵之禽也。今建以數千當單于數萬,力戰一日餘,士盡,不敢有二心,自歸。自歸而斬之,是示後無反意也。不當斬。大將軍曰,青幸得以肺腑待罪行閒,不患無威,而霸說我以明威,甚失臣意。且使臣職雖當斬將,以臣之尊寵而不敢自擅專誅於境外,而具歸天子,天子自裁之,於是以見為人臣不敢專權,不亦可乎。軍吏皆曰善。遂囚建詣行在所。入塞罷兵。

是歲也,大將軍姊子霍去病年十八,幸,為天子ヰ中。善騎射,再從大將軍,受詔與壯士,為剽姚校尉,與輕勇騎八百直棄大軍數百里赴利,斬捕首虜過當。於是天子曰,剽姚校尉去病斬首虜二千二十八級,及相國、當戶,斬單于大父行籍若侯產,生捕季父羅姑比,再冠軍,以千六百戶封去病為冠軍侯。上谷太守郝賢四從大將軍,捕斬首虜二千餘人,以千一百戶封賢為眾利侯。是歲,失兩將軍軍,亡翕侯,軍功不多,故大將軍不益封。右將軍建至,天子不誅,赦其罪,贖為庶人。

大將軍既還,賜千金。是時王夫人方幸於上,甯乘說大將軍曰,將軍所以功未甚多,身食萬戶,三子皆為侯者,徒以皇后故也。今王夫人幸而宗族未富貴,願將軍奉所賜千金為王夫人親壽。大將軍乃以五百金為壽。天子聞之,問大將軍,大將軍以實言,上乃拜甯乘為東海都尉。

張騫從大將軍,以嘗使大夏,留匈奴中久,導軍,知善水草處,軍得以無饑渴,因前使絕國功,封騫博望侯。

冠軍侯去病既侯三歲,元狩二年春,以冠軍侯去病為驃騎將軍,將萬騎出隴西,有功。天子曰,驃騎將軍率戎士踰烏盭,討遬濮,涉狐奴,歷五王國,輜重人眾懾慴者弗取,冀獲單于子。轉戰六日,過焉支山千有餘里,合短兵,殺折蘭王,斬盧胡王,誅全甲,執渾邪王子及相國、都尉,首虜八千餘級,收休屠祭天金人,益封去病二千戶。

其夏,驃騎將軍與合騎侯敖俱出北地,異道,博望侯張騫、郎中令李廣俱出右北平,異道,皆擊匈奴。郎中令將四千騎先至,博望侯將萬騎在後至。匈奴左賢王將數萬騎圍郎中令,郎中令與戰二日,死者過半,所殺亦過當。博望侯至,匈奴兵引去。博望侯坐行留,當斬,贖為庶人。而驃騎將軍出北地,已遂深入,與合騎侯失道,不相得,驃騎將軍踰居延至祁連山,捕首虜甚多。天子曰,驃騎將軍踰居延,遂過小月氏,攻祁連山,得酋涂王,以眾降者二千五百人,斬首虜三萬二百級,獲五王,五王母,單于閼氏、王子五十九人,相國、將軍、當戶、都尉六十三人,師大率減什三,益封去病五千戶。賜校尉從至小月氏爵左庶長。鷹擊司馬破奴再從驃騎將軍斬遬濮王,捕稽沮王,千騎將得王、王母各一人,王子以下四十一人,捕虜三千三百三十人,前行捕虜千四百人,以千五百戶封破奴為從驃侯。校尉句王高不識,從驃騎將軍捕呼于屠王王子以下十一人,捕虜千七百六十八人,以千一百戶封不識為宜冠侯。校尉仆多有功,封為煇渠侯。合騎侯敖坐行留不與驃騎會,當斬,贖為庶人。諸宿將所將士馬兵亦不如驃騎,驃騎所將常選,然亦敢深入,常與壯騎先其大軍,軍亦有天幸,未嘗困絕也。然而諸宿將常坐留落不遇。由此驃騎日以親貴,比大將軍。

其秋,單于怒渾邪王居西方數為漢所破,亡數萬人,以驃騎之兵也。單于怒,欲召誅渾邪王。渾邪王與休屠王等謀欲降漢,使人先要邊。是時大行李息將城河上,得渾邪王使,即馳傳以聞。天子聞之,於是恐其以詐降而襲邊,乃令驃騎將軍將兵往迎之。驃騎既渡河,與渾邪王眾相望。渾邪王裨將見漢軍而多欲不降者,頗遁去。驃騎乃馳入與渾邪王相見,斬其欲亡者八千人,遂獨遣渾邪王乘傳先詣行在所,盡將其眾渡河,降者數萬,號稱十萬。既至長安,天子所以賞賜者數十巨萬。封渾邪王萬戶,為漯陰侯。封其裨王呼毒尼為下摩侯,鷹庇為煇渠侯,禽為河綦侯,大當戶銅離為常樂侯。於是天子嘉驃騎之功曰,驃騎將軍去病率師攻匈奴西域王渾邪,王及厥眾萌咸相奔,率以軍糧接食,并將控弦萬有餘人,誅獟駻,獲首虜八千餘級,降異國之王三十二人,戰士不離傷,十萬之眾咸懷集服,仍與之勞,爰及河塞,庶幾無患,幸既永綏矣。以千七百戶益封驃騎將軍。減隴西、北地、上郡戍卒之半,以寬天下之繇。

居頃之,乃分徙降者邊五郡故塞外,而皆在河南,因其故俗,為屬國。其明年,匈奴入右北平、定襄,殺略漢千餘人。

其明年,天子與諸將議曰,翕侯趙信為單于畫計,常以為漢兵不能度幕輕留,今大發士卒,其勢必得所欲。是歲元狩四年也。

元狩四年春,上令大將軍青、驃騎將軍去病將各五萬騎,步兵轉者踵軍數十萬,而敢力戰深入之士皆屬驃騎。驃騎始為出定襄,當單于。捕虜言單于東,乃更令驃騎出代郡,令大將軍出定襄。郎中令為前將軍,太仆為左將軍,主爵趙食其為右將軍,平陽侯襄為後將軍,皆屬大將軍。兵即度幕,人馬凡五萬騎,與驃騎等咸擊匈奴單于。趙信為單于謀曰,漢兵既度幕,人馬罷,匈奴可坐收虜耳。乃悉遠北其輜重,皆以精兵待幕北。而適值大將軍軍出塞千餘里,見單于兵陳而待,於是大將軍令武剛車自環為營,而縱五千騎往當匈奴。匈奴亦縱可萬騎。會日且入,大風起,沙礫擊面,兩軍不相見,漢益縱左右翼繞單于。單于視漢兵多,而士馬尚彊,戰而匈奴不利,薄莫,單于遂乘六驘,壯騎可數百,直冒漢圍西北馳去。時已昏,漢匈奴相紛挐,殺傷大當。漢軍左校捕虜言單于未昏而去,漢軍因發輕騎夜追之,大將軍軍因隨其後。匈奴兵亦散走。遲明,行二百餘里,不得單于,頗捕斬首虜萬餘級,遂至窴顏山趙信城,得匈奴積粟食軍。軍留一日而還,悉燒其城餘粟以歸。

大將軍之與單于會也,而前將軍廣、右將軍食其軍別從東道,或失道,後擊單于。大將軍引還過幕南,乃得前將軍、右將軍。大將軍欲使使歸報,令長史簿責前將軍廣,廣自殺。右將軍至,下吏,贖為庶人。大將軍軍入塞,凡斬捕首虜萬九千級。

是時匈奴眾失單于十餘日,右谷蠡王聞之,自立為單于。單于後得其眾,右王乃去單于之號。

驃騎將軍亦將五萬騎,車重與大將軍軍等,而無裨將。悉以李敢等為大校,當裨將,出代、右北平千餘里,直左方兵,所斬捕功已多大將軍。軍既還,天子曰,驃騎將軍去病率師,躬將所獲葷粥之士,約輕齎,絕大幕,涉獲章渠,以誅比車耆,轉擊左大將,斬獲旗鼓,歷涉離侯。濟弓閭,獲屯頭王、韓王等三人,將軍、相國、當戶、都尉八十三人,封狼居胥山,禪於姑衍,登臨翰海。執鹵獲丑七萬有四百四十三級,師率減什三,取食於敵,閒行殊遠而糧不絕,以五千八百戶益封驃騎將軍。右北平太守路博德屬驃騎將軍,會與城,不失期,從至梼余山,斬首捕虜二千七百級,以千六百戶封博德為符離侯。北地都尉邢山從驃騎將軍獲王,以千二百戶封山為義陽侯。故歸義因淳王復陸支、樓專王伊即靬皆從驃騎將軍有功,以千三百戶封復陸支為壯侯,以千八百戶封伊即靬為眾利侯。從驃侯破奴、昌武侯安稽從驃騎有功,益封各三百戶。校尉敢得旗鼓,為關內侯,食邑二百戶。校尉自為爵大庶長。軍吏卒為官,賞賜甚多。而大將軍不得益封,軍吏卒皆無封侯者。

兩軍之出塞,塞閱官及私馬凡十四萬匹,而復入塞者不滿三萬匹。乃益置大司馬位,大將軍、驃騎將軍皆為大司馬。定令,令驃騎將軍秩祿與大將軍等。自是之後,大將軍青日退,而驃騎日益貴。舉大將軍故人門下多去事驃騎,輒得官爵,唯任安不肯。

驃騎將軍為人少言不泄,有氣敢任。天子嘗欲教之孫吳兵法,對曰,顧方略何如耳,不至學古兵法。天子為治第,令驃騎視之,對曰,匈奴未滅,無以家為也。由此上益重愛之。然少而侍中,貴,不省士。其從軍,天子為遣太官齎數十乘,既還,重車餘棄粱肉,而士有饑者。其在塞外,卒乏糧,或不能自振,而驃騎尚穿域蹋鞠。事多此類。大將軍為人仁善退讓,以和柔自媚於上,然天下未有稱也。

驃騎將軍自四年軍後三年,元狩六年而卒。天子悼之,發屬國玄甲軍,陳自長安至茂陵,為冢象祁連山。謚之,并武與廣地曰景桓侯。子嬗代侯。嬗少,字子侯,上愛之,幸其壯而將之。居六歲,元封元年,嬗卒,謚哀侯。無子,絕,國除。

自驃騎將軍死後,大將軍長子宜春侯伉坐法失侯。後五歲,伉弟二人,陰安侯不疑及發干侯登皆坐酎金失侯。失侯後二歲,冠軍侯國除。其後四年,大將軍青卒,謚為烈侯。子伉代為長平侯。

自大將軍圍單于之後,十四年而卒。竟不復擊匈奴者,以漢馬少,而方南誅兩越,東伐朝鮮,擊羌、西南夷,以故久不伐胡。

大將軍以其得尚平陽長公主故,長平侯伉代侯。六歲,坐法失侯。

左方兩大將軍及諸裨將名,

最大將軍青,凡七出擊匈奴,斬捕首虜五萬餘級。一與單于戰,收河南地,遂置朔方郡,再益封,凡萬一千八百戶。封三子為侯,侯千三百戶。并之,萬五千七百戶。其校尉裨將以從大將軍侯者九人。其裨將及校尉已為將者十四人。為裨將者曰李廣,自有傳。無傳者曰,

將軍公孫賀。賀,義渠人,其先胡種。賀父渾邪,景帝時為平曲侯,坐法失侯。賀,武帝為太子時舍人。武帝立八歲,以太仆為輕車將軍,軍馬邑。後四歲,以輕車將軍出雲中。後五歲,以騎將軍從大將軍有功,封為南窌侯。後一歲,以左將軍再從大將軍出定襄,無功。後四歲,以坐酎金失侯。後八歲,以浮沮將軍出五原二千餘里,無功。後八歲,以太仆為丞相,封葛繹侯。賀七為將軍,出擊匈奴無大功,而再侯,為丞相。坐子敬聲與陽石公主姦,為巫蠱,族滅,無後。

將軍李息,郁郅人。事景帝。至武帝立八歲,為材官將軍,軍馬邑,後六歲,為將軍,出代,後三歲,為將軍,從大將軍出朔方,皆無功。凡三為將軍,其後常為大行。

將軍公孫敖,義渠人。以郎事武帝。武帝立十二歲,為騎將軍,出代,亡卒七千人,當斬,贖為庶人。後五歲,以校尉從大將軍有功,封為合騎侯。後一歲,以中將軍從大將軍,再出定襄,無功。後二歲,以將軍出北地,後驃騎期,當斬,贖為庶人。後二歲,以校尉從大將軍,無功。後十四歲,以因杅將軍筑受降城。七歲,復以因杅將軍再出擊匈奴,至余吾,亡士卒多,下吏,當斬,詐死,亡居民閒五六歲。後發覺,復系。坐妻為巫蠱,族。凡四為將軍,出擊匈奴,一侯。

將軍李沮,雲中人。事景帝。武帝立十七歲,以左內史為彊弩將軍。後一歲,復為彊弩將軍。

將軍李蔡,成紀人也。事孝文帝、景帝、武帝。以輕車將軍從大將軍有功,封為樂安侯。已為丞相,坐法死。

將軍張次公,河東人。以校尉從衛將軍青有功,封為岸頭侯。其後太后崩,為將軍,軍北軍。後一歲,為將軍,從大將軍,再為將軍,坐法失侯。次公父隆,輕車武射也。以善射,景帝幸近之也。

將軍蘇建,杜陵人。以校尉從衛將軍青,有功,為平陵侯,以將軍筑朔方。後四歲,為游擊將軍,從大將軍出朔方。後一歲,以右將軍再從大將軍出定襄,亡翕侯,失軍,當斬,贖為庶人。其後為代郡太守,卒,冢在大猶鄉。

將軍趙信,以匈奴相國降,為翕侯。武帝立十七歲,為前將軍,與單于戰,敗,降匈奴。

將軍張騫,以使通大夏,還,為校尉。從大將軍有功,封為博望侯。後三歲,為將軍,出右北平,失期,當斬,贖為庶人。其後使通烏孫,為大行而卒,冢在漢中。

將軍趙食其,祋祤人也。武帝立二十二歲,以主爵為右將軍,從大將軍出定襄,迷失道,當斬,贖為庶人。

將軍曹襄,以平陽侯為後將軍,從大將軍出定襄。襄,曹參孫也。

將軍韓說,弓高侯庶孫也。以校尉從大將軍有功,為龍頟侯,坐酎金失侯。元鼎六年,以待詔為橫海將軍,擊東越有功,為按道侯。以太初三年為游擊將軍,屯於五原外列城。為光祿勳,掘蠱太子宮,衛太子殺之。

將軍郭昌,雲中人也。以校尉從大將軍。元封四年,以太中大夫為拔胡將軍,屯朔方。還擊昆明,毋功,奪印。

將軍荀彘,太原廣武人。以御見,侍中,為校尉,數從大將軍。以元封三年為左將軍擊朝鮮,毋功。以捕樓船將軍坐法死。

最驃騎將軍去病,凡六出擊匈奴,其四出以將軍,斬捕首虜十一萬餘級。及渾邪王以眾降數萬,遂開河西酒泉之地,西方益少胡寇。四益封,凡萬五千一百戶。其校吏有功為侯者凡六人,而後為將軍二人。

將軍路博德,平州人。以右北平太守從驃騎將軍有功,為符離侯。驃騎死後,博德以衛尉為伏波將軍,伐破南越,益封。其後坐法失侯。為彊弩都尉,屯居延,卒。

將軍趙破奴,故九原人。嘗亡入匈奴,已而歸漢,為驃騎將軍司馬。出北地時有功,封為從驃侯。坐酎金失侯。後一歲,為匈河將軍,攻胡至匈河水,無功。後二歲,擊虜樓蘭王,復封為浞野侯。後六歲,為浚稽將軍,將二萬騎擊匈奴左賢王,左賢王與戰,兵八萬騎圍破奴,破奴生為虜所得,遂沒其軍。居匈奴中十歲,復與其太子安國亡入漢。後坐巫蠱,族。

自衛氏興,大將軍青首封,其後枝屬為五侯。凡二十四歲而五侯盡奪,衛氏無為侯者。

太史公曰,蘇建語余曰,吾嘗責大將軍至尊重,而天下之賢大夫毋稱焉,願將軍觀古名將所招選擇賢者,勉之哉。大將軍謝曰,自魏其、武安之厚賓客,天子常切齒。彼親附士大夫,招賢絀不肖者,人主之柄也。人臣奉法遵職而已,何與招士。驃騎亦放此意,其為將如此。