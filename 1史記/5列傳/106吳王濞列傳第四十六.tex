\chapter{吳王濞列傳第四十六}

吳王濞者,高帝兄劉仲之子也。高帝已定天下七年,立劉仲為代王。而匈奴攻代,劉仲不能堅守,棄國亡,閒行走雒陽,自歸天子。天子為骨肉故,不忍致法,廢以為郃陽侯。高帝十一年秋,淮南王英布反,東并荊地,劫其國兵,西度淮,擊楚,高帝自將往誅之。劉仲子沛侯濞年二十,有氣力,以騎將從破布軍蘄西,會甀,布走。荊王劉賈為布所殺,無後。上患吳、會稽輕悍,無壯王以填之,諸子少,乃立濞於沛為吳王,王三郡五十三城。已拜受印,高帝召濞相之,謂曰,若狀有反相。心獨悔,業已拜,因拊其背,告曰,漢後五十年東南有亂者,豈若邪。然天下同姓為一家也,慎無反。濞頓首曰,不敢。

會孝惠、高后時,天下初定,郡國諸侯各務自拊循其民。吳有豫章郡銅山,濞則招致天下亡命者盜鑄錢,煮海水為鹽,以故無賦,國用富饒。

孝文時,吳太子入見,得侍皇太子飲博。吳太子師傅皆楚人,輕悍,又素驕,博,爭道,不恭,皇太子引博局提吳太子,殺之。於是遣其喪歸葬。至吳,吳王慍曰,天下同宗,死長安即葬長安,何必來葬為。復遣喪之長安葬。吳王由此稍失藩臣之禮,稱病不朝。京師知其以子故稱病不朝,驗問實不病,諸吳使來,輒系責治之。吳王恐,為謀滋甚。及後使人為秋請,上復責問吳使者,使者對曰,王實不病,漢系治使者數輩,以故遂稱病。且夫察見淵中魚,不祥。今王始詐病,及覺,見責急,愈益閉,恐上誅之,計乃無聊。唯上棄之而與更始。於是天子乃赦吳使者歸之,而賜吳王几杖,老,不朝。吳得釋其罪,謀亦益解。然其居國以銅鹽故,百姓無賦。卒踐更,輒與平賈。歲時存問茂材,賞賜閭里。佗郡國吏欲來捕亡人者,訟共禁弗予。如此者四十餘年,以故能使其眾。

晁錯為太子家令,得幸太子,數從容言吳過可削。數上書說孝文帝,文帝寬,不忍罰,以此吳日益橫。及孝景帝即位,錯為御史大夫,說上曰,昔高帝初定天下,昆弟少,諸子弱,大封同姓,故王孽子悼惠王王齊七十餘城,庶弟元王王楚四十餘城,兄子濞王吳五十餘城,封三庶孽,分天下半。今吳王前有太子之郄,詐稱病不朝,於古法當誅,文帝弗忍,因賜几杖。德至厚,當改過自新。乃益驕溢,即山鑄錢,煮海水為鹽,誘天下亡人,謀作亂。今削之亦反,不削之亦反。削之,其反亟,禍小,不削,反遲,禍大。三年冬,楚王朝,晁錯因言楚王戊往年為薄太后服,私姦服舍,請誅之。詔赦,罰削東海郡。因削吳之豫章郡、會稽郡。及前二年趙王有罪,削其河閒郡。膠西王卬以賣爵有姦,削其六縣。

漢廷臣方議削吳。吳王濞恐削地無已,因以此發謀,欲舉事。念諸侯無足與計謀者,聞膠西王勇,好氣,喜兵,諸齊皆憚畏,於是乃使中大夫應高誂膠西王。無文書,口報曰,吳王不肖,有宿夕之憂,不敢自外,使喻其驩心。王曰,何以教之。高曰,今者主上興於姦,飾於邪臣,好小善,聽讒賊,擅變更律令,侵奪諸侯之地,徵求滋多,誅罰良善,日以益甚。里語有之,舐糠及米。吳與膠西,知名諸侯也,一時見察,恐不得安肆矣。吳王身有內病,不能朝請二十餘年,嘗患見疑,無以自白,今脅肩累足,猶懼不見釋。竊聞大王以爵事有適,所聞諸侯削地,罪不至此,此恐不得削地而已。王曰,然,有之。子將柰何。高曰,同惡相助,同好相留,同情相成,同欲相趨,同利相死。今吳王自以為與大王同憂,願因時循理,棄軀以除患害於天下,億亦可乎。王瞿然駭曰,寡人何敢如是。今主上雖急,固有死耳,安得不戴。高曰,御史大夫晁錯,熒惑天子,侵奪諸侯,蔽忠塞賢,朝廷疾怨,諸侯皆有倍畔之意,人事極矣。彗星出,蝗蟲數起,此萬世一時,而愁勞聖人之所以起也。故吳王欲內以晁錯為討,外隨大王後車,彷徉天下,所鄉者降,所指者下,天下莫敢不服。大王誠幸而許之一言,則吳王率楚王略函谷關,守滎陽敖倉之粟,距漢兵。治次舍,須大王。大王有幸而臨之,則天下可并,兩主分割,不亦可乎。王曰,善。高歸報吳王,吳王猶恐其不與,乃身自為使,使於膠西,面結之。

膠西群臣或聞王謀,諫曰,承一帝,至樂也。今大王與吳西鄉,弟令事成,兩主分爭,患乃始結。諸侯之地不足為漢郡什二,而為畔逆以憂太后,非長策也。王弗聽。遂發使約齊、菑川、膠東、濟南、濟北,皆許諾,而曰城陽景王有義,攻諸呂,勿與,事定分之耳。

諸侯既新削罰,振恐,多怨晁錯。及削吳會稽、豫章郡書至,則吳王先起兵,膠西正月丙午誅漢吏二千石以下,膠東、菑川、濟南、楚、趙亦然,遂發兵西。齊王後悔,飲藥自殺,畔約。濟北王城壞未完,其郎中令劫守其王,不得發兵。膠西為渠率,膠東、菑川、濟南共攻圍臨菑。趙王遂亦反,陰使匈奴與連兵。

七國之發也,吳王悉其士卒,下令國中曰,寡人年六十二,身自將。少子年十四,亦為士卒先。諸年上與寡人比,下與少子等者,皆發。發二十餘萬人。南使閩越、東越,東越亦發兵從。

孝景帝三年正月甲子,初起兵於廣陵。西涉淮,因并楚兵。發使遺諸侯書曰,吳王劉濞敬問膠西王、膠東王、菑川王、濟南王、趙王、楚王、淮南王、衡山王、廬江王、故長沙王子,幸教寡人。以漢有賊臣,無功天下,侵奪諸侯地,使吏劾系訊治,以僇辱之為故,不以諸侯人君禮遇劉氏骨肉,絕先帝功臣,進任姦宄,詿亂天下,欲危社稷。陛下多病志失,不能省察。欲舉兵誅之,謹聞教。敝國雖狹,地方三千里,人雖少,精兵可具五十萬。寡人素事南越三十餘年,其王君皆不辭分其卒以隨寡人,又可得三十餘萬。寡人雖不肖,願以身從諸王。越直長沙者,因王子定長沙以北,西走蜀、漢中。告越、楚王、淮南三王,與寡人西面,齊諸王與趙王定河閒、河內,或入臨晉關,或與寡人會雒陽,燕王、趙王固與胡王有約,燕王北定代、雲中,摶胡眾入蕭關,走長安,匡正天子,以安高廟。願王勉之。楚元王子、淮南三王或不沐洗十餘年,怨入骨髓,欲一有所出之久矣,寡人未得諸王之意,未敢聽。今諸王茍能存亡繼絕,振弱伐暴,以安劉氏,社稷之所願也。敝國雖貧,寡人節衣食之用,積金錢,彊兵革,聚穀食,夜以繼日,三十餘年矣。凡為此,願諸王勉用之。能斬捕大將者,賜金五千斤,封萬戶,列將,三千斤,封五千戶,裨將,二千斤,封二千戶,二千石,千斤,封千戶,千石,五百斤,封五百戶,皆為列侯。其以軍若城邑降者,卒萬人,邑萬戶,如得大將,人戶五千,如得列將,人戶三千,如得裨將,人戶千,如得二千石,其小吏皆以差次受爵金。佗封賜皆倍軍法。其有故爵邑者,更益勿因。願諸王明以令士大夫,弗敢欺也。寡人金錢在天下者往往而有,非必取於吳,諸王日夜用之弗能盡。有當賜者告寡人,寡人且往遺之。敬以聞。

七國反書聞天子,天子乃遣太尉條侯周亞夫將三十六將軍,往擊吳楚,遣曲周侯酈寄擊趙,將軍欒布擊齊,大將軍竇嬰屯滎陽,監齊趙兵。

吳楚反書聞,兵未發,竇嬰未行,言故吳相袁盎。盎時家居,詔召入見。上方與晁錯調兵笇軍食,上問袁盎曰,君嘗為吳相,知吳臣田祿伯為人乎。今吳楚反,於公何如。對曰,不足憂也,今破矣。上曰,吳王即山鑄錢,煮海水為鹽,誘天下豪桀,白頭舉事。若此,其計不百全,豈發乎。何以言其無能為也。袁盎對曰,吳有銅鹽利則有之,安得豪桀而誘之。誠令吳得豪桀,亦且輔王為義,不反矣。吳所誘皆無賴子弟,亡命鑄錢姦人,故相率以反。晁錯曰,袁盎策之善。上問曰,計安出。盎對曰,願屏左右。上屏人,獨錯在。盎曰,臣所言,人臣不得知也。乃屏錯。錯趨避東廂,恨甚。上卒問盎,盎對曰,吳楚相遺書,曰高帝王子弟各有分地,今賊臣晁錯擅適過諸侯,削奪之地。故以反為名,西共誅晁錯,復故地而罷。方今計獨斬晁錯,發使赦吳楚七國,復其故削地,則兵可無血刃而俱罷。於是上嘿然良久,曰,顧誠何如,吾不愛一人以謝天下。盎曰,臣愚計無出此,願上孰計之。乃拜盎為太常,吳王弟子德侯為宗正。盎裝治行。後十餘日,上使中尉召錯,紿載行東市。錯衣朝衣斬東市。則遣袁盎奉宗廟,宗正輔親戚,使告吳如盎策。至吳,吳楚兵已攻梁壁矣。宗正以親故,先入見,諭吳王使拜受詔。吳王聞袁盎來,亦知其欲說己,笑而應曰,我已為東帝,尚何誰拜。不肯見盎而留之軍中,欲劫使將。盎不肯,使人圍守,且殺之,盎得夜出,步亡去,走梁軍,遂歸報。

條侯將乘六乘傳,會兵滎陽。至雒陽,見劇孟,喜曰,七國反,吾乘傳至此,不自意全。又以為諸侯已得劇孟,劇孟今無動。吾據滎陽,以東無足憂者。至淮陽,問父絳侯故客鄧都尉曰,策安出。客曰,吳兵銳甚,難與爭鋒。楚兵輕,不能久。方今為將軍計,莫若引兵東北壁昌邑,以梁委吳,吳必盡銳攻之。將軍深溝高壘,使輕兵絕淮泗口,塞吳馕道。彼吳梁相敝而糧食竭,乃以全彊制其罷極,破吳必矣。條侯曰,善。從其策,遂堅壁昌邑南,輕兵絕吳馕道。

吳王之初發也,吳臣田祿伯為大將軍。田祿伯曰,兵屯聚而西,無佗奇道,難以就功。臣願得五萬人,別循江淮而上,收淮南、長沙,入武關,與大王會,此亦一奇也。吳王太子諫曰,王以反為名,此兵難以藉人,藉人亦且反王,柰何。且擅兵而別,多佗利害,未可知也,徒自損耳。吳王即不許田祿伯。

吳少將桓將軍說王曰,吳多步兵,步兵利險,漢多車騎,車騎利平地。願大王所過城邑不下,直棄去,疾西據雒陽武庫,食敖倉粟,阻山河之險以令諸侯,雖毋入關,天下固已定矣。即大王徐行,留下城邑,漢軍車騎至,馳入梁楚之郊,事敗矣。吳王問諸老將,老將曰,此少年推鋒之計可耳,安知大慮乎。於是王不用桓將軍計。

吳王專并將其兵,未度淮,諸賓客皆得為將、校尉、候、司馬,獨周丘不得用。周丘者,下邳人,亡命吳,酤酒無行,吳王濞薄之,弗任。周丘上謁,說王曰,臣以無能,不得待罪行閒。臣非敢求有所將,願得王一漢節,必有以報王。王乃予之。周丘得節,夜馳入下邳。下邳時聞吳反,皆城守。至傳舍,召令。令入戶,使從者以罪斬令。遂召昆弟所善豪吏告曰,吳反兵且至,至,屠下邳不過食頃。今先下,家室必完,能者封侯矣。出乃相告,下邳皆下。周丘一夜得三萬人,使人報吳王,遂將其兵北略城邑。比至城陽,兵十餘萬,破城陽中尉軍。聞吳王敗走,自度無與共成功,即引兵歸下邳。未至,疽發背死。

二月中,吳王兵既破,敗走,於是天子制詔將軍曰,蓋聞為善者天報之以福,為非者天報之以殃。高皇帝親表功德,建立諸侯,幽王、悼惠王絕無後,孝文皇帝哀憐加惠,王幽王子遂、悼惠王子卬等,令奉其先王宗廟,為漢藩國,德配天地,明并日月。吳王濞倍德反義,誘受天下亡命罪人,亂天下幣,稱病不朝二十餘年,有司數請濞罪,孝文皇帝寬之,欲其改行為善。今乃與楚王戊、趙王遂、膠西王卬、濟南王辟光、菑川王賢、膠東王雄渠約從反,為逆無道,起兵以危宗廟,賊殺大臣及漢使者,迫劫萬民,夭殺無罪,燒殘民家,掘其丘冢,甚為暴虐。今卬等又重逆無道,燒宗廟,鹵御物,朕甚痛之。朕素服避正殿,將軍其勸士大夫擊反虜。擊反虜者,深入多殺為功,斬首捕虜比三百石以上者皆殺之,無有所置。敢有議詔及不如詔者,皆要斬。

初,吳王之度淮,與楚王遂西敗棘壁,乘勝前,銳甚。梁孝王恐,遣六將軍擊吳,又敗梁兩將,士卒皆還走梁。梁數使使報條侯求救,條侯不許。又使使惡條侯於上,上使人告條侯救梁,復守便宜不行。梁使韓安國及楚死事相弟張羽為將軍,乃得頗敗吳兵。吳兵欲西,梁城守堅,不敢西,即走條侯軍,會下邑。欲戰,條侯壁,不肯戰。吳糧絕,卒饑,數挑戰,遂夜奔條侯壁,驚東南。條侯使備西北,果從西北入。吳大敗,士卒多饑死,乃畔散。於是吳王乃與其麾下壯士數千人夜亡去,度江走丹徒,保東越。東越兵可萬餘人,乃使人收聚亡卒。漢使人以利啗東越,東越即紿吳王,吳王出勞軍,即使人鏦殺吳王,盛其頭,馳傳以聞。吳王子子華、子駒亡走閩越。吳王之棄其軍亡也,軍遂潰,往往稍降太尉、梁軍。楚王戊軍敗,自殺。

三王之圍齊臨菑也,三月不能下。漢兵至,膠西、膠東、菑川王各引兵歸。膠西王乃袒跣,席槁,飲水,謝太后。王太子德曰,漢兵遠,臣觀之已罷,可襲,願收大王餘兵擊之,擊之不勝,乃逃入海,未晚也。王曰,吾士卒皆已壞,不可發用。弗聽。漢將弓高侯穨當遺王書曰,奉詔誅不義,降者赦其罪,復故,不降者滅之。王何處,須以從事。王肉袒叩頭漢軍壁,謁曰,臣卬奉法不謹,驚駭百姓,乃苦將軍遠道至于窮國,敢請菹醢之罪。弓高侯執金鼓見之,曰,王苦軍事,願聞王發兵狀。王頓首膝行對曰,今者,晁錯天子用事臣,變更高皇帝法令,侵奪諸侯地。卬等以為不義,恐其敗亂天下,七國發兵,且以誅錯。今聞錯已誅,卬等謹以罷兵歸。將軍曰,王茍以錯不善,何不以聞。乃未有詔虎符,擅發兵擊義國。以此觀之,意非欲誅錯也。乃出詔書為王讀之。讀之訖,曰,王其自圖。王曰,如卬等死有餘罪。遂自殺。太后、太子皆死。膠東、菑川、濟南王皆死,國除,納于漢。酈將軍圍趙十月而下之,趙王自殺。濟北王以劫故,得不誅,徙王菑川。

初,吳王首反,并將楚兵,連齊趙。正月起兵,三月皆破,獨趙後下。復置元王少子平陸侯禮為楚王,續元王後。徙汝南王非王吳故地,為江都王。

太史公曰,吳王之王,由父省也。能薄賦斂,使其眾,以擅山海利。逆亂之萌,自其子興。爭技發難,卒亡其本,親越謀宗,竟以夷隕。晁錯為國遠慮,禍反近身。袁盎權說,初寵後辱。故古者諸侯地不過百里,山海不以封。毋親夷狄,以疏其屬,蓋謂吳邪。毋為權首,反受其咎,豈盎、錯邪。