\chapter{呂太后本紀第九}

呂太后者,高祖微時妃也,生孝惠帝、女魯元太后。及高祖為漢王,得定陶戚姬,愛幸,生趙隱王如意。孝惠為人仁弱,高祖以為不類我,常欲廢太子,立戚姬子如意,如意類我。戚姬幸,常從上之關東,日夜啼泣,欲立其子代太子。呂后年長,常留守,希見上,益疏。如意立為趙王後,幾代太子者數矣,賴大臣爭之,及留侯策,太子得毋廢。

呂后為人剛毅,佐高祖定天下,所誅大臣多呂后力。呂后兄二人,皆為將。長兄周呂侯死事,封其子呂臺為酈侯,子產為交侯,次兄呂釋之為建成侯。

高祖十二年四月甲辰,崩長樂宮,太子襲號為帝。是時高祖八子,長男肥,孝惠兄也,異母,肥為齊王,餘皆孝惠弟,戚姬子如意為趙王,薄夫人子恒為代王,諸姬子子恢為梁王,子友為淮陽王,子長為淮南王,子建為燕王。高祖弟交為楚王,兄子濞為吳王。非劉氏功臣番君吳芮子臣為長沙王。

呂后最怨戚夫人及其子趙王,乃令永巷囚戚夫人,而召趙王。使者三反,趙相建平侯周昌謂使者曰,高帝屬臣趙王,趙王年少。竊聞太后怨戚夫人,欲召趙王并誅之,臣不敢遣王。王且亦病,不能奉詔。呂后大怒,乃使人召趙相。趙相徵至長安,乃使人復召趙王。王來,未到。孝惠帝慈仁,知太后怒,自迎趙王霸上,與入宮,自挾與趙王起居飲食。太后欲殺之,不得閒。孝惠元年十二月,帝晨出射。趙王少,不能蚤起。太后聞其獨居,使人持酖飲之。犁明,孝惠還,趙王已死。於是乃徙淮陽王友為趙王。夏,詔賜酈侯父追謚為令武侯。太后遂斷戚夫人手足,去眼,煇耳,飲瘖藥,使居廁中,命曰人彘。居數日,乃召孝惠帝觀人彘。孝惠見,問,乃知其戚夫人,乃大哭,因病,歲餘不能起。使人請太后曰,此非人所為。臣為太后子,終不能治天下。孝惠以此日飲為淫樂,不聽政,故有病也。

二年,楚元王、齊悼惠王皆來朝。十月,孝惠與齊王燕飲太后前,孝惠以為齊王兄,置上坐,如家人之禮。太后怒,乃令酌兩卮酖,置前,令齊王起為壽。齊王起,孝惠亦起,取卮欲俱為壽。太后乃恐,自起泛孝惠卮。齊王怪之,因不敢飲,詳醉去。問,知其酖,齊王恐,自以為不得脫長安,憂。齊內史士說王曰,太后獨有孝惠與魯元公主。今王有七十餘城,而公主乃食數城。王誠以一郡上太后,為公主湯沐邑,太后必喜,王必無憂。於是齊王乃上城陽之郡,尊公主為王太后。呂后喜,許之。乃置酒齊邸,樂飲,罷,歸齊王。三年,方筑長安城,四年就半,五年六年城就。諸侯來會。十月朝賀。

七年秋八月戊寅,孝惠帝崩。發喪,太后哭,泣不下。留侯子張辟彊為侍中,年十五,謂丞相曰,太后獨有孝惠,今崩,哭不悲,君知其解乎。丞相曰,何解。辟彊曰,帝毋壯子,太后畏君等。君今請拜呂臺、呂產、呂祿為將,將兵居南北軍,及諸呂皆入宮,居中用事,如此則太后心安,君等幸得脫禍矣。丞相乃如辟彊計。太后說,其哭乃哀。呂氏權由此起。乃大赦天下。九月辛丑,葬。太子即位為帝,謁高廟。元年,號令一出太后。

太后稱制,議欲立諸呂為王,問右丞相王陵。王陵曰,高帝刑白馬盟曰非劉氏而王,天下共擊之。今王呂氏,非約也。太后不說。問左丞相陳平、絳侯周勃。勃等對曰,高帝定天下,王子弟,今太后稱制,王昆弟諸呂,無所不可。太后喜,罷朝。王陵讓陳平、絳侯曰,始與高帝啑血盟,諸君不在邪。今高帝崩,太后女主,欲王呂氏,諸君從欲阿意背約,何面目見高帝地下。陳平、絳侯曰,於今面折廷爭,臣不如君,夫全社稷,定劉氏之後,君亦不如臣。王陵無以應之。十一月,太后欲廢王陵,乃拜為帝太傅,奪之相權。王陵遂病免歸。乃以左丞相平為右丞相,以辟陽侯審食其為左丞相。左丞相不治事,令監宮中,如郎中令。食其故得幸太后,常用事,公卿皆因而決事。乃追尊酈侯父為悼武王,欲以王諸呂為漸。

四月,太后欲侯諸呂,乃先封高祖之功臣郎中令無擇為博城侯。魯元公主薨,賜謚為魯元太后。子偃為魯王。魯王父,宣平侯張敖也。封齊悼惠王子章為朱虛侯,以呂祿女妻之。齊丞相壽為平定侯。少府延為梧侯。乃封呂種為沛侯,呂平為扶柳侯,張買為南宮侯。

太后欲王呂氏,先立孝惠後宮子彊為淮陽王,子不疑為常山王,子山為襄城侯,子朝為軹侯,子武為壺關侯。太后風大臣,大臣請立酈侯呂臺為呂王,太后許之。建成康侯釋之卒,嗣子有罪,廢,立其弟呂祿為胡陵侯,續康侯後。二年,常山王薨,以其弟襄城侯山為常山王,更名義。十一月,呂王臺薨,謚為肅王,太子嘉代立為王。三年,無事。四年,封呂媭為臨光侯,呂他為俞侯,呂更始為贅其侯,呂忿為呂城侯,及諸侯丞相五人。

宣平侯女為孝惠皇后時,無子,詳為有身,取美人子名之,殺其母,立所名子為太子。孝惠崩,太子立為帝。帝壯,或聞其母死,非真皇后子,乃出言曰,后安能殺吾母而名我。我未壯,壯即為變。太后聞而患之,恐其為亂,乃幽之永卷中,言帝病甚,左右莫得見。太后曰,凡有天下治為萬民命者,蓋之如天,容之如地,上有歡心以安百姓,百姓欣然以事其上,歡欣交通而天下治。今皇帝病久不已,乃失惑惛亂,不能繼嗣奉宗廟祭祀,不可屬天下,其代之。群臣皆頓首言,皇太后為天下齊民計所以安宗廟社稷甚深,群臣頓首奉詔。帝廢位,太后幽殺之。五月丙辰,立常山王義為帝,更名曰弘。不稱元年者,以太后制天下事也。以軹侯朝為常山王。置太尉官,絳侯勃為太尉。五年八月,淮陽王薨,以弟壺關侯武為淮陽王。六年十月,太后曰呂王嘉居處驕恣,廢之,以肅王臺弟呂產為呂王。夏,赦天下。封齊悼惠王子興居為東牟侯。

七年正月,太后召趙王友。友以諸呂女為受后,弗愛,愛他姬,諸呂女妒,怒去,讒之於太后,誣以罪過,曰,呂氏安得王。太后百歲後,吾必擊之。太后怒,以故召趙王。趙王至,置邸不見,令衛圍守之,弗與食。其群臣或竊饋,輒捕論之,趙王餓,乃歌曰,諸呂用事兮劉氏危,迫脅王侯兮彊授我妃。我妃既妒兮誣我以惡,讒女亂國兮上曾不寤。我無忠臣兮何故棄國。自決中野兮蒼天舉直。于嗟不可悔兮寧蚤自財。為王而餓死兮誰者憐之。呂氏絕理兮託天報仇。丁丑,趙王幽死,以民禮葬之長安民冢次。

己丑,日食,晝晦。太后惡之,心不樂,乃謂左右曰,此為我也。

二月,徙梁王恢為趙王。呂王產徙為梁王,梁王不之國,為帝太傅。立皇子平昌侯太為呂王。更名梁曰呂,呂曰濟川。太后女弟呂媭有女為營陵侯劉澤妻,澤為大將軍。太后王諸呂,恐即崩後劉將軍為害,乃以劉澤為瑯邪王,以慰其心。

梁王恢之徙王趙,心懷不樂。太后以呂產女為趙王后。王后從官皆諸呂,擅權,微伺趙王,趙王不得自恣。王有所愛姬,王后使人酖殺之。王乃為歌詩四章,令樂人歌之。王悲,六月即自殺。太后聞之,以為王用婦人棄宗廟禮,廢其嗣。

宣平侯張敖卒,以子偃為魯王,敖賜謚為魯元王。

秋,太后使使告代王,欲徙王趙。代王謝,願守代邊。

太傅產、丞相平等言,武信侯呂祿上侯,位次第一,請立為趙王。太后許之,追尊祿父康侯為趙昭王。九月,燕靈王建薨,有美人子,太后使人殺之,無後,國除。八年十月,立呂肅王子東平侯呂通為燕王,封通弟呂莊為東平侯。

三月中,呂后祓,還過軹道,見物如蒼犬,據高后掖,忽弗復見。卜之,云趙王如意為祟。高后遂病掖傷。

高后為外孫魯元王偃年少,蚤失父母,孤弱,乃封張敖前姬兩子,侈為新都侯,壽為樂昌侯,以輔魯元王偃。及封中大謁者張釋為建陵侯,呂榮為祝茲侯。諸中宦者令丞皆為關內侯,食邑五百戶。

七月中,高后病甚,乃令趙王呂祿為上將軍,軍北軍,呂王產居南軍。呂太后誡產、祿曰,高帝已定天下,與大臣約,曰非劉氏王者,天下共擊之。今呂氏王,大臣弗平。我即崩,帝年少,大臣恐為變。必據兵衛宮,慎毋送喪,毋為人所制。辛巳,高后崩,遺詔賜諸侯王各千金,將相列侯郎吏皆以秩賜金。大赦天下。以呂王產為相國,以呂祿女為帝后。

高后已葬,以左丞相審食其為帝太傅。

朱虛侯劉章有氣力,東牟侯興居其弟也。皆齊哀王弟,居長安。當是時,諸呂用事擅權,欲為亂,畏高帝故大臣絳、灌等,未敢發。朱虛侯婦,呂祿女,陰知其謀。恐見誅,乃陰令人告其兄齊王,欲令發兵西,誅諸呂而立。朱虛侯欲從中與大臣為應。齊王欲發兵,其相弗聽。八月丙午,齊王欲使人誅相,相召平乃反,舉兵欲圍王,王因殺其相,遂發兵東,詐奪瑯邪王兵,并將之而西。語在齊王語中。

齊王乃遺諸侯王書曰,高帝平定天下,王諸子弟,悼惠王王齊。悼惠王薨,孝惠帝使留侯良立臣為齊王。孝惠崩,高后用事,春秋高,聽諸呂,擅廢帝更立,又比殺三趙王,滅梁、趙、燕以王諸呂,分齊為四。忠臣進諫,上惑亂弗聽。今高后崩,而帝春秋富,未能治天下,固恃大臣諸侯。而諸呂又擅自尊官,聚兵嚴威,劫列侯忠臣,矯制以令天下,宗廟所以危。寡人率兵入誅不當為王者。漢聞之,相國呂產等乃遣潁陰侯灌嬰將兵擊之。灌嬰至滎陽,乃謀曰,諸呂權兵關中,欲危劉氏而自立。今我破齊還報,此益呂氏之資也。乃留屯滎陽,使使諭齊王及諸侯,與連和,以待呂氏變,共誅之。齊王聞之,乃還兵西界待約。

呂祿、呂產欲發亂關中,內憚絳侯、朱虛等,外畏齊、楚兵,又恐灌嬰畔之,欲待灌嬰兵與齊合而發,猶豫未決。當是時,濟川王太、淮陽王武、常山王朝名為少帝弟,及魯元王呂后外孫,皆年少未之國,居長安。趙王祿、梁王產各將兵居南北軍,皆呂氏之人。列侯群臣莫自堅其命。

太尉絳侯勃不得入軍中主兵。曲周侯酈商老病,其子寄與呂祿善。絳侯乃與丞相陳平謀,使人劫酈商。令其子寄往紿說呂祿曰,高帝與呂后共定天下,劉氏所立九王,呂氏所立三王,皆大臣之議,事已布告諸侯,諸侯皆以為宜。今太后崩,帝少,而足下佩趙王印,不急之國守藩,乃為上將,將兵留此,為大臣諸侯所疑。足下何不歸印,以兵屬太尉。請梁王歸相國印,與大臣盟而之國,齊兵必罷,大臣得安,足下高枕而王千里,此萬世之利也。呂祿信然其計,欲歸將印,以兵屬太尉。使人報呂產及諸呂老人,或以為便,或曰不便,計猶豫未有所決。呂祿信酈寄,時與出游獵。過其姑呂媭,媭大怒,曰,若為將而棄軍,呂氏今無處矣。乃悉出珠玉寶器散堂下,曰,毋為他人守也

左丞相食其免。

八月庚申旦,平陽侯窋行御史大夫事,見相國產計事。郎中令賈壽使從齊來,因數產曰,王不蚤之國,今雖欲行,尚可得邪。具以灌嬰與齊楚合從,欲誅諸呂告產,乃趣產急入宮。平陽侯頗聞其語,乃馳告丞相、太尉。太尉欲入北軍,不得入。襄平侯通尚符節。乃令持節矯內太尉北軍。太尉復令酈寄與典客劉揭先說呂祿曰,帝使太尉守北軍,欲足下之國,急歸將印辭去,不然,禍且起。呂祿以為酈兄不欺己,遂解印屬典客,而以兵授太尉。太尉將之入軍門,行令軍中曰,為呂氏右袒,為劉氏左袒。軍中皆左袒為劉氏。太尉行至,將軍呂祿亦已解上將印去,太尉遂將北軍。

然尚有南軍。平陽侯聞之,以呂產謀告丞相平,丞相平乃召朱虛侯佐太尉。太尉令朱虛侯監軍門。令平陽侯告衛尉,毋入相國產殿門。呂產不知呂祿已去北軍,乃入未央宮,欲為亂,殿門弗得入,裴回往來。平陽侯恐弗勝,馳語太尉。太尉尚恐不勝諸呂,未敢訟言誅之,乃遣朱虛侯謂曰,急入宮衛帝。朱虛侯請卒,太尉予卒千餘人。入未央宮門,遂見產廷中。日餔時,遂擊產。產走,天風大起,以故其從官亂,莫敢鬬。逐產,殺之郎中府吏廁中。

朱虛侯已殺產,帝命謁者持節勞朱虛侯。朱虛侯欲奪節信,謁者不肯,朱虛侯則從與載,因節信馳走,斬長樂衛尉呂更始。還,馳入北軍,報太尉。太尉起,拜賀朱虛侯曰,所患獨呂產,今已誅,天下定矣。遂遣人分部悉捕諸呂男女,無少長皆斬之。辛酉,捕斬呂祿,而笞殺呂媭。使人誅燕王呂通,而廢魯王偃。壬戌,以帝太傅食其復為左丞相。戊辰,徙濟川王王梁,立趙幽王子遂為趙王。遣朱虛侯章以誅諸呂氏事告齊王,令罷兵。灌嬰兵亦罷滎陽而歸。

諸大臣相與陰謀曰,少帝及梁、淮陽、常山王,皆非真孝惠子也。呂后以計詐名他人子,殺其母,養後宮,令孝惠子之,立以為後,及諸王,以彊呂氏。今皆已夷滅諸呂,而置所立,即長用事,吾屬無類矣。不如視諸王最賢者立之。或言齊悼惠王高帝長子,今其適子為齊王,推本言之,高帝適長孫,可立也。大臣皆曰,呂氏以外家惡而幾危宗廟,亂功臣今齊王母家駟鈞,駟鈞,惡人也。即立齊王,則復為呂氏。欲立淮南王,以為少,母家又惡。乃曰,代王方今高帝見子,最長,仁孝寬厚。太后家薄氏謹良。且立長故順,以仁孝聞於天下,便。乃相與共陰使人召代王。代王使人辭謝。再反,然後乘六乘傳。後九月晦日己酉,至長安,舍代邸。大臣皆往謁,奉天子璽上代王,共尊立為天子。代王數讓,群臣固請,然後聽。

東牟侯興居曰,誅呂氏吾無功,請得除宮。乃與太仆汝陰侯滕公入宮,前謂少帝曰,足下非劉氏,不當立。乃顧麾左右執戟者掊兵罷去。有數人不肯去兵,宦者令張澤諭告,亦去兵。滕公乃召乘輿車載少帝出。少帝曰,欲將我安之乎。滕公曰出就舍。舍少府。乃奉天子法駕,迎代王於邸。報曰,宮謹除。代王即夕入未央宮。有謁者十人持戟衛端門,曰,天子在也,足下何為者而入。代王乃謂太尉。太尉往諭,謁者十人皆掊兵而去。代王遂入而聽政。夜,有司分部誅滅梁、淮陽、常山王及少帝於邸。代王立為天子。二十三年崩,謚為孝文皇帝。

太史公曰,孝惠皇帝、高后之時,黎民得離戰國之苦,君臣俱欲休息乎無為,故惠帝垂拱,高后女主稱制,政不出房戶,天下晏然。刑罰罕用,罪人是希。民務稼穡,衣食滋殖。