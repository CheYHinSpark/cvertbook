自古受命帝王,曷嘗不封禪。蓋有無其應而用事者矣,未有睹符瑞見而不臻乎泰山者也。雖受命而功不至,至梁父矣而德不洽,洽矣而日有不暇給,是以即事用希。傳曰,三年不為禮,禮必廢,三年不為樂,樂必壞。每世之隆,則封禪答焉,及衰而息。厥曠遠者千有餘載,近者數百載,故其儀闕然堙滅,其詳不可得而記聞云。

尚書曰,舜在璇璣玉衡,以齊七政。遂類于上帝,禋于六宗,望山川,遍群神。輯五瑞,擇吉月日,見四嶽諸牧,還瑞。歲二月,東巡狩,至于岱宗。岱宗,泰山也。柴,望秩于山川。遂覲東后。東后者,諸侯也。合時月正日,同律度量衡,修五禮,五玉三帛二生一死贄。五月,巡狩至南嶽。南嶽,衡山也。八月,巡狩至西嶽。西嶽,華山也。十一月,巡狩至北嶽。北岳,恒山也。皆如岱宗之禮。中嶽,嵩高也。五載一巡狩。

禹遵之。後十四世,至帝孔甲,淫德好神,神瀆,二龍去之。其後三世,湯伐桀,欲遷夏社,不可,作夏社。後八世,至帝太戊,有桑谷生於廷,一暮大拱,懼。伊陟曰,妖不勝德。太戊修德,桑谷死。伊陟贊巫咸,巫咸之興自此始。後十四世,帝武丁得傅說為相,殷復興焉,稱高宗。有雉登鼎耳雊,武丁懼。祖己曰,修德。武丁從之,位以永寧。後五世,帝武乙慢神而震死。後三世,帝紂淫亂,武王伐之。由此觀之,始未嘗不肅祗,後稍怠慢也。

周官曰,冬日至,祀天於南郊,迎長日之至,夏日至,祭地祗。皆用樂舞,而神乃可得而禮也。天子祭天下名山大川,五嶽視三公,四瀆視諸侯,諸侯祭其疆內名山大川。四瀆者,江、河、淮、濟也。天子曰明堂、辟雍,諸侯曰泮宮。

周公既相成王,郊祀后稷以配天,宗祀文王於明堂以配上帝。自禹興而修社祀,后稷稼穡,故有稷祠,郊社所從來尚矣。

自周克殷後十四世,世益衰,禮樂廢,諸侯恣行,而幽王為犬戎所敗,周東徙雒邑。秦襄公攻戎救周,始列為諸侯。秦襄公既侯,居西垂,自以為主少皞之神,作西畤,祠白帝,其牲用騮駒黃牛羝羊各一云。其後十六年,秦文公東獵汧渭之閒,卜居之而吉。文公夢黃蛇自天下屬地,其口止於鄜衍。文公問史敦,敦曰,此上帝之徵,君其祠之。於是作鄜畤,用三牲郊祭白帝焉。

自未作鄜畤也,而雍旁故有吳陽武畤,雍東有好畤,皆廢無祠。或曰,自古以雍州積高,神明之隩,故立畤郊上帝,諸神祠皆聚云。蓋黃帝時嘗用事,雖晚周亦郊焉。其語不經見,縉紳者不道。

作鄜畤後九年,文公獲若石雲,于陳倉北阪城祠之。其神或歲不至,或歲數來,來也常以夜,光輝若流星,從東南來集于祠城,則若雄雞,其聲殷雲,野雞夜雊。以一牢祠,命曰陳寶。

作鄜畤後七十八年,秦德公既立,卜居雍,後子孫飲馬於河,遂都雍。雍之諸祠自此興。用三百牢於鄜畤。作伏祠。磔狗邑四門,以御蠱菑。

德公立二年卒。其後四年,秦宣公作密畤於渭南,祭青帝。

其後十四年,秦繆公立,病臥五日不寤,寤,乃言夢見上帝,上帝命繆公平晉亂。史書而記藏之府。而後世皆曰秦繆公上天。

秦繆公即位九年,齊桓公既霸,會諸侯於葵丘,而欲封禪。管仲曰,古者封泰山禪梁父者七十二家,而夷吾所記者十有二焉。昔無懷氏封泰山,禪云云,虙羲封泰山,禪云云,神農封泰山,禪云云,炎帝封泰山,禪云云,黃帝封泰山,禪亭亭,顓頊封泰山,禪云云,帝嚳封泰山,禪云云,堯封泰山,禪云云,舜封泰山,禪云云,禹封泰山,禪會稽,湯封泰山,禪云云,周成王封泰山,禪社首,皆受命然後得封禪。桓公曰,寡人北伐山戎,過孤竹,西伐大夏,涉流沙,束馬懸車,上卑耳之山,南伐至召陵,登熊耳山以望江漢。兵車之會三,而乘車之會六,九合諸侯,一匡天下,諸侯莫違我。昔三代受命,亦何以異乎。於是管仲睹桓公不可窮以辭,因設之以事,曰,古之封禪,鄗上之黍,北里之禾,所以為盛,江淮之閒,一茅三脊,所以為藉也。東海致比目之魚,西海致比翼之鳥,然後物有不召而自至者十有五焉。今鳳皇麒麟不來,嘉穀不生,而蓬蒿藜莠茂,鴟梟數至,而欲封禪,毋乃不可乎。於是桓公乃止。是歲,秦繆公內晉君夷吾。其後三置晉國之君,平其亂。繆公立三十九年而卒。

其後百有餘年,而孔子論述六藝,傳略言易姓而王,封泰山禪乎梁父者七十餘王矣,其俎豆之禮不章,蓋難言之。或問禘之說,孔子曰,不知。知禘之說,其於天下也視其掌。詩云紂在位,文王受命,政不及泰山。武王克殷二年,天下未寧而崩。爰周德之洽維成王,成王之封禪則近之矣。及後陪臣執政,季氏旅於泰山,仲尼譏之。

是時萇弘以方事周靈王,諸侯莫朝周,周力少,萇弘乃明鬼神事,設射貍首。貍首者,諸侯之不來者。依物怪欲以致諸侯。諸侯不從,而晉人執殺萇弘。周人之言方怪者自萇弘。

其後百餘年,秦靈公作吳陽上畤,祭黃帝,作下畤,祭炎帝。

後四十八年,周太史儋見秦獻公曰,秦始與周合,合而離,五百歲當復合,合十七年而霸王出焉。櫟陽雨金,秦獻公自以為得金瑞,故作畦畤櫟陽而祀白帝。

其後百二十歲而秦滅周,周之九鼎入于秦。或曰宋太丘社亡,而鼎沒于泗水彭城下。

其後百一十五年而秦并天下。

秦始皇既并天下而帝,或曰,黃帝得土德,黃龍地螾見。夏得木德,青龍止於郊,草木暢茂。殷得金德,銀自山溢。周得火德,有赤烏之符。今秦變周,水德之時。昔秦文公出獵,獲黑龍,此其水德之瑞。於是秦更命河曰德水,以冬十月為年首,色上黑,度以六為名,音上大呂,事統上法。

即帝位三年,東巡郡縣,祠騶嶧山,頌秦功業。於是徵從齊魯之儒生博士七十人,至乎泰山下。諸儒生或議曰,古者封禪為蒲車,惡傷山之土石草木,埽地而祭,席用葅稭,言其易遵也。始皇聞此議各乖異,難施用,由此絀儒生。而遂除車道,上自泰山陽至巔,立石頌秦始皇帝德,明其得封也。從陰道下,禪於梁父。其禮頗采太祝之祀雍上帝所用,而封藏皆祕之,世不得而記也。

始皇之上泰山,中阪遇暴風雨,休於大樹下。諸儒生既絀,不得與用於封事之禮,聞始皇遇風雨,則譏之。

於是始皇遂東遊海上,行禮祠名山大川及八神,求僊人羨門之屬。八神將自古而有之,或曰太公以來作之。齊所以為齊,以天齊也。其祀絕莫知起時。八神,一曰天主,祠天齊。天齊淵水,居臨菑南郊山下者。二曰地主,祠泰山梁父。蓋天好陰,祠之必於高山之下,小山之上,命曰畤,地貴陽,祭之必於澤中圜丘云。三曰兵主,祠蚩尤。蚩尤在東平陸監鄉,齊之西境也。四曰陰主,祠三山。五曰陽主,祠之罘。六曰月主,祠之萊山。皆在齊北,并勃海。七曰日主,祠成山。成山斗入海,最居齊東北隅,以迎日出云。八曰四時主,祠瑯邪。瑯邪在齊東方,蓋歲之所始。皆各用一牢具祠,而巫祝所損益,珪幣雜異焉。

自齊威、宣之時,騶子之徒論著終始五德之運,及秦帝而齊人奏之,故始皇采用之。而宋毋忌、正伯僑、充尚、羨門高最後皆燕人,為方僊道,形解銷化,依於鬼神之事。騶衍以陰陽主運顯於諸侯,而燕齊海上之方士傳其術不能通,然則怪迂阿諛茍合之徒自此興,不可勝數也。

自威、宣、燕昭使人入海求蓬萊、方丈、瀛洲。此三神山者,其傅在勃海中,去人不遠,患且至,則船風引而去。蓋嘗有至者,諸僊人及不死之藥皆在焉。其物禽獸盡白,而黃金銀為宮闕。未至,望之如雲,及到,三神山反居水下。臨之,風輒引去,終莫能至云。世主莫不甘心焉。及至秦始皇并天下,至海上,則方士言之不可勝數。始皇自以為至海上而恐不及矣,使人乃齎童男女入海求之。船交海中,皆以風為解,曰未能至,望見之焉。其明年,始皇復游海上,至瑯邪,過恒山,從上黨歸。後三年,游碣石,考入海方士,從上郡歸。後五年,始皇南至湘山,遂登會稽,并海上,冀遇海中三神山之奇藥。不得,還至沙丘崩。

二世元年,東巡碣石,并海南,歷泰山,至會稽,皆禮祠之,而刻勒始皇所立石書旁,以章始皇之功德。其秋,諸侯畔秦。三年而二世弒死。

始皇封禪之後十二歲,秦亡。諸儒生疾秦焚詩書,誅僇文學,百姓怨其法,天下畔之,皆訛曰,始皇上泰山,為暴風雨所擊,不得封禪。此豈所謂無其德而用事者邪。

昔三代之居皆在河洛之閒,故嵩高為中嶽,而四嶽各如其方,四瀆咸在山東。至秦稱帝,都咸陽,則五嶽、四瀆皆并在東方。自五帝以至秦,軼興軼衰,名山大川或在諸侯,或在天子,其禮損益世殊,不可勝記。及秦并天下,令祠官所常奉天地名山大川鬼神可得而序也。

於是自殽以東,名山五,大川祠二。曰太室。太室,嵩高也。恒山,泰山,會稽,湘山。水曰濟,曰淮。春以脯酒為歲祠,因泮凍,秋涸凍,冬塞禱祠。其牲用牛犢各一,牢具珪幣各異。

自華以西,名山七,名川四。曰華山,薄山。薄山者,衰山也。岳山,岐山,吳岳,鴻冢,瀆山。瀆山,蜀之汶山。水曰河,祠臨晉,沔,祠漢中,湫淵,祠朝那,江水,祠蜀。亦春秋泮涸禱塞,如東方名山川,而牲牛犢牢具珪幣各異。而四大冢鴻、岐、吳、岳,皆有嘗禾。

陳寶節來祠。其河加有嘗醪。此皆在雍州之域,近天子之都,故加車一乘,騮駒四。

霸、產、長水、灃、澇、涇、渭皆非大川,以近咸陽,盡得比山川祠,而無諸加。

汧、洛二淵,鳴澤、蒲山、嶽胥山之屬,為小山川,亦皆歲禱塞泮涸祠,禮不必同。

而雍有日、月、參、辰、南北斗、熒惑、太白、歲星、填星、[辰星、二十八宿、風伯、雨師、四海、九臣、十四臣、諸布、諸嚴、諸逑之屬,百有餘廟。西亦有數十祠。於湖有周天子祠。於下邽有天神。灃、滈有昭明、天子辟池。於杜、亳有三社主之祠、壽星祠,而雍菅廟亦有杜主。杜主,故周之右將軍,其在秦中,最小表之神者。各以歲時奉祠。

唯雍四畤上帝為尊,其光景動人民唯陳寶。故雍四畤,春以為歲禱,因泮凍,秋涸凍,冬塞祠,五月嘗駒,及四仲之月祠,若陳寶節來一祠。春夏用騂,秋冬用駵。畤駒四匹,木禺龍欒車一駟,木禺車馬一駟,各如其帝色。黃犢羔各四,珪幣各有數,皆生瘞埋,無俎豆之具。三年一郊。秦以冬十月為歲首,故常以十月上宿郊見,通權火,拜於咸陽之旁,而衣上白,其用如經祠云。西畤、畦畤,祠如其故,上不親往。

諸此祠皆太祝常主,以歲時奉祠之。至如他名山川諸鬼及八神之屬,上過則祠,去則已。郡縣遠方神祠者,民各自奉祠,不領於天子之祝官。祝官有祕祝,即有菑祥,輒祝祠移過於下。

漢興,高祖之微時,嘗殺大蛇。有物曰,蛇,白帝子也,而殺者赤帝子。高祖初起,禱豐枌榆社。徇沛,為沛公,則祠蚩尤,釁鼓旗。遂以十月至灞上,與諸侯平咸陽,立為漢王。因以十月為年首,而色上赤。

二年,東擊項籍而還入關,問,故秦時上帝祠何帝也。對曰,四帝,有白、青、黃、赤帝之祠。高祖曰,吾聞天有五帝,而有四,何也。莫知其說。於是高祖曰,吾知之矣,乃待我而具五也。乃立黑帝祠,命曰北畤。有司進祠,上不親往。悉召故秦祝官,復置太祝、太宰,如其故儀禮。因令縣為公社。下詔曰,吾甚重祠而敬祭。今上帝之祭及山川諸神當祠者,各以其時禮祠之如故。

後四歲,天下已定,詔御史,令豐謹治枌榆社,常以四時春以羊彘祠之。令祝官立蚩尤之祠於長安。長安置祠祝官、女巫。其梁巫,祠天、地、天社、天水、房中、堂上之屬,晉巫,祠五帝、東君、雲中君、司命、巫社、巫祠、族人、先炊之屬,秦巫,祠社主、巫保、族纍之屬,荊巫,祠堂下、巫先、司命、施糜之屬,九天巫,祠九天,皆以歲時祠宮中。其河巫祠河於臨晉,而南山巫祠南山秦中。秦中者,二世皇帝。各有時日。

其後二歲,或曰周興而邑邰,立后稷之祠,至今血食天下。於是高祖制詔御史,其令郡國縣立靈星祠,常以歲時祠以牛。

高祖十年春,有司請令縣常以春二月及臘祠社稷以羊豕,民里社各自財以祠。制曰,可。

其後十八年,孝文帝即位。即位十三年,下詔曰,今祕祝移過于下,朕甚不取。自今除之。

始名山大川在諸侯,諸侯祝各自奉祠,天子官不領。及齊、淮南國廢,令太祝盡以歲時致禮如故。

是歲,制曰,朕即位十三年于今,賴宗廟之靈,社稷之福,方內艾安,民人靡疾。閒者比年登,朕之不德,何以饗此。皆上帝諸神之賜也。蓋聞古者饗其德必報其功,欲有增諸神祠。有司議增雍五畤路車各一乘,駕被具,西畤畦畤禺車各一乘,禺馬四匹,駕被具,其河、湫、漢水加玉各二,及諸祠,各增廣壇場,珪幣俎豆以差加之。而祝祕者歸福於朕,百姓不與焉。自今祝致敬,毋有所祈。

魯人公孫臣上書曰,始秦得水德,今漢受之,推終始傳,則漢當土德,土德之應黃龍見。宜改正朔,易服色,色上黃。是時丞相張蒼好律歷,以為漢乃水德之始,故河決金隄,其符也。年始冬十月,色外黑內赤,與德相應。如公孫臣言,非也。罷之。後三歲,黃龍見成紀。文帝乃召公孫臣,拜為博士,與諸生草改歷服色事。其夏,下詔曰,異物之神見于成紀,無害於民,歲以有年。朕祈郊上帝諸神,禮官議,無諱以勞朕。有司皆曰古者天子夏親郊,祀上帝於郊,故曰郊。於是夏四月,文帝始郊見雍五畤祠,衣皆上赤。

其明年,趙人新垣平以望氣見上,言長安東北有神氣,成五采,若人冠絻焉。或曰東北神明之舍,西方神明之墓也。天瑞下,宜立祠上帝,以合符應。於是作渭陽五帝廟,同宇,帝一殿,面各五門,各如其帝色。祠所用及儀亦如雍五畤。

夏四月,文帝親拜霸渭之會,以郊見渭陽五帝。五帝廟南臨渭,北穿蒲池溝水,權火舉而祠,若光煇然屬天焉。於是貴平上大夫,賜累千金。而使博士諸生刺六經中作王制,謀議巡狩封禪事。

文帝出長門,若見五人於道北,遂因其直北立五帝壇,祠以五牢具。

其明年,新垣平使人持玉杯,上書闕下獻之。平言上曰,闕下有寶玉氣來者。已視之,果有獻玉杯者,刻曰人主延壽。平又言臣候日再中。居頃之,日卻復中。於是始更以十七年為元年,令天下大酺。

平言曰,周鼎亡在泗水中,今河溢通泗,臣望東北汾陰直有金寶氣,意周鼎其出乎。兆見不迎則不至。於是上使使治廟汾陰南,臨河,欲祠出周鼎。

人有上書告新垣平所言氣神事皆詐也。下平吏治,誅夷新垣平。自是之後,文帝怠於改正朔服色神明之事,而渭陽、長門五帝使祠官領,以時致禮,不往焉。

明年,匈奴數入邊,興兵守御。後歲少不登。

數年而孝景即位。十六年,祠官各以歲時祠如故,無有所興,至今天子。

今天子初即位,尤敬鬼神之祀。

元年,漢興已六十餘歲矣,天下艾安,搢紳之屬皆望天子封禪改正度也,而上鄉儒術,招賢良,趙綰、王臧等以文學為公卿,欲議古立明堂城南,以朝諸侯。草巡狩封禪改歷服色事未就。會竇太后治黃老言,不好儒術,使人微伺得趙綰等姦利事,召案綰、臧,綰、臧自殺,諸所興為皆廢。

後六年,竇太后崩。其明年,徵文學之士公孫弘等。

明年,今上初至雍,郊見五畤。後常三歲一郊。是時上求神君,捨之上林中蹄氏觀。神君者,長陵女子,以子死,見神於先後宛若。宛若祠之其室,民多往祠。平原君往祠,其後子孫以尊顯。及今上即位,則厚禮置祠之內中。聞其言,不見其人云。

是時李少君亦以祠灶、穀道、卻老方見上,上尊之。少君者,故深澤侯舍人,主方。匿其年及其生長,常自謂七十,能使物,卻老。其游以方遍諸侯。無妻子。人聞其能使物及不死,更饋遺之,常餘金錢衣食。人皆以為不治生業而饒給,又不知其何所人,愈信,爭事之。少君資好方,善為巧發奇中。嘗從武安侯飲,坐中有九十餘老人,少君乃言與其大父游射處,老人為兒時從其大父,識其處,一坐盡驚。少君見上,上有故銅器,問少君。少君曰,此器齊桓公十年陳於柏寢。已而案其刻,果齊桓公器。一宮盡駭,以為少君神,數百歲人也。

少君言上曰,祠灶則致物,致物而丹沙可化為黃金,黃金成以為飲食器則益壽,益壽而海中蓬萊僊者乃可見,見之以封禪則不死,黃帝是也。臣嘗游海上,見安期生,安期生食巨棗,大如瓜。安期生僊者,通蓬萊中,合則見人,不合則隱。於是天子始親祠灶,遣方士入海求蓬萊安期生之屬,而事化丹沙諸藥齊為黃金矣。

居久之,李少君病死。天子以為化去不死,而使黃錘史寬舒受其方。求蓬萊安期生莫能得,而海上燕齊怪迂之方士多更來言神事矣。

亳人謬忌奏祠太一方,曰,天神貴者太一,太一佐曰五帝。古者天子以春秋祭太一東南郊,用太牢,七日,為壇開八通之鬼道。於是天子令太祝立其祠長安東南郊,常奉祠如忌方。其後人有上書,言古者天子三年壹用太牢祠神三一,天一、地一、太一。天子許之,令太祝領祠之於忌太一壇上,如其方。後人復有上書,言古者天子常以春解祠,祠黃帝用一梟破鏡,冥羊用羊祠,馬行用一青牡馬,太一、澤山君地長用牛,武夷君用乾魚,陰陽使者以一牛。令祠官領之如其方,而祠於忌太一壇旁。

其後,天子苑有白鹿,以其皮為幣,以發瑞應,造白金焉。

其明年,郊雍,獲一角獸,若麃然。有司曰,陛下肅祗郊祀,上帝報享,錫一角獸,蓋麟云。於是以薦五畤,畤加一牛以燎。錫諸侯白金,風符應合于天也。

於是濟北王以為天子且封禪,乃上書獻太山及其旁邑,天子以他縣償之。常山王有罪,遷,天子封其弟於真定,以續先王祀,而以常山為郡,然後五岳皆在天子之郡。

其明年,齊人少翁以鬼神方見上。上有所幸王夫人,夫人卒,少翁以方蓋夜致王夫人及灶鬼之貌云,天子自帷中望見焉。於是乃拜少翁為文成將軍,賞賜甚多,以客禮禮之。文成言曰,上即欲與神通,宮室被服非象神,神物不至。乃作畫雲氣車,及各以勝日駕車辟惡鬼。又作甘泉宮,中為臺室,畫天、地、太一諸鬼神,而置祭具以致天神。居歲餘,其方益衰,神不至。乃為帛書以飯牛,詳不知,言曰此牛腹中有奇。殺視得書,書言甚怪。天子識其手書,問其人,果是偽書,於是誅文成將軍,隱之。

其後則又作柏梁、銅柱、承露僊人掌之屬矣。

文成死明年,天子病鼎湖甚,巫醫無所不致,不愈。游水發根言上郡有巫,病而鬼神下之。上召置祠之甘泉。及病,使人問神君。神君言曰,天子無憂病。病少愈,彊與我會甘泉。於是病愈,遂起,幸甘泉,病良已。大赦,置壽宮神君。壽宮神君最貴者太一,其佐曰大禁、司命之屬,皆從之。非可得見,聞其言,言與人音等。時去時來,來則風肅然。居室帷中。時晝言,然常以夜。天子祓,然後入。因巫為主人,關飲食。所以言,行下。又置壽宮、北宮,張羽旗,設供具,以禮神君。神君所言,上使人受書其言,命之曰畫法。其所語,世俗之所知也,無絕殊者,而天子心獨喜。其事祕,世莫知也。

其後三年,有司言元宜以天瑞命,不宜以一二數。一元曰建,二元以長星曰光,三元以郊得一角獸曰狩云。

其明年冬,天子郊雍,議曰,今上帝朕親郊,而後土無祀,則禮不答也。有司與太史公、祠官寬舒議,天地牲角繭栗。今陛下親祠后土,后土宜於澤中圜丘為五壇,壇一黃犢太牢具,已祠盡瘞,而從祠衣上黃。於是天子遂東,始立后土祠汾陰脽丘,如寬舒等議。上親望拜,如上帝禮。禮畢,天子遂至滎陽而還。過雒陽,下詔曰,三代邈絕,遠矣難存。其以三十里地封周後為周子南君,以奉其先祀焉。是歲,天子始巡郡縣,侵尋於泰山矣。

其春,樂成侯上書言欒大。欒大,膠東宮人,故嘗與文成將軍同師,已而為膠東王尚方。而樂成侯姊為康王后,無子。康王死,他姬子立為王。而康后有淫行,與王不相中,相危以法。康后聞文成已死,而欲自媚於上,乃遣欒大因樂成侯求見言方。天子既誅文成,後悔其蚤死,惜其方不盡,及見欒大,大說。大為人長美,言多方略,而敢為大言處之不疑。大言曰,臣常往來海中,見安期、羨門之屬。顧以臣為賤,不信臣。又以為康王諸侯耳,不足與方。臣數言康王,康王又不用臣。臣之師曰,黃金可成,而河決可塞,不死之藥可得,僊人可致也。然臣恐效文成,則方士皆奄口,惡敢言方哉。上曰,文成食馬肝死耳。子誠能修其方,我何愛乎。大曰,臣師非有求人,人者求之。陛下必欲致之,則貴其使者,令有親屬,以客禮待之,勿卑,使各佩其信印,乃可使通言於神人。神人尚肯邪不邪。致尊其使,然後可致也。於是上使驗小方,鬬棋,棋自相觸擊。

是時上方憂河決,而黃金不就,乃拜大為五利將軍。居月餘,得四印,佩天士將軍、地士將軍、大通將軍印。制詔御史,昔禹疏九江,決四瀆。閒者河溢皋陸,隄繇不息。朕臨天下二十有八年,天若遺朕士而大通焉。乾稱蜚龍,鴻漸于般,朕意庶幾與焉。其以二千戶封地士將軍大為樂通侯。賜列侯甲第,僮千人。乘轝斥車馬帷幄器物以充其家。又以衛長公主妻之,齎金萬斤,更命其邑曰當利公主。天子親如五利之第。使者存問供給,相屬於道。自大主將相以下,皆置酒其家,獻遺之。於是天子又刻玉印曰天道將軍,使使衣羽衣,夜立白茅上,五利將軍亦衣羽衣,夜立白茅上受印,以示不臣也。而佩天道者,且為天子道天神也。於是五利常夜祠其家,欲以下神。神未至而百鬼集矣,然頗能使之。其後裝治行,東入海,求其師云。大見數月,佩六印,貴震天下,而海上燕齊之閒,莫不搤捥而自言有禁方,能神僊矣。

其夏六月中,汾陰巫錦為民祠魏脽后土營旁,見地如鉤狀,掊視得鼎。鼎大異於眾鼎,文鏤無款識,怪之,言吏。吏告河東太守勝,勝以聞。天子使使驗問巫得鼎無姦詐,乃以禮祠,迎鼎至甘泉,從行,上薦之。至中山,曣溫,有黃云蓋焉。有麃過,上自射之,因以祭云。至長安,公卿大夫皆議請尊寶鼎。天子曰,閒者河溢,歲數不登,故巡祭后土,祈為百姓育穀。今歲豐廡未報,鼎曷為出哉。有司皆曰,聞昔泰帝興神鼎一,一者壹統,天地萬物所系終也。黃帝作寶鼎三,象天地人。禹收九牧之金,鑄九鼎。皆嘗亨鬺上帝鬼神。遭聖則興,鼎遷于夏商。周德衰,宋之社亡,鼎乃淪沒,伏而不見。頌云自堂徂基,自羊徂牛,鼐鼎及鼒,不吳不驁,胡考之休。今鼎至甘泉,光潤龍變,承休無疆。合茲中山,有黃白云降蓋,若獸為符,路弓乘矢,集獲壇下,報祠大享。唯受命而帝者心知其意而合德焉。鼎宜見於祖禰,藏於帝廷,以合明應。制曰,可。

入海求蓬萊者,言蓬萊不遠,而不能至者,殆不見其氣。上乃遣望氣佐候其氣云。

其秋,上幸雍,且郊。或曰五帝,太一之佐也,宜立太一而上親郊之。上疑未定。齊人公孫卿曰,今年得寶鼎,其冬辛巳朔旦冬至,與黃帝時等。卿有札書曰,黃帝得寶鼎宛朐,問於鬼臾區。鬼臾區對曰,黃帝得寶鼎神策,是歲己酉朔旦冬至,得天之紀,終而復始。於是黃帝迎日推策,後率二十歲復朔旦冬至,凡二十推,三百八十年,黃帝僊登于天。卿因所忠欲奏之。所忠視其書不經,疑其妄書,謝曰,寶鼎事已決矣,尚何以為。卿因嬖人奏之。上大說,乃召問卿。對曰,受此書申公,申公已死。上曰,申公何人也。卿曰,申公,齊人。與安期生通,受黃帝言,無書,獨有此鼎書。曰漢興復當黃帝之時。曰漢之聖者在高祖之孫且曾孫也。寶鼎出而與神通,封禪。封禪七十二王,唯黃帝得上泰山封。申公曰,漢主亦當上封,上封能僊登天矣。黃帝時萬諸侯,而神靈之封居七千。天下名山八,而三在蠻夷,五在中國。中國華山、首山、太室、泰山、東萊,此五山黃帝之所常游,與神會。黃帝且戰且學僊。患百姓非其道者,乃斷斬非鬼神者。百餘歲然後得與神通。黃帝郊雍上帝,宿三月。鬼臾區號大鴻,死葬雍,故鴻冢是也。其後黃帝接萬靈明廷。明廷者,甘泉也。所謂寒門者,谷口也。黃帝采首山銅,鑄鼎於荊山下。鼎既成,有龍垂胡髯下迎黃帝。黃帝上騎,群臣後宮從上者七十餘人,龍乃上去。餘小臣不得上,乃悉持龍髯,龍髯拔,墮,墮黃帝之弓。百姓仰望黃帝既上天,乃抱其弓與胡髯號,故後世因名其處曰鼎湖,其弓曰烏號。於是天子曰,嗟乎。吾誠得如黃帝,吾視去妻子如脫屣耳。乃拜卿為郎,東使候神於太室。

上遂郊雍,至隴西,西登崆峒,幸甘泉。令祠官寬舒等具太一祠壇,祠壇放薄忌太一壇,壇三垓。五帝壇環居其下,各如其方,黃帝西南,除八通鬼道。太一,其所用如雍一畤物,而加醴棗脯之屬,殺一貍牛以為俎豆牢具。而五帝獨有俎豆醴進。其下四方地,為醊食群神從者及北斗云。已祠,胙餘皆燎之。其牛色白,鹿居其中,彘在鹿中,水而洎之。祭日以牛,祭月以羊彘特。太一祝宰則衣紫及繡。五帝各如其色,日赤,月白。

十一月辛巳朔旦冬至,昧爽,天字始郊拜太一。朝朝日,夕夕月,則揖,而見太一如雍郊禮。其贊饗曰,天始以寶鼎神策授皇帝,朔而又朔,終而復始,皇帝敬拜見焉。而衣上黃。其祠列火滿壇,壇旁亨炊具。有司云祠上有光焉。公卿言皇帝始郊見太一云陽,有司奉瑄玉嘉牲薦饗。是夜有美光,及晝,黃氣上屬天。太史公、祠官寬舒等曰,神靈之休,祐福兆祥,宜因此地光域立太畤壇以明應。令太祝領,秋及臘閒祠。三歲天子一郊見。

其秋,為伐南越,告禱太一。以牡荊畫幡日月北斗登龍,以象太一三星,為太一鋒,命曰靈旗。為兵禱,則太史奉以指所伐國。而五利將軍使不敢入海,之泰山祠。上使人隨驗,實毋所見。五利妄言見其師,其方盡,多不讎。上乃誅五利。

其冬,公孫卿候神河南,言見僊人跡緱氏城上,有物如雉,往來城上。天子親幸緱氏城視跡。問卿,得毋效文成、五利乎。卿曰,僊者非有求人主,人主者求之。其道非少寬假,神不來。言神事,事如迂誕,積以歲乃可致也。於是郡國各除道,繕治宮觀名山神祠所,以望幸矣。

其春,既滅南越,上有嬖臣李延年以好音見。上善之,下公卿議,曰,民閒祠尚有鼓舞樂,今郊祀而無樂,豈稱乎。公卿曰,古者祠天地皆有樂,而神祇可得而禮。或曰,太帝使素女鼓五十弦瑟,悲,帝禁不止,故破其瑟為二十五弦。於是塞南越,禱祠太一、后土,始用樂舞,益召歌兒,作二十五弦及空侯琴瑟自此起。

其來年冬,上議曰,古者先振兵澤旅,然後封禪。乃遂北巡朔方,勒兵十餘萬,還祭黃帝冢橋山,釋兵須如。上曰,吾聞黃帝不死,今有冢,何也。或對曰,黃帝已僊上天,群臣葬其衣冠。既至甘泉,為且用事泰山,先類祠太一。

自得寶鼎,上與公卿諸生議封禪。封禪用希曠絕,莫知其儀禮,而群儒采封禪尚書、周官、王制之望祀射牛事。齊人丁公年九十餘,曰,封禪者,合不死之名也。秦皇帝不得上封,陛下必欲上,稍上即無風雨,遂上封矣。上於是乃令諸儒習射牛,草封禪儀。數年,至且行。天子既聞公孫卿及方士之言,黃帝以上封禪,皆致怪物與神通,欲放黃帝以上接神僊人蓬萊士,高世比德於九皇,而頗采儒術以文之。群儒既已不能辨明封禪事,又牽拘於詩書古文而不能騁。上為封禪祠器示群儒,群儒或曰不與古同,徐偃又曰太常諸生行禮不如魯善,周霸屬圖封禪事,於是上絀偃、霸,而盡罷諸儒不用。

三月,遂東幸緱氏,禮登中嶽太室。從官在山下聞若有言萬歲云。問上,上不言,問下,下不言。於是以三百戶封太室奉祠,命曰崇高邑。東上泰山,泰山之草木葉未生,乃令人上石立之泰山巔。

上遂東巡海上,行禮祠八神。齊人之上疏言神怪奇方者以萬數,然無驗者。乃益發船,令言海中神山者數千人求蓬萊神人。公孫卿持節常先行候名山,至東萊,言夜見大人,長數丈,就之則不見,見其跡甚大,類禽獸云。群臣有言見一老父牽狗,言吾欲見巨公,已忽不見。上即見大跡,未信,及群臣有言老父,則大以為僊人也。宿留海上,予方士傳車及閒使求僊人以千數。

四月,還至奉高。上念諸儒及方士言封禪人人殊,不經,難施行。天子至梁父,禮祠地主。乙卯,令侍中儒者皮弁薦紳,射牛行事。封泰山下東方,如郊祠太一之禮。封廣丈二尺,高九尺,其下則有玉牒書,書祕。禮畢,天子獨與侍中奉車子侯上泰山,亦有封。其事皆禁。明日,下陰道。丙辰,禪泰山下阯東北肅然山,如祭后土禮。天子皆親拜見,衣上黃而盡用樂焉。江淮閒一茅三脊為神藉。五色土益雜封。縱遠方奇獸蜚禽及白雉諸物,頗以加禮。兕牛犀象之屬不用。皆至泰山祭后土。封禪祠,其夜若有光,晝有白雲起封中。

天子從禪還,坐明堂,群臣更上壽。於是制詔御史,朕以眇眇之身承至尊,兢兢焉懼不任。維德菲薄,不明于禮樂。修祠太一,若有象景光,屑如有望,震於怪物,欲止不敢,遂登封太山,至于梁父,而後禪肅然。自新,嘉與士大夫更始,賜民百戶牛一酒十石,加年八十孤寡布帛二匹。復博、奉高、蛇丘、歷城,無出今年租稅。其大赦天下,如乙卯赦令。行所過毋有復作。事在二年前,皆勿聽治。又下詔曰,古者天子五載一巡狩,用事泰山,諸侯有朝宿地。其令諸侯各治邸泰山下。

天子既已封泰山,無風雨災,而方士更言蓬萊諸神若將可得,於是上欣然庶幾遇之,乃復東至海上望,冀遇蓬萊焉。奉車子侯暴病,一日死。上乃遂去,并海上,北至碣石,巡自遼西,歷北邊至九原。五月,反至甘泉。有司言寶鼎出為元鼎,以今年為元封元年。

其秋,有星茀于東井。後十餘日,有星茀于三能。望氣王朔言,候獨見填星出如瓜,食頃復入焉。有司皆曰,陛下建漢家封禪,天其報德星雲。

其來年冬,郊雍五帝。還,拜祝祠太一。贊饗曰,德星昭衍,厥維休祥。壽星仍出,淵耀光明。信星昭見,皇帝敬拜太祝之享。

其春,公孫卿言見神人東萊山,若云欲見天子。天子於是幸緱氏城,拜卿為中大夫。遂至東萊,宿留之數日,無所見,見大人跡云。復遣方士求神怪采芝藥以千數。是歲旱。於是天子既出無名,乃禱萬里沙,過祠泰山。還至瓠子,自臨塞決河,留二日,沈祠而去。使二卿將卒塞決河,徙二渠,復禹之故跡焉。

是時既滅兩越,越人勇之乃言越人俗鬼,而其祠皆見鬼,數有效。昔東甌王敬鬼,壽百六十歲。後世怠慢,故衰秏。乃令越巫立越祝祠,安臺無壇,亦祠天神上帝百鬼,而以雞卜。上信之,越祠雞卜始用。

公孫卿曰,僊人可見,而上往常遽,以故不見。今陛下可為觀,如緱城,置脯棗,神人宜可致也。且僊人好樓居。於是上令長安則作蜚廉桂觀,甘泉則作益延壽觀,使卿持節設具而候神人。乃作通天莖臺,置祠具其下,將招來僊神人之屬。於是甘泉更置前殿,始廣諸宮室。夏,有芝生殿房內中。天子為塞河,興通天臺,若見有光云,乃下詔,甘泉房中生芝九莖,赦天下,毋有復作。

其明年,伐朝鮮。夏,旱。公孫卿曰,黃帝時封則天旱,乾封三年。上乃下詔曰,天旱,意乾封乎。其令天下尊祠靈星焉。

其明年,上郊雍,通回中道,巡之。春,至鳴澤,從西河歸。

其明年冬,上巡南郡,至江陵而東。登禮灊之天柱山,號曰南岳。浮江,自尋陽出樅陽,過彭蠡,禮其名山川。北至瑯邪,并海上。四月中,至奉高修封焉。

初,天子封泰山,泰山東北阯古時有明堂處,處險不敞。上欲治明堂奉高旁,未曉其制度。濟南人公玊帶上黃帝時明堂圖。明堂圖中有一殿,四面無壁,以茅蓋,通水,圜宮垣為複道,上有樓,從西南入,命曰昆侖,天子從之入,以拜祠上帝焉。於是上令奉高作明堂汶上,如帶圖。及五年修封,則祠太一、五帝於明堂上坐,令高皇帝祠坐對之。祠后土於下房,以二十太牢。天子從昆侖道入,始拜明堂如郊禮。禮畢,燎堂下。而上又上泰山,自有祕祠其巔。而泰山下祠五帝,各如其方,黃帝并赤帝,而有司侍祠焉。山上舉火,下悉應之。

其後二歲,十一月甲子朔旦冬至,推歷者以本統。天子親至泰山,以十一月甲子朔旦冬至日祠上帝明堂,毋修封禪。其贊饗曰,天增授皇帝太元神策,周而復始。皇帝敬拜太一。東至海上,考入海及方士求神者,莫驗,然益遣,冀遇之。

十一月乙酉,柏梁災。十二月甲午朔,上親禪高里,祠后土。臨勃海,將以望祀蓬萊之屬,冀至殊廷焉。

上還,以柏梁災故,朝受計甘泉。公孫卿曰,黃帝就青靈臺,十二日燒,黃帝乃治明廷。明廷,甘泉也。方士多言古帝王有都甘泉者。其後天子又朝諸侯甘泉,甘泉作諸侯邸。勇之乃曰,越俗有火災,復起屋必以大,用勝服之。於是作建章宮,度為千門萬戶。前殿度高未央。其東則鳳闕,高二十餘丈。其西則唐中,數十里虎圈。其北治大池,漸臺高二十餘丈,命曰太液池,中有蓬萊、方丈、瀛洲、壺梁,象海中神山龜魚之屬。其南有玉堂、璧門、大鳥之屬。乃立神明臺、井幹樓,度五十丈,輦道相屬焉。

夏,漢改歷,以正月為歲首,而色上黃,官名更印章以五字,為太初元年。是歲,西伐大宛。蝗大起。丁夫人、雒陽虞初等以方祠詛匈奴、大宛焉。

其明年,有司上言雍五畤無牢熟具,芬芳不備。乃令祠官進畤犢牢具,色食所勝,而以木禺馬代駒焉。獨五月嘗駒,行親郊用駒。及諸名山川用駒者,悉以木禺馬代。行過,乃用駒。他禮如故。

其明年,東巡海上,考神僊之屬,未有驗者。方士有言黃帝時為五城十二樓,以候神人於執期,命曰迎年。上許作之如方,命曰明年。上親禮祠上帝焉。

公玊帶曰,黃帝時雖封泰山,然風后、封巨、岐伯令黃帝封東泰山,禪凡山,合符,然後不死焉。天子既令設祠具,至東泰山,東泰山卑小,不稱其聲,乃令祠官禮之,而不封禪焉。其後令帶奉祠候神物。夏,遂還泰山,修五年之禮如前,而加以禪祠石閭。石閭者,在泰山下阯南方,方士多言此僊人之閭也,故上親禪焉。

其後五年,復至泰山修封。還過祭恒山。

今天子所興祠,太一、后土,三年親郊祠,建漢家封禪,五年一修封。薄忌太一及三一、冥羊、馬行、赤星,五,寬舒之祠官以歲時致禮。凡六祠,皆太祝領之。至如八神諸神,明年、凡山他名祠,行過則祠,行去則已。方士所興祠,各自主,其人終則已,祠官不主。他祠皆如其故。今上封禪,其後十二歲而還,遍於五岳、四瀆矣。而方士之候祠神人,入海求蓬萊,終無有驗。而公孫卿之候神者,猶以大人之跡為解,無有效。天子益怠厭方士之怪迂語矣,然羈縻不絕,冀遇其真。自此之後,方士言神祠者彌眾,然其效可睹矣。太史公曰,余從巡祭天地諸神名山川而封禪焉。入壽宮侍祠神語,究觀方士祠官之意,於是退而論次自古以來用事於鬼神者,具見其表裏。後有君子,得以覽焉。若至俎豆珪幣之詳,獻酬之禮,則有司存。