\chapter{平準書第八}

漢興,接秦之獘,丈夫從軍旅,老弱轉糧馕,作業劇而財匱,自天子不能具鈞駟,而將相或乘牛車,齊民無藏蓋。於是為秦錢重難用,更令民鑄錢,一黃金一斤,約法省禁。而不軌逐利之民,蓄積餘業以稽市物,物踴騰糶,米至石萬錢,馬一匹則百金。

天下已平,高祖乃令賈人不得衣絲乘車,重租稅以困辱之。孝惠、高后時,為天下初定,復弛商賈之律,然市井之子孫亦不得仕宦為吏。量吏祿,度官用,以賦於民。而山川園池市井租稅之入,自天子以至于封君湯沐邑,皆各為私奉養焉,不領於天下之經費。漕轉山東粟,以給中都官,歲不過數十萬石。

至孝文時,莢錢益多,輕,乃更鑄四銖錢,其文為半兩,令民縱得自鑄錢。故吳諸侯也,以即山鑄錢,富埒天子,其後卒以叛逆。鄧通,大夫也,以鑄錢財過王者。故吳、鄧氏錢布天下,而鑄錢之禁生焉。

匈奴數侵盜北邊,屯戍者多,邊粟不足給食當食者。於是募民能輸及轉粟於邊者拜爵,爵得至大庶長。

孝景時,上郡以西旱,亦復修賣爵令,而賤其價以招民,及徒復作,得輸粟縣官以除罪。益造苑馬以廣用,而宮室列觀輿馬益增修矣。

至今上即位數歲,漢興七十餘年之閒,國家無事,非遇水旱之災,民則人給家足,都鄙廩庾皆滿,而府庫餘貨財。京師之錢累巨萬,貫朽而不可校。太倉之粟陳陳相因,充溢露積於外,至腐敗不可食。眾庶街巷有馬,阡陌之閒成群,而乘字牝者儐而不得聚會。守閭閻者食粱肉,為吏者長子孫,居官者以為姓號。故人人自愛而重犯法,先行義而後絀恥辱焉。當此之時,網疏而民富,役財驕溢,或至兼并豪黨之徒,以武斷於鄉曲。宗室有土公卿大夫以下,爭于奢侈,室廬輿服僭于上,無限度。物盛而衰,固其變也。

自是之後,嚴助、朱買臣等招來東甌,事兩越,江淮之閒蕭然煩費矣。唐蒙、司馬相如開路西南夷,鑿山通道千餘里,以廣巴蜀,巴蜀之民罷焉。彭吳賈滅朝鮮,置滄海之郡,則燕齊之閒靡然發動。及王恢設謀馬邑,匈奴絕和親,侵擾北邊,兵連而不解,天下苦其勞,而干戈日滋。行者齎,居者送,中外騷擾而相奉,百姓抏獘以巧法,財賂衰秏而不贍。入物者補官,出貨者除罪,選舉陵遲,廉恥相冒,武力進用,法嚴令具。興利之臣自此始也。

其後漢將歲以數萬騎出擊胡,及車騎將軍衛青取匈奴河南地,筑朔方。當是時,漢通西南夷道,作者數萬人,千里負擔饋糧,率十餘鐘致一石,散幣於邛僰以集之。數歲道不通,蠻夷因以數攻,吏發兵誅之。悉巴蜀租賦不足以更之,乃募豪民田南夷,入粟縣官,而內受錢於都內。東至滄海之郡,人徒之費擬於南夷。又興十萬餘人筑衛朔方,轉漕甚遼遠,自山東咸被其勞,費數十百巨萬,府庫益虛。乃募民能入奴婢得以終身復,為郎增秩,及入羊為郎,始於此。

其後四年,而漢遣大將將六將軍,軍十餘萬,擊右賢王,獲首虜萬五千級。明年,大將軍將六將軍仍再出擊胡,得首虜萬九千級。捕斬首虜之士受賜黃金二十餘萬斤,虜數萬人皆得厚賞,衣食仰給縣官,而漢軍之士馬死者十餘萬,兵甲之財轉漕之費不與焉。於是大農陳藏錢經秏,賦稅既竭,猶不足以奉戰士。有司言,天子曰朕聞五帝之教不相復而治,禹湯之法不同道而王,所由殊路,而建德一也。北邊未安,朕甚悼之。日者,大將軍攻匈奴,斬首虜萬九千級,留蹛無所食。議令民得買爵及贖禁錮免減罪。請置賞官,命曰武功爵。級十七萬,凡直三十餘萬金。諸買武功爵官首者試補吏,先除,千夫如五大夫,其有罪又減二等,爵得至樂卿,以顯軍功。軍功多用越等,大者封侯卿大夫,小者郎吏。吏道雜而多端,則官職秏廢。

自公孫弘以春秋之義繩臣下取漢相,張湯用唆文決理為廷尉,於是見知之法生,而廢格沮誹窮治之獄用矣。其明年,淮南、衡山、江都王謀反跡見,而公卿尋端治之,竟其黨與,而坐死者數萬人,長吏益慘急而法令明察。

當是之時,招尊方正賢良文學之士,或至公卿大夫。公孫弘以漢相,布被,食不重味,為天下先。然無益於俗,稍騖於功利矣。

其明年,驃騎仍再出擊胡,獲首四萬。其秋,渾邪王率數萬之眾來降,於是漢發車二萬乘迎之。既至,受賞,賜及有功之士。是歲費凡百餘巨萬。

初,先是往十餘歲河決觀,梁楚之地固已數困,而緣河之郡隄塞河,輒決壞,費不可勝計。其後番系欲省底柱之漕,穿汾、河渠以為溉田,作者數萬人,鄭當時為渭漕渠回遠,鑿直渠自長安至華陰,作者數萬人,朔方亦穿渠,作者數萬人,各歷二三期,功未就,費亦各巨萬十數。

天子為伐胡,盛養馬,馬之來食長安者數萬匹,卒牽掌者關中不足,乃調旁近郡。而胡降者皆衣食縣官,縣官不給,天子乃損膳,解乘輿駟,出御府禁藏以贍之。

其明年,山東被水菑,民多饑乏,於是天子遣使者虛郡國倉廥以振貧民。猶不足,又募豪富人相貸假。尚不能相救,乃徙貧民於關以西,及充朔方以南新秦中,七十餘萬口,衣食皆仰給縣官。數歲,假予產業,使者分部護之,冠蓋相望。其費以億計,不可勝數。於是縣官大空。

而富商大賈或蹛財役貧,轉轂百數,廢居居邑,封君皆低首仰給。冶鑄煮鹽,財或累萬金,而不佐國家之急,黎民重困。於是天子與公卿議,更錢造幣以贍用,而摧浮淫并兼之徒。是時禁苑有白鹿而少府多銀錫。自孝文更造四銖錢,至是歲四十餘年,從建元以來,用少,縣官往往即多銅山而鑄錢,民亦閒盜鑄錢,不可勝數。錢益多而輕,物益少而貴。有司言曰,古者皮幣,諸侯以聘享。金有三等,黃金為上,白金為中,赤金為下。今半兩錢法重四銖,而姦或盜摩錢裏取鋊,錢益輕薄而物貴,則遠方用幣煩費不省。乃以白鹿皮方尺,緣以藻繢,為皮幣,直四十萬。王侯宗室朝覲聘享,必以皮幣薦璧,然後得行。

又造銀錫為白金。以為天用莫如龍,地用莫如馬,人用莫如龜,故白金三品,其一曰重八兩,圜之,其文龍,名曰白選,直三千,二曰以重差小,方之,其文馬,直五百,三曰復小,撱之,其文龜,直三百。令縣官銷半兩錢,更鑄三銖錢,文如其重。盜鑄諸金錢罪皆死,而吏民之盜鑄白金者不可勝數。

於是以東郭咸陽、孔僅為大農丞,領鹽鐵事,桑弘羊以計算用事,侍中。咸陽,齊之大煮鹽,孔僅,南陽大冶,皆致生累千金,故鄭當時進言之。弘羊,雒陽賈人子,以心計,年十三侍中。故三人言利事析秋豪矣。

法既益嚴,吏多廢免。兵革數動,民多買復及五大夫,徵發之士益鮮。於是除千夫五大夫為吏,不欲者出馬,故吏皆適令伐棘上林,作昆明池。

其明年,大將軍、驃騎大出擊胡,得首虜八九萬級,賞賜五十萬金,漢軍馬死者十餘萬匹,轉漕車甲之費不與焉。是時財匱,戰士頗不得祿矣。

有司言三銖錢輕,易姦詐,乃更請諸郡國鑄五銖錢,周郭其下,令不可磨取鋊焉。

大農上鹽鐵丞孔僅、咸陽言,山海,天地之藏也,皆宜屬少府,陛下不私,以屬大農佐賦。願募民自給費,因官器作煮鹽,官與牢盆。浮食奇民欲擅管山海之貨,以致富羨,役利細民。其沮事之議,不可勝聽。敢私鑄鐵器煮鹽者,釱左趾,沒入其器物。郡不出鐵者,置小鐵官,便屬在所縣。使孔僅、東郭咸陽乘傳舉行天下鹽鐵,作官府,除故鹽鐵家富者為吏。吏道益雜,不選,而多賈人矣。

商賈以幣之變,多積貨逐利。於是公卿言,郡國頗被菑害,貧民無產業者,募徙廣饒之地。陛下損膳省用,出禁錢以振元元,寬貸賦,而民不齊出於南畝,商賈滋眾。貧者畜積無有,皆仰縣官。異時算軺車賈人緡錢皆有差,請算如故。諸賈人末作貰貸賣買,居邑稽諸物,及商以取利者,雖無市籍,各以其物自占,率緡錢二千而一算。諸作有租及鑄,率緡錢四千一算。非吏比者三老、北邊騎士,軺車以一算,商賈人軺車二算,船五丈以上一算。匿不自占,占不悉,戍邊一歲,沒入緡錢。有能告者,以其半畀之。賈人有市籍者,及其家屬,皆無得籍名田,以便農。敢犯令,沒入田僮。

天子乃思卜式之言,召拜式為中郎,爵左庶長,賜田十頃,布告天下,使明知之。

初,卜式者,河南人也,以田畜為事。親死,式有少弟,弟壯,式脫身出分,獨取畜羊百餘,田宅財物盡予弟。式入山牧十餘歲,羊致千餘頭,買田宅。而其弟盡破其業,式輒復分予弟者數矣。是時漢方數使將擊匈奴,卜式上書,願輸家之半縣官助邊。天子使使問式,欲官乎。式曰,臣少牧,不習仕宦,不願也。使問曰,家豈有冤,欲言事乎。式曰,臣生與人無分爭。式邑人貧者貸之,不善者教順之,所居人皆從式,式何故見冤於人。無所欲言也。使者曰,茍如此,子何欲而然。式曰,天子誅匈奴,愚以為賢者宜死節於邊,有財者宜輸委,如此而匈奴可滅也。使者具其言入以聞。天子以語丞相弘。弘曰,此非人情。不軌之臣,不可以為化而亂法,願陛下勿許。於是上久不報式,數歲,乃罷式。式歸,復田牧。歲餘,會軍數出,渾邪王等降,縣官費眾,倉府空。其明年,貧民大徙,皆仰給縣官,無以盡贍。卜式持錢二十萬予河南守,以給徙民。河南上富人助貧人者籍,天子見卜式名,識之,曰是固前而欲輸其家半助邊,乃賜式外繇四百人。式又盡復予縣官。是時富豪皆爭匿財,唯式尤欲輸之助費。天子於是以式終長者,故尊顯以風百姓。

初,式不願為郎。上曰,吾有羊上林中,欲令子牧之。式乃拜為郎,布衣屩而牧羊。歲餘,羊肥息。上過見其羊,善之。式曰,非獨羊也,治民亦猶是也。以時起居,惡者輒斥去,毋令敗群。上以式為奇,拜為緱氏令試之,緱氏便之。遷為成皋令,將漕最。上以為式樸忠,拜為齊王太傅。

而孔僅之使天下鑄作器,三年中拜為大農,列於九卿。而桑弘羊為大農丞,筦諸會計事,稍稍置均輸以通貨物矣。

始令吏得入穀補官,郎至六百石。

自造白金五銖錢後五歲,赦吏民之坐盜鑄金錢死者數十萬人。其不發覺相殺者,不可勝計。赦自出者百餘萬人。然不能半自出,天下大抵無慮皆鑄金錢矣。犯者眾,吏不能盡誅取,於是遣博士褚大、徐偃等分曹循行郡國,舉兼并之徒守相為利者。而御史大夫張湯方隆貴用事,減宣、杜周等為中丞,義縱、尹齊、王溫舒等用慘急刻深為九卿,而直指夏蘭之屬始出矣。

而大農顏異誅。初,異為濟南亭長,以廉直稍遷至九卿。上與張湯既造白鹿皮幣,問異。異曰,今王侯朝賀以蒼璧,直數千,而其皮薦反四十萬,本末不相稱。天子不說。張湯又與異有卻,及有人告異以它議,事下張湯治異。異與客語,客語初令下有不便者,異不應,微反脣。湯奏當異九卿見令不便,不入言而腹誹,論死。自是之後,有腹誹之法比,而公卿大夫多諂諛取容矣。

天子既下緡錢令而尊卜式,百姓終莫分財佐縣官,於是告緡錢縱矣。

郡國多姦鑄錢,錢多輕,而公卿請令京師鑄鐘官赤側,一當五,賦官用非赤側不得行。白金稍賤,民不寶用,縣官以令禁之,無益。歲餘,白金終廢不行。

是歲也,張湯死而民不思。

其後二歲,赤側錢賤,民巧法用之,不便,又廢。於是悉禁郡國無鑄錢,專令上林三官鑄。錢既多,而令天下非三官錢不得行,諸郡國所前鑄錢皆廢銷之,輸其銅三官。而民之鑄錢益少,計其費不能相當,唯真工大姦乃盜為之。

卜式相齊,而楊可告緡遍天下,中家以上大抵皆遇告。杜周治之,獄少反者。乃分遣御史廷尉正監分曹往,即治郡國緡錢,得民財物以億計,奴婢以千萬數,田大縣數百頃,小縣百餘頃,宅亦如之。於是商賈中家以上大率破,民偷甘食好衣,不事畜藏之產業,而縣官有鹽鐵緡錢之故,用益饒矣。

益廣關,置左右輔。

初,大農筦鹽鐵官布多,置水衡,欲以主鹽鐵,及楊可告緡錢,上林財物眾,乃令水衡主上林。上林既充滿,益廣。是時越欲與漢用船戰逐,乃大修昆明池,列觀環之。治樓船,高十餘丈,旗幟加其上,甚壯。於是天子感之,乃作柏梁臺,高數十丈。宮室之修,由此日麗。

乃分緡錢諸官,而水衡、少府、大農、太仆各置農官,往往即郡縣比沒入田田之。其沒入奴婢,分諸苑養狗馬禽獸,及與諸官。諸官益雜置多,徒奴婢眾,而下河漕度四百萬石,及官自糴乃足。

所忠言,世家子弟富人或鬬雞走狗馬,弋獵博戲,亂齊民。乃徵諸犯令,相引數千人,命曰株送徒。入財者得補郎,郎選衰矣。

是時山東被河菑,及歲不登數年,人或相食,方一二千里。天子憐之,詔曰,江南火耕水耨,令饑民得流就食江淮閒,欲留,留處。遣使冠蓋相屬於道,護之,下巴蜀粟以振之。

其明年,天子始巡郡國。東度河,河東守不意行至,不辨,自殺。行西踰隴,隴西守以行往卒,天子從官不得食,隴西守自殺。於是上北出蕭關,從數萬騎,獵新秦中,以勒邊兵而歸。新秦中或千里無亭徼,於是誅北地太守以下,而令民得畜牧邊縣,官假馬母,三歲而歸,及息什一,以除告緡,用充仞新秦中。

既得寶鼎,立后土、太一祠,公卿議封禪事,而天下郡國皆豫治道橋,繕故宮,及當馳道縣,縣治官儲,設供具,而望以待幸。

其明年,南越反,西羌侵邊為桀。於是天子為山東不贍,赦天下囚,因南方樓船卒二十餘萬人擊南越,數萬人發三河以西騎擊西羌,又數萬人度河筑令居。初置張掖、酒泉郡,而上郡、朔方、西河、河西開田官,斥塞卒六十萬人戍田之。中國繕道餽糧,遠者三千,近者千餘里,皆仰給大農。邊兵不足,乃發武庫工官兵器以贍之。車騎馬乏絕,縣官錢少,買馬難得,乃著令,令封君以下至三百石以上吏,以差出牝馬天下亭,亭有畜牸馬,歲課息。

齊相卜式上書曰,臣聞主憂臣辱。南越反,臣願父子與齊習船者往死之。天子下詔曰,卜式雖躬耕牧,不以為利,有餘輒助縣官之用。今天下不幸有急,而式奮願父子死之,雖未戰,可謂義形於內。賜爵關內侯,金六十斤,田十頃。布告天下,天下莫應。列侯以百數,皆莫求從軍擊羌、越。至酎,少府省金,而列侯坐酎金失侯者百餘人。乃拜式為御史大夫。

式既在位,見郡國多不便縣官作鹽鐵,鐵器苦惡,賈貴,或彊令民賣買之。而船有算,商者少,物貴,乃因孔僅言船算事。上由是不悅卜式。

漢連兵三歲,誅羌,滅南越,番禺以西至蜀南者置初郡十七,且以其故俗治,毋賦稅。南陽、漢中以往郡,各以地比給初郡吏卒奉食幣物,傳車馬被具。而初郡時時小反,殺吏,漢發南方吏卒往誅之,閒歲萬餘人,費皆仰給大農。大農以均輸調鹽鐵助賦,故能贍之。然兵所過縣,為以訾給毋乏而已,不敢言擅賦法矣。

其明年,元封元年,卜式貶秩為太子太傅。而桑弘羊為治粟都尉,領大農,盡代僅筦天下鹽鐵。弘羊以諸官各自市,相與爭,物故騰躍,而天下賦輸或不償其僦費,乃請置大農部丞數十人,分部主郡國,各往往縣置均輸鹽鐵官,令遠方各以其物貴時商賈所轉販者為賦,而相灌輸。置平準于京師,都受天下委輸。召工官治車諸器,皆仰給大農。大農之諸官盡籠天下之貨物,貴即賣之,賤則買之。如此,富商大賈無所牟大利,則反本,而萬物不得騰踴。故抑天下物,名曰平準。天子以為然,許之。於是天子北至朔方,東到太山,巡海上,并北邊以歸。所過賞賜,用帛百餘萬匹,錢金以巨萬計,皆取足大農。

弘羊又請令吏得入粟補官,及罪人贖罪。令民能入粟甘泉各有差,以復終身,不告緡。他郡各輸急處,而諸農各致粟,山東漕益歲六百萬石。一歲之中,太倉、甘泉倉滿。邊餘穀諸物均輸帛五百萬匹。民不益賦而天下用饒。於是弘羊賜爵左庶長,黃金再百斤焉。

是歲小旱,上令官求雨,卜式言曰,縣官當食租衣稅而已,今弘羊令吏坐市列肆,販物求利。亨弘羊,天乃雨。

太史公曰,農工商交易之路通,而龜貝金錢刀布之幣興焉。所從來久遠,自高辛氏之前尚矣,靡得而記云。故書道唐虞之際,詩述殷周之世,安寧則長庠序,先本絀末,以禮義防于利,事變多故而亦反是。是以物盛則衰,時極而轉,一質一文,終始之變也。禹貢九州,各因其土地所宜,人民所多少而納職焉。湯武承獘易變,使民不倦,各兢兢所以為治,而稍陵遲衰微。齊桓公用管仲之謀,通輕重之權,徼山海之業,以朝諸侯,用區區之齊顯成霸名。魏用李克,盡地力,為彊君。自是以後,天下爭於戰國,貴詐力而賤仁義,先富有而後推讓。故庶人之富者或累巨萬,而貧者或不厭糟糠,有國彊者或并群小以臣諸侯,而弱國或絕祀而滅世。以至於秦,卒并海內。虞夏之幣,金為三品,或黃,或白,或赤,或錢,或布,或刀,或龜貝。及至秦,中一國之幣為二等,黃金以溢名,為上幣,銅錢識曰半兩,重如其文,為下幣。而珠玉、龜貝、銀錫之屬為器飾寶藏,不為幣。然各隨時而輕重無常。於是外攘夷狄,內興功業,海內之士力耕不足糧馕,女子紡績不足衣服。古者嘗竭天下之資財以奉其上,猶自以為不足也。無異故云,事勢之流,相激使然,曷足怪焉。