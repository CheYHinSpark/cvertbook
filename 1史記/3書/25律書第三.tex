\chapter{律書第三}

王者制事立法,物度軌則,壹稟於六律,六律為萬事根本焉。

其於兵械尤所重,故云望敵知吉凶,聞聲效勝負,百王不易之道也。

武王伐紂,吹律聽聲,推孟春以至于季冬,殺氣相并,而音尚宮。同聲相從,物之自然,何足怪哉。

兵者,聖人所以討彊暴,平亂世,夷險阻,救危殆。自含血戴角之獸見犯則校,而況於人懷好惡喜怒之氣。喜則愛心生,怒則毒螫加,情性之理也。

昔黃帝有涿鹿之戰,以定火災,顓頊有共工之陳,以平水害,成湯有南巢之伐,以殄夏亂。遞興遞廢,勝者用事,所受於天也。

自是之後,名士迭興,晉用咎犯,而齊用王子,吳用孫武,申明軍約,賞罰必信,卒伯諸侯,兼列邦土,雖不及三代之誥誓,然身寵君尊,當世顯揚,可不謂榮焉。豈與世儒闇於大較,不權輕重,猥云德化,不當用兵,大至君辱失守,小乃侵犯削弱,遂執不移等哉。笔教笞不可廢於家,刑罰不可捐於國,誅伐不可偃於天下,用之有巧拙,行之有逆順耳。

夏桀、殷紂手搏豺狼,足追四馬,勇非微也,百戰克勝,諸侯懾服,權非輕也。秦二世宿軍無用之地,連兵於邊陲,力非弱也,結怨匈奴,絓禍於越,勢非寡也。及其威盡勢極,閭巷之人為敵國,咎生窮武之不知足,甘得之心不息也。

高祖有天下,三邊外畔,大國之王雖稱蕃輔,臣節未盡。會高祖厭苦軍事,亦有蕭、張之謀,故偃武一休息,羈縻不備。

歷至孝文即位,將軍陳武等議曰,南越、朝鮮自全秦時內屬為臣子,後且擁兵阻阸,選蠕觀望。高祖時天下新定,人民小安,未可復興兵。今陛下仁惠撫百姓,恩澤加海內,宜及士民樂用,征討逆黨,以一封疆。孝文曰,朕能任衣冠,念不到此。會呂氏之亂,功臣宗室共不羞恥,誤居正位,常戰戰慄慄,恐事之不終。且兵凶器,雖克所願,動亦秏病,謂百姓遠方何。又先帝知勞民不可煩,故不以為意。朕豈自謂能。今匈奴內侵,軍吏無功,邊民父子荷兵日久,朕常為動心傷痛,無日忘之。今未能銷距,願且堅邊設候,結和通使,休寧北陲,為功多矣。且無議軍。故百姓無內外之繇,得息肩於田畝,天下殷富,粟至十餘錢,鳴雞吠狗,煙火萬里,可謂和樂者乎。

太史公曰,文帝時,會天下新去湯火,人民樂業,因其欲然,能不擾亂,故百姓遂安。自年六七十翁亦未嘗至市井,游敖嬉戲如小兒狀。孔子所稱有德君子者邪。

書曰七正,二十八舍。律歷,天所以通五行八正之氣,天所以成孰萬物也。舍者,日月所舍。舍者,舒氣也。

不周風居西北,主殺生。東壁居不周風東,主辟生氣而東之。至於營室。營室者,主營胎陽氣而產之。東至于危。危,垝也。言陽氣之垝,故曰危。十月也,律中應鐘。應鐘者,陽氣之應,不用事也。其於十二子為亥。亥者,該也。言陽氣藏於下,故該也。

廣莫風居北方。廣莫者,言陽氣在下,陰莫陽廣大也,故曰廣莫。東至於虛。虛者,能實能虛,言陽氣冬則宛藏於虛,日冬至則一陰下藏,一陽上舒,故曰虛。東至于須女。言萬物變動其所,陰陽氣未相離,尚相如胥也,故曰須女。十一月也,律中黃鐘。黃鐘者,陽氣踵黃泉而出也。其於十二子為子。子者,滋也,滋者,言萬物滋於下也。其於十母為壬癸。壬之為言任也,言陽氣任養萬物於下也。癸之為言揆也,言萬物可揆度,故曰癸。東至牽牛。牽牛者,言陽氣牽引萬物出之也。牛者,冒也,言地雖凍,能冒而生也。牛者,耕植種萬物也。東至於建星。建星者,建諸生也。十二月也,律中大呂。大呂者。其於十二子為醜。

條風居東北,主出萬物。條之言條治萬物而出之,故曰條風。南至於箕。箕者,言萬物根棋,故曰箕。正月也,律中泰蔟。泰蔟者,言萬物蔟生也,故曰泰蔟。其於十二子為寅。寅言萬物始生螾然也,故曰寅。南至於尾,言萬物始生如尾也。南至於心,言萬物始生有華心也。南至於房。房者,言萬物門戶也,至于門則出矣。

明庶風居東方。明庶者,明眾物盡出也。二月也,律中夾鐘。夾鐘者,言陰陽相夾廁也。其於十二子為卯。卯之為言茂也,言萬物茂也。其於十母為甲乙。甲者,言萬物剖符甲而出也,乙者,言萬物生軋軋也。南至于氐者。氐者,言萬物皆至也。南至於亢。亢者,言萬物亢見也。南至于角。角者,言萬物皆有枝格如角也。三月也,律中姑洗。姑洗者,言萬物洗生。其於十二子為辰。辰者,言萬物之蜄也。

清明風居東南維,主風吹萬物而西之。至於軫。軫者,言萬物益大而軫軫然。西至於翼。翼者,言萬物皆有羽翼也。四月也,律中中呂。中呂者,言萬物盡旅而西行也。其於十二子為巳。巳者,言陽氣之已盡也。西至于七星。七星者,陽數成於七,故曰七星。西至于張。張者,言萬物皆張也。西至于注。注者,言萬物之始衰,陽氣下注,故曰注。五月也,律中蕤賓。蕤賓者,言陰氣幼少,故曰蕤,痿陽不用事,故曰賓。

景風居南方。景者,言陽氣道竟,故曰景風。其於十二子為午。午者,陰陽交,故曰午。其於十母為丙丁。丙者,言陽道著明,故曰丙,丁者,言萬物之丁壯也,故曰丁。西至于弧。弧者,言萬物之吳落且就死也。西至于狼。狼者,言萬物可度量,斷萬物,故曰狼。

涼風居西南維,主地。地者,沈奪萬物氣也。六月也,律中林鐘。林鐘者,言萬物就死氣林林然。其於十二子為未。未者,言萬物皆成,有滋味也。北至於罰。罰者,言萬物氣奪可伐也。北至於參。參言萬物可參也,故曰參。七月也,律中夷則。夷則,言陰氣之賊萬物也。其於十二子為申。申者,言陰用事,申賊萬物,故曰申。北至於濁。濁者,觸也,言萬物皆觸死也,故曰濁。北至於留。留者,言陽氣之稽留也,故曰留。八月也,律中南呂。南呂者,言陽氣之旅入藏也。其於十二子為酉。酉者,萬物之老也,故曰酉。

閶闔風居西方。閶者,倡也,闔者,藏也。言陽氣道萬物,闔黃泉也。其於十母為庚辛。庚者,言陰氣庚萬物,故曰庚,辛者,言萬物之辛生,故曰辛。北至於胃。胃者,言陽氣就藏,皆胃胃也。北至於婁。婁者,呼萬物且內之也。北至於奎。奎者,主毒螫殺萬物也,奎而藏之。九月也,律中無射。無射者,陰氣盛用事,陽氣無餘也,故曰無射。其於十二子為戌。戌者,言萬物盡滅,故曰戌。

律數,九九八十一以為宮。三分去一,五十四以為徵。三分益一,七十二以為商。三分去一,四十八以為羽。三分益一,六十四以為角。黃鐘長八寸十分一宮。大呂長七寸五分三分一。太蔟長七寸七分二角。夾鐘長六寸一分三分一。姑洗長六寸七分四羽。仲呂長五寸九分三分二徵。蕤賓長五寸六分三分一。林鐘長五寸七分四角。夷則長五寸四分三分二商。南呂長四寸七分八徵。無射長四寸四分三分二。應鐘長四寸二分三分二羽。

生鐘分,子一分。丑三分二。寅九分八。卯二十七分十六。辰八十一分六十四。巳二百四十三分一百二十八。午七百二十九分五百一十二。未二千一百八十七分一千二十四。申六千五百六十一分四千九十六。酉一萬九千六百八十三分八千一百九十二。戌五萬九千四十九分三萬二千七百六十八。亥十七萬七千一百四十七分六萬五千五百三十六。

生黃鐘術曰,以下生者,倍其實,三其法。以上生者,四其實,三其法。上九,商八,羽七,角六,宮五,徵九。置一而九三之以為法。實如法,得長一寸。凡得九寸,命曰黃鐘之宮。故曰音始於宮,窮於角,數始於一,終於十,成於三,氣始於冬至,周而復生。

神生於無,形成於有,形然後數,形而成聲,故曰神使氣,氣就形。形理如類有可類。或未形而未類,或同形而同類,類而可班,類而可識。聖人知天地識之別,故從有以至未有,以得細若氣,微若聲。然聖人因神而存之,雖妙必效情,核其華道者明矣。非有聖心以乘聰明,孰能存天地之神而成形之情哉。神者,物受之而不能知其去來,故聖人畏而欲存之。唯欲存之,神之亦存。其欲存之者,故莫貴焉。

太史公曰,在旋璣玉衡以齊七政,即天地二十八宿。十母,十二子,鐘律調自上古。建律運歷造日度,可據而度也。合符節,通道德,即從斯之謂也。