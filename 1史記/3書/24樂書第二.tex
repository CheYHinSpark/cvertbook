\chapter{樂書第二}

太史公曰,余每讀虞書,至於君臣相敕,維是几安,而股肱不良,萬事墮壞,未嘗不流涕也。成王作頌,推己懲艾,悲彼家難,可不謂戰戰恐懼,善守善終哉。君子不為約則修德,滿則棄禮,佚能思初,安能惟始,沐浴膏澤而歌詠勤苦,非大德誰能如斯。傳曰治定功成,禮樂乃興。海內人道益深,其德益至,所樂者益異。滿而不損則溢,盈而不持則傾。凡作樂者,所以節樂。君子以謙退為禮,以損減為樂,樂其如此也。以為州異國殊,情習不同,故博采風俗,協比聲律,以補短移化,助流政教。天子躬於明堂臨觀,而萬民咸蕩滌邪穢,斟酌飽滿,以飾厥性。故云雅頌之音理而民正,嘄噭之聲興而士奮,鄭衛之曲動而心淫。及其調和諧合,鳥獸盡感,而況懷五常,含好惡,自然之勢也。

治道虧缺而鄭音興起,封君世辟,名顯鄰州,爭以相高。自仲尼不能與齊優遂容於魯,雖退正樂以誘世,作五章以剌時,猶莫之化。陵遲以至六國,流沔沈佚,遂往不返,卒於喪身滅宗,并國於秦。

秦二世尤以為娛。丞相李斯進諫曰,放棄詩書,極意聲色,祖伊所以懼也,輕積細過,恣心長夜,紂所以亡也。趙高曰,五帝、三王樂各殊名,示不相襲。上自朝廷,下至人民,得以接歡喜,合殷勤,非此和說不通,解澤不流,亦各一世之化,度時之樂,何必華山之騄耳而後行遠乎。二世然之。

高祖過沛詩三侯之章,令小兒歌之。高祖崩,令沛得以四時歌儛宗廟。孝惠、孝文、孝景無所增更,於樂府習常肄舊而已。

至今上即位,作十九章,令侍中李延年次序其聲,拜為協律都尉。通一經之士不能獨知其辭,皆集會五經家,相與共講習讀之,乃能通知其意,多爾雅之文。

漢家常以正月上辛祠太一甘泉,以昏時夜祠,到明而終。常有流星經於祠壇上。使僮男僮女七十人俱歌。春歌青陽,夏歌朱明,秋歌西暤,冬歌玄冥。世多有,故不論。

又嘗得神馬渥洼水中,復次以為太一之歌。歌曲曰,太一貢兮天馬下,霑赤汗兮沫流赭。騁容與兮跇萬里,今安匹兮龍為友。後伐大宛得千里馬,馬名蒲梢,次作以為歌。歌詩曰,天馬來兮從西極,經萬里兮歸有德。承靈威兮降外國,涉流沙兮四夷服。中尉汲黯進曰,凡王者作樂,上以承祖宗,下以化兆民。今陛下得馬,詩以為歌,協於宗廟,先帝百姓豈能知其音邪。上默然不說。丞相公孫弘曰,黯誹謗聖制,當族。

凡音之起,由人心生也。人心之動,物使之然也。感於物而動,故形於聲,聲相應,故生變,變成方,謂之音,比音而樂之,及干戚羽旄,謂之樂也。樂者,音之所由生也,其本在人心感於物也。是故其哀心感者,其聲噍以殺,其樂心感者,其聲啴以緩,其喜心感者,其聲發以散,其怒心感者,其聲麤以厲,其敬心感者,其聲直以廉,其愛心感者,其聲和以柔。六者非性也,感於物而後動,是故先王慎所以感之。故禮以導其志,樂以和其聲,政以壹其行,刑以防其姦。禮樂刑政,其極一也,所以同民心而出治道也。

凡音者,生人心者也。情動於中,故形於聲,聲成文謂之音。是故治世之音安以樂,其正和,亂世之音怨以怒,其正乖,亡國之音哀以思,其民困。聲音之道,與正通矣。宮為君,商為臣,角為民,徵為事,羽為物。五者不亂,則無惉懘之音矣。宮亂則荒,其君驕,商亂則搥,其臣壞,角亂則憂,其民怨,徵亂則哀,其事勤,羽亂則危,其財匱。五者皆亂,迭相陵,謂之慢。如此則國之滅亡無日矣。鄭衛之音,亂世之音也,比於慢矣。桑閒濮上之音,亡國之音也,其政散,其民流,誣上行私而不可止。

凡音者,生於人心者也,樂者,通於倫理者也。是故知聲而不知音者,禽獸是也,知音而不知樂者,眾庶是也。唯君子為能知樂。是故審聲以知音,審音以知樂,審樂以知政,而治道備矣。是故不知聲者不可與言音,不知音者不可與言樂知樂則幾於禮矣。禮樂皆得,謂之有德。德者得也。是故樂之隆,非極音也,食饗之禮,非極味也。清廟之瑟,朱弦而疏越,一倡而三嘆,有遺音者矣。大饗之禮,尚玄酒而俎腥魚,大羹不和,有遺味者矣。是故先王之制禮樂也,非以極口腹耳目之欲也,將以教民平好惡而反人道之正也。

人生而靜,天之性也,感於物而動,性之頌也。物至知知,然後好惡形焉。好惡無節於內,知誘於外,不能反己,天理滅矣。夫物之感人無窮,而人之好惡無節,則是物至而人化物也。人化物也者,滅天理而窮人欲者也。於是有悖逆詐偽之心,有淫佚作亂之事。是故彊者脅弱,眾者暴寡,知者詐愚,勇者苦怯,疾病不養,老幼孤寡不得其所,此大亂之道也。是故先王制禮樂,人為之節,衰麻哭泣,所以節喪紀也,鐘鼓干戚,所以和安樂也,婚姻冠笄,所以別男女也,射鄉食饗,所以正交接也。禮節民心,樂和民聲,政以行之,刑以防之。禮樂刑政四達而不悖,則王道備矣。

樂者為同,禮者為異。同則相親,異則相敬。樂勝則流,禮勝則離。合情飾貌者,禮樂之事也。禮義立,則貴賤等矣,樂文同,則上下和矣,好惡著,則賢不肖別矣,刑禁暴,爵舉賢,則政均矣。仁以愛之,義以正之,如此則民治行矣。

樂由中出,禮自外作。樂由中出,故靜,禮自外作,故文。大樂必易,大禮必簡。樂至則無怨,禮至則不爭。揖讓而治天下者,禮樂之謂也。暴民不作,諸侯賓服,兵革不試,五刑不用,百姓無患,天子不怒,如此則樂達矣。合父子之親,明長幼之序,以敬四海之內。天子如此,則禮行矣。

大樂與天地同和,大禮與天地同節。和,故百物不失,節,故祀天祭地。明則有禮樂,幽則有鬼神,如此則四海之內合敬同愛矣。禮者,殊事合敬者也,樂者,異文合愛者也。禮樂之情同,故明王以相沿也。故事與時并,名與功偕。故鐘鼓管磬羽籥干戚,樂之器也,詘信俯仰級兆舒疾,樂之文也。簠簋俎豆制度文章,禮之器也,升降上下周旋裼襲,禮之文也。故知禮樂之情者能作,識禮樂之文者能術。作者之謂聖,術者之謂明。明聖者,術作之謂也。

樂者,天地之和也,禮者,天地之序也。和,故百物皆化,序,故群物皆別。樂由天作,禮以地制。過制則亂,過作則暴。明於天地,然後能興禮樂也。論倫無患,樂之情也,欣喜驩愛,樂之容也。中正無邪,禮之質也,莊敬恭順,禮之制也。若夫禮樂之施於金石,越於聲音,用於宗廟社稷,事于山川鬼神,則此所以與民同也。

王者功成作樂,治定制禮。其功大者其樂備,其治辨者其禮具。干戚之舞,非備樂也,亨孰而祀,非達禮也。五帝殊時,不相沿樂,三王異世,不相襲禮。樂極則憂,禮粗則偏矣。及夫敦樂而無憂,禮備而不偏者,其唯大聖乎。天高地下,萬物散殊,而禮制行也,流而不息,合同而化,而樂興也。春作夏長,仁也,秋斂冬藏,義也。仁近於樂,義近於禮。樂者敦和,率神而從天,禮者辨宜,居鬼而從地。故聖人作樂以應天,作禮以配地。禮樂明備,天地官矣。

天尊地卑,君臣定矣。高卑已陳,貴賤位矣。動靜有常,小大殊矣。方以類聚,物以群分,則性命不同矣。在天成象,在地成形,如此則禮者天地之別也。地氣上隮,天氣下降,陰陽相摩,天地相蕩,鼓之以雷霆,奮之以風雨,動之以四時,煖之以日月,而百物化興焉,如此則樂者天地之和也。

化不時則不生,男女無別則亂登,此天地之情也。及夫禮樂之極乎天而蟠乎地,行乎陰陽而通乎鬼神,窮高極遠而測深厚,樂著太始而禮居成物。著不息者天也,著不動者地也。一動一靜者,天地之閒也。故聖人曰禮云樂云。

昔者舜作五弦之琴,以歌南風,夔始作樂,以賞諸侯。故天子之為樂也,以賞諸侯之有德者也。德盛而教尊,五穀時孰,然後賞之以樂。故其治民勞者,其舞行級遠,其治民佚者,其舞行級短。故觀其舞而知其德,聞其謚而知其行。大章,章之也,咸池,備也,韶,繼也,夏,大也,殷周之樂盡也。

天地之道,寒暑不時則疾,風雨不節則饑。教者,民之寒暑也,教不時則傷世。事者,民之風雨也,事不節則無功。然則先王之為樂也,以法治也,善則行象德矣。夫豢豕為酒,非以為禍也,而獄訟益煩,則酒之流生禍也。是故先王因為酒禮,一獻之禮,賓主百拜,終日飲酒而不得醉焉,此先王之所以備酒禍也。故酒食者,所以合歡也。

樂者,所以象德也,禮者,所以閉淫也。是故先王有大事,必有禮以哀之,有大福,必有禮以樂之,哀樂之分,皆以禮終。

樂也者,施也,禮也者,報也。樂,樂其所自生,而禮,反其所自始。樂章德,禮報情反始也。所謂大路者,天子之輿也,龍旂九旒,天子之旌也,青黑緣者,天子之葆龜也,從之以牛羊之群,則所以贈諸侯也。

樂也者,情之不可變者也,禮也者,理之不可易者也。樂統同,禮別異,禮樂之說貫乎人情矣。窮本知變,樂之情也,著誠去偽,禮之經也。禮樂順天地之誠,達神明之德,降興上下之神,而凝是精粗之體,領父子君臣之節。

是故大人舉禮樂,則天地將為昭焉。天地欣合,陰陽相得,煦嫗覆育萬物,然後草木茂,區萌達,羽翮奮,角觡生,蟄蟲昭穌,羽者嫗伏,毛者孕鬻,胎生者不殰而卵生者不殈,則樂之道歸焉耳。

樂者,非謂黃鐘大呂弦歌干揚也,樂之末節也,故童者舞之,布筵席,陳樽俎,列籩豆,以升降為禮者,禮之末節也,故有司掌之。樂師辯乎聲詩,故北面而弦,宗祝辯乎宗廟之禮,故後尸,商祝辯乎喪禮,故後主人。是故德成而上,藝成而下,行成而先,事成而後。是故先王有上有下,有先有後,然後可以有制於天下也。

樂者,聖人之所樂也,而可以善民心。其感人深,其風移俗易,故先王著其教焉。

夫人有血氣心知之性,而無哀樂喜怒之常,應感起物而動,然後心術形焉。是故志微焦衰之音作,而民思憂,啴緩慢易繁文簡節之音作,而民康樂,粗厲猛起奮末廣賁之音作,而民剛毅,廉直經正莊誠之音作,而民肅敬,寬裕肉好順成和動之音作,而民慈愛,流辟邪散狄成滌濫之音作,而民淫亂。

是故先王本之情性,稽之度數,制之禮義,合生氣之和,道五常之行,使之陽而不散,陰而不密,剛氣不怒,柔氣不懾,四暢交於中而發作於外,皆安其位而不相奪也。然後立之學等,廣其節奏,省其文采,以繩德厚也。類小大之稱,比終始之序,以象事行,使親疏貴賤長幼男女之理皆形見於樂,故曰樂觀其深矣。

土敝則草木不長,水煩則魚鱉不大,氣衰則生物不育,世亂則禮廢而樂淫。是故其聲哀而不莊,樂而不安,慢易以犯節,流湎以忘本。廣則容姦。狹則思欲,感滌蕩之氣而滅平和之德,是以君子賤之也。

凡姦聲感人而逆氣應之,逆氣成象而淫樂興焉。正聲感人而順氣應之,順氣成象而和樂興焉。倡和有應,回邪曲直各歸其分,而萬物之理以類相動也。

是故君子反情以和其志,比類以成其行。姦聲亂色不留聰明,淫樂廢禮不接於心術,惰慢邪辟之氣不設於身體,使耳目鼻口心知百體皆由順正,以行其義。然後發以聲音,文以琴瑟,動以干戚,飾以羽旄,從以簫管,奮至德之光,動四氣之和,以著萬物之理。是故清明象天,廣大象地,終始象四時,周旋象風雨,五色成文而不亂,八風從律而不姦,百度得數而有常,小大相成,終始相生,倡和清濁,代相為經。故樂行而倫清,耳目聰明,血氣和平,移風易俗,天下皆寧。故曰樂者樂也。君子樂得其道,小人樂得其欲。以道制欲,則樂而不亂,以欲忘道,則惑而不樂。是故君子反情以和其志,廣樂以成其教,樂行而民鄉方,可以觀德矣。

德者,性之端也,樂者,德之華也,金石絲竹,樂之器也。詩,言其志也,歌,詠其聲也,舞,動其容也,三者本乎心,然後樂氣從之。是故情深而文明,氣盛而化神,和順積中而英華發外,唯樂不可以為偽。

樂者,心之動也,聲者,樂之象也,文采節奏,聲之飾也。君子動其本,樂其象,然後治其飾。是故先鼓以警戒,三步以見方,再始以著往,復亂以飭歸,奮疾而不拔,極幽而不隱。獨樂其志,不厭其道,備舉其道,不私其欲。是以情見而義立,樂終而德尊,君子以好善,小人以息過,故曰生民之道,樂為大焉。

君子曰,禮樂不可以斯須去身。致樂以治心,則易直子諒之心油然生矣。易直子諒之心生則樂,樂則安,安則久,久則天,天則神。天則不言而信,神則不怒而威。致樂,以治心者也,致禮,以治躬者也。治躬則莊敬,莊敬則嚴威。心中斯須不和不樂,而鄙詐之心入之矣,外貌斯須不莊不敬,而慢易之心入之矣。故樂也者,動於內者也,禮也者,動於外者也。樂極和,禮極順。內和而外順,則民瞻其顏色而弗與爭也,望其容貌而民不生易慢焉。德煇動乎內而民莫不承聽,理發乎外而民莫不承順,故曰知禮樂之道,舉而錯之天下無難矣。

樂也者,動於內者也,禮也者,動於外者也。故禮主其謙,樂主其盈。禮謙而進,以進為文,樂盈而反,以反為文。禮謙而不進,則銷,樂盈而不反,則放。故禮有報而樂有反。禮得其報則樂,樂得其反則安。禮之報,樂之反,其義一也。

夫樂者樂也,人情之所不能免也。樂必發諸聲音,形於動靜,人道也。聲音動靜,性術之變,盡於此矣。故人不能無樂,樂不能無形。形而不為道,不能無亂。先王惡其亂,故制雅頌之聲以道之,使其聲足以樂而不流,使其文足以綸而不息,使其曲直繁省廉肉節奏,足以感動人之善心而已矣,不使放心邪氣得接焉,是先王立樂之方也。是故樂在宗廟之中,君臣上下同聽之,則莫不和敬,在族長鄉里之中,長幼同聽之,則莫不和順,在閨門之內,父子兄弟同聽之,則莫不和親。故樂者,審一以定和,比物以飾節,節奏合以成文,所以合和父子君臣,附親萬民也,是先王立樂之方也。故聽其雅頌之聲,志意得廣焉,執其干戚,習其俯仰詘信,容貌得莊焉,行其綴兆,要其節奏,行列得正焉,進退得齊焉。故樂者天地之齊,中和之紀,人情之所不能免也。

夫樂者,先王之所以飾喜也,軍旅鈇鉞者,先王之所以飾怒也。故先王之喜怒皆得其齊矣。喜則天下和之,怒則暴亂者畏之。先王之道禮樂可謂盛矣。

魏文侯問於子夏曰,吾端冕而聽古樂則唯恐臥,聽鄭衛之音則不知倦。敢問古樂之如彼,何也。新樂之如此,何也。

子夏答曰,今夫古樂,進旅而退旅,和正以廣,弦匏笙簧合守拊鼓,始奏以文,止亂以武,治亂以相,訊疾以雅。君子於是語,於是道古,修身及家,平均天下,此古樂之發也。今夫新樂,進俯退俯,姦聲以淫,溺而不止,及優侏儒,獶雜子女,不知父子。樂終不可以語,不可以道古,此新樂之發也。今君之所問者樂也,所好者音也。夫樂之與音,相近而不同。

文侯曰,敢問如何。

子夏答曰,夫古者天地順而四時當,民有德而五穀昌,疾疢不作而無祅祥,此之謂大當。然後聖人作為父子君臣以為之紀綱,紀綱既正,天下大定,天下大定,然後正六律,和五聲,弦歌詩頌,此之謂德音,德音之謂樂。詩曰,莫其德音,其德克明,克明克類,克長克君。王此大邦,克順克俾。俾於文王,其德靡悔。既受帝祉,施于孫子。此之謂也。今君之所好者,其溺音與。

文侯曰,敢問溺音者何從出也。

子夏答曰,鄭音好濫淫志,宋音燕女溺志,衛音趣數煩志,齊音驁辟驕志,四者皆淫於色而害於德,是以祭祀不用也。詩曰,肅雍和鳴,先祖是聽。夫肅肅,敬也,雍雍,和也。夫敬以和,何事不行。為人君者,謹其所好惡而已矣。君好之則臣為之,上行之則民從之。詩曰,誘民孔易,此之謂也。然後聖人作為鞉鼓椌楬壎篪,此六者,德音之音也。然後鐘磬竽瑟以和之,干戚旄狄以舞之。此所以祭先王之廟也,所以獻醻酳酢也,所以官序貴賤各得其宜也,此所以示後世有尊卑長幼序也。鐘聲鏗,鏗以立號,號以立橫,橫以立武。君子聽鐘聲則思武臣。石聲硁,硁以立別,別以致死。君子聽磬聲則思死封疆之臣。絲聲哀,哀以立廉,廉以立志。君子聽琴瑟之聲則思志義之臣。竹聲濫,濫以立會,會以聚眾。君子聽竽笙簫管之聲則思畜聚之臣。鼓鼙之聲讙,讙以立動,動以進眾。君子聽鼓鼙之聲則思將帥之臣。君子之聽音,非聽其鏗鎗而已也,彼亦有所合之也。

賓牟賈侍坐於孔子,孔子與之言,及樂,曰,夫武之備戒之已久,何也。

答曰,病不得其眾也。

永嘆之,淫液之,何也。

答曰,恐不逮事也。

發揚蹈厲之已蚤,何也。

答曰,及時事也。

武坐致右憲左,何也。

答曰,非武坐也。

聲淫及商,何也。

答曰,非武音也。

子曰,若非武音,則何音也。

答曰,有司失其傳也。如非有司失其傳,則武王之志荒矣。

子曰,唯丘之聞諸萇弘,亦若吾子之言是也。

賓牟賈起,免席而請曰,夫武之備戒之已久,則既聞命矣。敢問遲之遲而又久,何也。

子曰,居,吾語汝。夫樂者,象成者也。總干而山立,武王之事也,發揚蹈厲,太公之志也,武亂皆坐,周召之治也。且夫武,始而北出,再成而滅商,三成而南,四成而南國是疆,五成而分陜,周公左,召公右,六成復綴,以崇天子,夾振之而四伐,盛威於中國也。分夾而進,事蚤濟也。久立於綴,以待諸侯之至也。且夫女獨未聞牧野之語乎。武王克殷反商,未及下車,而封黃帝之後於薊,封帝堯之後於祝,封帝舜之後於陳,下車而封夏后氏之後於杞,封殷之後於宋,封王子比干之墓,釋箕子之囚,使之行商容而復其位。庶民弛政,庶士倍祿。濟河而西,馬散華山之陽而弗復乘,牛散桃林之野而不復服,車甲弢而藏之府庫而弗復用,倒載干戈,苞之以虎皮,將率之士,使為諸侯,名之曰建櫜,然後天下知武王之不復用兵也。散軍而郊射,左射貍首,右射騶虞,而貫革之射息也,裨冕搢笏,而虎賁之士稅劍也,祀乎明堂,而民知孝,朝覲,然後諸侯知所以臣,耕藉,然後諸侯知所以敬,五者天下之大教也。食三老五更於太學,天子袒而割牲,執醬而饋,執爵而酳,冕而總干,所以教諸侯之悌也。若此,則周道四達,禮樂交通,則夫武之遲久,不亦宜乎。

子貢見師乙而問焉,曰,賜聞聲歌各有宜也,如賜者宜何歌也。

師乙曰,乙,賤工也,何足以問所宜。請誦其所聞,而吾子自執焉。寬而靜,柔而正者宜歌頌,廣大而靜,疏達而信者宜歌大雅,恭儉而好禮者宜歌小雅,正直清廉而謙者宜歌風,肆直而慈愛者宜歌商,溫良而能斷者宜歌齊。夫歌者,直己而陳德,動己而天地應焉,四時和焉,星辰理焉,萬物育焉。故商者,五帝之遺聲也,商人志之,故謂之商,齊者,三代之遺聲也,齊人志之,故謂之齊。明乎商之詩者,臨事而屢斷,明乎齊之詩者,見利而讓也。臨事而屢斷,勇也,見利而讓,義也。有勇有義,非歌孰能保此。故歌者,上如抗,下如隊,曲如折,止如槁木,居中矩,句中鉤,累累乎殷如貫珠。故歌之為言也,長言之也。說之,故言之,言之不足,故長言之,長言之不足,故嗟嘆之,嗟嘆之不足,故不知手之舞之足之蹈之。子貢問樂。

凡音由於人心,天之與人有以相通,如景之象形,響之應聲。故為善者天報之以福,為惡者天與之以殃,其自然者也。

故舜彈五弦之琴,歌南風之詩而天下治,紂為朝歌北鄙之音,身死國亡。舜之道何弘也。紂之道何隘也。夫南風之詩者生長之音也,舜樂好之,樂與天地同意,得萬國之驩心,故天下治也。夫朝歌者不時也,北者敗也,鄙者陋也,紂樂好之,與萬國殊心,諸侯不附,百姓不親,天下畔之,故身死國亡。

而衛靈公之時,將之晉,至於濮水之上舍。夜半時聞鼓琴聲,問左右,皆對曰不聞。乃召師涓曰,吾聞鼓琴音,問左右,皆不聞。其狀似鬼神,為我聽而寫之。師涓曰,諾。因端坐援琴,聽而寫之。明日,曰,臣得之矣,然未習也,請宿習之。靈公曰,可。因復宿。明日,報曰,習矣。即去之晉,見晉平公。平公置酒於施惠之臺。酒酣,靈公曰,今者來,聞新聲,請奏之。平公曰,可。即令師涓坐師曠旁,援琴鼓之。未終,師曠撫而止之曰,此亡國之聲也,不可遂。平公曰,何道出。師曠曰,師延所作也。與紂為靡靡之樂,武王伐紂,師延東走,自投濮水之中,故聞此聲必於濮水之上,先聞此聲者國削。平公曰,寡人所好者音也,願遂聞之。師涓鼓而終之。

平公曰,音無此最悲乎。師曠曰,有。平公曰,可得聞乎。師曠曰,君德義薄,不可以聽之。平公曰,寡人所好者音也,願聞之。師曠不得已,援琴而鼓之。一奏之,有玄鶴二八集乎廊門,再奏之,延頸而鳴,舒翼而舞。

平公大喜,起而為師曠壽。反坐,問曰,音無此最悲乎。師曠曰,有。昔者黃帝以大合鬼神,今君德義薄,不足以聽之,聽之將敗。平公曰,寡人老矣,所好者音也,願遂聞之。師曠不得已,援琴而鼓之。一奏之,有白雲從西北起,再奏之,大風至而雨隨之,飛廊瓦,左右皆奔走。平公恐懼,伏於廊屋之閒。晉國大旱,赤地三年。

聽者或吉或凶。夫樂不可妄興也。

太史公曰,夫上古明王舉樂者,非以娛心自樂,快意恣欲,將欲為治也。正教者皆始於音,音正而行正。故音樂者,所以動蕩血脈,通流精神而和正心也。故宮動脾而和正聖,商動肺而和正義,角動肝而和正仁,徵動心而和正禮,羽動腎而和正智。故樂所以內輔正心而外異貴賤也,上以事宗廟,下以變化黎庶也。琴長八尺一寸,正度也。弦大者為宮,而居中央,君也。商張右傍,其餘大小相次,不失其次序,則君臣之位正矣。故聞宮音,使人溫舒而廣大,聞商音,使人方正而好義,聞角音,使人惻隱而愛人,聞徵音,使人樂善而好施,聞羽音,使人整齊而好禮。夫禮由外入,樂自內出。故君子不可須臾離禮,須臾離禮則暴慢之行窮外,不可須臾離樂,須臾離樂則姦邪之行窮內。故樂音者,君子之所養義也。夫古者,天子諸侯聽鐘磬未嘗離於庭,卿大夫聽琴瑟之音未嘗離於前,所以養行義而防淫佚也。夫淫佚生於無禮,故聖王使人耳聞雅頌之音,目視威儀之禮,足行恭敬之容,口言仁義之道。故君子終日言而邪辟無由入也。