\chapter{禮書第一}

太史公曰,洋洋美德乎。宰制萬物,役使群眾,豈人力也哉。余至大行禮官,觀三代損益,乃知緣人情而制禮,依人性而作儀,其所由來尚矣。

人道經緯萬端,規矩無所不貫,誘進以仁義,束縛以刑罰,故德厚者位尊,祿重者寵榮,所以總一海內而整齊萬民也。人體安駕乘,為之金輿錯衡以繁其飾,目好五色,為之黼黻文章以表其能,耳樂鐘磬,為之調諧八音以蕩其心,口甘五味,為之庶羞酸咸以致其美,情好珍善,為之琢磨圭璧以通其意。故大路越席,皮弁布裳,朱弦洞越,大羹玄酒,所以防其淫侈,救其彫敝。是以君臣朝廷尊卑貴賤之序,下及黎庶車輿衣服宮室飲食嫁娶喪祭之分,事有宜適,物有節文。仲尼曰,褅自既灌而往者,吾不欲觀之矣。

周衰,禮廢樂壞,大小相踰,管仲之家,兼備三歸。循法守正者見侮於世,奢溢僭差者謂之顯榮。自子夏,門人之高弟也,猶云出見紛華盛麗而說,入聞夫子之道而樂,二者心戰,未能自決,而況中庸以下,漸漬於失教,被服於成俗乎。孔子曰必也正名,於衛所居不合。仲尼沒後,受業之徒沈湮而不舉,或適齊、楚,或入河海,豈不痛哉。

至秦有天下,悉內六國禮儀,采擇其善,雖不合聖制,其尊君抑臣,朝廷濟濟,依古以來。至于高祖,光有四海,叔孫通頗有所增益減損,大抵皆襲秦故。自天子稱號下至佐僚及宮室官名,少所變改。孝文即位,有司議欲定儀禮,孝文好道家之學,以為繁禮飾貌,無益於治,躬化謂何耳,故罷去之。孝景時,御史大夫晁錯明於世務刑名,數干諫孝景曰,諸侯藩輔,臣子一例,古今之制也。今大國專治異政,不稟京師,恐不可傳後。孝景用其計,而六國畔逆,以錯首名,天子誅錯以解難。事在袁盎語中。是後官者養交安祿而已,莫敢復議。

今上即位,招致儒術之士,令共定儀,十餘年不就。或言古者太平,萬民和喜,瑞應辨至,乃采風俗,定制作。上聞之,制詔御史曰,蓋受命而王,各有所由興,殊路而同歸,謂因民而作,追俗為制也。議者咸稱太古,百姓何望。漢亦一家之事,典法不傳,謂子孫何。化隆者閎博,治淺者褊狹,可不勉與。乃以太初之元改正朔,易服色,封太山,定宗廟百官之儀,以為典常,垂之於後云。

禮由人起。人生有欲,欲而不得則不能無忿,忿而無度量則爭,爭則亂。先王惡其亂,故制禮義以養人之欲,給人之求,使欲不窮於物,物不屈於欲,二者相待而長,是禮之所起也。故禮者養也。稻粱五味,所以養口也,椒蘭芬茝,所以養鼻也,鐘鼓管弦,所以養耳也,刻鏤文章,所以養目也,疏房床笫几席,所以養體也,故禮者養也。

君子既得其養,又好其辨也。所謂辨者,貴賤有等,長少有差,貧富輕重皆有稱也。故天子大路越席,所以養體也,側載臭茝,所以養鼻也,前有錯衡,所以養目也,和鸞之聲,步中武象,驟中韶濩,所以養耳也,龍旂九斿,所以養信也,寢兕持虎,鮫韅彌龍,所以養威也。故大路之馬,必信至教順,然後乘之,所以養安也。孰知夫士出死要節之所以養生也。孰知夫輕費用之所以養財也,孰知夫恭敬辭讓之所以養安也,孰知夫禮義文理之所以養情也。

人茍生之為見,若者必死,茍利之為見,若者必害,怠惰之為安,若者必危,情勝之為安,若者必滅。故聖人一之於禮義,則兩得之矣,一之於情性,則兩失之矣。故儒者將使人兩得之者也,墨者將使人兩失之者也。是儒墨之分。

治辨之極也,彊固之本也,威行之道也,功名之總也。王公由之,所以一天下,臣諸侯也,弗由之,所以捐社稷也。故堅革利兵不足以為勝,高城深池不足以為固,嚴令繁刑不足以為威。由其道則行,不由其道則廢。楚人鮫革犀兕,所以為甲,堅如金石,宛之鉅鐵施,鉆如蜂蠆,輕利剽遫,卒如熛風。然而兵殆於垂涉,唐昧死焉,莊蹻起,楚分而為四參。是豈無堅革利兵哉。其所以統之者非其道故也。汝潁以為險,江漢以為池,阻之以鄧林,緣之以方城。然而秦師至鄢郢,舉若振槁。是豈無固塞險阻哉。其所以統之者非其道故也。紂剖比干,囚箕子,為炮格,刑殺無辜,時臣下懔然,莫必其命。然而周師至,而令不行乎下,不能用其民。是豈令不嚴,刑不陖哉。其所以統之者非其道故也。

古者之兵,戈矛弓矢而已,然而敵國不待試而詘。城郭不集,溝池不掘,固塞不樹,機變不張,然而國晏然不畏外而固者,無他故焉,明道而均分之,時使而誠愛之,則下應之如景響。有不由命者,然後俟之以刑,則民知罪矣。故刑一人而天下服。罪人不尤其上,知罪之在己也。是故刑罰省而威行如流,無他故焉,由其道故也。故由其道則行,不由其道則廢。古者帝堯之治天下也,蓋殺一人刑二人而天下治。傳曰威厲而不試,刑措而不用。

天地者,生之本也,先祖者,類之本也,君師者,治之本也。無天地惡生。無先祖惡出。無君師惡治。三者偏亡,則無安人。故禮,上事天,下事地,尊先祖而隆君師,是禮之三本也。

故王者天太祖,諸侯不敢懷,大夫士有常宗,所以辨貴賤。貴賤治,得之本也。

郊疇乎天子,社至乎諸侯,函及士大夫,所以辨尊者事尊,卑者事卑,宜鉅者鉅,宜小者小。

故有天下者事七世,有一國者事五世,有五乘之地者事三世,有三乘之地者事二世,有特牲而食者不得立宗廟,所以辨積厚者流澤廣,積薄者流澤狹也。

大饗上玄尊,俎上腥魚,先大羹,貴食飲之本也。

大饗上玄尊而用薄酒,食先黍稷而飯稻粱,祭嚌先大羹而飽庶羞,貴本而親用也。貴本之謂文,親用之謂理,兩者合而成文,以歸太一,是謂大隆。

故尊之上玄尊也,俎之上腥魚也,豆之先大羹,一也。
		
利爵弗啐也,成事俎弗嘗也,三侑之弗食也。
		
大昏之未廢齊也,大廟之未內尸也,始絕之未小斂,一也。
		
大路之素幬也,郊之麻絻,喪服之先散麻,一也。
		
三年哭之不反也,清廟之歌一倡而三嘆,縣一鐘尚拊膈,朱弦而通越,一也。

凡禮始乎脫,成乎文,終乎稅。故至備,情文俱盡,其次,情文代勝,其下,復情以歸太一。

天地以合,日月以明,四時以序,星辰以行,江河以流,萬物以昌,好惡以節,喜怒以當。以為下則順,以為上則明。

太史公曰,至矣哉。立隆以為極,而天下莫之能益損也。本末相順,終始相應,至文有以辨,至察有以說。天下從之者治,不從者亂,從之者安,不從者危。小人不能則也。

禮之貌誠深矣,堅白同異之察,入焉而弱。其貌誠大矣,擅作典制褊陋之說,入焉而望。其貌誠高矣,暴慢恣睢,輕俗以為高之屬,入焉而隊。故繩誠陳,則不可欺以曲直,衡誠縣,則不可欺以輕重,規矩誠錯,則不可欺以方員,君子審禮,則不可欺以詐偽。故繩者,直之至也,衡者,平之至也,規矩者,方員之至也,禮者,人道之極也。然而不法禮者不足禮,謂之無方之民,法禮足禮,謂之有方之士。禮之中,能思索,謂之能慮,能慮勿易,謂之能固。能慮能固,加好之焉,聖矣。天者,高之極也,地者,下之極也,日月者,明之極也,無窮者,廣大之極也,聖人者,道之極也。

以財物為用,以貴賤為文,以多少為異,以隆殺為要。文貌繁,情欲省,禮之隆也,文貌省,情欲繁,禮之殺也,文貌情欲相為內外表裏,并行而雜,禮之中流也。君子上致其隆,下盡其殺,而中處其中。步驟馳騁廣騖不外,是以君子之性守宮庭也。人域是域,士君子也。外是,民也。於是中焉,房皇周浹,曲直得其次序,聖人也。故厚者,禮之積也,大者,禮之廣也,高者,禮之隆也,明者,禮之盡也。