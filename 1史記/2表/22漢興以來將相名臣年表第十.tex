\chapter{漢興以來將相名臣年表第十}

\biao[\hline 紀年 & 大事紀 & 相位 & 將位 & 御史大夫位 \\ \hline
\multicolumn{5}{}{}]
{|p{5em}|p{15em}|p{12em}|p{21em}|p{14em}|}
{\hline
紀年 & 大事紀 & 相位 & 將位 & 御史大夫位 \\ \hline
高皇帝元年 & 春,沛公為漢王,之南鄭。秋,還定雍。 & 一。丞相蕭何守漢中。 &  & 御史大夫周苛守滎陽。 \\ \hline
二 & 春,定塞、翟、魏、河南、韓、殷國。夏,伐項籍,至彭城。立太子。還據滎陽。 & 二。守關中。 & 一。太尉長安侯盧綰。 &  \\ \hline
三 & 魏豹反。使韓信別定魏,伐趙。楚圍我滎陽。 & 三 & 二 &  \\ \hline
四 & 使韓信別定齊及燕,太公自楚歸,與楚界洪渠。 & 四 & 三。周苛守滎陽,死。 & 御史大夫汾陰侯周昌。 \\ \hline
五 & 冬,破楚垓下,殺項籍。春,王踐皇帝位定陶。入都關中。 & 五。罷太尉官。 & 四。後九月,綰為燕王。 &  \\ \hline
六 & 尊太公為太上皇。劉仲為代王。立大市。更命咸陽曰長安。 & 六。封為酇侯。張蒼為計相。 &  &  \\ \hline
七 & 長樂宮成,自櫟陽徙長安。伐匈奴,匈奴圍我平城。 & 七 &  &  \\ \hline
八 & 擊韓信反虜於趙城。貫高作亂,明年覺,誅之。匈奴攻代王,代王棄國亡,廢為郃陽侯。 & 八 &  &  \\ \hline
九 & 未央宮成,置酒前殿,太上皇輦上坐,帝奉玉卮上壽,曰,始常以臣不如仲力,今臣功孰與仲多。太上皇笑,殿上稱萬歲。徙齊田,楚昭、屈、景于關中。 & 九。遷為相國。 &  & 御史大夫昌為趙丞相。 \\ \hline
十 & 太上皇崩。陳豨反代地。 & 十 &  & 御史大夫江邑侯趙堯。 \\ \hline
十一 & 誅淮陰、彭越。黥布反。 & 十一 & 周勃為太尉。攻代。後官省。 &  \\ \hline
十二 & 冬,擊布。還過沛。夏,上崩,葬長陵。 & 十二 &  &  \\ \hline
孝惠元年 & 趙隱王如意死。始作長安城西北方。除諸侯丞相為相。 & 十三 &  &  \\ \hline
二 & 楚元王、齊悼惠王來朝。七月辛未,何薨。 & 十四。七月癸巳,齊相平陽侯曹參為相國。 &  &  \\ \hline
三 & 初作長安城。蜀湔氐反。擊之。 & 二 &  &  \\ \hline
四 & 三月甲子,赦,無所復作。 & 三 &  &  \\ \hline
五 & 為高祖立廟於沛城成,置歌兒一百二十人。八月乙丑,參卒。 & 四 &  &  \\ \hline
六 & 七月,齊悼惠王薨。立太倉、西市。 & 一。十月己巳,安國侯王陵為右丞相。曲逆侯陳平為左丞相。 & 堯抵罪。 & 廣阿侯任敖為御史大夫。 \\ \hline
七 & 上崩。大臣用張辟彊計,呂氏權重,以呂台為呂王。立少帝。九月辛巳,葬安陵。 & 二 &  &  \\ \hline
高后元年 & 王孝惠諸子。置孝悌力田。 & 三。十一月甲子,徙平為右丞相。辟陽侯審食其為左丞相。 &  &  \\ \hline
二 & 十二月,呂王台薨,子嘉代立為呂王。行八銖錢。 & 四。平。二。食其。 &  & 平陽侯曹窋為御史大夫。 \\ \hline
三 &  & 五。三 &  &  \\ \hline
四 & 廢少帝,更立常山王弘為帝。 & 六。四。置太尉官。 & 一。絳侯周勃為太尉。 &  \\ \hline
五 & 八月,淮陽王薨,以其弟壺關侯武為淮陽王。令戍卒歲更。 & 七。五 & 二 &  \\ \hline
六 & 以呂產為呂王。四月丁酉,赦天下。晝昏。 & 八。六 & 三 &  \\ \hline
七 & 趙王幽死,以呂祿為趙王。梁王徙趙,自殺。 & 九。七 & 四 &  \\ \hline
八 & 七月,高后崩。九月,誅諸呂。後九月,代王至,踐皇帝位。後九月,食其免相。 & 十。七月辛巳,為帝太傅。九月壬戌,復為丞相。八 & 五。隆慮侯灶為將軍,擊南越。 & 御史大夫蒼。 \\ \hline
孝文元年 & 除收孥相坐律。立太子。賜民爵。 & 十一。十一月辛巳,平徙為左丞相。太尉絳侯周勃為右丞相。 & 六。勃為相,潁陰侯灌嬰為太尉。 &  \\ \hline
二 & 除誹謗律。皇子武為代王,參為太原王,揖為梁王。十月,丞相平薨。 & 一。十一月乙亥,絳侯勃復為丞相。 & 一 &  \\ \hline
三 & 徙代王武為淮陽王。上幸太原。濟北王反。匈奴大入上郡。以地盡與太原,太原更號代。十一月壬子,勃免相,之國。 & 一。十二月乙亥,太尉潁陰侯灌嬰為丞相。罷太尉官。 & 二。棘蒲侯陳武為大將軍,擊濟北。昌侯盧卿、共侯盧罷師、甯侯鸱、深澤侯將夜皆為將軍,屬武祁侯賀,將兵屯滎陽。 &  \\ \hline
四 & 十二月己巳,嬰卒。 & 一。正月甲午,御史大夫北平侯張蒼為丞相。 & 安丘侯張說為將軍,擊胡,出代。 & 關中侯申屠嘉為御史大夫。 \\ \hline
五 & 除錢律,民得鑄錢 &  &  &  \\ \hline
六 & 廢淮南王,遷嚴道,道死雍。 & 三 &  &  \\ \hline
七 & 四月丙子,初置南陵。 & 四 &  &  \\ \hline
八 & 太僕汝陰侯滕公卒。 & 五 &  &  \\ \hline
九 & 溫室鐘自鳴。以芷陽鄉為霸陵。 & 六 &  & 御史大夫敬。 \\ \hline
十 & 諸侯王皆至長安。 & 七 &  &  \\ \hline
十一 & 上幸代。地動。 & 八 &  &  \\ \hline
十二 & 河決東郡金隄。徙淮陽王為梁王。 & 九 &  &  \\ \hline
十三 & 除肉刑及田租稅律、戍卒令。 & 十 &  &  \\ \hline
十四 & 匈奴大入蕭關,發兵擊之,及屯長安旁。 & 十一 & 成侯董赤、內史欒布、昌侯盧卿、隆慮侯灶、甯侯鸱皆為將軍,東陽侯張相如為大將軍,皆擊匈奴。中尉周舍、郎中令張武皆為將軍,屯長安旁。 &  \\ \hline
十五 & 黃龍見成紀。上始郊見雍五帝。 & 十二 &  &  \\ \hline
十六 & 上始郊見渭陽五帝。 & 十三 &  &  \\ \hline
後元年 & 新垣平詐言方士,覺,誅之。 & 十四 &  &  \\ \hline
二 & 匈奴和親。地動。八月戊辰,蒼免相。 & 十五。八月庚午,御史大夫申屠嘉為丞相,封故安侯。 &  & 御史大夫青。 \\ \hline
三 & 置谷口邑。 & 二 &  &  \\ \hline
四 &  & 三 &  &  \\ \hline
五 & 上幸雍。 & 四 &  &  \\ \hline
六 & 匈奴三萬人入上郡,二萬人入雲中。 & 五 & 以中大夫令免為車騎將軍,軍飛狐,故楚相蘇意為將軍,軍句注。將軍張武屯北地,河內守周亞夫為將軍,軍細柳,宗正劉禮軍霸上,祝茲侯徐厲軍棘門,以備胡。數月,胡去,亦罷。 &  \\ \hline
七 & 六月己亥,孝文皇帝崩。丁未,太子立。民出臨三日,葬霸陵。 & 六 & 中尉亞夫為車騎將軍,郎中令張武為復土將軍,屬國捍為將屯將軍。詹事戎奴為車騎將軍,侍太后。 &  \\ \hline
孝景元年 & 立孝文皇帝廟,郡國為太宗廟。 & 七。置司徒官。 &  &  \\ \hline
二 & 立皇子德為河閒王,閼為臨江王,餘為淮陽王,非為汝南王,彭祖為廣川王,發為長沙王。四月中,孝文太后崩。嘉卒。 & 八。開封侯陶青為丞相。 &  & 御史大夫錯。 \\ \hline
三 & 吳楚七國反,發兵擊,皆破之。皇子端為膠西王,勝為中山王。 & 二。置太尉官。 & 中尉條侯周亞夫。為太尉,擊吳楚,曲周侯酈寄為將軍,擊趙,竇嬰為大將軍,屯滎陽,欒布為將軍,擊齊。 &  \\ \hline
四 & 立太子。 & 三 & 二。太尉亞夫。 & 御史大夫蚡。 \\ \hline
五 & 置陽陵邑。丞相北平侯張蒼卒。 & 四 & 三 &  \\ \hline
六 & 徙廣川王彭祖為趙王。 & 五 & 四 & 御史大夫陽陵侯岑邁。 \\ \hline
七 & 廢太子榮為臨江王。四月丁巳,膠東王立為太子。青罷相。 & 六月乙巳,太尉條侯亞夫為丞相。罷太尉官。 & 五。遷為丞相。 & 御史大夫舍。 \\ \hline
中元年 &  & 二 &  &  \\ \hline
二 & 皇子越為廣川王,寄為膠東王。 & 三 &  &  \\ \hline
三 & 皇子乘為清河王。亞夫免相。 & 四。御史大夫桃侯劉舍為丞相。 &  & 御史大夫綰。 \\ \hline
四 & 臨江王徵,自殺,葬藍田,燕數萬為銜士置冢上。 & 二 &  &  \\ \hline
五 & 皇子舜為常山王。 & 三 &  &  \\ \hline
六 & 梁孝王武薨。分梁為五國,王諸子,子買為梁王,明為濟川王,彭離為濟東王,定為山陽王,不識為濟陰王。 & 四 &  &  \\ \hline
後元年 & 五月,地動。七月乙巳,日蝕。舍免相。 & 五。八月壬辰,御史大夫建陵侯衛綰為丞相。 &  & 御史大夫不疑。 \\ \hline
二 &  & 二 & 六月丁丑,御史大夫岑邁卒。 &  \\ \hline
三 & 正月甲子,孝景皇帝崩。二月丙子,太子立。 & 三 &  &  \\ \hline
孝武建元元年 & 綰免相。 & 四。魏其侯竇嬰為丞相。置太尉。 & 武安侯田蚡為太尉。 & 御史大夫抵。 \\ \hline
二 & 置茂陵。嬰免相。 & 二月乙未,太常柏至侯許昌為丞相。蚡免太尉。罷太尉官。 &  & 御史大夫趙綰。 \\ \hline
三 & 東甌王廣武侯望率其眾四萬餘人來降,處廬江郡。 & 二 &  &  \\ \hline
四 &  & 三 &  & 御史大夫青翟。 \\ \hline
五 & 行三分錢。 & 四 &  &  \\ \hline
六 & 正月,閩越王反。孝景太后崩。昌免相。 & 五。六月癸巳,武安侯田蚡為丞相。 & 青翟為太子太傅。 & 御史大夫安國。 \\ \hline
元光元年 &  & 二 &  &  \\ \hline
二 & 帝初之雍,郊見五畤。 & 三 & 夏,御史大夫韓安國為護軍將軍,衛尉李廣為驍騎將軍,太僕公孫賀為輕車將軍,大行王恢為將屯將軍,太中大夫李息為材官將軍,篡單于馬邑,不合,誅恢。 &  \\ \hline
三 & 五月丙子,河決于瓠子。 & 四 &  &  \\ \hline
四 & 十二月丁亥,地動。蚡卒。 & 五。平棘侯薛澤為丞相。 &  & 御史大夫歐。 \\ \hline
五 & 十月,族灌夫家,棄魏其侯市。 & 二 &  &  \\ \hline
六 & 南夷始置郵亭。 & 三 & 太中大夫衛青為車騎將軍,出上谷,衛尉李廣為驍騎將軍,出鴈門,大中大夫公孫敖為騎將軍,出代,太僕公孫賀為輕車將軍,出雲中,皆擊匈奴。 &  \\ \hline
元朔元年 & 衛夫人立為皇后。 & 四 & 車騎將軍青出雁門,擊匈奴。衛尉韓安國為將屯將軍,軍代,明年,屯漁陽卒。 &  \\ \hline
二 &  & 五 & 春,車騎將軍衛青出雲中,至高闕,取河南地。 &  \\ \hline
三 & 匈奴殺代太守友。 & 六 &  & 御史大夫弘。 \\ \hline
四 & 匈奴入定襄、代、上郡。 & 七 &  &  \\ \hline
五 & 匈奴殺代都尉朱英。澤免相。 & 八。十一月乙丑,御史大夫公孫弘為丞相,封平津侯。 & 春,長平侯衛青為大將軍,擊右賢。衛尉蘇建為游擊將軍,屬青。左內史李沮為強弩將軍,太僕賀為車騎將軍,代相李蔡為輕車將軍,岸頭侯張次公為將軍,大行息為將軍,皆屬大將軍,擊匈奴。 &  \\ \hline
六 &  & 二 & 大將軍青再出定襄擊胡。合騎侯公孫敖為中將軍,太僕賀為左將軍,郎中令李廣為後將軍。翕侯趙信為前將軍,敗降匈奴。衛尉蘇建為右將軍,敗,身脫。左內史沮為彊弩將軍。皆屬青。 &  \\ \hline
元狩元年 & 十月中,淮南王安、衡山王賜謀反,皆自殺,國除。 & 三 &  & 御史大夫蔡。 \\ \hline
二 & 匈奴入鴈門、代郡。江都王建反。膠東王子慶立為六安王。弘卒。 & 四。御史大夫樂安侯李蔡為丞相。 & 冠軍侯霍去病為驃騎將軍,擊胡,至祁連,合騎侯敖為將軍。出北地,博望侯張騫、郎中令李廣為將軍,出右北平。 & 御史大夫湯。 \\ \hline
三 & 匈奴入右北平、定襄。 & 二 &  &  \\ \hline
四 &  & 三 & 大將軍青出定襄,郎中令李廣為前將軍,太僕公孫賀為左將軍,主爵趙食其為右將軍,平陽侯曹襄為後將軍,擊單于。 &  \\ \hline
五 & 蔡坐侵園堧,自殺。 & 四。太子少傅武彊侯莊青翟為丞相。 &  &  \\ \hline
六 & 四月乙巳,皇子閎為齊王,旦為燕王,胥為廣陵王。 & 二 &  &  \\ \hline
元鼎元年 &  & 三 &  &  \\ \hline
二 & 青翟有罪,自殺。 & 四。太子太傅高陵侯趙周為丞相。 & 湯有罪,自殺。 & 御史大夫慶。 \\ \hline
三 &  & 二 &  &  \\ \hline
四 & 立常山憲王子平為真定王,商為泗水王。六月中,河東汾陰得寶鼎。 & 三 &  &  \\ \hline
五 & 三月中,南越相嘉反,殺其王及漢使者。八月,周坐酎金,自殺。 & 四。九月辛巳,御史大夫石慶為丞相,封牧丘侯。 & 衛尉路博德為伏波將軍,出桂陽,主爵楊僕為樓船將軍,出豫章,皆破南越。 &  \\ \hline
六 & 十二月,東越反。 & 二 & 故龍镪侯韓說為橫海將軍,出會稽,樓船將軍楊僕出豫章,中尉王溫舒出會稽,皆破東越。 & 御史大夫式。 \\ \hline
元封元年。 &  & 三 &  & 御史大夫寬。 \\ \hline
二 &  & 四 & 秋,樓船將軍楊僕、左將軍荀彘出遼東,擊朝鮮。 &  \\ \hline
三 &  & 五 &  &  \\ \hline
四 &  & 六 &  &  \\ \hline
五 &  & 七 &  &  \\ \hline
六 &  & 八 &  &  \\ \hline
太初元年 & 改曆,以正月為歲首。 & 九 &  &  \\ \hline
二 & 正月戊寅,慶卒。 & 十。三月丁卯,太僕公孫賀為丞相,封葛繹侯。 &  &  \\ \hline
三 &  & 二 &  & 御史大夫延廣。 \\ \hline
四 &  & 三 &  &  \\ \hline
天漢元年 &  & 四 &  & 御史大夫卿。 \\ \hline
二 &  & 五 &  &  \\ \hline
三 &  & 六 &  & 御史大夫周。 \\ \hline
四 &  & 七 & 春,貳師將軍李廣利出朔方,至余吾水上,游擊將軍韓說出五原,因杅將軍公孫敖,皆擊匈奴。 &  \\ \hline
太始元年 &  & 八 &  &  \\ \hline
二 &  & 九 &  &  \\ \hline
三 &  & 十 &  & 御史大夫勝之。 \\ \hline
四 &  & 十一 &  &  \\ \hline
征和元年 & 冬,賀坐為蠱死。 & 十二 &  &  \\ \hline
二 & 七月壬午,太子發兵,殺游擊將軍說、使者江充。 & 三月丁巳,涿郡太守劉屈氂為丞相,封彭城侯。 &  & 御史大夫成。 \\ \hline
三 & 六月,劉屈氂因蠱斬。 & 二 & 春,貳師將軍李廣利出朔方,以兵降胡。重合侯莽通出酒泉,御史大夫商丘成出河西,擊匈奴。 &  \\ \hline
四 &  & 六月丁巳,大鴻臚田千秋為丞相,封富民侯。 &  &  \\ \hline
後元元年 &  & 二 &  &  \\ \hline
二 &  & 三 & 二月己巳,光祿大夫霍光為大將軍,博陸侯,都尉金日磾為車騎將軍,秺侯,太僕安陽侯上官舛為大將軍。 &  \\ \hline
孝昭始元元年 &  & 四。九月,日磾卒。 &  &  \\ \hline
二 &  & 五 &  &  \\ \hline
三 &  & 六 &  &  \\ \hline
四 &  & 七 & 三月癸酉,衛尉王莽為左將軍,騎都尉上官安為車騎將軍。 &  \\ \hline
五 &  & 八 &  &  \\ \hline
六 &  & 九 &  &  \\ \hline
元鳳元年 &  & 十 & 九月庚午,光祿勳張安世為右將軍。 & 御史大夫訢。 \\ \hline
二 &  & 十一 &  &  \\ \hline
三 &  & 十二 & 十二月庚寅,中郎將范明友為度遼將軍,擊烏丸。 &  \\ \hline
四 & 三月甲戌,千秋卒。 & 三月乙丑,御史大夫王訢為丞相,封富春侯。 &  & 御史大夫楊敞。 \\ \hline
五 & 十二月庚戌,訢卒。 & 二 &  &  \\ \hline
六 &  & 十一月乙丑,御史大夫楊敞為丞相,封安平侯。 & 九月庚寅,衛尉平陵侯范明友為度遼將軍,擊烏丸。 &  \\ \hline
元平元年 & 敞卒。 & 九月戊戌,御史大夫蔡義為丞相,封陽平侯。 & 四月甲申,光祿大夫龍镪侯韓曾為前將軍。五月丁酉,水衡都尉趙充國為後將軍,右將軍張安世為車騎將軍。 & 御史大夫昌水侯田廣明。 \\ \hline
孝宣本始元年 &  & 二 &  &  \\ \hline
二 &  & 三 & 七月庚寅,御史大夫田廣明為祁連將軍,龍镪侯韓曾為後將軍,營平侯趙充國為蒲類將軍,度遼將軍平陵侯范明友為雲中太守,富民侯田順為虎牙將軍,皆擊匈奴。 &  \\ \hline
三 & 三月戊子,皇后崩。六月乙丑,義薨。 & 六月甲辰,長信少府韋賢為丞相,封扶陽侯。田廣明、田順擊胡還,皆自殺。充國奪將軍印。 &  & 御史大夫魏相。 \\ \hline
四 & 十月乙卯,立霍后。 & 二 &  &  \\ \hline
地節元年 &  & 三 &  &  \\ \hline
二 &  & 四。三月庚午,將軍光卒。 & 二月丁卯,侍中、中郎將霍禹為右將軍。 &  \\ \hline
三 & 立太子。五月甲申,賢老,賜金百斤。 & 六月壬辰,御史大夫魏相為丞相,封高平侯。 & 七月,安世為大司馬、衛將軍。禹為大司馬。 & 御史大夫邴吉。 \\ \hline
四 &  & 二。七月壬寅,禹腰斬。 &  &  \\ \hline
元康元年 &  & 三 &  &  \\ \hline
二 &  & 四 &  &  \\ \hline
三 &  & 五 &  &  \\ \hline
四 &  & 六。八月丙寅,安世卒。 &  &  \\ \hline
神爵元年 & 上郊甘泉太畤、汾陰后土。 & 七 & 四月,樂成侯許延壽為強弩將軍。後將軍充國擊羌。酒泉太守辛武賢為破羌將軍。韓曾為大司馬、車騎將軍。 &  \\ \hline
二 & 上郊雍五畤。祋祤出寶璧玉器。 & 八 &  &  \\ \hline
三 & 三月,相卒。 & 四月戊戌,御史大夫邴吉為丞相,封博陽侯。 &  & 御史大夫望之。 \\ \hline
四 &  & 二 &  &  \\ \hline
五鳳元年 &  & 三 &  &  \\ \hline
二 &  & 四。五月己丑,曾卒。 & 五月,延壽為大司馬、車騎將軍。 & 御史大夫霸。 \\ \hline
三 & 正月,吉卒。 & 三月壬申,御史大夫黃霸為丞相,封建成侯。 &  & 御史大夫延年。 \\ \hline
四 &  & 二 &  &  \\ \hline
甘露元年 &  & 三。三月丁未,延壽卒。 &  &  \\ \hline
二 & 赦殊死,賜高年及鰥寡孤獨帛,女子牛酒。 & 四 &  & 御史大夫定國。 \\ \hline
三 & 三月己丑,霸薨。 & 七月丁巳,御史大夫于定國為丞相,封西平侯。 &  & 太僕陳萬年為御史大夫。 \\ \hline
四 &  & 二 &  &  \\ \hline
黃龍元年 &  & 三 & 樂陵侯史子長為大司馬、車騎將軍。太子太傅蕭望之為前將軍。 &  \\ \hline
孝元初元元年 &  & 四 &  &  \\ \hline
二 &  & 五 &  &  \\ \hline
三 &  & 六 & 十二月,執金吾馮奉世為右將軍。 &  \\ \hline
四 &  & 七 &  &  \\ \hline
五 &  & 八 & 二月丁巳,平恩侯許嘉為左將軍。 & 中少府貢禹為御史大夫。十二月丁未,長信少府薛廣德為御史大夫。 \\ \hline
永光元年 & 十月戊寅,定國免。 & 九。七月,子長免,就第。 & 九月,衛尉平昌侯王接為大司馬、車騎將軍。二月,廣德免。 & 七月,太子太傅韋玄成為御史大夫。 \\ \hline
二 & 三月壬戌朔,日蝕。 & 二月丁酉,御史大夫韋玄成為丞相,封扶陽侯。丞相賢子。 & 七月,太常任千秋為奮武將軍,擊西羌,雲中太守韓次君為建威將軍,擊羌。後不行。 & 二月丁酉,右扶風鄭弘為御史大夫。 \\ \hline
三 &  & 二 & 右將軍平恩侯許嘉為車騎將軍,侍中、光祿大夫樂昌侯王商為右將軍,右將軍馮奉世為左將軍。 &  \\ \hline
四 &  & 三 &  &  \\ \hline
五 &  & 四 &  &  \\ \hline
建昭元年 &  & 五 &  &  \\ \hline
二 &  & 六 & 弘免。 & 光祿勳匡衡為御史大夫。 \\ \hline
三 & 六月甲辰,玄成薨。 & 七月癸亥,御史大夫匡衡為丞相,封樂安侯。 &  & 衛尉繁延壽為御史大夫。 \\ \hline
四 &  & 二 &  &  \\ \hline
五 &  & 三 &  &  \\ \hline
竟寧元年 &  & 四 & 六月己未,衛尉楊平侯王鳳為大司馬、大將軍。延壽卒。 & 三月丙寅,太子少傅張譚為御史大夫。 \\ \hline
孝成建始元年 &  & 五 &  &  \\ \hline
二 &  & 六 &  &  \\ \hline
三 & 十二月丁丑,衡免。 & 七。八月癸丑,遣光祿勳詔嘉上印綬免,賜金二百斤。 & 十月,右將軍樂昌侯王商為光祿大夫、右將軍,執金吾弋陽侯任千秋為右將軍。譚免。 & 廷尉尹忠為御史大夫。 \\ \hline
四 &  & 三月甲申,右將軍樂昌侯王商為右丞相。 & 任千秋為左將軍,長樂衛尉史丹為右將軍。十月己亥,尹忠自剌殺。 & 少府張忠為御史大夫。 \\ \hline
河平元年 &  & 二 &  &  \\ \hline
二 &  & 三 &  &  \\ \hline
三 &  & 四 & 十月辛卯,史丹為左將軍,太僕平安侯王章為右將軍。 &  \\ \hline
四 & 四月壬寅,丞相商免。 & 六月丙午,諸吏散騎光祿大夫張禹為丞相。 &  &  \\ \hline
陽朔元年 &  & 二 &  &  \\ \hline
二 &  & 三 & 張忠卒。 & 六月,太僕王音為御史大夫。 \\ \hline
三 &  &  & 九月甲子,御史大夫王音為車騎將軍。 & 十月乙卯,光祿勳于永為御史大夫。 \\ \hline
四 &  & 七月乙丑,右將軍光祿勳平安侯王章卒。 & 閏月壬戌,永卒。 &  \\ \hline
鴻嘉元年 & 三月,禹卒。 & 四月庚辰,薛宣為丞相。 &  &  \\ \hline
}