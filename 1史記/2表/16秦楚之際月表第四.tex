\chapter{秦楚之際月表第四}
			
太史公讀秦楚之際,曰,初作難,發於陳涉,虐戾滅秦,自項氏,撥亂誅暴,平定海內,卒踐帝祚,成於漢家。五年之閒,號令三嬗。自生民以來,未始有受命若斯之亟也。

昔虞、夏之興,積善累功數十年,德洽百姓,攝行政事,考之于天,然後在位。湯、武之王,乃由契、后稷修仁行義十餘世,不期而會孟津八百諸侯,猶以為未可,其后乃放弒。秦起襄公,章於文、繆,獻、孝之後,稍以蠶食六國,百有餘載,至始皇乃能并冠帶之倫。以德若彼,用力如此,蓋一統若斯之難也。

秦既稱帝,患兵革不休,以有諸侯也,於是無尺土之封,墮壞名城,銷鋒鏑,鉏豪桀,維萬世之安。然王跡之興,起於閭巷,合從討伐,軼於三代,鄉秦之禁,適足以資賢者為驅除難耳。故憤發其所為天下雄,安在無土不王。此乃傳之所謂大聖乎。豈非天哉,豈非天哉。非大聖孰能當此受命而帝者乎。

\biao[\hline 秦 & 楚 & 項 & 趙 & 齊 & 漢 & 燕 & 魏 & 韓 \\ \hline
\multicolumn{9}{}{}]
{|p{4em}|p{7em}|p{11em}|p{8em}|p{10em}|p{10em}|p{6em}|p{5em}|p{5em}|}{
\hline
秦 & 楚 & 項 & 趙 & 齊 & 漢 & 燕 & 魏 & 韓 \\ \hline
二世元年 &  &  &  &  &  &  &  &  \\ \hline
七月 & 楚隱王陳涉起兵入秦。 &  &  &  &  &  &  &  \\ \hline
八月 & 二。葛嬰為涉徇九江,立襄彊為楚王。 &  & 武臣始至邯鄲,自立為趙王,始。 &  &  &  &  &  \\ \hline
九月。楚兵至戲。 & 三。周文兵至戲,敗。而葛嬰聞涉王,即殺彊。 & 項梁號武信君。 & 二 & 齊王田儋始。儋,狄人。諸田宗彊。從弟榮,榮弟橫。 & 沛公初起。 & 韓廣為趙略地至薊,自立為燕王始。 & 魏王咎始。咎在陳,不得歸國。 &  \\ \hline
二年十月 & 四。誅葛嬰。 & 二 & 三 & 二。儋之起,殺狄令自王。 & 二。擊胡陵、方與,破秦監軍。 & 二 & 二 &  \\ \hline
十一月 & 五。周文死。 & 三 & 四。李良殺武臣,張耳、陳餘走。 & 三 & 三。殺泗水守。拔薛西。周市東略地豐沛閒。 & 三 & 三。齊、趙共立周市,市不肯,曰必立魏咎云。 &  \\ \hline
十二月 & 六。陳涉死。 & 四 &  & 四 & 四。雍齒叛沛公,以豐降魏。沛公還攻豐,不能下。 & 四 & 四。咎自陳歸,立。 &  \\ \hline
端月 & 楚王景駒始,秦嘉立之。 & 五。涉將召平矯拜項梁為楚柱國,急西擊秦。 & 趙王歇始,張耳、陳餘立之。 & 五。讓景駒以擅自王不請我。 & 五。沛公聞景駒王在留,往從,與擊秦軍碭西。 & 五 & 五。章邯已破涉,圍咎臨濟。 &  \\ \hline
二月 & 二。嘉為上將軍。 & 六。梁渡江,陳嬰、黥布皆屬。 & 二 & 六。景駒使公孫慶讓齊,誅慶。 & 六。攻下碭,收得兵六千,與故凡九千人。 & 六 & 六 &  \\ \hline
三月 & 三 & 七 & 三 & 七 & 七。攻拔下邑,遂擊豐,豐不拔。聞項梁兵眾,往請擊豐。 & 七 & 七 &  \\ \hline
四月 & 四 & 八。梁擊殺景駒、秦嘉,遂入薛,兵十餘萬眾。 & 四 & 八 & 八。沛公如薛見項梁,梁益沛公卒五千,擊豐,拔之。雍齒奔魏。 & 八 & 八。臨濟急,周市如齊、楚請救。 &  \\ \hline
五月 &  & 九 & 五 & 九 & 九 & 九 & 九 &  \\ \hline
六月 & 楚懷王始,都盱台,故懷王孫,梁立之。 & 十。梁求楚懷王孫,得之民閒,立為楚王。 & 六 & 十。儋救臨濟,章邯殺田儋。榮走東阿。 & 十。沛公如薛,共立楚懷王。 & 十 & 十。咎自殺,臨濟降秦。 & 韓王成始。 \\ \hline
七月 & 二。陳嬰為柱國。 & 十一。天大雨,三月不見星。 & 七 & 齊立田假為王,秦急圍東阿。 & 十一。沛公與項羽北救東阿,破秦軍濮陽,東屠城陽。 & 十一 & 咎弟豹走東阿。 & 二 \\ \hline
八月 & 三 & 十二。救東阿,破秦軍,乘勝至定陶,項梁有驕色。 & 八 & 楚救榮,得解歸,逐田假,立儋子市為齊王,始。 & 十二。沛公與項羽西略地,斬三川守李由於雍丘。 & 十二 &  & 三 \\ \hline
九月 & 四。徙都彭城。 & 十三。章邯破殺項梁於定陶,項羽恐,還軍彭城。 & 九 & 二。田假走楚,楚趨齊救趙。田榮以假故,不肯,謂楚殺假乃出兵。項羽怒田榮。 & 十三。沛公聞項梁死,還軍,從懷王,軍於碭。 & 十三 & 魏豹自立為魏王,都平陽,始。 & 四 \\ \hline
後九月。 & 五。拜宋義為上將軍。 & 懷王封項羽於魯,為次將,屬宋義,北救趙。 & 十。秦軍圍歇鉅鹿,陳餘出收兵。 & 三 & 十四。懷王封沛公為武安侯,將碭郡兵西,約先至咸陽王之。 & 十四 & 二 & 五 \\ \hline
三年十月 & 六 & 二 & 十一。章邯破邯鄲,徙其民於河內。 & 四。齊將田都叛榮,往助項羽救趙。 & 十五。攻破東郡尉及王離軍於成武南。 & 十五。使將臧荼救趙。 & 三 & 六 \\ \hline
十一月 & 七。拜籍上將軍。 & 三。羽矯殺宋義,將其兵渡河救鉅鹿。 & 十二 & 五 & 十六 & 十六 & 四 & 七 \\ \hline
十二月 & 八 & 四。大破秦軍鉅鹿下,諸侯將皆屬項羽。 & 十三。楚救至,秦圍解。 & 六。故齊王建孫田安下濟北,從項羽救趙。 & 十七。至栗,得皇訢、武蒲軍。與秦軍戰,破之。 & 十七 & 五。豹救趙。 & 八 \\ \hline
端月 & 九 & 五。虜秦將王離。 & 十四。張耳怒陳餘,棄將印去。 & 七 & 十八 & 十八 & 六 & 九 \\ \hline
二月 & 十 & 六。攻破章邯,章邯軍卻。 & 十五 & 八 & 十九。得彭越軍昌邑,襲陳留。用酈食其策,軍得積粟。 & 十九 & 七 & 十 \\ \hline
三月 & 十一 & 七 & 十六 & 九 & 二十。攻開封,破秦將楊熊,熊走滎陽,秦斬熊以徇。 & 二十 & 八 & 十一 \\ \hline
四月 & 十二 & 八。楚急攻章邯,章邯恐,使長史欣歸秦請兵,趙高讓之。 & 十七 & 十 & 二十一。攻潁陽,略韓地,北絕河津。 & 二十一 & 九 & 十二 \\ \hline
五月 & 二年一月 & 九。趙高欲誅欣,欣恐,亡走,告章邯謀叛秦。 & 十八 & 十一 & 二十二 & 二十二 & 十 & 十三 \\ \hline
六月 & 二 & 十。章邯與楚約降,未定,項羽許而擊之。 & 十九 & 十二 & 二十三。攻南陽守齮,破之陽城郭東。 & 二十三 & 十一 & 十四 \\ \hline
七月 & 三 & 十一。項羽與章邯期殷虛,章邯等已降,與盟,以邯為雍王。 & 二十 & 十三 & 二十四。降下南陽,封其守齮。 & 二十四 & 十二 & 十五。申陽下河南,降楚。 \\ \hline
八月。趙高殺二世。 & 四 & 十二。以秦降都尉翳、長史欣為上將,將秦降軍。 & 二十一。趙王歇留國。陳餘亡居南皮。 & 十四 & 二十五。攻武關,破之。 & 二十五 & 十三 & 十六 \\ \hline
九月子嬰為王。 & 五 & 十三 & 二十二 & 十五 & 二十六。攻下嶢及藍田。以留侯策,不戰皆降。 & 二十六 & 十四 & 十七 \\ \hline
十月 & 六 & 十四。項羽將諸侯兵四十餘萬,行略地,西至於河南。 & 二十三。張耳從楚西入秦。 & 十六 & 二十七。漢元年,秦王子嬰降。沛公入破咸陽,平秦,還軍霸上,待諸侯約。 & 二十七 & 十五。從項羽略地,遂入關 & 十八 \\ \hline
十一月 & 七 & 十五。羽詐阬殺秦降卒二十萬人於新安。 & 二十四 & 十七 & 二十八。沛公出令三章,秦民大悅。 & 二十八 & 十六 & 十九 \\ \hline
十二月 & 八。分楚為四。 & 十六。至關中,誅秦王子嬰,屠燒咸陽。分天下,立諸侯。 & 二十五。分趙為代國。 & 十八。項羽怨榮,分齊為三國。 & 二十九。與項羽有纳,見之戲下,講解。羽倍約,分關中為四國。 & 二十九。臧荼從入,分燕為二國。 & 十七。分魏為殷國。 & 二十。分韓為河南國。 \\ \hline
}

\biao[\hline 秦 & 楚 &  & 項 &  & 趙 &  & 齊 &  &  & 漢 &  &  &  & 燕 &  & 魏 &  & 韓 &  \\ \hline
\multicolumn{15}{}{}]
{|p{3em}|p{4em}|p{3em}|p{3em}|p{3em}|p{3em}|p{3em}|p{3em}|p{3em}|p{3em}|p{4em}|p{3em}|p{3em}|p{3em}|p{3em}|p{3em}|p{3em}|p{3em}|p{3em}|p{3em}|}{
\hline
秦 & 楚 &  & 項 &  & 趙 &  & 齊 &  &  & 漢 &  &  &  & 燕 &  & 魏 &  & 韓 &  \\ \hline
九。義帝元年。諸侯尊懷王為義帝。 & 十七。項籍自立為西楚霸王。 & 分為衡山。 & 分為臨江。 & 分為九江。 & 二十六。更名為常山。 & 分為代。 & 十九。更名為臨菑。 & 分為濟北。 & 分為膠東。 & 正月。分關中為漢。 & 分關中為雍。 & 分關中為塞。 & 分關中為翟。 & 三十。燕 & 分為遼東。 & 十八。更為西魏。 & 分為殷。 & 二十一。韓 & 分為河南。 \\ \hline
二。徙都江南郴。 & 西楚主伯,項籍始,為天下主命,主十八王。 & 王吳芮始,故番君。 & 王共敖始,故楚柱國。 & 王英布始,故楚將。 & 王張耳始,故楚將。 & 二十七。王趙歇始,故趙王。 & 王田都始,故齊將。 & 王田安始,故齊將。 & 二十。王田市始,故齊王。 & 二月。漢王始,故沛公。 & 王章邯始,故秦將。 & 王司馬欣始,故秦將。 & 王董翳始,故秦將。 & 王臧荼始,故燕將。 & 三十一。王韓廣始,故燕王。 & 十九。王魏豹始,故魏王。 & 王司馬卬始,故趙將。 & 二十二。王韓成始,故韓將。 & 王申陽始,故楚將。 \\ \hline
三 & 二。都彭城。 & 二。都邾。 & 二。都江陵。 & 二。都六。 & 二。都襄國。 & 二十八。都代。 & 二。都臨菑。 & 二。都博陽。 & 二十一。都即墨。 & 三月。都南鄭。 & 二。都廢丘。 & 二。都櫟陽。 & 二。都高奴。 & 二。都薊。 & 三十二。都無終。 & 二十。都平陽。 & 二。都朝歌。 & 二十三。都陽翟。 & 二。都洛陽。 \\ \hline
四 & 三。諸侯罷戲下兵,皆之國。 & 三 & 三 & 三 & 三 & 二十九 & 三 & 三 & 二十二 & 四月 & 三 & 三 & 三 & 三 & 三十三 & 二十一 & 三 & 二十四 & 三 \\ \hline
五 & 四 & 四 & 四 & 四 & 四 & 三十 & 四。田榮擊都,都降楚。 & 四 & 二十三 & 五月 & 四 & 四 & 四 & 四 & 三十四 & 二十二 & 四 & 二十五 & 四 \\ \hline
六 & 五 & 五 & 五 & 五 & 五 & 三十一 & 齊王田榮始,故齊相。 & 五 & 二十四。田榮擊殺市。 & 六月 & 五 & 五 & 五 & 五 & 三十五 & 二十三 & 五 & 二十六 & 五 \\ \hline
七 & 六 & 六 & 六 & 六 & 六 & 三十二 & 二 & 六。田榮擊殺安。 & 屬齊。 & 七月 & 六 & 六 & 六 & 六 & 三十六 & 二十四 & 六 & 二十七。項羽誅成。 & 六 \\ \hline
八 & 七 & 七 & 七 & 七 & 七 & 三十三 & 三 & 屬齊。 &  & 八月 & 七。邯守廢丘,漢圍之。 & 七。欣降漢,國除。 & 七。翳降漢,國除。 & 七 & 三十七。臧荼擊廣無終,滅之。 & 二十五 & 七 & 韓王鄭昌始,項羽立之。 & 七 \\ \hline
九 & 八 & 八 & 八 & 八 & 八 & 三十四 & 四 &  &  & 九月 & 八 & 屬漢,為渭南、河上郡。 & 屬漢為上郡。 & 八 & 屬燕。 & 二十六 & 八 & 二 & 八 \\ \hline
十。項羽滅義帝。 & 九 & 九 & 九 & 九 & 九。耳降漢。 & 三十五。歇復王趙。 & 五 &  &  & 十月。王至陝。 & 九 &  &  & 九 &  & 二十七 & 九 & 三 & 九 \\ \hline
 & 十 & 十 & 十 & 十 &  & 三十六 & 六 &  &  & 十一月 & 十。漢拔我隴西。 &  &  & 十 &  & 二十八 & 十 & 韓王信始,漢立之。 & 屬漢,為河南郡。 \\ \hline
 & 十一 & 十一 & 十一 & 十一 & 歇以陳餘為代王,故成安君。 & 三十七 & 七 &  &  & 十二月 & 十一 &  &  & 十一 &  & 二十九 & 十一 & 二 &  \\ \hline
 & 十二 & 十二 & 十二 & 十二 & 二 & 三十八 & 八。項籍擊榮,走平原平原民殺之。 &  &  & 正月 & 十二。漢拔我北地。 &  &  & 十二 &  & 三十 & 十二 & 三 &  \\ \hline
 & 二年一月 & 二年一月 & 十三 & 二年一月 & 三 & 三十九 & 項籍立故齊王田假為齊王。 &  &  & 二月 & 二年一月 &  &  & 二年一月 &  & 三十一 & 十三 & 四 &  \\ \hline
 & 二 & 二 & 十四 & 二 & 四 & 四十 & 二。田榮弟橫反城陽,擊假,走楚,楚殺假。 &  &  & 三月。王擊殷。 & 二 &  &  & 二 &  & 三十二。降漢。 & 十四。降漢,卬廢。 & 五 &  \\ \hline
 & 三。項羽以兵三萬破漢兵五十六萬。 & 三 & 十五 & 三 & 五 & 四十一 & 齊王田廣始。廣,榮子,橫立之。 &  &  & 四月。王伐楚至彭城,壞走。 & 三 &  &  & 三 &  & 三十三。從漢伐楚。 & 為河內郡,屬漢。 & 六。從漢伐楚。 &  \\ \hline
 & 四 & 四 & 十六 & 四 & 六 & 四十二 & 二 &  &  & 五月。王走滎陽。 & 四 &  &  & 四 &  & 三十四。豹歸,叛漢。 &  & 七 &  \\ \hline
 & 五 & 五 & 十七 & 五 & 七 & 四十三 & 三 &  &  & 六月。王入關,立太子。復如滎陽。 & 五。漢殺邯廢丘。 &  &  & 五 &  & 三十五 &  & 八 &  \\ \hline
 & 六 & 六 & 十八 & 六 & 八 & 四十四 & 四 &  &  & 七月 & 屬漢,為隴西、北地、中地郡。 &  &  & 六 &  & 三十六 &  & 九 &  \\ \hline
 & 七 & 七 & 十九 & 七 & 九 & 四十五 & 五 &  &  & 八月 &  &  &  & 七 &  & 三十七 &  & 十 &  \\ \hline
 & 八 & 八 & 二十 & 八 & 十 & 四十六 & 六 &  &  & 九月 &  &  &  & 八 &  & 三十八。漢將信虜豹。 &  & 十一 &  \\ \hline
 & 九 & 九 & 二十一 & 九 & 十一 & 四十七 & 七 &  &  & 後九月。 &  &  &  & 九 &  & 屬漢,為河東、上黨郡。 &  & 十二 &  \\ \hline
 & 十 & 十 & 二十二 & 十 & 十二。漢將韓信斬陳餘。 & 四十八。漢滅歇。 & 八 &  &  & 三年十月 &  &  &  & 十 &  &  &  & 二年一月 &  \\ \hline
 & 十一 & 十一 & 二十三 & 十一 & 屬漢,為太原郡。 & 屬漢,為郡。 & 九 &  &  & 十一月 &  &  &  & 十一 &  &  &  & 二 &  \\ \hline
 & 十二 & 十二 & 二十四 & 十二。布身降漢,地屬項籍。 &  &  & 十 &  &  & 十二月 &  &  &  & 十二 &  &  &  & 三 &  \\ \hline
 & 三年一月 & 三年一月 & 二十五 &  &  &  & 十一 &  &  & 正月 &  &  &  & 三年一月 &  &  &  & 四 &  \\ \hline
 & 二 & 二 & 二十六 &  &  &  & 十二 &  &  & 二月 &  &  &  & 二 &  &  &  & 五 &  \\ \hline
 & 三 & 三 & 二十七 &  &  &  & 十三 &  &  & 三月 &  &  &  & 三 &  &  &  & 六 &  \\ \hline
 & 四 & 四 & 二十八 &  &  &  & 十四 &  &  & 四月。楚圍王滎陽。 &  &  &  & 四 &  &  &  & 七 &  \\ \hline
 & 五 & 五 & 二十九 &  &  &  & 十五 &  &  & 五月 &  &  &  & 五 &  &  &  & 八 &  \\ \hline
 & 六 & 六 & 三十 &  &  &  & 十六 &  &  & 六月 &  &  &  & 六 &  &  &  & 九 &  \\ \hline
 & 七 & 七 & 三十一。王敖薨。 &  &  &  & 十七 &  &  & 七月。王出滎陽。 &  &  &  & 七 &  &  &  & 十 &  \\ \hline
 & 八 & 八 & 臨江王驩。始,敖子。 &  &  &  & 十八 &  &  & 八月。周苛、樅公殺魏豹。 &  &  &  & 八 &  &  &  & 十一 &  \\ \hline
 & 九 & 九 & 二 &  &  &  & 十九 &  &  & 九月 &  &  &  & 九 &  &  &  & 十二 &  \\ \hline
 & 十 & 十 & 三 &  &  &  & 二十 &  &  & 四年十月 &  &  &  & 十 &  &  &  & 三年一月 &  \\ \hline
 & 十一。漢將韓信破殺龍且。 & 十一 & 四 &  & 趙王張耳始,漢立之。 &  & 二十一。漢將韓信擊殺廣。 &  &  & 十一月 &  &  &  & 十一 &  &  &  & 二 &  \\ \hline
 & 十二 & 十二 & 五 &  & 二 &  & 屬漢,為郡。 &  &  & 十二月 &  &  &  & 十二 &  &  &  & 三 &  \\ \hline
 & 四年一月 & 四年一月 & 六 &  & 三 &  &  &  &  & 正月 &  &  &  & 四年一月 &  &  &  & 四 &  \\ \hline
 & 二 & 二 & 七 &  & 四 &  & 齊王韓信始,漢立之。 &  &  & 二月。立信王齊。 &  &  &  & 二 &  &  &  & 五 &  \\ \hline
 & 三。漢御史周苛入楚,死。 & 三 & 八 &  & 五 &  & 二 &  &  & 三月。周苛入楚。 &  &  &  & 三 &  &  &  & 六 &  \\ \hline
 & 四 & 四 & 九 &  & 六 &  & 三 &  &  & 四月。王出滎陽。豹死。 &  &  &  & 四 &  &  &  & 七 &  \\ \hline
 & 五 & 五 & 十 &  & 七 &  & 四 &  &  & 五月 &  &  &  & 五 &  &  &  & 八 &  \\ \hline
 & 六 & 六 & 十一 &  & 八 &  & 五 &  &  & 六月 &  &  &  & 六 &  &  &  & 九 &  \\ \hline
 & 七 & 七 & 十二 & 淮南王英布始,漢立之。 & 九 &  & 六 &  &  & 七月。立布為淮南王。 &  &  &  & 七 &  &  &  & 十 &  \\ \hline
 & 八 & 八 & 十三 & 二 & 十 &  & 七 &  &  & 八月 &  &  &  & 八 &  &  &  & 十一 &  \\ \hline
 & 九 & 九 & 十四 & 三 & 十一 &  & 八 &  &  & 九月。太公、呂后歸自楚。 &  &  &  & 九 &  &  &  & 十二 &  \\ \hline
 & 十 & 十 & 十五 & 四 & 十二 &  & 九 &  &  & 五年十月 &  &  &  & 十 &  &  &  & 四年一月 &  \\ \hline
 & 十一 & 十一 & 十六 & 五 & 二年一月 &  & 十 &  &  & 十一月 &  &  &  & 十一 &  &  &  & 二 &  \\ \hline
 & 十二。誅籍。 & 十二 & 十七。漢虜驩。 & 六 & 二 &  & 十一 &  &  & 十二月 &  &  &  & 十二 &  &  &  & 三 &  \\ \hline
 & 齊王韓信徙楚王。 & 十三。徙王長沙。 & 屬漢,為南郡。 & 七。淮南國 & 三。趙國 &  & 十二。徙王楚,屬漢,為四郡。 &  &  & 正月。殺項籍,天下平,諸侯臣屬漢。 &  &  &  & 五年一月。燕國 &  & 復置梁國。 &  & 四。韓王信徙王代,都馬邑。 & 分臨江為長沙國。 \\ \hline
 & 二 & 屬淮南國。 &  & 八 & 四 &  &  &  &  & 二月。甲午,王更號,即皇帝位於定陶。 &  &  &  & 二 &  & 梁王彭越始。 &  & 五 & 衡山王吳芮為為長沙王。 \\ \hline
 & 三 &  &  & 九 & 五 &  &  &  &  & 三月 &  &  &  & 三 &  & 二 &  & 六 & 二 \\ \hline
 & 四 &  &  & 十 & 六 &  &  &  &  & 四月 &  &  &  & 四 &  & 三 &  & 七 & 三 \\ \hline
 & 五 &  &  & 十一 & 七 &  &  &  &  & 五月 &  &  &  & 五 &  & 四 &  & 八 & 四 \\ \hline
 & 六 &  &  & 十二 & 八 &  &  &  &  & 六月。帝入關。 &  &  &  & 六 &  & 五 &  & 九 & 五 \\ \hline
 & 七 &  &  & 二年一月 & 九。耳薨,謚景王。 &  &  &  &  & 七月 &  &  &  & 七 &  & 六 &  & 十 & 六。薨,謚文王。 \\ \hline
 & 八 &  &  & 二 & 趙王張敖始,耳子。 &  &  &  &  & 八月。帝自將誅燕。 &  &  &  & 八 &  & 七 &  & 十一 & 長沙成王臣始,芮子。 \\ \hline
 & 九。王得故項羽將鍾離辚,斬之以聞。 &  &  & 三 & 二 &  &  &  &  & 九月 &  &  &  & 九。反漢,虜荼。 &  & 八 &  & 十二 & 二 \\ \hline
 & 十 &  &  & 四 & 三 &  &  &  &  & 後九月。 &  &  &  & 燕王盧綰始,漢太尉。 &  & 九 &  & 五年一月 & 三 \\ \hline
 }