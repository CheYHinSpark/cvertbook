\chapter{六國年表第三}

太史公讀秦記,至犬戎敗幽王,周東徙洛邑,秦襄公始封為諸侯,作西畤用事上帝,僭端見矣。禮曰,天子祭天地,諸侯祭其域內名山大川。今秦雜戎翟之俗,先暴戾,後仁義,位在藩臣而臚於郊祀,君子懼焉。及文公踰隴,攘夷狄,尊陳寶,營岐雍之閒,而穆公修政,東竟至河,則與齊桓、晉文中國侯伯侔矣。是後陪臣執政,大夫世祿,六卿擅晉權,征伐會盟,威重於諸侯。及田常殺簡公而相齊國,諸侯晏然弗討,海內爭於戰功矣。三國終之卒分晉,田和亦滅齊而有之,六國之盛自此始。務在彊兵并敵,謀詐用而從衡短長之說起。矯稱蜂出,誓盟不信,雖置質剖符猶不能約束也。秦始小國僻遠,諸夏賓之,比於戎翟,至獻公之後常雄諸侯。論秦之德義不如魯衛之暴戾者,量秦之兵不如三晉之彊也,然卒并天下,非必險固便形勢利也,蓋若天所助焉。
	
或曰東方物所始生,西方物之成孰。夫作事者必於東南,收功實者常於西北。故禹興於西羌,湯起於亳,周之王也以豐鎬伐殷,秦之帝用雍州興,漢之興自蜀漢。

秦既得意,燒天下詩書,諸侯史記尤甚,為其有所刺譏也。詩書所以復見者,多藏人家,而史記獨藏周室,以故滅。惜哉,惜哉。獨有秦記,又不載日月,其文略不具。然戰國之權變亦有可頗采者,何必上古。秦取天下多暴,然世異變,成功大。傳曰法後王,何也。以其近己而俗變相類,議卑而易行也。學者牽於所聞,見秦在帝位日淺,不察其終始,因舉而笑之,不敢道,此與以耳食無異。悲夫。

余於是因秦記,踵春秋之後,起周元王,表六國時事,訖二世,凡二百七十年,著諸所聞興壞之端。後有君子,以覽觀焉。

\biao[\hline 周&秦&魏&韓&趙&楚&燕&齊\\\hline
\multicolumn{8}{}{}]
{|p{3em}|p{18em}|p{8em}|p{8em}|p{8em}|p{9em}|p{8em}|p{5em}|}{\hline
周元王元年。&秦厲共公元年。&魏獻子。衛出公輒後元年。&韓宣子&趙簡子。四十二。&楚惠王章十三年。吳伐我。&燕獻公十七年。&齊平公驁五年。\\\hline
二&二。蜀人來賂。&晉定公卒。&&四十三&十四。越圍吳,吳怨。&十八&六\\\hline
三&三&晉出公錯元年。&&四十四&十五&十九&七。越人始來。\\\hline
四&四&&&四十五&十六。越滅吳。&二十&八\\\hline
五&五。楚人來賂。&&&四十六&十七。蔡景侯卒。&二十一&九。晉知伯瑤來伐我。\\\hline
六&六。義渠來賂。綿諸乞援。&&&四十七&十八。蔡聲侯元年。&二十二&十\\\hline
七&七。彗星見。&衛出公飲,大夫不解襪,公怒,即攻公,公奔宋。&&四十八&十九。王子英奔秦。&二十三&十一\\\hline
八&八&&&四十九&二十&二十四&十二\\\hline
定王元年。&九&&&五十&二十一&二十五&十三\\\hline
二&十。庶長將兵拔魏城。彗星見。&&&五十一&二十二。魯哀公卒。&二十六&十四\\\hline
三&十一&&&五十二&二十三。魯悼公元年。三桓勝,魯如小侯。&二十七&十五\\\hline
四&十二&&&五十三&二十四&二十八&十六\\\hline
五&十三&&知伯伐鄭,駟桓子如齊求救。&五十四。知伯謂簡子,欲廢太子襄子,襄子怨知伯。&二十五&燕孝公元年&十七。救鄭,晉師去。中行文子謂田常,乃今知所以亡。\\\hline
六&十四。晉人、楚人來賂。&&鄭聲公卒。&五十五&二十六&二&十八\\\hline
七&十五&&鄭哀公元年。&五十六&二十七&三&十九\\\hline
八&十六。塹阿旁。伐大荔。補龐戲城。&&&五十七&二十八&四&二十\\\hline
九&十七&&&五十八&二十九&五&二十一\\\hline
十&十八&&&五十九&三十&六&二十二\\\hline
十一&十九&&&六十&三十一&七&二十三\\\hline
十二&二十。公將師與綿諸戰。&&&襄子。元年未除服,登夏屋,誘代王,以金斗殺代王。封伯魯子周為代成君。&三十二。蔡聲侯卒。&八&二十四\\\hline
十三&二十一&晉哀公忌元年。&&二&三十三。蔡元侯元年。&九&二十五\\\hline
十四&二十二&衛悼公黔元年。&&三&三十四&十&齊宣公就匝元年。\\\hline
十五&二十三&&&四。與智伯分范、中行地。&三十五&十一&二\\\hline
十六&二十四&魏桓子敗智伯于晉陽。&韓康子敗智伯于晉陽。&五。襄子敗智伯晉陽,與魏、韓三分其地。&三十六&十二&三\\\hline
十七&二十五。晉大夫智開率其邑來奔。&&&六&三十七&十三&四\\\hline
十八&二十六。左庶長城南鄭。&&&七&三十八&十四&五。宋景公卒。\\\hline
十九&二十七&衛敬公元年。&&八&三十九。蔡侯齊元年。&十五&六。宋昭公元年。\\\hline
二十&二十八。越人來迎女。&&&九&四十&燕成公元年&七\\\hline
二十一&二十九。晉大夫智寬率其邑人來奔。&&&十&四十一&二&八\\\hline
二十二&三十&&&十一&四十二。楚滅蔡。&三&九\\\hline
二十三&三十一&&&十二&四十三&四&十\\\hline
二十四&三十二&&&十三&四十四。滅杞。杞,夏之後。&五&十一\\\hline
二十五&三十三。伐義渠,虜其王。&&&十四&四十五&六&十二\\\hline
二十六&三十四。日蝕,晝晦。星見。&&&十五&四十六&七&十三\\\hline
二十七&秦躁公元年&&&十六&四十七&八&十四\\\hline
二十八&二。南鄭反。&&&十七&四十八&九&十五\\\hline
考王元年。&三&&&十八&四十九&十&十六\\\hline
二&四&&&十九&五十&十一&十七\\\hline
三&五&&&二十&五十一&十二&十八\\\hline
四&六&晉幽公柳元年。服韓、魏。&&二十一&五十二&十三&十九\\\hline
五&七&&&二十二&五十三&十四&二十\\\hline
六&八。六月,雨雪。日、月蝕。&&&二十三&五十四&十五&二十一\\\hline
七&九&&&二十四&五十五&十六&二十二\\\hline
八&十&&&二十五&五十六&燕湣公元年&二十三\\\hline
九&十一&&&二十六&五十七&二&二十四\\\hline
十&十二&衛昭公元年。&&二十七&楚簡王仲元年。滅莒。&三&二十五\\\hline
十一&十三。義渠伐秦,侵至渭陽。&&&二十八&二&四&二十六\\\hline
十二&十四&&&二十九&三。魯悼公卒。&五&二十七\\\hline
十三&秦懷公元年。生靈公。&&&三十&四。魯元公元年。&六&二十八\\\hline
十四&二&&&三十一&五&七&二十九\\\hline
十五&三&&&三十二&六&八&三十\\\hline
威烈王元年。&四。庶長晁殺懷公。太子蚤死,大臣立太子之子,為靈公。&衛悼公亹元年。&&三十三。襄子卒。&七&九&三十一\\\hline
二&秦靈公元年。生獻公。&魏文侯斯元年。&韓武子元年。&趙桓子元年。&八&十&三十二\\\hline
三&二&二&二。鄭幽公元年。韓殺之。&趙獻侯元年&九&十一&三十三\\\hline
四&三。作上下畤。&三&三。鄭立幽公子,為繻公,元年。&二&十&十二&三十四\\\hline
五&四&四&四&三&十一&十三&三十五\\\hline
六&五&五。魏誅晉幽公,立其弟止。&五&四&十二&十四&三十六\\\hline
七&六&六。晉烈公止元年。魏城少梁。&六&五&十三&十五&三十七\\\hline
八&七。與魏戰少梁。&七&七&六&十四&十六&三十八\\\hline
九&八。城塹河瀕。初以君主妻河。&八。復城少梁。&八&七&十五&十七&三十九\\\hline
十&九&九&九&八&十六&十八&四十\\\hline
十一&十。補龐,城籍姑。靈公卒,立其季父悼子,是為簡公。&十&十&九&十七&十九&四十一\\\hline
十二&秦簡公元年&十一。衛慎公元年。&十一&十。中山武公初立。&十八&二十&四十二\\\hline
十三&二。與晉戰,敗鄭下。&十二&十二&十一&十九&二十一&四十三。伐晉,毀黃城,圍陽狐。\\\hline
十四&三&十三。公子擊圍繁龐,出其民。&十三&十二&二十&二十二&四十四。伐魯、莒及安陽。\\\hline
十五&四&十四&十四&十三。城平邑。&二十一&二十三&四十五。伐魯,取都。\\\hline
十六&五。日蝕。&十五&十五&十四&二十二&二十四&四十六\\\hline
十七&六。初令吏帶劍。&十六。伐秦,築臨晉、元里。&十六&十五&二十三&二十五&四十七\\\hline
十八&七。塹洛,城重泉。初租禾。&十七。擊守中山。伐秦至鄭,還築洛陰、合陽。&韓景侯虔元年。伐鄭,取雍丘。鄭城京。&趙烈侯籍元年。魏使太子伐中山。&二十四。簡王卒。&二十六&四十八。取魯郕。\\\hline
十九&八&十八。文侯受經子夏。過段干木之閭常式。&二。鄭敗韓于負黍。&二&楚聲王當元年。魯穆公元年&二十七&四十九。與鄭會于西城。伐衛,取毌。\\\hline
二十&九&十九&三&三&二&二十八&五十\\\hline
二十一&十&二十。卜相,李克、翟璜爭。&四&四&三&二十九&五十一。田會以廩丘反。\\\hline
二十二&十一&二十一&五&五&四&三十&齊康公貸元年\\\hline
二十三。九鼎震。&十二&二十二。初為侯。&六。初為侯。&六。初為侯。&五。魏、韓、趙始列為諸侯。&三十一&二。宋悼公元年。\\\hline
二十四&十三&二十三&七&七。烈侯好音,欲賜歌者田,徐越侍以仁義,乃止。&六。盜殺聲王。&燕釐公元年&三\\\hline
安王元年。&十四。伐魏,至陽狐。&二十四。秦伐我,至陽狐。&八&八&楚悼王類元年&二&四\\\hline
二&十五&二十五。太子罃生。&九。鄭圍陽翟。&九&二。三晉來伐我,至乘丘。&三&五\\\hline
三。王子定奔晉。&秦惠公元年。&二十六。虢山崩,壅河。&韓烈侯元年。&趙武公元年&三。歸榆關于鄭。&四&六\\\hline
四&二&二十七&二。鄭殺其相駟子陽。&二&四。敗鄭師,圍鄭。鄭人殺子陽。&五&七\\\hline
五&三。日蝕。&二十八&三。三月,盜殺韓相俠累。&三&五&六&八\\\hline
六&四&二十九&四。鄭相子陽之徒殺其君繻公。&四&六&七&九\\\hline
七&五。伐綿諸。&三十&五。鄭康公元年。&五&七&八&十。宋休公元年。\\\hline
八&六&三十一&六。救魯。鄭負黍反。&六&八&九&十一。伐魯,取最。\\\hline
九&七&三十二。伐鄭,城酸棗。&七&七&九。伐韓,取負黍。&十&十二\\\hline
十&八&三十三。晉孝公傾元年。&八&八&十&十一&十三\\\hline
十一&九。伐韓宜陽,取六邑。&三十四&九。秦伐宜陽,取六邑。&九&十一&十二&十四\\\hline
十二&十。與晉戰武城。縣陝。&三十五。齊伐取襄陵。&十&十&十二&十三&十五。魯敗我平陸。\\\hline
十三&十一。太子生。&三十六。秦侵陰晉。&十一&十一&十三&十四&十六。與晉、衛會濁澤。\\\hline
十四&十二&三十七&十二&十二&十四&十五&十七\\\hline
十五&十三。蜀取我南鄭。&三十八&十三&十三&十五&十六&十八\\\hline
十六&秦出公元年。&魏武侯。元年。襲邯鄲,敗焉。&韓文侯元年&趙敬侯元年。武公子朝作亂,奔魏。&十六&十七&十九。田常曾孫田和始列為諸侯。遷康公海上,食一城。\\\hline
十七&二。庶長改迎靈公太子,立為獻公。誅出公。&二。城安邑、王垣。&二。伐鄭,取陽城。伐宋,到彭城,執宋君。&二&十七&十八&二十。伐魯,破之。田和卒。\\\hline
十八&秦獻公元年。&三&三&三&十八&十九&二十一。田和子桓公午立。\\\hline
十九&二。城櫟陽。&四&四&四。魏敗我兔臺。&十九&二十&二十二\\\hline
二十&三。日蝕,晝晦。&五&五&五&二十&二十一&二十三\\\hline
二十一&四。孝公生。&六&六&六&二十一&二十二&二十四\\\hline
二十二&五&七。伐齊,至桑丘。&七。伐齊,至桑丘。鄭敗晉。&七。伐齊,至桑丘。&楚肅王臧元年&二十三&二十五。伐燕,取桑丘。\\\hline
二十三&六。初縣蒲、藍田、善明氏。&八&八&八。襲衛,不克。&二&二十四&二十六。康公卒,田氏遂并齊而有之。太公望之後絕祀。\\\hline
二十四&七&九。翟敗我澮。伐齊,至靈丘。&九。伐齊,至靈丘。&九。伐齊,至靈丘。&三&二十五&齊威王因元年。自田常至威王,威王始以齊彊天下。\\\hline
二十五&八&十。晉靜公俱酒元年。&十&十&四。蜀伐我茲方。&二十六&二\\\hline
二十六&九&十一。魏、韓、趙滅晉,絕無後。&韓哀侯元年。分晉國。&十一。分晉國。&五。魯共公元年。&二十七&三。三晉滅其君。\\\hline
烈王元年。&十。日蝕。&十二&二。滅鄭。康公二十年滅,無後。&十二&六&二十八&四\\\hline
二&十一。縣櫟陽。&十三&三&趙成侯元年&七&二十九&五\\\hline
三&十二&十四&四&二&八&三十。敗齊林孤。&六。魯伐入陽關。晉伐到鱄陵。\\\hline
四&十三&十五。衛聲公元年。敗趙北藺。&五&三。伐衛,取都鄙七十三。魏敗我藺。&九&燕桓公元年&七。宋辟公元年。\\\hline
五&十四&十六。伐楚,取魯陽。&六。韓嚴殺其君。&四&十。魏取我魯陽。&二&八\\\hline
六&十五&惠王元年&莊侯元年。&五。伐齊于甄。魏敗我懷。&十一&三&九。趙伐我甄。\\\hline
七&十六。民大疫。日蝕。&二。敗韓馬陵。&二。魏敗我馬陵。&六。敗魏涿澤,圍惠王。&楚宣王良夫元年&四&十。宋剔成元年。\\\hline
顯王元年。&十七。櫟陽雨金,四月至八月。&三。齊伐我觀。&三&七。侵齊,至長城。&二&五&十一。伐魏取觀。趙侵我長城。\\\hline
二&十八&四&四&八&三&六&十二\\\hline
三&十九。敗韓、魏洛陰。&五。與韓會宅陽。城武都。&五&九&四&七&十三\\\hline
四&二十&六。伐宋,取儀臺。&六&十&五&八&十四\\\hline
五。賀秦。&二十一。章蟜與晉戰石門。斬首六萬,天子賀。&七&七&十一&六&九&十五\\\hline
六&二十二&八&八&十二&七&十&十六\\\hline
七&二十三。與魏戰少梁,虜其太子。&九。與秦戰少梁,虜我太子。&九。魏敗我于澮。大雨三月。&十三。魏敗我于澮。&八&十一&十七\\\hline
八&秦孝公元年。彗星見西方。&十。取趙皮牢。衛成侯元年。&十&十四&九&燕文公元年&十八\\\hline
九。致胙于秦。&二。天子致胙。&十一&十一&十五&十&二&十九\\\hline
十&三&十二。星晝墮,有聲。&十二&十六&十一&三&二十\\\hline
十一&四&十三&韓昭侯元年。秦敗我西山。&十七&十二&四&二十一。鄒忌以鼓琴見威王。\\\hline
十二&五&十四。與趙會鄗。&二。宋取我黃池。魏取我朱。&十八。趙孟如齊。&十三。君尹黑迎女秦。&五&二十二。封鄒忌為成侯。\\\hline
十三&六&十五。魯、衛、宋、鄭侯來。&三&十九。與燕會阿。與齊、宋會平陸。&十四&六&二十三。與趙會平陸。\\\hline
十四&七。與魏王會杜平。&十六。與秦孝公會杜平。侵宋黃池,宋復取之。&四&二十&十五&七&二十四。與魏會田於郊。\\\hline
十五&八。與魏戰元里,斬首七千,取少梁。&十七。與秦戰元里,秦取我少梁。&五&二十一。魏圍我邯鄲。&十六&八&二十五\\\hline
十六&九&十八。邯鄲降。齊敗我桂陵。&六。伐東周,取陵觀、廩丘。&二十二。魏拔邯鄲。&十七&九&二十六。敗魏桂陵。\\\hline
十七&十。衛公孫鞅為大良造,伐安邑,降之。&十九。諸侯圍我襄陵。築長城,塞固陽。&七&二十三&十八。魯康公元年&十&二十七\\\hline
十八&十一。城商塞。衛鞅圍固陽,降之。&二十。歸趙邯鄲。&八。申不害相。&二十四。魏歸邯鄲,與魏盟漳水上。&十九&十一&二十八\\\hline
十九&十二。初聚小邑為三十一縣,令。為田開阡陌。&二十一。與秦遇彤。&九&二十五&二十&十二&二十九\\\hline
二十&十三。初為縣,有秩史。&二十二&十。韓姬弒其君悼公。&趙肅侯元年。&二十一&十三&三十\\\hline
二十一&十四。初為賦。&二十三&十一。昭侯如秦。&二&二十二&十四&三十一\\\hline
二十二&十五&二十四&十二&三。公子范襲邯鄲,不勝,死。&二十三&十五&三十二\\\hline
二十三&十六&二十五&十三&四&二十四&十六&三十三。殺其大夫牟辛。\\\hline
二十四&十七&二十六&十四&五&二十五&十七&三十四\\\hline
二十五。諸侯會。&十八&二十七。丹封名會。丹,魏大臣。&十五&六&二十六&十八&三十五。田忌襲齊,不勝。\\\hline
二十六。致伯秦。&十九。城武城。從東方牡丘來歸。天子致伯。&二十八&十六&七&二十七。魯景公偃元年。&十九&三十六\\\hline
二十七&二十。諸侯畢賀。會諸侯于澤。朝天子。&二十九。中山君為相。&十七&八&二十八&二十&齊宣王辟彊元年\\\hline
二十八&二十一。馬生人。&三十。齊虜我太子申,殺將軍龐涓。&十八&九&二十九&二十一&二。敗魏馬陵。田忌、田嬰、田馊將,孫子為師。\\\hline
二十九&二十二。封大良造商鞅。&三十一。秦商君伐我,虜我公子卬。&十九&十&三十&二十二&三。與趙會,伐魏。\\\hline
三十&二十三。與晉戰岸門。&三十二。公子赫為太子。&二十&十一&楚威王熊商元年&二十三&四\\\hline
三十一&二十四。大荔圍合陽。孝公薨。商君反,死彤地。&三十三。衛鞅亡歸我,我恐,弗內。&二十一&十二&二&二十四&五\\\hline
三十二&秦惠文王元年。楚、韓、趙、蜀人來。&三十四&二十二。申不害卒。&十三&三&二十五&六\\\hline
三十三。賀秦。&二。天子賀。行錢。宋太丘社亡。&三十五。孟子來,王問利國,對曰,君不可言利。&二十三&十四&四&二十六&七。與魏會平阿南。\\\hline
三十四&三。王冠。拔韓宜陽。&三十六&二十四。秦拔我宜陽&十五&五&二十七&八。與魏會于甄。\\\hline
三十五&四。天子致文武胙。魏夫人來。&魏襄王元年。與諸侯會徐州,以相王。&二十五。旱。作高門,屈宜臼曰昭侯不出此門。&十六&六&二十八。蘇秦說燕。&九。與魏會徐州,諸侯相王。\\\hline
三十六&五。陰晉人犀首為大良造。&二。秦敗我彫陰。&二十六。高門成,昭侯卒,不出此門。&十七&七。圍齊于徐州。&二十九&十。楚圍我徐州。\\\hline
三十七&六。魏以陰晉為和,命曰寧秦。&三。伐趙。衛平侯元年。&韓宣惠王元年&十八。齊、魏伐我,我決河水浸之。&八&燕易王元年&十一。與魏伐趙。\\\hline
三十八&七。義渠內亂,庶長操將兵定之。&四&二&十九&九&二&十二\\\hline
三十九&八。魏入河西地于秦。&五。與秦河西地少梁。秦圍我焦、曲沃。&三&二十&十&三&十三\\\hline
四十&九。度河,取汾陰、皮氏。圍焦,降之。與魏會應。&六。與秦會應。秦取汾陰、皮氏。&四&二十一&十一。魏敗我陘山。&四&十四\\\hline
四十一&十。張儀相。公子桑圍蒲陽,降之。魏納上郡。&七。入上郡于秦。&五&二十二&楚懷王槐元年&五&十五。宋君偃元年。\\\hline
四十二&十一。義渠君為臣。歸魏焦、曲沃。&八。秦歸我焦、曲沃。&六&二十三&二&六&十六\\\hline
四十三&十二。初臘。會龍門。&九&七&二十四&三&七&十七\\\hline
四十四&十三。四月戊午,君為王。&十&八。魏敗我韓舉。&趙武靈王元年。魏敗我趙護。&四&八&十八\\\hline
四十五&相張儀將兵取陝。初更元年&十一。衛嗣君元年。&九&二。城鄗。&五&九&十九\\\hline
四十六&二。相張儀與齊、楚會齧桑。&十二&十。君為王。&三&六。敗魏襄陵。&十。君為王。&齊湣王地元年\\\hline
四十七&三。張儀免相,相魏。&十三。秦取曲沃。平周女化為丈夫。&十一&四。與韓會區鼠。&七&十一&二\\\hline
四十八&四&十四&十二&五。取韓女為夫人。&八&十二&三。封田嬰於薛。\\\hline
慎靚王元年。&五。王北遊戎地,至河上。&十五&十三&六&九&燕王噲元年&四。迎婦于秦。\\\hline
二&六&十六&十四。秦來擊我,取鄢。&七&十。城廣陵。&二&五\\\hline
三&七。五國共擊秦,不勝而還。&魏哀王元年。擊秦不勝。&十五。擊秦不勝。&八。擊秦不勝。&十一。擊秦不勝。&三。擊秦不勝。&六。宋自立為王。\\\hline
四&八。與韓、趙戰,斬首八萬。張儀復相。&二。齊敗我觀澤。&十六。秦敗我脩魚,得將軍申差。&九。與韓、魏擊秦。齊敗我觀澤。&十二&四&七。敗魏、趙觀澤。\\\hline
五&九。擊蜀,滅之。取趙中都、西陽。&三&十七&十。秦取我中都、西陽。&十三&五。君讓其臣子之國,顧為臣。&八\\\hline
六&十&四&十八&十一。秦敗我將軍英。&十四&六&九\\\hline
周赧王元年。&十一。侵義渠,得二十五城。&五。秦拔我曲沃,歸其人。走犀首岸門。&十九&十二&十五。魯平公元年。&七。君噲及太子相子之皆死。&十\\\hline
二&十二。樗里子擊藺陽,虜趙將。公子繇通封蜀。&六。秦來立公子政為太子。與秦王會臨晉。&二十&十三。秦拔我藺,虜將趙莊。&十六。張儀來相。&八&十一\\\hline
三&十三。庶長章擊楚,斬首八萬。&七。擊齊,虜聲子於濮。與秦擊燕。&二十一。我秦攻楚,圍景座。&十四&十七。秦敗我將屈饨。&九。燕人共立公子平。&十二\\\hline
四&十四。蜀相殺蜀侯。&八。圍衛。&韓襄王元年&十五&十八&燕昭王元年&十三\\\hline
五&秦武王元年。誅蜀相壯。張儀、魏章皆出之魏。&九。與秦會臨晉。&二&十六。吳廣入女,生子何,立為惠王后。&十九&二&十四\\\hline
六&二。初置丞相,樗里子、甘茂為丞相。&十。張儀死。&三&十七&二十&三&十五\\\hline
七&三&十一。與秦會應。&四。與秦會臨晉。秦擊我宜陽。&十八&二十一&四&十六\\\hline
八&四。拔宜陽城,斬首六萬。涉河,城武遂。&十二。太子往朝秦。&五。秦拔我宜陽,斬首六萬。&十九。初胡服。&二十二&五&十七\\\hline
九&秦昭襄王元年&十三。秦擊皮氏,未拔而解。&六。秦復與我武遂。&二十&二十三&六&十八\\\hline
十&二。彗星見。桑君為亂,誅。&十四。秦武王后來歸。&七&二十一&二十四。秦來迎婦。&七&十九\\\hline
十一&三&十五&八&二十二&二十五。與秦王會黃觉,秦復歸我上庸。&八&二十\\\hline
十二&四。彗星見。&十六。秦拔我蒲阪、晉陽、封陵。&九。秦取武遂。&二十三&二十六。太子質秦。&九&二十一\\\hline
十三&五。魏王來朝。&十七。與秦會臨晉,復歸我蒲阪。&十。太子嬰與秦王會臨晉,因至咸陽而歸。&二十四&二十七&十&二十二\\\hline
十四&六。蜀反,司馬錯往誅蜀守煇,定蜀。日蝕,晝晦。伐楚。&十八。與秦擊楚。&十一。秦取我穰。與秦擊楚。&二十五。趙攻中山。惠后卒。&二十八。秦、韓、魏、齊敗我將軍唐眛於重丘。&十一&二十三。與秦擊楚,使公子將,大有功。\\\hline
十五&七。樗里疾卒。擊楚,斬首三萬。魏冉為相。&十九&十二&二十六&二十九。秦取我襄城,殺景缺。&十二&二十四。秦使涇陽君來為質。\\\hline
十六&八。楚王來,因留之。&二十。與齊王會于韓。&十三。齊、魏王來。立咎為太子。&二十七&三十。王入秦。秦取我八城。&十三&二十五。涇陽君復歸秦。薛文入相秦。\\\hline
十七&九&二十一。與齊、韓共擊秦于函谷。河、渭絕一日。&十四。與齊、魏共擊秦。&趙惠文王元年。以公子勝為相,封平原君。&楚頃襄王元年。秦取我十六城&十四&二十六。與魏、韓共擊秦。孟嘗君歸相齊。\\\hline
十八&十。楚懷王亡之趙,趙弗內。&二十二&十五&二。楚懷王亡來,弗內。&二&十五&二十七\\\hline
十九&十一。彗星見。復與魏封陵。&二十三&十六。秦與我武遂和。&三&三。懷王卒于秦,來歸葬。&十六&二十八\\\hline
二十&十二。樓緩免。穰侯魏冉為丞相。&魏昭王元年。秦尉錯來擊我襄。&韓釐王咎元年&四。圍殺主父。與齊、燕共滅中山。&四。魯文公元年。&十七&二十九。佐趙滅中山。\\\hline
二十一&十三。任鄙為漢中守。&二。與秦戰,我不利。&二&五&五&十八&三十。田甲劫王,相薛文走。\\\hline
二十二&十四。白起擊伊闕,斬首二十四萬。&三。佐韓擊秦,秦敗我兵伊闕。&三。秦敗我伊闕,斬首二十四萬,虜將喜。&六&六&十九&三十一\\\hline
二十三&十五。魏冉免相。&四&四&七&七。迎婦秦。&二十&三十二\\\hline
二十四&十六&五&五。秦拔我宛城。&八&八&二十一&三十三\\\hline
二十五&十七。魏入河東四百里。&六。芒卯以詐見重。&六。與秦武遂地方二百里。&九&九&二十二&三十四\\\hline
二十六&十八。客卿錯擊魏,至軹,取城大小六十一。&七。秦擊我。取城大小六十一。&七&十&十&二十三&三十五\\\hline
二十七&十九。十月為帝,十二月復為王。任鄙卒。&八&八&十一。秦拔我桂陽。&十一&二十四&三十六。為東帝二月,復為王。\\\hline
二十八&二十&九。秦拔我新垣、曲陽之城。&九&十二&十二&二十五&三十七\\\hline
二十九&二十一。魏納安邑及河內。&十。宋王死我溫。&十。秦敗我兵夏山。&十三&十三&二十六&三十八。齊滅宋。\\\hline
三十&二十二。蒙武擊齊。&十一&十一&十四。與秦會中陽。&十四。與秦會宛。&二十七&三十九。秦拔我列城九。\\\hline
三十一&二十三。尉斯離與韓、魏、燕、趙共擊齊,破之。&十二。與秦擊齊濟西。與秦王會西周。&十二。與秦擊齊濟西。與秦王會西周。&十五。取齊昔陽。&十五。取齊淮北。&二十八。與秦、三晉擊齊,燕獨入至臨菑,取其寶器。&四十。五國共擊湣王,王走莒。\\\hline
三十二&二十四。與楚會穰。&十三。秦拔我安城,兵至大梁而還。&十三&十六&十六。與秦王會穰。&二十九&齊襄王法章元年\\\hline
三十三&二十五&十四。大水。衛懷君元年。&十四。與秦會兩周閒。&十七。秦拔我兩城。&十七&三十&二\\\hline
三十四&二十六。魏冉復為丞相。&十五&十五&十八。秦拔我石城。&十八&三十一&三\\\hline
三十五&二十七。擊趙,斬首三萬。地動,壞城。&十六&十六&十九。秦敗我軍,斬首三萬。&十九。秦擊我,與秦漢北及上庸地。&三十二&四\\\hline
三十六&二十八&十七&十七&二十。與秦會黽池,藺相如從。&二十。秦拔衝、西陵。&三十三&五。殺燕騎劫。\\\hline
三十七&二十九。白起擊楚,拔郢,更東至竟陵,以為南郡。&十八&十八&二十一&二十一。秦拔我郢,燒夷陵,王亡走陳。&燕惠王元年&六\\\hline
三十八&三十。白起封為武安君。&十九&十九&二十二&二十二。秦拔我巫、黔中。&二&七\\\hline
三十九&三十一&魏安釐王元年。秦拔我兩城。封弟公子無忌為信陵君。&二十&二十三&二十三。秦所拔我江旁反秦。&三&八\\\hline
四十&三十二&二。秦拔我兩城,軍大梁下,韓來救,與秦溫以和。&二十一。暴鳶救魏,為秦所敗,走開封。&二十四&二十四&四&九\\\hline
四十一&三十三&三。秦拔我四城,斬首四萬。&二十二&二十五&二十五&五&十\\\hline
四十二&三十四。白起擊魏華陽軍,芒卯走,得三晉將,斬首十五萬。&四。與秦南陽以和。&二十三&二十六&二十六&六&十一\\\hline
四十三&三十五&五。擊燕。&韓桓惠王元年&二十七&二十七。擊燕。魯頃公元年。&七&十二\\\hline
四十四&三十六&六&二&二十八。藺相如攻齊,至平邑。&二十八&燕武成王元年&十三\\\hline
四十五&三十七&七&三。秦擊我閼與城,不拔。&二十九。秦攻韓閼與。趙奢將擊秦,大敗之,賜號曰馬服。&二十九&二&十四。秦、楚擊我剛壽。\\\hline
四十六&三十八&八&四&三十&三十&三&十五\\\hline
四十七&三十九&九。秦拔我懷城。&五&三十一&三十一&四&十六\\\hline
四十八&四十。太子質於魏者死,歸葬芷陽。&十&六&三十二&三十二&五&十七\\\hline
四十九&四十一&十一。秦拔我廩丘。&七&三十三&三十三&六&十八\\\hline
五十&四十二。宣太后薨。安國君為太子。&十二&八&趙孝成王元年。秦拔我三城。平原君相。&三十四&七。齊田單拔中陽。&十九\\\hline
五十一&四十三&十三&九。秦拔我陘。城汾旁。&二&三十五&八&齊王建元年\\\hline
五十二&四十四。攻韓,取南陽。&十四&十。秦擊我太行。&三&三十六&九&二\\\hline
五十三&四十五。攻韓,取十城。&十五&十一&四&楚考烈王元年。秦取我州。黃歇為相。&十&三\\\hline
五十四&四十六。王之南鄭。&十六&十二&五。使廉頗拒秦於長平。&二&十一&四\\\hline
五十五&四十七。白起破趙長平,殺卒四十五萬。&十七&十三&六。使趙括代廉頗將。白起破括四十五萬。&三&十二&五\\\hline
五十六&四十八&十八&十四&七&四&十三&六\\\hline
五十七&四十九&十九&十五&八&五&十四&七\\\hline
五十八&五十。王齕、鄭安平圍邯鄲,及齕還軍,拔新中。&二十。公子無忌救邯鄲,秦兵解去。&十六&九。秦圍我邯鄲,楚、魏救我。&六。春申君救趙。&燕孝王元年&八\\\hline
五十九。赧王卒。&五十一&二十一。韓、魏、楚救趙新中,秦兵罷。&十七。秦擊我陽城,救趙新中。&十&七。救趙新中。&二&九\\\hline
&五十二。取西周。王稽棄市。&二十二&十八&十一&八。取魯,魯君封於莒。&三&十\\\hline
&五十三&二十三&十九&十二&九&燕王喜元年&十一\\\hline
&五十四&二十四&二十&十三&十。徙於鉅陽。&二&十二\\\hline
&五十五&二十五。衛元君元年。&二十一&十四&十一&三&十三\\\hline
&五十六&二十六&二十二&十五。平原君卒。&十二。柱國景伯死。&四。伐趙,趙破我軍,殺栗腹。&十四\\\hline
&秦孝文王元年。&二十七&二十三&十六&十三&五&十五\\\hline
&秦莊襄王楚元年。蒙驁取成皋、滎陽。初置三川郡。呂不韋相。取東周。&二十八&二十四。秦拔我成皋、滎陽。&十七&十四。楚滅魯,頃公遷卞,為家人,絕祀。&六&十六\\\hline
&二。蒙驁擊趙榆次、新城、狼孟,得三十七城。日蝕。&二十九&二十五&十八&十五。春申君徙封於吳。&七&十七\\\hline
&三。王齮擊上黨。初置太原郡。魏公子無忌率五國卻我軍河外,蒙驁解去,&三十。無忌率五國兵敗秦軍河外。&二十六。秦拔我上黨。&十九&十六&八&十八\\\hline
&始皇帝元年。擊取晉陽,作鄭國渠。&三十一&二十七&二十。秦拔我晉陽。&十七&九&十九\\\hline
&二&三十二&二十八&二十一&十八&十&二十\\\hline
&三。蒙驁擊韓,取十三城。王齮死。&三十三&二十九。秦拔我十三城。&趙悼襄王偃元年&十九&十一&二十一\\\hline
&四。七月,蝗蔽天下。百姓納粟千石,拜爵一級。&三十四。信陵君死。&三十&二。太子從質秦歸。&二十&十二。趙拔我武遂、方城。&二十二\\\hline
&五。蒙驁取魏酸棗二十城。初置東郡。&魏景湣王元年。秦拔我二十城。&三十一&三。趙相、魏相會柯,盟。&二十一&十三。劇辛死於趙。&二十三\\\hline
&六。五國共擊秦。&二。秦拔我朝歌。衛從濮陽徙野王。&三十二&四&二十二。王東徙壽春,命曰郢。&十四&二十四\\\hline
&七。彗星見北方西方。夏太后薨。蒙驁死。&三。秦拔我汲。&三十三&五&二十三&十五&二十五\\\hline
&八。嫪毐封長信侯。&四&三十四&六&二十四&十六&二十六\\\hline
&九。彗星見,竟天。嫪毐為亂,遷其舍人于蜀。彗星復見。&五。秦拔我垣、蒲陽、衍。&韓王安元年&七&二十五。李園殺春申君。&十七&二十七\\\hline
&十。相國呂不韋免。齊、趙來,置酒。太后入咸陽。大索。&六&二&八。入秦,置酒。&楚幽王悼元年&十八&二十八。入秦,置酒。\\\hline
&十一。呂不韋之河南。王翦擊鄴、閼與,取九城。&七&三&九。秦拔我閼與、鄴,取九城。&二&十九&二十九\\\hline
&十二。發四郡兵助魏擊楚。呂不韋卒。復嫪毐舍人遷蜀者。&八。秦助我擊楚。&四&趙王遷元年。&三。秦、魏擊我。&二十&三十\\\hline
&十三。桓齮擊平陽,殺趙扈輒,斬首十萬,因東擊。趙王之河南。彗星見。&九&五&二。秦拔我平陽,敗扈輒。斬首十萬。&四&二十一&三十一\\\hline
&十四。桓齮定平陽、武城、宜安。韓使非來,我殺非。韓王請為臣。&十&六&三。秦拔我宜安。&五&二十二&三十二\\\hline
&十五。興軍至鄴。軍至太原。取狼孟。&十一&七&四。秦拔我狼孟、鄱吾。軍鄴。&六&二十三。太子丹質於秦,亡來歸。&三十三\\\hline
&十六。置麗邑。發卒受韓南陽。&十二。獻城秦。&八。秦來受地。&五。地大動。&七&二十四&三十四\\\hline
&十七。內史騰擊得韓王安,盡取其地,置潁川郡。華陽太后薨。&十三&九。秦虜王安,秦滅韓。&六&八&二十五&三十五\\\hline
&十八&十四。衛君角元年。&&七&九&二十六&三十六\\\hline
&十九。王翦拔趙,虜王遷邯鄲。帝太后薨。&十五&&八。秦王翦虜王遷邯鄲。公子嘉自立為代王。&十。幽王卒,弟郝立,為哀王。三月,負芻殺哀王。&二十七&三十七\\\hline
&二十。燕太子使荊軻刺王,覺之。王翦將擊燕。&魏王假元年&&代王嘉元年&楚王負芻元年。負芻,哀王庶兄。&二十八。太子丹使荊軻刺秦王,秦伐我。&三十八\\\hline
&二十一。王賁擊楚。&二&&二&二。秦大破我,取十城。&二十九。秦拔我薊,得太子丹。王徙遼東。&三十九\\\hline
&二十二。王賁擊魏,得其王假,盡取其地。&三。秦虜王假。&&三&三&三十&四十\\\hline
&二十三。王翦、蒙武擊破楚軍,殺其將項燕。&&&四&四。秦破我將項燕。&三十一&四十一\\\hline
&二十四。王翦、蒙武破楚,虜其王負芻。&&&五&五。秦虜王負芻。秦滅楚。&三十二&四十二\\\hline
&二十五。王賁擊燕,虜王喜。又擊得代王嘉。五月,天下大酺。&&&六。秦將王賁虜王嘉,秦滅趙。&&三十三。秦虜王喜,拔遼東,秦滅燕。&四十三\\\hline
&二十六。王賁擊齊,虜王建。初并天下,立為皇帝。&&&&&&四十四。秦虜王建。秦滅齊。\\\hline
&二十七更命河為德水。為金人十二。命民曰黔首。同天下書。分為三十六郡。&&&&&&\\\hline
&二十八為阿房宮。之衡山。治馳道。帝之琅邪,道南郡入。為太極廟。賜戶三十,爵一級。&&&&&&\\\hline
&二十九郡縣大索十日。帝之琅邪,道上黨入。&&&&&&\\\hline
&三十&&&&&&\\\hline
&三十一更命臘曰嘉平。賜黔首里六石米二羊,以嘉平。大索二十日。&&&&&&\\\hline
&三十二帝之碣石,道上郡入。&&&&&&\\\hline
&三十三遣諸逋亡及賈人贅婿略取陸梁,為桂林、南海、象郡,以適戍。西北取戎為三十四縣。築長城河上,蒙恬將三十萬。&&&&&&\\\hline
&三十四適治獄不直者築長城。取南方越地。覆獄故失。&&&&&&\\\hline
&三十五為直道,道九原,通甘泉。&&&&&&\\\hline
&三十六徙民於北河、榆中,耐徙三處,拜爵一級。石畫下東郡,有文言地分。&&&&&&\\\hline
&三十七十月,帝之會稽、琅邪,還至沙丘崩。子胡亥立,為二世皇帝。殺蒙恬。道九原入。復行錢。&&&&&&\\\hline
&二世元年十月戊寅,大赦罪人。十一月,為兔園。十二月,就阿房宮。其九月,郡縣皆反。楚兵至戲,章邯擊卻之。出衛君角為庶人。&&&&&&\\\hline
&二將軍章邯、長史司馬欣、都尉董翳追楚兵至河。誅丞相斯、去疾,將軍馮劫。&&&&&&\\\hline
&三趙高反,二世自殺,高立二世兄子嬰。子嬰立,剌殺高,夷三族。諸侯入秦,嬰降,為項羽所殺。尋誅羽,天下屬漢。&&&&&&\\\hline
}