\onecolumn
\chapter{高祖功臣侯者年表第六}

太史公曰,古者人臣功有五品,以德立宗廟定社稷曰勳,以言曰勞,用力曰功,明其等曰伐,積日曰閱。封爵之誓曰,使河如帶,泰山若厲。國以永寧,爰及苗裔。始未嘗不欲固其根本,而枝葉稍陵夷衰微也。

余讀高祖侯功臣,察其首封,所以失之者,曰,異哉新聞。書曰協和萬國,遷于夏商,或數千歲。蓋周封八百,幽厲之後,見於春秋。尚書有唐虞之侯伯,歷三代千有餘載,自全以蕃衛天子,豈非篤於仁義,奉上法哉。漢興,功臣受封者百有餘人。天下初定,故大城名都散亡,戶口可得而數者十二三,是以大侯不過萬家,小者五六百戶。後數世,民咸歸鄉里,戶益息,蕭、曹、絳、灌之屬或至四萬,小侯自倍,富厚如之。子孫驕溢,忘其先,淫嬖。至太初百年之閒,見侯五,餘皆坐法隕命亡國,秏矣。罔亦少密焉,然皆身無兢兢於當世之禁云。

居今之世,志古之道,所以自鏡也,未必盡同。帝王者各殊禮而異務,要以成功為統紀,豈可緄乎。觀所以得尊寵及所以廢辱,亦當世得失之林也,何必舊聞?於是謹其終始,表其文,頗有所不盡本末;著其明,疑者闕之。後有君子,欲推而列之,得以覽焉。

\twocolumn