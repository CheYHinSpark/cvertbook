\onecolumn
\chapter{高祖功臣侯者年表第六}

太史公曰,古者人臣功有五品,以德立宗廟定社稷曰勳,以言曰勞,用力曰功,明其等曰伐,積日曰閱。封爵之誓曰,使河如帶,泰山若厲。國以永寧,爰及苗裔。始未嘗不欲固其根本,而枝葉稍陵夷衰微也。

余讀高祖侯功臣,察其首封,所以失之者,曰,異哉新聞。書曰協和萬國,遷于夏商,或數千歲。蓋周封八百,幽厲之後,見於春秋。尚書有唐虞之侯伯,歷三代千有餘載,自全以蕃衛天子,豈非篤於仁義,奉上法哉。漢興,功臣受封者百有餘人。天下初定,故大城名都散亡,戶口可得而數者十二三,是以大侯不過萬家,小者五六百戶。後數世,民咸歸鄉里,戶益息,蕭、曹、絳、灌之屬或至四萬,小侯自倍,富厚如之。子孫驕溢,忘其先,淫嬖。至太初百年之閒,見侯五,餘皆坐法隕命亡國,秏矣。罔亦少密焉,然皆身無兢兢於當世之禁云。

居今之世,志古之道,所以自鏡也,未必盡同。帝王者各殊禮而異務,要以成功為統紀,豈可緄乎。觀所以得尊寵及所以廢辱,亦當世得失之林也,何必舊聞?於是謹其終始,表其文,頗有所不盡本末;著其明,疑者闕之。後有君子,欲推而列之,得以覽焉。
\biao{|l|l|l|l|l|l|l|l|}{
\hline
侯功 & 高祖十二 & 孝惠七 & 高后八 & 孝文二十三 & 孝景十六 & 建元至元封六年三十六,太初元年盡後元二年十八。 & 侯第。 \\ \hline
以中涓從起沛,至霸上,侯。以將軍入漢,以左丞相出征齊、魏,以右丞相為平陽侯,萬六百戶。 & 七。六年十二月甲申,懿侯曹參元年。 & \begin{tabular}[c]{@{}l@{}}五。其二年為相國。\\ 。二。六年十月,靖侯窋元年。\end{tabular} & 八 & \begin{tabular}[c]{@{}l@{}}十九\\ 。四。後四年,簡侯奇元年。\end{tabular} & \begin{tabular}[c]{@{}l@{}}三\\ 。十三。四年,夷侯時。。元年。\end{tabular} & \begin{tabular}[c]{@{}l@{}}十\\ 。十六。元光五年,恭侯襄元年。元鼎三年,今侯宗元年。\end{tabular} & 二。 \\ \hline
}
\twocolumn