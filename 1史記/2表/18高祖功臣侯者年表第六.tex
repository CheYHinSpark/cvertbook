\chapter{高祖功臣侯者年表第六}

太史公曰,古者人臣功有五品,以德立宗廟定社稷曰勳,以言曰勞,用力曰功,明其等曰伐,積日曰閱。封爵之誓曰,使河如帶,泰山若厲。國以永寧,爰及苗裔。始未嘗不欲固其根本,而枝葉稍陵夷衰微也。

余讀高祖侯功臣,察其首封,所以失之者,曰,異哉新聞。書曰協和萬國,遷于夏商,或數千歲。蓋周封八百,幽厲之後,見於春秋。尚書有唐虞之侯伯,歷三代千有餘載,自全以蕃衛天子,豈非篤於仁義,奉上法哉。漢興,功臣受封者百有餘人。天下初定,故大城名都散亡,戶口可得而數者十二三,是以大侯不過萬家,小者五六百戶。後數世,民咸歸鄉里,戶益息,蕭、曹、絳、灌之屬或至四萬,小侯自倍,富厚如之。子孫驕溢,忘其先,淫嬖。至太初百年之閒,見侯五,餘皆坐法隕命亡國,秏矣。罔亦少密焉,然皆身無兢兢於當世之禁云。

居今之世,志古之道,所以自鏡也,未必盡同。帝王者各殊禮而異務,要以成功為統紀,豈可緄乎。觀所以得尊寵及所以廢辱,亦當世得失之林也,何必舊聞。於是謹其終始,表其文,頗有所不盡本末;著其明,疑者闕之。後有君子,欲推而列之,得以覽焉。

\biao{|p{2em}|p{15em}|p{7em}|p{7em}|p{7em}|p{9em}|p{7em}|p{9em}|p{3em}|}{
\hline
國名 & 侯功 & 高祖十二 & 孝惠七 & 高后八 & 孝文二十三 & 孝景十六 & 建元至元封六年三十六,太初元年盡後元二年十八。 & 侯第 \\ \hline
平陽 & 以中涓從起沛,至霸上,侯。以將軍入漢,以左丞相出征齊、魏,以右丞相為平陽侯,萬六百戶。 & 七。六年十二月甲申,懿侯曹參元年。 & 五。其二年為相國。二。六年十月,靖侯窋元年。 & 八 & 十九。四。後四年,簡侯奇元年。 & 三。十三。四年,夷侯時。元年。 & 十。十六。元光五年,恭侯襄元年。元鼎三年,今侯宗元年。 & 二 \\ \hline
信武 & 以中涓從起宛、朐,入漢,以騎都尉定三秦,擊項羽,別定江陵,侯,五千三百戶。以車騎將軍攻黥布、陳豨。 & 七。六年十二月甲申,肅侯靳歙元年。 & 七 & 五。三。六年,夷侯亭元年。 & 十八。後三年,侯亭坐事國人過律,奪侯,國除。 &  &  & 十一 \\ \hline
清陽 & 以中涓從起豐,至霸上,為騎郎將,入漢,以將軍擊項羽功,侯,三千一百戶。 & 七。六年十二月甲申,定侯王吸元年。 & 七 & 八 & 七。元年,哀侯彊元年。十六。八年,孝侯伉元年。 & 四。十二。五年,哀侯不害元年。 & 七。元光二年,侯不害薨,無後,國除。 & 十四 \\ \hline
汝陰 & 以令史從降沛,為太僕,常奉車,為滕公,竟定天下,入漢中,全孝惠、魯元,侯,六千九百戶。常為太僕。 & 七。六年十二月甲申,文侯夏侯嬰元年。 & 七 & 八 & 八。七。九年,夷侯灶元年。八。十六年,恭侯賜元年。 & 十六 & 七。元光二年,侯頗元年。十九。元鼎二年,侯頗坐尚公主,與父御婢姦罪,自殺,國除。 & 八 \\ \hline
陽陵 & 以舍人從起橫陽,至霸上,為騎將,入漢,定三秦,屬淮陰,定齊,為齊丞相,侯,二千六百戶。 & 七。六年十二月甲申,景侯傅寬元年。 & 五。二。六年,頃侯靖元年。 & 八 & 十四。九。十五年,恭侯則元年。 & 三。十三。前四年,侯偃元年。 & 十八。元狩元年,偃坐與淮南王謀反,國除。 & 十 \\ \hline
廣嚴 & 以中涓從起沛,至霸上,為連敖,入漢,以騎將定燕、趙,得將軍,侯,二千二百戶。 & 七。六年十二月甲申,壯侯召歐元年。 & 七 & 八 & 十九。二年,戴侯勝元年。十三。十一年,恭侯嘉元年。至後七年嘉薨,無後,國除。 &  &  & 二十八 \\ \hline
廣平 & 以舍人從起豐,至霸上,為郎中,入漢,以將軍擊項羽、鍾離眛功,侯,四千五百戶。 & 七。六年十二月甲申,敬侯薛歐元年。 & 七 & 八。元年,靖侯山元年。 & 十八。五。後三年,侯澤元年。 & 八。中二年,有罪,絕。平棘五。中五年,復封節侯澤元年。 & 十五。其十年,為丞相。三。元朔四年,侯穰元年。元狩元年,穰受淮南王財物,稱臣,在赦前,詔問謾罪,國除。 & 十五 \\ \hline
博陽 & 以舍人從起碭,以刺客將,入漢,以都尉擊項羽滎陽,絕甬道,擊殺追卒功,侯。 & 七。六年十二月甲申,壯侯陳濞元年。 & 七 & 八 & 十八。五。後三年,侯始元年。 & 四。前五年,侯始有罪,國除。塞二。中五年,復封始。後元年,始有罪,國除。 &  & 十九 \\ \hline
曲逆 & 以故楚都尉,漢王二年初從修武,為都尉,遷為護軍中尉;出六奇計,定天下,侯,五千戶。 & 七。六年十二月甲申,獻侯陳平元年。 & 七。其五年,為左丞相。 & 八。其元年,徙為右丞相;後專為丞相,相孝文二年。 & 二。二。三年,恭侯買元年。十九。五年,簡侯悝元年。 & 四。十二。五年,侯何元年。 & 十。元光五年,侯何坐略人妻,棄市,國除。 & 四十七 \\ \hline
堂邑 & 以自定東陽,為將,屬項梁,為楚柱國。四歲,項羽死,屬漢,定豫章、浙江都浙自立為王壯息,侯,千八百戶。復相楚元王十一年。 & 七。六年十二月甲申,安侯陳嬰元年。 & 七 & 四。四。五年,恭侯祿元年。 & 二。二十一。三年,夷侯午元年。 & 十六 & 十一。元光六年,季須元年。十三。元鼎元年,侯須坐母長公主卒,未除服姦,兄弟爭財,當死,自殺,國除。 & 八十六 \\ \hline
周呂 & 以呂后兄初起以客從,入漢為侯。還定三秦,將兵先入碭。漢王之解彭城,往從之,復發兵佐高祖定天下,功侯。 & 三。六年正月丙戌,令武侯呂澤元年。四。九年,子台封酈侯元年。 & 七 &  &  &  &  &  \\ \hline
建成 & 以呂后兄初起以客從,擊三秦。漢王入漢,而釋之還豐沛,奉衛呂宣王、太上皇。天下已平,封釋之為建成侯。 & 七。六年正月丙戌,康侯釋之元年。 & 二。五。三年,侯則元年。有罪。 & 胡陵七。元年,五月丙寅,封則弟大中大夫呂祿元年。七年,祿為趙王,國除。追尊康侯為昭王。祿以趙王謀為不善,大臣誅祿,遂滅呂。 &  &  &  &  \\ \hline
留 & 以廄將從起下邳,以韓申徒下韓國,言上張旗志,秦王恐,降,解上與項羽之纳,為漢王請漢中地,常計謀平天下,侯,萬戶。 & 七。六年正月丙午,文成侯張良。元年。 & 七 & 二。六。三年,不疑元年。 & 四。五年,侯不疑坐與門大夫謀殺故楚內史,當死,贖為城旦,國除。 &  &  & 六十二 \\ \hline
射陽 & 兵初起,與諸侯共擊秦,為楚左令尹,漢王與項羽有纳於鴻門,項伯纏解難,以破羽纏嘗有功,封射陽侯。 & 七。六年正月丙午,侯項纏元年。賜姓劉氏。 & 二。三年,侯纏卒。嗣子睢有罪,國除。 &  &  &  &  &  \\ \hline
酇 & 以客初起從入漢,為丞相,備守蜀及關中,給軍食,佐上定諸侯,為法令,立宗廟,侯,八千戶。 & 七。六年正月丙午,文終侯蕭何元年。元年,為丞相;九年,為相國。 & 二。五。三年,哀侯祿元年。 & 一。七。二年,懿侯同元年。同,祿弟。 & 筑陽十九。元年,同有罪,封何小子延元年。一。後四年,煬侯遺元年。三。後五年,侯則元年。 & 一。有罪。武陽七。前二年,封煬侯弟幽侯嘉元年。八。中二年,侯勝元年。 & 十。元朔二年,侯勝坐不敬,絕。三。元狩三年,封何曾孫恭侯慶元年。酇三。元狩六年,侯壽成元年。十。元封四年,壽成為太常,犧牲不如令,國除。 & 一 \\ \hline
曲周 & 以將軍從起岐,攻長社以南,別定漢中及蜀,定三秦,擊項羽,侯,四千八百戶。 & 七。六年正月丙午,景侯酈商元年。 & 七 & 八 & 二十三。元年,侯寄元年。 & 九。有罪。繆七。中三年,封商他子靖侯堅元年。 & 九。元光四年,康侯遂元年。五。元朔三年,侯宗元年。十一。元鼎二年,侯終根元年。二十八。後元二年,侯終根坐咒詛誅,國除。 & 六 \\ \hline
絳 & 以中涓從起沛,至霸上,為侯。定三秦,食邑,為將軍。入漢,定隴西,擊項羽,守嶢關,定泗水、東海。八千一百戶。 & 七。六年正月丙午,武侯周勃元年。 & 七 & 八。其四年為太尉。 & 十一。元年,為右丞相,三年,免。復為丞相。六。十二年,侯勝之元年。條六。後二年,封勃子亞夫元年。 & 十三。其三年,為太尉;七年,為丞相。有罪,國除。平曲三。後元年,封勃子恭侯堅元年。 & 十六。元朔五年,侯建德元年。十二。元鼎五年,侯建德坐酎金,國除。 & 四 \\ \hline
舞陽 & 以舍人起沛,從至霸上,為侯。入漢,定三秦,為將軍,擊項籍,再益封。從破燕,執韓信,侯,五千戶。 & 七。六年正月丙午,武侯樊噲元年。其七年,為將軍、相國三月。 & 六。一。七年,侯伉元年。呂須子。 & 八。坐呂氏誅,族。 & 二十三。元年,封樊噲子荒侯市人元年。 & 六。七年,侯它廣元年。六。中六年,侯它廣非市人子,國除。 &  & 五 \\ \hline
潁陰 & 以中涓從起碭,至霸上,為昌文君。入漢,定三秦,食邑。以車騎將軍屬淮陰,定齊、淮南及下邑,殺項籍,侯,五千戶。 & 七。六年正月丙午,懿侯灌嬰元年。 & 七 & 八 & 四。其一,為太尉;三,為丞相。十九。五年,平侯何元年。 & 九。七。中三年,侯彊元年。 & 六。有罪,絕。九。元光二年,封嬰孫賢為臨汝侯。侯賢元年。元朔五年,侯賢行賕罪,國除。 & 九 \\ \hline
汾陰 & 初起以職志擊破秦,入漢,出關,以內史堅守敖倉,以御史大夫定諸侯,比清陽侯,二千八百戶。 & 七。六年正月丙午,悼侯周昌元年。 & 三。建平四。四年,哀侯開方元年。 & 八 & 四。前五年,侯意元年。十三。有罪,絕。 & 安陽八。中二年,封昌孫左車。 & 建元元年,有罪,國除。 & 十六 \\ \hline
梁鄒 & 兵初起,以謁者從擊破秦,入漢,以將軍擊定諸侯功,比博陽侯,二千八百戶。 & 七。六年正月丙午,孝侯武儒元年。 & 四。三。五年,侯最元年。 & 八 & 二十三 & 十六 & 六。元光元年,頃侯嬰齊元年。三。元光四年,侯山柎元年。二十。元鼎五年,侯山柎坐酎金,國除。 & 二十 \\ \hline
成 & 兵初起,以舍人從擊秦,為都尉;入漢,定三秦。出關,以將軍定諸侯功,比厭次侯,二千八百戶。 & 七。六年正月丙午,敬侯董渫元年。 & 七。元年,康侯赤元年。 & 八 & 二十三 & 六。有罪,絕。節氏五。中五年,復封康侯赤元年。 & 三。建元四年,恭侯罷軍元年。五。元光三年,侯朝元年。十二。元狩三年,侯朝為濟南太守,與成陽王女通,不敬,國除。 & 二十五 \\ \hline
蓼 & 以執盾前元年從起碭,以左司馬入漢,為將軍,三以都尉擊項羽,屬韓信,功侯。 & 七。六年正月丙午,侯孔藂元年。 & 七 & 八 & 八。十五。九年,侯臧元年。 & 十六 & 十四。元朔三年,侯臧坐為太常,南陵橋壞,衣冠車不得度,國除。 & 三十 \\ \hline
費 & 以舍人前元年從起碭,以左司馬入漢,用都尉屬韓信,擊項羽有功,為將軍,定會稽、浙江、湖陽,侯。 & 七。六年正月丙午,圉侯陳賀元年。 & 七 & 八 & 二十三。元年,共侯常元年。 & 一。二年,侯偃元年。中二年,有罪,絕。八。中六年,封賀子侯最元年。巢四。後三年,最薨,無後,國除。 &  &  \\ \hline
陽夏 & 以特將將卒五百人,前元年從起宛、朐,至霸上,為侯,以游擊將軍別定代,已破臧荼,封豨為陽夏侯。 & 五。六年,正月丙午,侯陳豨元年。十年,八月,豨以趙相國將兵守代。漢使召豨,豨反,以其兵與王黃等略代,自立為王。漢殺豨靈丘。 &  &  &  &  &  &  \\ \hline
隆慮 & 以卒從起碭,以連敖入漢,以長鈹都尉。擊項羽,有功,侯。 & 七。六年正月丁未,哀侯周灶元年。 & 七 & 八 & 十七。六。後二年,侯通元年。 & 七。中元年,侯通有罪,國除。 &  & 三十四 \\ \hline
陽都 & 以趙將從起鄴,至霸上,為樓煩將,入漢,定三秦,別降翟王,屬悼武王,殺龍且彭城,為大司馬;破羽軍葉,拜為將軍,忠臣,侯,七千八百戶。 & 七。六年正月戊申,敬侯丁復元年。 & 七 & 五。三。六年,趮侯甯元年。 & 九。十四。十年,侯安成元年。 & 一。二年,侯安成有罪,國除。 &  & 十七 \\ \hline
新陽 & 以漢五年用左令尹初從,功比堂邑侯,千戶。 & 七。六年正月壬子,胡侯呂清元年。 & 三。四。四年,頃侯臣元年。 & 八 & 六。二。七年,懷侯義元年。十五。九年,惠侯它元年。 & 四。五。五年,恭侯善元年。七。中三年,侯譚元年。 & 二十八。元鼎五年,侯譚坐酎金,國除。 & 八十一 \\ \hline
東武 & 以戶衛。起薛,屬悼武王,破秦軍杠里,楊熊軍曲遇,入漢,為越。將軍,定三秦,以都尉堅守敖倉,為將軍,破籍軍,功侯,二千戶。 & 七。六年正月戊午,貞侯郭蒙元年。 & 七 & 五。三。六年,侯它元年。 & 二十三 & 五。六年,侯它棄市,國除。 &  & 四十一 \\ \hline
汁方 & 以趙將前三年從定諸侯,侯,二千五百戶,功比平定侯。齒故沛豪,有力,與上有纳,故晚從。 & 七。六年三月戊子,肅侯雍齒元年。 & 二。五。三年,荒侯巨元年。 & 八 & 二十三 & 二。十。三年,侯野元年。四。中六年,終侯桓元年。 & 二十八。元鼎五年,終侯桓坐酎金,國除。 & 五十七 \\ \hline
棘蒲 & 以將軍前元年率將二千五百人起薛,別救東阿,至霸上,二歲十月入漢,擊齊歷下軍田既,功侯。 & 七。六年三月丙申,剛侯陳武元年。 & 七 & 八 & 十六。後元年,侯武薨。嗣子奇反,不得置後,國除。 &  &  & 十三 \\ \hline
都昌 & 以舍人前元年從起沛,以騎隊率先降翟王,虜章邯,功侯。 & 七。六年三月庚子,莊侯朱軫元年。 & 七 & 八。元年,剛侯率元年。 & 七。十六。八年,夷侯詘元年。 & 二。元年,恭侯偃元年。五。三年,侯辟彊元年。中元年,辟彊薨,無後,國除。 &  & 二十三 \\ \hline
武彊 & 以舍人從至霸上,以騎將入漢。還擊項羽,屬丞相甯,功侯,用將軍擊黥布,侯。 & 七。六年三月庚子,莊侯莊不識元年。 & 七 & 六。二。七年,簡侯嬰元年。 & 十七。六。後二年,侯青翟元年。 & 十六 & 二十五。元鼎二年,侯青翟坐為丞相與長史朱買臣等逮御史大夫湯不直,國除。 & 三十三 \\ \hline
貰 & 以越戶將從破秦,入漢,定三秦,以都尉擊項羽,千六百戶,功比臺侯。 & 二。六年,三月庚子,齊侯呂元年。五。八年,恭侯方山元年。 & 七 & 八 & 二。元年,煬侯赤元年。十二。十二年,康侯遺元年。 & 十六 & 十六。元朔五年,侯倩。元年。八。元鼎元年,侯倩坐殺人棄市,國除。 & 三十六 \\ \hline
海陽 & 以越隊將從破秦,入漢定三秦,以都尉擊頨羽,侯,千八百戶。 & 七。六年三月庚子,齊信侯搖毋餘元年。 & 二。五。三年,哀侯招攘元年。 & 四。四。五年,康侯建元年。 & 二十三 & 三。四年,哀侯省元年。十。中六年,侯省薨,無後,國除。 &  & 三十七 \\ \hline
南安 & 以河南將軍漢王三年降晉陽,以亞將破臧荼,侯,九百戶。 & 七。六年三月庚子,莊侯宣虎元年。 & 七 & 八 & 八。十一。九年,共侯戎元年。四。後四年,侯千秋元年。 & 七。中元年,千秋坐傷人免。 &  & 六十三 \\ \hline
肥如 & 以魏太僕三年初從,以車騎都尉破龍且及彭城,侯,千戶。 & 七。六年三月庚子,敬侯蔡寅元年。 & 七 & 八 & 二。十四。三年,莊侯成元年。七。後元年,侯奴元年。 & 元年,侯奴薨,無後,國除。 &  & 六十六 \\ \hline
曲城 & 以曲城戶將卒三十七人初從起碭,至霸上,為執珪,為二隊將,屬悼武王,入漢,定三秦,以都尉破項羽軍陳下,功侯,四千戶。為將軍,擊燕、代,拔之。 & 七。六年三月庚子,圉侯蠱逢元年。 & 七 & 八 & 八。元年,侯捷元年。有罪,絕。五。後三年,復封恭侯捷元年。 & 十三。有罪,絕。垣五。中五年,復封恭侯捷元年。 & 一。建元二年,侯皋柔元年。二十五。元鼎三年,侯皋柔坐為汝南太守知民不用赤側錢為賦。國除。 & 十八 \\ \hline
河陽 & 以卒前元年起碭從,以二隊將入漢,擊項羽,身得郎將處,功侯。以丞相定齊地。 & 七。六年三月庚子,莊侯陳涓元年。 & 七 & 八 & 三。元年,侯信元年。四年,侯信坐不償人責過六月,奪侯,國除。 &  &  & 二十九 \\ \hline
淮陰 & 兵初起,以卒從項梁,梁死屬項羽為郎中,至咸陽,亡從入漢,為連敖典客,蕭何言為大將軍,別定魏、齊,為王,徙楚,坐擅發兵,廢為淮陰侯。 & 五。六年四月,侯韓信元年。十一年,信謀反關中,呂后誅信,夷三族,國除。 &  &  &  &  &  &  \\ \hline
芒 & 以門尉前元年初起碭,至霸上,為武定君,入漢,還定三秦,以都尉擊項羽,侯。 & 三。六年,侯昭元年。九年,侯昭有罪,國除。 &  &  &  & 張十一。孝景三年,昭以故芒侯將兵從太尉亞夫擊吳楚有功,復侯。三。後元年三月,侯申元年。 & 十七。元朔六年,侯申坐尚南宮公主。不敬,國除。 &  \\ \hline
故市 & 以執盾初起,入漢,為河上守,遷為假相,擊項羽,侯,千戶,功比平定侯。 & 三。六年四月癸未,侯閻澤赤元年。四。九年,夷侯毋害元年。 & 七 & 八 & 十九。四。後四年,戴侯續元年。 & 四。十二。孝景五年,侯穀嗣。 & 二十八。元鼎五年,侯穀坐酎金,國除。 & 五十五 \\ \hline
柳丘 & 以連敖從起薛,以二隊將入漢,定三秦,以都尉破項籍軍,為將軍。侯,千戶。 & 七。六年六月丁亥,齊侯戎賜元年。 & 七 & 四。四。五年,定侯安國元年。 & 二十三 & 三。四年,敬侯嘉成元年。十。後元年,侯角嗣,有罪,國除。 &  & 二十六 \\ \hline
魏其 & 以舍人從沛,以郎中入漢,為周信侯,定三秦,遷為郎中騎將,破籍東城,侯,千戶。 & 七。六年六月丁亥,莊侯周定元年。 & 七 & 四。四。五年,侯閒元年。 & 二十三 & 二。前三年,侯閒反,國除。 &  & 四十四 \\ \hline
祁 & 以執盾漢王三年初起從晉陽,以連敖擊項籍,漢王敗走,賀方將軍擊楚,追騎以故不得進。漢王顧謂賀:「子留彭城,用執圭東擊羽,急絕其近壁。」侯,千四百戶。 & 七。六年六月丁亥,穀侯繒賀元年。 & 七 & 八 & 十一。十二。十二年,頃侯湖元年。 & 五。十一。六年,侯它元年。 & 八。元光二年,侯它坐從射擅罷,不敬,國除。 & 五十一 \\ \hline
平 & 兵初起,以舍人從擊秦,以郎中入漢,以將軍定諸侯,守洛陽,功侯,比費侯賀,千三百戶。 & 六。六年六月丁亥,悼侯沛嘉元年。一。十二年,靖侯奴元年。 & 七 & 八 & 十五。八。十六年,侯執元年。 & 十一。中五年,侯執有罪,國除。 &  & 三十二 \\ \hline
魯 & 以舍人從起沛,至咸陽為郎中,入漢,以將軍從定諸侯,侯,四千八百戶,功比舞陽侯。死事,母代侯。 & 七。六年中,母侯疵元年。 & 七 & 四。五年,母侯疵薨,無後,國除。 &  &  &  & 七 \\ \hline
故城 & 兵初起,以謁者從,入漢,以將軍擊諸侯,以右丞相備守淮陽功,比厭次侯,二千戶。 & 七。六年中,莊侯尹恢元年。 & 二。五。三年,侯開方元年。 & 二。三年,侯方奪侯,為關內侯。 &  &  &  & 二十六 \\ \hline
任 & 以騎都尉漢五年從起東垣,擊燕、代,屬雍齒,有功,侯。為車騎將軍。 & 七。六年,侯張越。元年。 & 七 & 二。三年,侯越坐匿死罪,免為庶人,國除。 &  &  &  &  \\ \hline
棘丘 & 以執盾隊史前元年從起碭,破秦,以治粟內史入漢,以上郡守擊定西魏地,功侯。 & 七。六年,侯襄。元年。 & 七 & 四。四年,侯襄奪侯,為士伍,國除。 &  &  &  &  \\ \hline
阿陵 & 以連敖前元年從起單父,以塞疏入漢。 & 七。六年七月庚寅,頃侯郭亭元年。 & 七 & 八 & 二。二十一。三年,惠侯歐元年。 & 一八。前二年,侯勝客元年。有罪,絕。南四。中六年,靖侯延居元年。 & 十一。元光六年,侯則元年。十七。元鼎五年,侯則坐酎金,國除。 & 二十七 \\ \hline
昌武 & 初起以舍人從,以郎中入漢,定三秦,以郎中將擊諸侯,侯,九百八十戶,比魏其侯。 & 七。六年七月庚寅,靖信侯單甯元年。 & 五。二。六年,夷侯如意元年。 & 八 & 二十三 & 十。六。中四年,康侯賈成元年。 & 十。元光五年,侯得元年。四。元朔三年,侯得坐傷人二旬內死,棄市,國除。 & 四十五 \\ \hline
高苑 & 初起以舍人從,入漢,定三秦,以中尉破籍,侯,千六百戶,比斥丘侯。 & 七。六年七月戊戌,制侯丙倩元年。 & 七。元年,簡侯得元年。 & 八 & 十五。八。十六年,孝侯武元年。 & 十六 & 二。建元元年,侯信元年。建元三年,侯信坐出入屬車閒,奪侯,國除。 & 四十一 \\ \hline
宣曲 & 以卒從起留,以騎將入漢,定三秦,破籍軍滎陽,為郎騎將,破鍾離昧軍固陵,侯,六百七十戶。 & 七。六年七月戊戌,齊侯丁義元年。 & 七 & 八 & 十。十三。十一年,侯通元年。 & 四。有罪,除。發婁。中五年,復封侯通元年。中六年,侯通有罪,國除。 &  & 四十三 \\ \hline
絳陽 & 以越將從起留,入漢,定三秦,擊臧荼,侯,七百四十戶。從攻馬邑及布。 & 七。六年七月戊戌,齊侯華無害元年。 & 七 & 八 & 三。十六。四年,恭侯勃齊元年。四。後四年,侯祿元年。 & 三。四年,侯祿坐出界,有罪,國除。 &  & 四十六 \\ \hline
東茅 & 以舍人從起碭,至霸上,以二隊入漢,定三秦,以都尉擊項羽,破臧荼,侯。捕韓信,為將軍,益邑千戶。 & 七。六年八月丙辰,敬侯劉釗元年。 & 七 & 八 & 二。三年,侯吉元年。十三。十六年,侯吉奪爵,國除。 &  &  & 四十八 \\ \hline
斥丘 & 以舍人從起豐,以左司馬入漢,以亞將攻籍,剋敵,為東郡都尉,擊破籍武城,侯,為漢中尉,擊布,為斥丘侯。千戶。 & 七。六年八月丙辰,懿侯唐厲元年。 & 七 & 八 & 八。十三。九年,恭侯晁元年。二。後六年,侯賢元年。 & 十六 & 二十五。元鼎二年,侯尊元年。三。元鼎五年,侯尊坐酎金,國除。 & 四十 \\ \hline
臺 & 以舍人從起碭,用隊率入漢,以都尉擊籍,籍死,轉擊臨江,屬將軍賈,功侯。以將軍擊燕。 & 七。六年八月甲子,定侯戴野元年。 & 七 & 八 & 三。二十。四年,侯才元年。 & 二。三年,侯才反,國除。 &  & 三十五 \\ \hline
安國 & 以客從起豐,以廄將別定東郡、南陽,從至霸上。入漢,守豐。上東,因從戰不利,奉孝惠、魯元出睢水中,及堅守豐,封雍侯,五千戶。 & 七。六年八月甲子,武侯王陵元年。定侯安國。 & 七。其六年,為右丞相。 & 七。一。八年,哀侯忌元年。 & 二十三。元年,終侯游元年。 & 十六 & 二十。建元元年,三月,安侯辟方元年。八。元狩三年,侯定元年。元鼎五年,侯定坐酎金,國除。 & 十二 \\ \hline
樂成 & 以中涓騎從起碭中,為騎將,入漢,定三秦,侯。以都尉擊籍,屬灌嬰,殺龍且,更為樂成侯,千戶。 & 七。六年八月甲子,節侯丁禮元年。 & 七 & 八 & 四。十八。五年,夷侯馬從元年。一。後七年,武侯客元年。 & 十六 & 二十五。元鼎二年,侯義元年。三。元鼎五年,侯義坐言五利侯不道,棄市,國除。 & 四十二 \\ \hline
辟陽 & 以舍人初起,侍呂后、孝惠沛三歲十月,呂后入楚,食其從一歲,侯。 & 七。六年八月甲子,幽侯審食其元年。 & 七 & 八 & 三。二十。四年,侯平元年。 & 二。三年,平坐反,國除。 &  & 五十九 \\ \hline
安平 & 以謁者漢王三年初從,定諸侯,有功秩,舉蕭何,功侯,二千戶。 & 七。六年八月甲子,敬侯諤千秋元年。 & 二。五。孝惠三年,簡侯嘉元年。 & 七。一。八年,頃侯應元年。 & 十三。十。十四年,煬侯寄元年。 & 十五。一。後三年,侯但元年。 & 十八。元狩元年,坐與淮南王女陵通,遺淮南書稱臣盡力,棄市,國除。 & 六十一 \\ \hline
蒯成 & 以舍人從起沛,至霸上,侯。入漢,定三秦,食邑池陽。擊項羽軍滎陽,絕甬道,從出,度平陰,遇淮陰侯軍襄國。楚漢約分鴻溝,以惞為信,戰不利,不敢離上,侯,三千三百戶。 & 七。六年八月甲子,尊侯周惞元年。十二年十月乙未,定蒯成。 & 七 & 八 & 五。惞薨,子昌代。有罪,絕,國除。 & 鄲一。中元年,封惞子康侯應元年。八。中二年,侯中居元年。 & 二十六。元鼎三年,居坐為太常有罪,國除。 & 二十一 \\ \hline
北平 & 以客從起陽武,至霸上,為常山守,得陳餘,為代相,徙趙相,侯。為計相四歲,淮南相十四歲。千三百戶。 & 七。六年八月丁丑,文侯張倉元年。 & 七 & 八 & 二十三。其四為丞相。五歲罷。 & 五。八。六年,康侯奉元年。三。後元年,侯預元年。 & 四。建元五年,侯預坐臨諸侯喪後,不敬,國除。 & 六十五 \\ \hline
高胡 & 以卒從起杠里,入漢,以都尉擊籍,以都尉定燕,侯,千戶。 & 七。六年中,侯陳夫乞元年。 & 七 & 八 & 四。五年,殤侯程嗣。薨,無後,國除。 &  &  & 八十二 \\ \hline
厭次 & 以慎將前元年從起留,入漢,以都尉守廣武,功侯。 & 七。六年中,侯元頃元年。 & 七 & 八 & 五。元年,侯賀元年。六年,侯賀謀反,國除。 &  &  & 二十四 \\ \hline
平皋 & 項它,漢六年以碭郡長初從,賜姓為劉氏;功比戴侯彭祖,五百八十戶。 & 六。七年十月癸亥,煬侯劉它元年。 & 四。三。五年,恭侯遠元年。 & 八 & 二十三 & 十六。元年,節侯光元年。 & 二十八。建元元年,侯勝元年。元鼎五年,侯勝坐酎金,國除。 & 百二十一 \\ \hline
復陽 & 以卒從起薛,以將軍入漢,以右司馬擊項籍,侯,千戶。 & 六。七年十月甲子,剛侯陳胥元年。 & 七 & 八 & 十。十三。十一年,恭侯嘉元年。 & 五。十一。六年,康侯拾元年。 & 十二。元朔元年,侯彊元年。七。元狩二年,坐父拾非嘉子,國除。 & 四十九 \\ \hline
陽河 & 以中謁者從入漢,以郎中騎從定諸侯,侯五百戶,功比高胡侯。 & 三。七年十月甲子,齊哀侯元年。三。十年,侯安國元年。 & 七 & 八 & 二十三 & 十。六。中四年,侯午元年。中絕。 & 二十七。元鼎四年,恭侯章元年。埤山三。元封元年,侯仁元年。二十。征和三年,十月,仁與母坐祝詛,大逆無道,國除。 & 八十三 \\ \hline
朝陽 & 以舍人從起薛,以連敖入漢,以都尉擊項羽,後攻韓王信,侯,千戶。 & 六。七年三月壬寅,齊侯華寄元年。 & 七 & 八。元年,文侯要元年。 & 十三。十。十四年,侯當元年。 & 十六 & 十三。元朔二年,侯當坐教人上書枉法罪,國除。 & 六十九 \\ \hline
棘陽 & 以卒從起胡陵,入漢,以郎將迎左丞相軍以擊項籍,侯,千戶。 & 六。七年七月丙申,莊侯。杜得臣元年。 & 七 & 八 & 五。十八。六年,質侯但元年。 & 十六 & 九。元光四年,懷侯武元年。七。元朔五年,薨,無後,國除。 & 八十一 \\ \hline
涅陽 & 以騎士漢王二年從出關,以郎將擊斬項羽,侯,千五百戶,比杜衍侯。 & 六。七年中,莊侯。呂勝元年。 & 七 & 八 & 四。五年,莊侯子成實非子,不當為侯,國除。 &  &  & 百四 \\ \hline
平棘 & 以客從起亢父,斬章邯所署蜀守,用燕相侯,千戶。 & 六。七年中,懿侯執元年。 & 七 & 七。一。八年,侯辟彊元年。 & 五。六年,侯辟彊有罪,為鬼薪,國除。 &  &  & 六十四 \\ \hline
羹頡 & 以高祖兄子從軍,擊反韓王信,為郎中將。信母嘗有罪高祖微時,太上憐之,故封為羹頡侯。 & 六。七年中,侯劉信元年。 & 七 & 元年,信有罪,削爵一級,為關內侯。 &  &  &  &  \\ \hline
深澤 & 以趙將漢王三年降,屬淮陰侯,定趙、齊、楚,以擊平城,侯,七百戶。 & 五。八年十月癸丑,齊侯趙將夜元年。 & 七 & 一。奪,絕。三年復封,一年絕。 & 四。十四年,復封將夜元年。六。後二年,戴侯頭元年。 & 二。七。三年,侯循元年。罪,絕。更五。中五年,封頭子夷侯胡元年。 & 十六。元朔五年,夷侯胡薨,無後,國除。 & 九十八 \\ \hline
柏至 & 以駢憐從起昌邑,以說衛入漢,以中尉擊籍,侯,千戶。 & 六。七年十月戊辰,靖侯許溫元年。 & 七 & 一。二年,有罪,絕。六。三年,復封溫如故。 & 十四。元年,簡侯祿元年。九。十五年,哀侯昌元年。 & 十六 & 七。元光二年,共侯安如元年。十三。元狩三年,侯福元年。五。元鼎二年,侯福有罪,國除。 & 五十八 \\ \hline
中水 & 以郎中騎將漢王元年從起好畤,以司馬擊龍且,復共斬項羽,侯,千五百戶。 & 六。七年正月己酉,莊侯。呂馬童元年。 & 七 & 八 & 九。三。十年,夷侯假元年。十一。十三年,共侯青肩元年。 & 十六 & 五。建元六年,靖侯德元年。一。元光元年,侯宜成元年。二十三。元鼎五年,宜成坐酎金,國除。 & 百一 \\ \hline
杜衍 & 以郎中騎漢王三年從起下邳,屬淮陰,從灌嬰共斬項羽,侯,千七百戶。 & 六。七年正月己酉,莊侯王翳元年。 & 七 & 五。三。六年,共侯福元年。 & 四。七。五年,侯市臣元年。十二。十二年,侯翕元年。 & 十二。有罪,絕。三。後元年,復封翳子彊侯郢人元年。 & 九。元光四年,侯定國元年。十二。元狩四年,侯定國有罪,國除。 & 百二 \\ \hline
赤泉 & 以郎中騎漢王二年從起杜,屬淮陰,後從灌嬰共斬項羽,侯,千九百戶。 & 六。七年正月己酉,莊侯楊喜元年。 & 七 & 元年,奪,絕。七。二年,復封。 & 十一。十二。十二年,定侯殷元年。 & 三。四年,侯無害元年。六。有罪,絕。臨汝五。中五年,復封侯無害元年。 & 七。元光二年,侯無害有罪,國除。 & 百三 \\ \hline
栒 & 以燕將軍漢王四年從曹咎軍,為燕相,告燕王荼反,侯,以燕相國定盧奴,千九百戶。 & 五。八年十月丙辰,頃侯溫疥元年。 & 七 & 八 & 五。十七。六年,文侯仁元年。一。後七年,侯河元年。 & 十。中四年,侯河有罪,國除。 &  & 九十一 \\ \hline
武原 & 漢七年,以梁將軍初從擊韓信、陳豨、黥布功,侯,二千八百戶,功比高陵。 & 五。八年十二月丁未,靖侯衛胠元年。 & 三。四。四年,共侯寄元年。 & 八 & 二十三 & 三。四年,侯不害元年。十三。後二年,不害坐葬過律,國除。 &  & 九十三 \\ \hline
磨 & 以趙衛將軍漢王三年從起盧奴,擊項羽敖倉下,為將軍,攻臧荼有功,侯,千戶。 & 五。八年七月癸酉,簡侯程黑元年。 & 七 & 二。六。三年,孝侯釐元年。 & 十六。七。後元年,侯灶元年。 & 七。中元年,灶有罪,國除。 &  & 九十二 \\ \hline
槁 & 高帝七年,為將軍,從擊代陳豨有功,侯,六百戶。 & 五。八年十二月丁未,祗侯陳錯元年。 & 二。五。三年,懷侯嬰元年。 & 八 & 六。十四。七年,共侯應元年。三。後五年,侯安元年。 & 十六 & 十二。不得,千秋父。七。元狩二年,侯千秋元年。九。元鼎五年,侯千秋坐酎金,國除。 & 百二十四 \\ \hline
宋子 & 以漢三年以趙羽林將初從,擊定諸侯,功比磨侯,五百四十戶。 & 四。八年十二月丁卯,惠侯許瘛元年。一。十二年,共侯不疑元年。 & 七 & 八 & 九。十四。十年,侯九元年。 & 八。中二年,侯九坐買塞外禁物罪,國除。 &  & 九十九 \\ \hline
猗氏 & 以舍人從起豐,入漢,以都尉擊項羽,侯,二千四百戶。 & 五。八年三月丙戌,敬侯陳鸱元年。 & 六。一。七年,靖侯交元年。 & 八 & 二十三 & 二。三年,頃侯差元年。薨,無後,國除。 &  & 五十 \\ \hline
清 & 以弩將初起,從入漢,以都尉擊項羽、代,侯,比彭侯,千戶。 & 五。八年三月丙戌,簡侯空中元年。 & 七。元年,頃侯聖元年。 & 八 & 七。十六。八年,康侯鮒元年。 & 十六 & 二十。元狩三年,恭侯石元年。七。元鼎四年,侯生元年。一。元鼎五年,生坐酎金,國除。 & 七十一 \\ \hline
彊 & 以客吏初起,從入漢,以都尉擊項羽、代,侯,比彭侯千戶。 & 三。八年三月丙戌,簡侯留勝元年。二。十一年,戴侯章元年。 & 七 & 八 & 十二。十三年,侯服元年。二。十五年,侯服有罪,國除。 &  &  & 七十二 \\ \hline
彭 & 以卒從起薛,以弩將入漢,以都尉擊項羽、代,侯,千戶。 & 五。八年三月丙戌,簡侯秦同元年。 & 七 & 八 & 二。二十一。三年,戴侯執元年。 & 二。三年,侯武元年。十一。後元年,侯武有罪,國除。 &  & 七十 \\ \hline
吳房 & 以郎中騎將漢王元年從起下邽,擊陽夏,以都尉斬項羽,有功,侯,七百戶。 & 五。八年三月辛卯,莊侯楊武元年。 & 七 & 八 & 十二。十一。十三年,侯去疾元年。 & 十四。後元年,去疾有罪,國除。 &  & 九十四 \\ \hline
甯 & 以舍人從起碭,入漢,以都尉擊臧荼功,侯,千戶。 & 五。八年四月辛酉,莊侯魏選元年。 & 七 & 八 & 十五。八。十六年,恭侯連元年。 & 三。元年,侯指元年。四年,侯指坐出國界,有罪,國除。 &  & 七十八 \\ \hline
昌 & 以齊將漢王四年從淮陰侯起無鹽,定齊,擊籍及韓王信於代,侯,千戶。 & 五。八年六月戊申,圉侯盧卿元年。 & 七 & 八 & 十四。九。十五年,侯通元年。 & 二。三年,侯通反,國除。 &  & 百九 \\ \hline
共 & 以齊將漢王四年從淮陰侯起臨淄,擊籍及韓王信於平城,有功,侯,千二百戶。 & 五。八年六月壬子,莊侯盧罷師元年。 & 七 & 八 & 六。七年,惠侯黨元年。八。十五年,懷侯商元年。五。後四年,侯商薨,無後,國除。 &  &  & 百十四 \\ \hline
閼氏 & 以代太尉漢王三年降,為鴈門守,以特將平代反寇,侯,千戶。 & 四。八年六月壬子,節侯馮解敢元年。一。十二年,恭侯它元年。薨,無後,絕。 &  &  & 十四。二年,封恭侯遺腹子文侯遺元年。八。十六年,恭侯勝之元年。 & 五。十一。前六年,侯平元年。 & 二十八。元鼎五年,侯平坐酎金,國除。 & 百 \\ \hline
安丘 & 以卒從起方與,屬魏豹,二歲五月,以執鈹入漢,以司馬擊籍,以將軍定代,侯,三千戶。 & 五。八年七月癸酉,懿侯張說元年。 & 七 & 八 & 十二。十一。十三年,恭侯奴元年。 & 二。一。三年,敬侯執元年。十三。四年,康侯訢元年。 & 十八。元狩元年,侯指元年。九。元鼎四年,侯指坐入上林謀盜鹿,國除。 & 六十七 \\ \hline
合陽 & 高祖兄。兵初起,侍太公守豐,天下已平,以六年正月立仲為代王。高祖八年,匈奴攻代,王棄國亡,廢為合陽侯。 & 五。八年九月丙子,侯劉仲元年。 & 二。仲子濞,為吳王。以子吳王故,尊仲謚為代頃侯。 &  &  &  &  &  \\ \hline
襄平 & 兵初起,紀成以將軍從擊破秦,入漢,定三秦,功比平定侯。戰好畤,死事。子通襲成功,侯。 & 五。八年後九月丙午,侯紀通元年。 & 七 & 八 & 二十三 & 九。七。中三年,康侯相夫元年。 & 十二。元朔元年,侯夷吾元年。十九。元封元年,夷吾薨,無後,國除。 &  \\ \hline
龍 & 以卒從,漢王元年起霸上,以謁者擊籍,斬曹咎,侯,千戶。 & 五。八年後九月己未,敬侯陳署元年。 & 七 & 六。二。七年,侯堅元年。 & 十六。後元年,侯堅奪侯,國除。 &  &  & 八十四 \\ \hline
繁 & 以趙騎將從,漢三年,從擊諸侯,侯,比吳房侯,千五百戶。 & 四。九年十一月壬寅,莊侯彊瞻元年。 & 四。三。五年,康侯昫獨元年。 & 八 & 二十三 & 三。六。四年侯寄元年。七。中三年,侯安國元年。 & 十八。元狩元年,安國為人所殺,國除。 & 九十五 \\ \hline
陸梁 & 詔以為列侯,自置吏,受令長沙王。 & 三。九年三月丙辰,侯須毋元年。一。十二年,共侯桑元年。 & 七 & 八 & 十八。五。後三年,康侯慶忌元年。 & 元年,侯冉元年。十六。 & 二十八。元鼎五年,侯冉坐酎金,國除。 & 百三十七 \\ \hline
高京 & 周苛起兵,以內史從,擊破秦,為御史大夫,入漢,圍取諸侯,堅守滎陽,功比辟陽。苛以御史大夫死事。子成為後,襲侯。 & 四。九年四月戊寅,侯周成元年。 & 七 & 八 & 二十。後五年,坐謀反,繫死,國除,絕。 & 繩。中元年,封成孫應元年。侯平嗣,不得元。 & 元狩四年,平坐為太常不繕治園陵,不敬,國除。 & 六十 \\ \hline
離 & 失此侯始所起及所絕。 & 九年四月戊寅,鄧弱元年。 &  &  &  &  &  &  \\ \hline
義陵 & 以長沙柱國侯,千五百戶。 & 四。九年九月丙子,侯吳程元年。 & 三。四。四年,侯種元年。 & 六。七年,侯種薨,無後,國除。皆失謚。 &  &  &  & 百三十四 \\ \hline
宣平 & 兵初起,張耳誅秦,為相,合諸侯兵鉅鹿,破秦定趙,為常山王。陳餘反,襲耳,棄國,與大臣歸漢,漢定趙,為王。卒,子敖嗣。其臣貫高不善,廢為侯。 & 四。九年四月,武侯張敖元年。 & 七 & 六。信平薨,子偃為魯王,國除。 & 十五。元年,以故魯王為南宮侯。八。十六年,哀侯歐元年。 & 九。七。中三年,侯生元年。 & 七。罪,絕。睢陽十八。元光三年,封偃孫侯廣元年。十三。元鼎二年,侯昌元年。太初三年,侯昌為太常,乏祠,國除。 & 三 \\ \hline
東陽 & 高祖六年,為中大夫,以河閒守擊陳豨力戰功,侯,千三百戶。 & 二。十一年十二月癸巳,武侯張相如元年。 & 七 & 八 & 十五。五。十六年,共侯殷元年。三。後五年,戴侯安國元年。 & 三。十三。四年,哀侯彊元年。 & 建元元年,侯彊薨,無後,國除。 & 百十八 \\ \hline
開封 & 以右司馬漢王五年初從,以中尉擊燕,定代,侯,比共侯,二千戶。 & 一。十一年十二月丙辰,閔侯陶舍元年。一。十二年,夷侯青元年。 & 七 & 八 & 二十三 & 九。景帝時,為丞相。七。中三年,節侯偃元年。 & 十。元光五年,侯睢元年。十八。元鼎五年,侯睢坐酎金,國除。 & 百十五 \\ \hline
沛 & 高祖兄合陽侯劉仲子,侯。 & \begin{tabular}[c]{@{}l@{}}一。十一年十二月癸巳,侯劉濞元年。\\ 十二年十月辛丑,侯濞為吳王,國除。\end{tabular} &  &  &  &  &  &  \\ \hline
慎陽 & 為淮陰舍人,告淮陰侯信反,侯,二千戶。 & 二。十一年十二月甲寅,侯欒說元年。 & 七 & 八 & 二十二 & 十二。四。中六年,靖侯願之元年。 & 二十二。建元元年,侯買之元年。元狩五年,侯買之坐鑄白金棄市,國除。 & 百三十一 \\ \hline
禾成 & 以卒漢五年初從,以郎中擊代,斬陳豨,侯,千九百戶。 & 二。十一年正月己未,孝侯公孫耳元年。 & 七 & 八 & 四。五年,懷侯漸元年。九。十四年,侯漸薨,無後,國除。 &  &  & 百十七 \\ \hline
堂陽 & 以中涓從起沛,以郎入漢,以將軍擊籍,為惠侯。坐守滎陽降楚免。後復來,以郎擊籍,為上黨守,擊豨,侯,八百戶。 & 二。十一年正月己未,哀侯孫赤元年。 & 七 & 八。元年,侯德元年。 & 二十三 & 十二。中六年,侯德有罪,國除。 &  & 七十七 \\ \hline
祝阿 & 以客從起齧桑,以上隊將入漢,以將軍定魏太原,破井陘,屬淮陰侯,以莳度軍擊籍及攻豨,侯,八百戶。 & 二。十一年正月己未,孝侯高邑元年。 & 七 & 八 & 四。五年,侯成元年。十四。後三年,侯成坐事國人過律,國除。 &  &  & 七十四 \\ \hline
長脩 & 以漢二年用御史初從出關,以內史擊諸侯,功比須昌侯,以廷尉死事,千九百戶。 & 二。十一年正月丙辰,平侯杜恬元年。 & 二。五。三年,懷侯中元年。 & 八 & 四。十九。五年,侯喜元年。 & 八。罪絕。陽平五。中五年,復封;侯相夫元年。 & 三十三。元封四年,侯相夫坐為太常與樂令無可當鄭舞人擅繇不如令,闌出函谷關,國除。 & 百八 \\ \hline
江邑 & 以漢五年為御史,用奇計徙御史大夫周昌為趙相而代之,從擊陳豨,功侯,六百戶。 & 二。十一年正月辛未,侯趙堯元年。 & 七 & 元年,侯堯有罪,國除。 &  &  &  &  \\ \hline
營陵 & 以漢三年為郎中,擊項羽,以將軍擊陳豨,得王黃,為侯。與高祖疏屬劉氏,世為衛尉。萬二千戶。 & 二。十一年,侯劉澤元年。 & 七 & 五。六年,侯澤為琅邪王,國除。 &  &  &  & 八十八 \\ \hline
土軍 & 高祖六年為中地守,以廷尉擊陳豨,侯,千二百戶。就國,後為燕相。 & 二。十一年二月丁亥,武侯宣義元年。 & 五。二。六年,孝侯莫如元年。 & 八 & 二十三 & 二。十四。三年,康侯平元年。 & 五。建元六年,侯生元年。八。元朔二年,生坐與人妻姦罪,國除。 & 百一十二 \\ \hline
廣阿 & 以客從起沛,為御史,守豐二歲,擊籍,為上黨守,陳豨反,堅守,侯,千八百戶。後遷御史大夫。 & 二。十一年二月丁亥,懿侯任敖元年。 & 七 & 八 & 二。一。三年,夷侯竟元年。二十。四年,敬侯但元年。 & 十六 & 四。建元五年,侯越元年。二十一。元鼎二年,侯越坐為太常廟酒酸,不敬,國除。 & 八十九 \\ \hline
須昌 & 以謁者漢王元年初起漢中,雍軍塞陳,謁上,上計欲還,衍言從他道,道通,後為河閒守,陳豨反,誅都尉相如,功侯,千四百戶。 & 二。十一年二月己酉,貞侯趙衍元年。 & 七 & 八 & 十五。四。十六年,戴侯福元年。四。後四年,侯不害元年。 & 四。五年,侯不害有罪,國除。 &  & 百七 \\ \hline
臨轅 & 初起從為郎,以都尉守蘄城,以中尉侯,五百戶。 & 二。十一年二月乙酉,堅侯戚鰓元年。 & 四。三。五年,夷侯觸龍元年。 & 八 & 二十三 & 三。十三。四年,共侯忠元年。 & 三。建元四年,侯賢元年。二十五。元鼎五年,侯賢坐酎金,國除。 & 百十六 \\ \hline
汲 & 高祖六年為太僕,擊代豨,有功,侯,千二百戶。為趙太傅。 & 二。十一年二月己巳,終侯公上不害元年。 & 一。六。二年,夷侯武元年。 & 八 & 十三。十。十四年,康侯通元年。 & 十六 & 一。九。建元二年,侯廣德元年。元光五年,廣德坐妻精大逆罪,頗連廣德,棄市,國除。 & 百二十三 \\ \hline
寧陵 & 以舍人從陳留,以郎入漢,破曹咎成皋,為上解隨馬,以都尉擊陳豨,功侯,千戶。 & 二。十一年二月辛亥,夷侯呂臣元年。 & 七 & 八 & 十。十三。十一年,戴侯射元年。 & 三。四年,惠侯始元年。一。五年,侯始薨,無後,國除。 &  & 七十三 \\ \hline
汾陽 & 以郎中騎千人前二年從起陽夏,擊項羽,以中尉破鍾離眛,功侯。 & 二。十一年二月辛亥,侯靳彊元年。 & 七 & 二。六。三年,共侯解元年。 & 二十三 & 四。十二。五年,康侯胡元年。絕。 & 江鄒十九。元鼎五年,侯石元年。太始四年五月丁卯,侯石坐為太常,行太僕事,治嗇夫可年,益縱年,國除。 & 九十六 \\ \hline
戴 & 以卒從起沛,以卒開沛城門,為太公僕;以中廄令擊豨,侯,千二百戶。 & 二。十一年三月癸酉,敬侯彭祖元年。 & 七 & 二。六。三年,共侯悼元年。 & 七。十六。八年,夷侯安國元年。 & 十六 & 十六。元朔五年,侯安期元年。十二。元鼎五年,侯蒙元年。二十五。後元元年五月甲戌,坐祝詛,無道,國除。 & 百一十六 \\ \hline
衍 & 以漢二年為燕令,以都尉下楚九城,堅守燕,侯,九百戶。 & 二。十一年七月乙巳,簡侯翟盱。元年。 & 七 & 三。二。四年,祗侯山元年。三。六年,節侯嘉元年。 & 二十三 & 十六 & 二。建元三年,侯不疑元年。十。元朔元年,不疑坐挾詔書論罪,國除。 & 百三十 \\ \hline
平州 & 漢王四年,以燕相從擊籍,還擊荼,以故二千石將為列侯,千戶。 & 二。十一年八月甲辰,共侯昭涉掉尾元年。 & 七 & 八 & 一。三。二年,戴侯福元年。四。五年,懷侯它人元年。十五。九年,孝侯馬童元年。 & 十四。二。後二年,侯昧元年。 & 三十三。元狩五年,侯昧坐行馳中更呵馳去罪,國除。 & 百十一 \\ \hline
中牟 & 以卒從起沛,入漢以郎中擊布,功侯,二千三百戶。始高祖微時,有急,給高祖一馬,故得侯。 & 一。十二年十月乙未,共侯單父聖元年。 & 七 & 八 & 七。五。八年,敬侯繒元年。十一。十三年,戴侯終根元年。 & 十六 & 十。元光五年,侯舜元年。十八。元鼎五年,侯舜坐酎金,國除。 & 百二十五 \\ \hline
邔 & 以故群盜長為臨江將,已而為漢擊臨江王及諸侯,破布,功侯,千戶。 & 十二年十月戊戌,莊侯黃極中元年。 & 七 & 八 & 十一。九。十二年,慶侯榮盛元年。三。後五年,共侯明元年。 & 十六 & 十六。元朔五年,侯遂元年。八。元鼎元年,遂坐賣宅縣官故貴,國除。 & 百十三 \\ \hline
博陽 & 以卒從起豐,以隊卒入漢,擊籍成皋,有功,為將,布反,定吳郡,侯,千四百戶。 & 一。十二年十月辛丑,節侯周聚元年。 & 七 & 八 & 八。十五。九年,侯鸱元年。 & 十一。中五年,侯鸱奪爵一級,國除。 &  & 五十三 \\ \hline
陽義 & 以荊令尹漢王五年初從,擊鍾離眛及陳公利幾,破之,徙為漢大夫,從至陳,取韓信,還為中尉,從擊布,功侯,二千戶。 & 一。十二年十月壬寅,定侯靈常元年。 & 七 & 六。二。七年,共侯賀元年。 & 六。七年,哀侯勝元年。六。十二年,侯勝薨,無後,國除。 &  &  & 百十九 \\ \hline
下相 & 以客從起沛,用兵從擊破齊田解軍,以楚丞相堅守彭城,距布軍,功侯,二千戶。 & 一。十二年十月己酉,莊侯冷耳元年。 & 七 & 八 & 二。二十一。三年侯慎元年 & 二。三年三月,侯慎反,國除。 &  & 八十五 \\ \hline
德 & 以代頃王子侯。頃王,吳王濞父也;廣,濞之弟也。 & 一。十二年十一月庚辰,哀侯劉廣元年。 & 七 & 二。六。三年,頃侯通元年。 & 二十三 & 五。十一。六年,侯齕元年。 & 二十七。元鼎四年,侯何元年。一。元鼎五年,侯何坐酎金,國除。 & 百二十七 \\ \hline
高陵 & 以騎司馬漢王元年從起廢丘,以都尉破田橫、龍且,追籍至東城,以將軍擊布,九百戶。 & 一。十二年十二月丁亥,圉侯王周元年。 & 七 & 二。六。三年,惠侯并弓元年。 & 十二。十一。十三年,侯行元年。 & 二。三年,反,國除。 &  & 九十二 \\ \hline
期思 & 淮南王布中大夫,有纳,上書告布反,侯,二千戶。布盡殺其宗族。 & 一。十二年十二月癸卯,康侯賁赫元年。 & 七 & 八 & 十三。十四年,赫薨,無後,國除。 &  &  & 百三十二 \\ \hline
穀陵 & 以卒從,前二年起柘,擊籍,定代,為將軍,功侯。 & 一。十二年正月乙丑,定侯馮谿元年。 & 七 & 八 & 六。十七。七年,共侯熊元年。 & 二。二。三年,隱侯卬元年。十二。五年,獻侯解元年。 & 三。建元四年,侯偃元年。 & 百五 \\ \hline
戚 & 以都尉漢二年初起櫟陽,攻廢丘,破之,因擊項籍,別屬韓信破齊軍,攻臧荼,遷為將軍,擊信,侯,千戶。 & 一。十二年十二月癸卯,圉侯季必元年。 & 七 & 八 & 三。二十。四年,齊侯班元年。 & 十六 & 二。建元三年,侯信成元年。二十。元狩五年,侯信成坐為太常,縱丞相侵神道壖,不敬,國除。 & 九十 \\ \hline
壯 & 以楚將漢王三年降,起臨濟,以郎中擊籍、陳豨,功侯,六百戶。 & 一。十二年正月乙丑,敬侯許倩元年。 & 七 & 八 & 二十三 & 一。十五。二年,共侯恢元年。 & 一。建元二年,殤侯則元年。九。元光五年,侯廣宗元年。十五。元鼎元年,侯廣宗坐酎金,國除。 & 百十二 \\ \hline
成陽 & 以魏郎漢王二年從起陽武,擊籍,屬魏豹,豹反,屬相國彭越,以太原尉定代,侯,六百戶。 & 一。十二年正月乙酉,定侯意。元年。 & 七 & 八 & 十。十三。十一年,侯信元年。 & 十六 & 建元元年,侯信罪鬼薪,國除。 & 百一十 \\ \hline
桃 & 以客從漢王二年從起定陶,以大謁者擊布,侯,千戶。為淮陰守。項氏親也,賜姓。 & 一。十二年三月丁巳,安侯劉襄元年。 & 七 & 一。奪,絕。七。二年,復封襄。 & 九。十四。十年,哀侯舍元年。 & 十六。景帝時,為丞相。 & 十三。建元元年,厲侯申元年。十五。元朔二年,侯自為元年。元鼎五年,侯自為坐酎金,國除。 & 百三十五 \\ \hline
高梁 & 食其,兵起以客從擊破秦,以列侯入漢,還定諸侯,常使約和諸侯列卒兵聚,侯,功比平侯嘉;以死事,子疥襲食其功侯,九百戶。 & 一。十二年三月丙寅,共侯酈疥元年。 & 七 & 八 & 二十三 & 十六 & 八。元光三年,侯勃元年。十。元狩元年,坐詐詔衡山王取金,當死,病死,國除。 & 六十六 \\ \hline
紀 & 以中涓從起豐,以騎將入漢,以將軍擊籍,後攻盧綰,侯,七百戶。 & 一。十二年六月壬辰,匡侯陳倉元年。 & 七 & 二。六。三年,夷侯開元年。 & 十七。六。後二年,侯陽元年。 & 二。三年,陽反,國除。 &  & 八十 \\ \hline
甘泉 & 以車司馬漢王元年初從起高陵,屬劉賈,以都尉從軍,侯。 & 一。十二年六月壬辰,侯王竟。元年。 & 六。一。七年,戴侯莫搖元年。 & 八 & 十。十三。十一年,侯嫖。元年。 & 九。十年,侯嫖有罪,國除。 &  & 百六 \\ \hline
煮棗 & 以越連敖從起豐,別以郎將入漢,擊諸侯,以都尉侯,九百戶。 & 一。十二年六月壬辰,靖侯赤元年。 & 七 & 八 & 一。二十二。二年,赤子康侯武元年。 & 八。中二年,侯昌元年。二。中四年,有罪,國除。 &  & 七十五 \\ \hline
張 & 以中涓騎從起豐,以郎將入漢,從擊諸侯,七百戶。 & 一。十二年六月壬辰,節侯毛澤元年。 & 七 & 八 & 十。二。十一年,夷侯慶元年。十一。十三年,侯舜元年。 & 十二。中六年,侯舜有罪,國除。 &  & 七十九 \\ \hline
鄢陵 & 以卒從起豐,入漢,以都尉擊籍、荼,侯,七百戶。 & 一。十二年中,莊侯朱濞元年。 & 七 & 三。五。四年,恭侯慶元年。 & 六。七年,恭侯慶薨,無後,國除。 &  &  & 五十二 \\ \hline
菌 & 以中涓前元年從起單父,不入關,以擊籍、布、燕王綰,得南陽,侯,二千七百戶。 & 一。十二年,莊侯張平元年。 & 七 & 四。四。五年,侯勝元年。 & 三。四年,侯勝有罪,國除。 &  &  & 四十八 \\ \hline
}