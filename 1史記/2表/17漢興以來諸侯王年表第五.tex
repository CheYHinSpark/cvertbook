\chapter{漢興以來諸侯王年表第五}
			
太史公曰,殷以前尚矣。周封五等,公,侯,伯,子,男。然封伯禽、康叔於魯、衛,地各四百里,親親之義,褒有德也,太公於齊,兼五侯地,尊勤勞也。武王、成、康所封數百,而同姓五十五,地上不過百里,下三十里,以輔衛王室。管、蔡、康叔、曹、鄭,或過或損。厲、幽之後,王室缺,侯伯彊國興焉,天子微,弗能正。非德不純,形勢弱也。

漢興,序二等。高祖末年,非劉氏而王者,若無功上所不置而侯者,天下共誅之。高祖子弟同姓為王者九國,雖獨長沙異姓,而功臣侯者百有餘人。自雁門、太原以東至遼陽,為燕代國,常山以南,大行左轉,度河、濟,阿、甄以東薄海,為齊、趙國,自陳以西,南至九疑,東帶江、淮、谷、泗,薄會稽,為梁、楚、淮南、長沙國,皆外接於胡、越。而內地北距山以東盡諸侯地,大者或五六郡,連城數十,置百官宮觀,僭於天子。漢獨有三河、東郡、潁川、南陽,自江陵以西至蜀,北自雲中至隴西,與內史凡十五郡,而公主列侯頗食邑其中。何者。天下初定,骨肉同姓少,故廣彊庶孽,以鎮撫四海,用承衛天子也。

漢定百年之閒,親屬益疏,諸侯或驕奢,忕邪臣計謀為淫亂,大者叛逆,小者不軌于法,以危其命,殞身亡國。天子觀於上古,然後加惠,使諸侯得推恩分子弟國邑,故齊分為七,趙分為六,梁分為五,淮南分三,及天子支庶子為王,王子支庶為侯,百有餘焉。吳楚時,前后諸侯或以適削地,是以燕、代無北邊郡,吳、淮南、長沙無南邊郡,齊、趙、梁、楚支郡名山陂海咸納於漢。諸侯稍微,大國不過十餘城,小侯不過數十里,上足以奉貢職,下足以供養祭祀,以蕃輔京師。而漢郡八九十,形錯諸侯閒,犬牙相臨,秉其阸塞地利,彊本干,弱枝葉之勢,尊卑明而萬事各得其所矣。

臣遷謹記高祖以來至太初諸侯,譜其下益損之時,令時世得覽。形勢雖彊,要之以仁義為本。

\biao[\hline  & 楚 &  &  & 齊 &  &  &  &  &  &  & 荊 & 淮南 & 燕 & 趙 &  &  &  &  &  & 梁 &  &  &  & 淮陽 & 代 & 長沙 \\ \hline
\multicolumn{27}{}{}]
{|p{1em}|p{2em}|p{2em}|p{2em}|p{2em}|p{2em}|p{2em}|p{2em}|p{2em}|p{2em}|p{2em}|p{2em}|p{2em}|p{2em}|p{2em}|p{2em}|p{2em}|p{2em}|p{2em}|p{2em}|p{2em}|p{2em}|p{2em}|p{2em}|p{2em}|p{2em}|p{2em}|}
{\hline
高祖元年 & 楚 &  &  & 齊 &  &  &  &  &  &  & 荊 & 淮南 & 燕 & 趙 &  &  &  &  &  & 梁 &  &  &  & 淮陽 & 代 & 長沙 \\ \hline
二 & 都彭城。 &  &  & 都臨菑。 &  &  &  &  &  &  & 都吳。 & 都壽春。 & 都薊。 & 都邯鄲。 &  &  &  &  &  & 都淮陽。 &  &  &  & 都陳。 & 十一月,初王韓信元年。都馬邑。 &  \\ \hline
三 &  &  &  &  &  &  &  &  &  &  &  &  &  &  &  &  &  &  &  &  &  &  &  &  & 二 &  \\ \hline
四 &  &  &  & 初王信元年。故相國。 &  &  &  &  &  &  &  & 十月乙丑,初王英布元年。 &  & 初王張耳元年。薨。 &  &  &  &  &  &  &  &  &  &  & 三 &  \\ \hline
五 & 齊王信徙為楚王元年。反,廢。 &  &  & 二。徙楚。 &  &  &  &  &  &  &  & 二 & 後九月壬子,初王盧綰元年。 & 王敖元年。敖,耳子。 &  &  &  &  &  & 初王彭越元年 &  &  &  &  & 四。降匈奴,國除為郡。 & 二月乙未,初王文王吳芮元年。薨。 \\ \hline
六 & 正月丙午,初王交元年。交,高祖弟。 &  &  & 正月甲子,初王悼惠王肥元年。肥,高祖子。 &  &  &  &  &  &  & 正月丙午,初王劉賈元年。 & 三 & 二 & 二 &  &  &  &  &  & 二 &  &  &  &  &  & 成王臣元年 \\ \hline
七 & 二 &  &  & 二 &  &  &  &  &  &  & 二 & 四 & 三 & 三 &  &  &  &  &  & 三 &  &  &  &  &  & 二 \\ \hline
八 & 三 &  &  & 三 &  &  &  &  &  &  & 三 & 五 & 四 & 四。廢。 &  &  &  &  &  & 四 &  &  &  &  &  & 三 \\ \hline
九 & 四。來朝。 &  &  & 四。來朝。 &  &  &  &  &  &  & 四 & 六。來朝。 & 五 & 初王隱王如意元年。如意,高祖子。 &  &  &  &  &  & 五。來朝。 &  &  &  &  &  & 四 \\ \hline
十 & 五。來朝。 &  &  & 五。來朝。 &  &  &  &  &  &  & 五。來朝。 & 七。來朝。反,誅。 & 六。來朝。 & 二 &  &  &  &  &  & 六。來朝。反,誅。 &  &  &  &  & 復置代,都中都。 & 五。來朝。 \\ \hline
十一 & 六 &  &  & 六 &  &  &  &  &  &  & 六。為英布所殺,國除為郡。 & 十二月庚午,厲王長元年。長,高祖子。 & 七。 & 三 &  &  &  &  &  & 二月丙午,初王恢元年。恢,高祖子。 &  &  &  & 三月丙寅,初王友元年。友,高祖子。 & 正月丙子初王元年。 & 六 \\ \hline
十二 & 七 &  &  & 七 &  &  &  &  &  &  & 更為吳國。十月辛丑,初王濞元年。濞,高祖兄仲子,故沛侯。 & 二 & 二月甲午,初王靈王建元年。建,高祖子。 & 四。死。 &  &  &  &  &  & 二 &  &  &  & 二 & 二 & 七 \\ \hline
孝惠元年 & 八 &  &  & 八 &  &  &  &  &  &  & 二 & 三 & 二 & 淮陽王徙於趙,名友,元年。是為幽王。 &  &  &  &  &  & 三 &  &  &  & 為郡。 & 三 & 八 \\ \hline
二 & 九。來朝。 &  &  & 九。來朝。 &  &  &  &  &  &  & 三 & 四 & 三 & 二 &  &  &  &  &  & 四 &  &  &  &  & 四 & 哀王回元年 \\ \hline
三 & 十 &  &  & 十 &  &  &  &  &  &  & 四 & 五 & 四 & 三 &  &  &  &  &  & 五 &  &  &  &  & 五 & 二 \\ \hline
四 & 十一。來朝。 &  &  & 十一。來朝。 &  &  &  &  &  &  & 五 & 六。來朝。 & 五 & 四。來朝。 &  &  &  &  &  & 六 &  &  &  &  & 六 & 三 \\ \hline
五 & 十二 &  &  & 十二 &  &  &  &  &  &  & 六。來朝。 & 七 & 六。來朝。 & 五 &  &  &  &  &  & 七 &  &  &  &  & 七 & 四 \\ \hline
六 & 十三 &  &  & 十三。薨。 &  &  &  &  &  &  & 七 & 八 & 七 & 六 &  &  &  &  &  & 八 &  &  &  &  & 八 & 五 \\ \hline
七 & 十四。來朝。 & 初置魯國。 &  & 哀王襄元年 &  &  &  &  &  &  & 八。來朝。 & 九。來朝。 & 八。來朝。 & 七。來朝。 &  &  &  &  & 初置常山國。 & 九。來朝。 &  &  & 初置呂國。 & 復置淮陽國。 & 九 & 六 \\ \hline
高后元年 & 十五 & 四月初王張偃元年。偃,高后外孫,故趙王敖子。 &  & 二 &  &  &  &  &  &  & 九 & 十 & 九 & 八 &  &  &  &  & 四月辛卯,哀王不疑元年。薨。 & 十 &  &  & 四月辛卯,呂王台元年。薨。 & 四月辛卯,初王懷王強元年。強,惠帝子。 & 十 & 七 \\ \hline
二 & 十六 & 二 &  & 三 &  &  &  &  &  &  & 十 & 十一 & 十 & 九 &  &  &  &  & 七月癸巳,初王義元年。哀王弟。義,孝惠子,故襄城侯,立為帝。 & 十一 &  &  & 十一月癸亥,王呂嘉元年。嘉,肅王子。 & 二 & 十一 & 恭王右元年 \\ \hline
三 & 十七 & 三 &  & 四。來朝。 &  &  &  &  &  &  & 十一 & 十二 & 十一 & 十 &  &  &  &  & 二 & 十二 &  &  & 二 & 三 & 十二 & 二。來朝。 \\ \hline
四 & 十八 & 四 &  & 五 &  &  &  &  &  &  & 十二 & 十三 & 十二 & 十一 &  &  &  &  & 五月丙辰,初王朝元年。朝,惠帝子,故軹侯。 & 十三 &  &  & 三 & 四 & 十三 & 三 \\ \hline
五 & 十九 & 五 &  & 六 &  &  &  &  &  &  & 十三 & 十四。來朝。 & 十三 & 十二 &  &  &  &  & 二 & 十四 &  &  & 四 & 五。無嗣。 & 十四 & 四 \\ \hline
六 & 二十 & 六 &  & 七 &  &  &  & 初置琅邪國。 &  &  & 十四 & 十五 & 十四 & 十三 &  &  &  &  & 三 & 十五 &  &  & 嘉廢。七月丙辰,呂產元年。產,肅王弟,故洨侯。 & 初王武元年。武,孝惠帝子,故壺關侯。 & 十五 & 五 \\ \hline
七 & 二十一 & 七 &  & 八 &  &  &  & 王澤元年。故營陵侯。 &  &  & 十五 & 十六 & 十五。絕。 & 。 &  &  &  &  & 四 & 。徙王趙,自殺。王呂產元年。 &  &  & 呂產徙王梁。二月丁巳,王太元年。惠帝子。 & 二 & 十六 & 六 \\ \hline
八 & 二十二 & 八 &  & 九 &  &  &  & 二 &  &  & 十六 & 十七 & 十月辛丑,初王呂通元年。肅王子,故東平侯。九月誅,國除。 & 初王呂祿元年。呂后兄子,胡陵侯。誅,國除。 &  &  &  &  & 五。非子,誅,國除為郡。 & 二。有罪,誅,為郡。 &  &  & 二 & 三。武誅,國除。 & 十七 & 七 \\ \hline
孝文元年 & 二十三 & 九。廢為侯。 &  & 十。薨。 & 初置城陽邵。 & 初置濟北。 &  & 三。徙燕。 &  &  & 十七 & 十八 & 十月庚戌,琅邪王澤徙燕元年。是為敬王。 & 十月庚戌,趙王遂元年。幽王子。 & 分為河閒,都樂成。 &  &  &  & 初置太原,都晉陽。 & 復置梁國。 &  &  &  &  & 十八。為文帝。 & 八 \\ \hline
二 & 夷王郢元年 &  &  & 文王則元年 & 二月乙卯景王章元年。章,悼惠王子,故朱虛侯。 & 二月乙卯,王興居元年。興居,悼惠王子,故東牟侯。 &  & 國除為郡。 &  &  & 十八 & 十九 & 二。薨。 & 二 & 二月乙卯,初王文王辟強元年。辟強,趙幽王子。 &  &  &  & 二月乙卯,初王參元年。參,文帝子。 & 二月乙卯,初王懷王勝元年。勝,文帝子。 &  &  &  &  & 二月乙卯,初王武元年。武,文帝子。 & 九 \\ \hline
三 & 二 &  &  & 二 & 二 & 為郡。 &  &  &  &  & 十九。來朝。 & 二十。來朝。 & 康王嘉元年 & 三 & 二 &  &  &  & 二 & 二 &  &  &  & 復置淮陽國。 & 二。徙淮陽。 & 靖王著元年 \\ \hline
四 & 三 &  &  & 三 & 共王喜元年 &  &  &  &  &  & 二十 & 二十一 & 二 & 四 & 三 &  &  &  & 三。更為代王。 & 三 &  &  &  & 代王武徙淮陽三年。 & 三。太原王參更號為代王三年,實居太原,是為孝王。 & 二 \\ \hline
五 & 四。薨。 &  &  & 四 & 二 &  &  &  &  &  & 二十一 & 二十二 & 三 & 五 & 四 &  &  &  &  & 四 &  &  &  & 四 & 四 & 三 \\ \hline
六 & 王戊元年 &  &  & 五 & 三 &  &  &  &  &  & 二十二 & 二十三。王無道,遷蜀,死雍,為郡。 & 四 & 六 & 五 &  &  &  &  & 五 &  &  &  & 五 & 五 & 四 \\ \hline
七 & 二 &  &  & 六 & 四 &  &  &  &  &  & 二十三 &  & 五 & 七。來朝。 & 六 &  &  &  &  & 六。來朝。 &  &  &  & 六。來朝。 & 六。來朝。 & 五 \\ \hline
八 & 三 &  &  & 七。來朝。 & 五 &  &  &  &  &  & 二十四 &  & 六。來朝。 & 八 & 七。來朝。 &  &  &  &  & 七 &  &  &  & 七 & 七 & 六 \\ \hline
九 & 四 &  &  & 八 & 六。來朝。 &  &  &  &  &  & 二十五 &  & 七 & 九 & 八 &  &  &  &  & 八 &  &  &  & 八。來朝。 & 八 & 七 \\ \hline
十 & 五 &  &  & 九 & 七 &  &  &  &  &  & 二十六 &  & 八 & 十 & 九 &  &  &  &  & 九 &  &  &  & 九 & 九 & 八。來朝。 \\ \hline
十一 & 六 &  &  & 十 & 八。徙淮南。為郡,屬齊。 &  &  &  &  &  & 二十七 &  & 九 & 十一 & 十 &  &  &  &  & 十。來朝。薨,無後。 &  &  &  & 十。來朝。徙梁。為郡。 & 十。來朝。 & 九 \\ \hline
十二 & 七 &  &  & 十一。來朝。 &  &  &  &  &  &  & 二十八 & 城陽王喜徙淮南元年 & 十 & 十二。來朝。 & 十一。來朝。 &  &  &  &  & 十一。淮陽王武徙梁年,是為孝王。 &  &  &  &  & 十一 & 十 \\ \hline
十三 & 八。來朝。 &  &  & 十二 &  &  &  &  &  &  & 二十九 & 二 & 十一 & 十三 & 十二 &  &  &  &  & 十二 &  &  &  &  & 十二 & 十一 \\ \hline
十四 & 九 &  &  & 十三 &  &  &  &  &  &  & 三十 & 三 & 十二。來朝。 & 十四 & 十三。薨。 &  &  &  &  & 十三 &  &  &  &  & 十三 & 十二 \\ \hline
十五 & 十 &  & 初置衡山。 & 十四。薨。無後。 & 復置城陽國。 & 復置濟北國。 & 分為濟南國。 & 分為菑川,都劇。 & 分為膠西、都宛。 & 分為膠東,都即墨。 & 三十一 & 四。徙城陽。 & 十三。來朝。 & 十五 & 哀王福元年。薨,無後,國除為郡。 &  &  & 初置廬江國。 &  & 十四。來朝。 &  &  &  &  & 十四 & 十三 \\ \hline
十六 & 十一 &  & 四月丙寅,王勃元年。淮南厲王子,故安陽侯。 & 四月丙寅,孝王將閭元年。齊悼惠王子,故陽虛侯。 & 淮南王喜徙城陽十三年。 & 四月丙寅,初王志元年。齊悼惠王子,故安都侯。 & 四月丙寅,初王辟光元年。齊悼惠王子,故扐侯。 & 四月丙寅,初王賢元年。齊悼惠王子,故武城侯。 & 四月丙寅,初王卬元年。齊悼惠王子,故平昌侯。 & 四月丙寅,初王雄渠元年。齊悼惠王子,故白石侯。 & 三十二 & 四月丙寅,王安元年。淮南厲王子,故阜陵侯。 & 十四 & 十六 &  &  &  & 四月丙寅,王賜元年。淮南厲王子,故陽周侯。 &  & 十五 &  &  &  &  & 十五 & 十四 \\ \hline
後元年 & 十二 &  & 二 & 二 & 十四 & 二 & 二 & 二 & 二 & 二 & 三十三 & 二 & 十五 & 十七 &  &  &  & 二 &  & 十六 &  &  &  &  & 十六 & 十五 \\ \hline
二 & 十三 &  & 三 & 三 & 十五 & 三 & 三 & 三 & 三 & 三 & 三十四 & 三 & 十六 & 十八 &  &  &  & 三 &  & 十七 &  &  &  &  & 十七。薨。 & 十六 \\ \hline
三 & 十四 &  & 四 & 四。來朝。 & 十六 & 四。來朝。 & 四。來朝。 & 四 & 四 & 四 & 三十五 & 四 & 十七 & 十九 &  &  &  & 四 &  & 十八。來朝。 &  &  &  &  & 恭王登元年 & 十七 \\ \hline
四 & 十五 &  & 五 & 五 & 十七 & 五。來朝。 & 五 & 五 & 五 & 五 & 三十六 & 五 & 十八。來朝。 & 二十。來朝。 &  &  &  & 五 &  & 十九 &  &  &  &  & 二 & 十八 \\ \hline
五 & 十六。來朝。 &  & 六 & 六 & 十八。來朝。 & 六 & 六。來朝。 & 六 & 六。來朝。 & 六 & 三十七 & 六 & 十九 & 二十一 &  &  &  & 六 &  & 二十 &  &  &  &  & 三 & 十九 \\ \hline
六 & 十七 &  & 七 & 七 & 十九 & 七 & 七 & 七 & 七 & 七 & 三十八 & 七。來朝。 & 二十 & 二十二 &  &  &  & 七 &  & 二十一。來朝。 &  &  &  &  & 四 & 二十。來朝。 \\ \hline
七 & 十八 &  & 八 & 八 & 二十 & 八 & 八 & 八 & 八 & 八 & 三十九 & 八 & 二十一 & 二十三 &  &  &  & 八 &  & 二十二 &  &  &  &  & 五 & 二十一。來朝。薨,無後,國除。 \\ \hline
孝景元年 & 十九 &  & 九 & 九 & 二十一 & 九 & 九 & 九 & 九 & 九 & 四十 & 九 & 二十二 & 二十四 & 復置河閒國。 & 初置廣川,都信都。 &  & 九 &  & 二十三 &  & 初置臨江,都江陵。 & 初置汝南國。 & 復置淮陽國。 & 六 & 復置長沙國。 \\ \hline
二 & 二十。來朝。 & 分楚復置魯國。 & 十 & 十 & 二十二 & 十。來朝。 & 十 & 十 & 十 & 十 & 四十一 & 十 & 二十三 & 二十五。來朝。 & 三月甲寅,初王獻王德元年。景帝子。 & 三月甲寅,王彭祖元年。景帝子。 & 初置中山,都盧奴。 & 十 &  & 二十四。來朝。 &  & 三月甲寅,初王閼于元年。景帝子。 & 三月甲寅,初王非元年。景帝子。 & 三月甲寅,初王餘元年,景帝子。 & 七 & 三月甲寅,定王發元年。景帝子。 \\ \hline
三 & 二十一。反,誅。 & 六月乙亥淮陽王徙魯元年。是為恭王。 & 十一 & 十一 & 二十三 & 十一。徙菑川。 & 十一。反,誅。為郡。 & 十一。反,誅。濟北王志徙菑川十一年。是為懿王。 & 十一。反,誅。六月乙亥,于王端元年。景帝子。 & 十一。反,誅。 & 四十二。反,誅。 & 十一 & 二十四 & 二十六。反,誅。為郡。 & 二。來朝。 & 二。來朝。 & 六月乙亥,靖王勝元年。景帝子。 & 十一 &  & 二十五。來朝。 &  & 二 & 二 & 徙魯。為郡。 & 八 & 二 \\ \hline
四。四月己巳立太子 & 文王禮元年。元王子,故平陸侯。 & 二。來朝。 & 十二。徙濟北。廬江王賜徙衡山元年。 & 懿王壽元年 & 二十四 & 衡山王勃徙濟北十二年。是為貞王。 &  & 十二 & 二 & 四月己巳,初王元年。是為孝武帝。 & 初置江都。六月乙亥,汝南王非為江都王元年。是為易王。 & 十二 & 二十五 &  & 三 & 三 & 二 & 十二。徙衡山,國除為郡。 &  & 二十六 &  & 三。薨,無後,國除為郡。 & 三。徙江都。 &  & 九 & 三 \\ \hline
五 & 二 & 三 & 二 & 二。來朝。 & 二十五 & 十三。薨。 &  & 十三 & 三 & 二 & 二 & 十三。來朝。 & 二十六。薨。 & 廣川王彭祖徙趙四年。是為敬肅王。 & 四 & 四。徙趙,國除為信都郡。 & 三 &  &  & 二十七 &  &  &  &  & 十 & 四 \\ \hline
六 & 三。來朝。薨。 & 四 & 三 & 三 & 二十六 & 武王胡元年 &  & 十四 & 四 & 三 & 三 & 十四 & 王定國元年 & 五 & 五 &  & 四 &  &  & 二十八 &  & 復置臨江國。 &  &  & 十一 & 五。來朝。 \\ \hline
七。十一月乙丑太子廢 & 安王道元年 & 五 & 四 & 四 & 二十七 & 二 &  & 十五 & 五 & 四。四月丁巳,為太子。 & 四 & 十五 & 二 & 六 & 六 &  & 五。來朝。 &  &  & 二十九。來朝。 &  & 十一月乙丑,初王閔王榮元年。景帝太子,廢。 &  &  & 十二 & 六。來朝。 \\ \hline
中元年 & 二。來朝。 & 六。來朝。 & 五 & 五 & 二十八 & 三 &  & 十六。來朝。 & 六。來朝。 & 復置膠東國。 & 五 & 十六 & 三 & 七 & 七 & 復置廣川國。 & 六 &  &  & 三十 &  & 二 &  &  & 十三 & 七 \\ \hline
二 & 三 & 七 & 六 & 六 & 二十九。來朝。 & 四 &  & 十七。來朝。 & 七 & 四月乙巳,初王康王寄元年。景帝子。 & 六 & 十七 & 四 & 八。來朝。 & 八。來朝。 & 四月乙巳,惠王越元年。景帝子。 & 七 & 初置清河,都清陽。 &  & 三十一。來朝。 &  & 三 &  &  & 十四 & 八 \\ \hline
三 & 四 & 八 & 七。來朝。 & 七 & 三十 & 五 &  & 十八 & 八 & 二 & 七 & 十八 & 五。來朝。 & 九 & 九 & 二 & 八 & 三月丁巳,哀王乘元年。景帝子。 &  & 三十二 &  & 四。坐侵廟壖垣為宮,自殺。國除為南郡。 &  &  & 十五。來朝。 & 九 \\ \hline
四 & 五 & 九 & 八 & 八 & 三十一 & 六 &  & 十九 & 九 & 三 & 八 & 十九。來朝。 & 六 & 十 & 十 & 三 & 九。來朝。 & 二 & 復置常山國。 & 三十三 &  &  &  &  & 十六 & 十。來朝。 \\ \hline
五 & 六。來朝。 & 十 & 九 & 九 & 三十二 & 七 &  & 二十 & 十 & 四。來朝。 & 九 & 二十 & 七 & 十一 & 十一 & 四 & 十 & 三 & 四月丁巳,初王憲王舜元年。孝景子。 & 三十四 & 分為濟川國。 & 分為濟東國。 & 分為山陽國。 & 分為濟陰國。 & 十七 & 十一。來朝。 \\ \hline
六 & 七 & 十一 & 十 & 十 & 三十三。薨。 & 八 &  & 二十一 & 十一 & 五 & 十 & 二十一 & 八 & 十二 & 十二 & 五 & 十一 & 四 & 二 & 三十五。來朝。薨。 & 五月丙戌,初王明元年。梁孝王子。 & 五月丙戌,初王彭離元年。梁孝王子。 & 五月丙戌,初王定元年。梁孝王子。 & 五月丙戌,初王不識元年。梁孝王子。 & 十八 & 十二 \\ \hline
後元年 & 八 & 十二 & 十一 & 十一 & 頃王延元年。 & 九 &  & 二十二。來朝。 & 十二 & 六 & 十一 & 二十二 & 九。來朝。 & 十三。來朝。 & 六 & 十二 & 五 & 三 & 恭王買元年。孝王子。 & 二 & 二 & 二 & 二。薨,無後,國除。 & 十九 & 十三 &  \\ \hline
二 & 九 & 十三 & 十二 & 十二。來朝。 & 二 & 十。來朝。 &  & 二十三 & 十三 & 七 & 十二 & 二十三 & 十。來朝。 & 十四 & 十四 & 七 & 十三 & 六 & 四 & 二 & 三 & 三 & 三 &  & 二十 & 十四 \\ \hline
三 & 十 & 十四 & 十三 & 十三 & 三 & 十一 &  & 二十四 & 十四 & 八。來朝。 & 十三 & 二十四 & 十一 & 十五 & 十五 & 八 & 十四 & 七 & 五 & 三 & 四 & 四 & 四 &  & 二十一 & 十五 \\ \hline
孝武建元元年 & 十一 & 十五 & 十四 & 十四 & 四 & 十二 &  & 二十五 & 十五 & 九 & 十四 & 二十五 & 十二 & 十六 & 十六 & 九 & 十五 & 八 & 六 & 四 & 五 & 五 & 五 &  & 二十二 & 十六 \\ \hline
二 & 十二。來朝。 & 十六。來朝。 & 十五 & 十五 & 五 & 十三 &  & 二十六 & 十六 & 十 & 十五 & 二十六。來朝。 & 十三 & 十七 & 十七 & 十 & 十六 & 九。來朝。 & 七 & 五 & 六 & 六 & 六 &  & 二十三 & 十七 \\ \hline
三 & 十三 & 十七 & 十六 & 十六 & 六 & 十四 & 二十七 & 十七 & 十一 & 十六 & 二十七 & 十四 & 十八 & 十八 & 十一 & 十七。來朝。 & 十 & 八 & 六 & 七。明殺中傅。廢遷房陵。 & 七 & 七 &  & 二十四。來朝。 & 十八。來朝。 &  \\ \hline
四 & 十四 & 十八 & 十七 & 十七 & 七 & 十五 &  & 二十八 & 十八 & 十二 & 十七。來朝。 & 二十八 & 十五 & 十九 & 十九 & 十二 & 十八 & 十一 & 九。來朝。 & 七。薨。 & 為郡。 & 八 & 八 &  & 二十五 & 十九 \\ \hline
五 & 十五 & 十九 & 十八 & 十八 & 八 & 十六 &  & 二十九 & 十九 & 十三 & 十八 & 二十九 & 十六 & 二十 & 二十 & 繆王元年。 & 十九 & 十二。薨,無後,國除為郡。 & 十 & 平王襄元年 &  & 九 & 九。薨,無後,國除為郡。 &  & 二十六 & 二十 \\ \hline
六 & 十六 & 二十 & 十九 & 十九 & 九 & 十七 &  & 三十 & 二十。來朝。 & 十四 & 十九 & 三十 & 十七 & 二十一。來朝。 & 二十一 & 二 & 二十 &  & 十一 & 二 &  & 十 &  &  & 二十七 & 二十一 \\ \hline
元光元年 & 十七 & 二十一 & 二十 & 二十 & 十。來朝。 & 十八 &  & 三十一 & 二十一 & 十五。來朝。 & 二十 & 三十一 & 十八。來朝。 & 二十二 & 二十二 & 三 & 二十一 &  & 十二 & 三 &  & 十一 &  &  & 二十八 & 二十二 \\ \hline
二 & 十八。來朝。 & 二十二 & 二十一 & 二十一 & 十一 & 十九 &  & 三十二 & 二十二 & 十六 & 二十一 & 三十二 & 十九 & 二十三 & 二十三 & 四 & 二十二。來朝。 &  & 十三 & 四 &  & 十二 &  &  & 二十九 & 二十三。來朝。 \\ \hline
三 & 十九。來朝。 & 二十三 & 二十二 & 二十二。卒。 & 十二 & 二十 &  & 三十三 & 二十三 & 十七 & 二十二 & 三十三 & 二十 & 二十四 & 二十四 & 五 & 二十三。來朝。 &  & 十四 & 五 &  & 十三 &  &  & 王義元年 & 二十四。來朝。 \\ \hline
四 & 二十 & 二十四 & 二十三 & 厲王次昌元年 & 十三 & 二十一 &  & 三十四 & 二十四 & 十八 & 二十三 & 三十四 & 二十一 & 二十五 & 二十五 & 六 & 二十四 &  & 十五 & 六 &  & 十四。來朝。 &  &  & 二 & 二十五 \\ \hline
五 & 二十一 & 二十五 & 二十四 & 二 & 十四。來朝。 & 二十二 &  & 三十五。薨。 & 二十五 & 十九 & 二十四 & 三十五 & 二十二 & 二十六 & 二十六。來朝。 & 七 & 二十五 &  & 十六 & 七 &  & 十五 &  &  & 三 & 二十六 \\ \hline
六 & 二十二。薨。 & 二十六。薨。 & 二十五 & 三 & 十五 & 二十三 &  & 靖王建元年 & 二十六 & 二十 & 二十五 & 三十六 & 二十三 & 二十七。來朝。 & 恭王不害元年 & 八 & 二十六 &  & 十七 & 八 &  & 十六 &  &  & 四 & 二十七 \\ \hline
元朔元年 & 襄王注元年 & 安王光元年 & 二十六 & 四 & 十六 & 二十四。來朝。 &  & 二 & 二十七 & 二十一 & 二十六 & 三十七 & 二十四。坐禽獸行自殺。國除為郡。 & 二十八 & 二 & 九 & 二十七 &  & 十八 & 九 &  & 十七 &  &  & 五 & 康王庸元年 \\ \hline
二 & 二 & 二 & 二十七 & 五。薨,無後,國除為郡。 & 十七 & 二十五 &  & 三 & 二十八。來朝。 & 二十二 & 王建元年 & 三十八 &  & 二十九 & 三 & 十 & 二十八 &  & 十九 & 十。來朝。 &  & 十八 &  &  & 六 & 二 \\ \hline
三 & 三 & 三 & 二十八 &  & 十八 & 二十六 &  & 四 & 二十九 & 二十三 & 二 & 三十九 &  & 三十 & 四。薨。 & 十一 & 二十九。來朝。 &  & 二十 & 十一 &  & 十九 &  &  & 七 & 三 \\ \hline
四 & 四。來朝。 & 四 & 二十九 &  & 十九 & 二十七 &  & 五 & 三十 & 二十四 & 三 & 四十 &  & 三十一 & 剛王堪元年 & 十二 & 三十 &  & 二十一 & 十二 &  & 二十。來朝。 &  &  & 八 & 四 \\ \hline
五 & 五 & 五 & 三十 &  & 二十 & 二十八 &  & 六 & 三十一 & 二十五。來朝。 & 四 & 四十一。安有罪,削國二縣。 &  & 三十二 & 二 & 十三 & 三十一 &  & 二十二。來朝。 & 十三 &  & 二十一 &  &  & 九 & 五 \\ \hline
六 & 六 & 六 & 三十一 &  & 二十一。來朝。 & 二十九 &  & 七 & 三十二 & 二十六 & 五 & 四十二 &  & 三十三 & 三 & 十四。來朝。 & 三十二 &  & 二十三 & 十四 &  & 二十二 &  &  & 十 & 六 \\ \hline
元狩元年 & 七 & 七 & 三十二。反,自殺,國除。 &  & 二十二 & 三十 &  & 八 & 三十三 & 二十七 & 六 & 四十三。反,自殺。 &  & 三十四。來朝。 & 四 & 十五 & 三十三 &  & 二十四 & 十五 &  & 二十三 &  &  & 十一 & 七 \\ \hline
二 & 八 & 八。來朝。 &  &  & 二十三 & 三十一 &  & 九 & 三十四 & 二十八 & 七。反,自殺,國除為廣陵郡。 & 置六安國,以故陳為都。七月丙子。初王恭王慶元年。膠東王子。 &  & 三十五 & 五 & 十六 & 三十四 &  & 二十五 & 十六 &  & 二十四 &  &  & 十二。來朝。 & 八。來朝。 \\ \hline
三 & 九 & 九 &  &  & 二十四 & 三十二。來朝。 &  & 十 & 三十五 & 哀王賢元年 &  & 二 &  & 三十六 & 六 & 十七 & 三十五。來朝。 &  & 二十六 & 十七 &  & 二十五 &  &  & 十三 & 九 \\ \hline
四 & 十。來朝。 & 十 &  &  & 二十五 & 三十三 &  & 十一 & 三十六 & 二 &  & 三 &  & 三十七 & 七 & 十八 & 三十六 &  & 二十七 & 十八 &  & 二十六。來朝。 &  &  & 十四 & 十 \\ \hline
五 & 十一 & 十一 &  & 復置齊國。 & 二十六。來朝。薨。 & 三十四 &  & 十二。來朝。 & 三十七 & 三 & 更為廣陵國。 & 四 & 復置燕國。 & 三十八 & 八 & 十九 & 三十七 &  & 二十八 & 十九 &  & 二十七 &  &  & 十五 & 十一 \\ \hline
六 & 十二 & 十二 &  & 四月乙巳,初王懷王閎元年。武帝子。 & 敬王義元年 & 三十五 &  & 十三 & 三十八 & 四 & 四月乙巳,初王胥元年。武帝子。 & 五 & 四月乙巳,初王剌王旦元年。武帝子。 & 三十九 & 九。來朝。 & 二十 & 三十八 &  & 二十九。來朝。 & 二十 &  & 二十八 &  &  & 十六 & 十二 \\ \hline
元鼎元年 & 十三 & 十三 &  & 二 & 二 & 三十六 &  & 十四 & 三十九 & 五 & 二 & 六 & 二 & 四十 & 十 & 二十一。來朝。 & 三十九 &  & 三十 & 二十一 &  & 二十九。剽攻殺人,遷上庸,國為大河郡。 &  &  & 十七 & 十三 \\ \hline
二 & 十四。薨。 & 十四。來朝。 &  & 三 & 三 & 三十七 &  & 十五 & 四十 & 六 & 三 & 七 & 三 & 四十一 & 十一 & 二十二 & 四十 &  & 三十一 & 二十二 &  &  &  &  & 十八。來朝。 & 十四 \\ \hline
三 & 節王純元年 & 十五 & 初置泗水,都郯。 & 四 & 四 & 三十八 &  & 十六 & 四十一 & 七 & 四 & 八 & 四 & 四十二 & 十二。薨。 & 二十三 & 四十一。來朝。 & 復置清河國。 & 三十二。薨,子為王。 & 二十三 &  &  &  &  & 十九。徙清河。為太原郡。 & 十五。來朝。 \\ \hline
四 & 二 & 十六 & 思王商元年。商,常山憲王子。 & 五 & 五 & 三十九 &  & 十七 & 四十二 & 八 & 五 & 九 & 五 & 四十三 & 頃王授元年 & 二十四 & 四十二。薨。 & 二十。代王義徙清河年。是為剛王。 & 更為真定國。頃王平元年。常山憲王子。 & 二十四 &  &  &  &  &  & 十六 \\ \hline
五 & 三 & 十七 & 二 & 六 & 六 & 四十 &  & 十八 & 四十三 & 九 & 六 & 十 & 六 & 四十四 & 二 & 二十五。來朝。 & 哀王昌元年。即年薨。 & 二十一 & 二 & 二十五 &  &  &  &  &  & 十七 \\ \hline
六 & 四 & 十八 & 三 & 七 & 七 & 四十一。來朝。 &  & 十九 & 四十四 & 十 & 七 & 十一。來朝。 & 七 & 四十五 & 三 & 二十六 & 康王昆侈元年。 & 二十二 & 三 & 二十六 &  &  &  &  &  & 十八 \\ \hline
元封元年 & 五 & 十九 & 四 & 八。薨,無後,國除為郡。 & 八。來朝。 & 四十二 &  & 二十 & 四十五 & 十一 & 八 & 十二 & 八 & 四十六 & 四 & 二十七 & 二 & 二十三 & 四。來朝。 & 二十七 &  &  &  &  &  & 十九 \\ \hline
二 & 六 & 二十 & 五 &  & 九。薨。 & 四十三 &  & 頃王遺元年。 & 四十六 & 十二 & 九 & 十三 & 九 & 四十七 & 五 & 二十八 & 三 & 二十四 & 五 & 二十八 &  &  &  &  &  & 二十 \\ \hline
三 & 七 & 二十一。來朝。 & 六 &  & 慧王武元年 & 四十四 &  & 二 & 四十七。薨,無後,國除。 & 十三 & 十 & 十四 & 十 & 四十八 & 六 & 二十九 & 四 & 二十五。來朝。 & 六 & 二十九 &  &  &  &  &  & 二十一 \\ \hline
四 & 八 & 二十二 & 七 &  & 二 & 四十五 &  & 三 &  & 十四 & 十一 & 十五 & 十一 & 四十九 & 七 & 三十 & 五 & 二十六 & 七 & 三十 &  &  &  &  &  & 二十二 \\ \hline
五 & 九 & 二十三。朝泰山。 & 八 &  & 三 & 四十六。朝泰山。 &  & 四 &  & 戴王通平元年 & 十二 & 十六 & 十二 & 五十 & 八 & 三十一 & 六 & 二十七 & 八 & 三十一 &  &  &  &  &  & 二十三 \\ \hline
六 & 十 & 二十四 & 九 &  & 四 & 四十七 &  & 五 &  & 二 & 十三 & 十七 & 十三 & 五十一 & 九 & 三十二 & 七 & 二十八 & 九。來朝。 & 三十二 &  &  &  &  &  & 二十四 \\ \hline
太初元年 & 十一 & 二十五 & 十。薨。 &  & 五 & 四十八 &  & 六 &  & 三 & 十四 & 十八。來朝。 & 十四 & 五十二 & 十 & 三十三 & 八 & 二十九 & 十 & 三十三 &  &  &  &  &  & 二十五 \\ \hline
二 & 十二 & 二十六 & 哀王安世元年。即戴王賀元年。安世子。 &  & 六 & 四十九 &  & 七 &  & 四 & 十五 & 十九 & 十五 & 五十三 & 十一 & 三十四 & 九。來朝。 & 三十 & 十一 & 三十四 &  &  &  &  &  & 二十六 \\ \hline
三 & 十三 & 二十七 & 二 &  & 七。 & 五十 &  & 八 &  & 五 & 十六 & 二十 & 十六 & 五十四 & 十二 & 三十五 & 十 & 三十一 & 十二 & 三十五 &  &  &  &  &  & 二十七 \\ \hline
四 & 十四 & 二十八 & 三 &  &  & 五十一 &  & 九 &  & 六 & 十七 & 二十一 & 十七 & 五十五 & 十三 & 三十六 & 十一 & 三十二 & 十三 & 三十六。來朝。 &  &  &  &  &  & 二十八。來朝。 \\ \hline}