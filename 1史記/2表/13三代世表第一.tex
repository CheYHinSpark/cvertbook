\chapter{三代世表第一}

太史公曰,五帝、三代之記,尚矣。自殷以前諸侯不可得而譜,周以來乃頗可著。孔子因史文次春秋,紀元年,正時日月,蓋其詳哉。至於序尚書則略,無年月,或頗有,然多闕,不可錄。故疑則傳疑,蓋其慎也。

余讀諜記,黃帝以來皆有年數。稽其歷譜諜終始五德之傳,古文咸不同,乖異。夫子之弗論次其年月,豈虛哉。於是以五帝系諜、尚書集世紀黃帝以來訖共和為世表。

\biao{|p{20em}|p{5em}|p{5em}|p{5em}|p{5em}|p{5em}|p{5em}|p{8em}|}
{\hline
帝王世國號&顓頊屬&黼屬&堯屬&舜屬&夏屬&殷屬&周屬\\\hline
黃帝號有熊。&黃帝生昌意。&黃帝生玄囂。&黃帝生玄囂。&黃帝生昌意。&黃帝生昌意。&黃帝生玄囂。&黃帝生玄囂。\\\hline
帝顓頊,黃帝孫。起黃帝,至顓頊三世,號高陽。&昌意生顓頊。為高陽氏。&玄囂生蟜極。&玄囂生蟜極。&昌意生顓頊。顓頊生窮蟬。&昌意生顓頊。&玄囂生蟜極。蟜極生高辛。&玄囂生蟜極。蟜極生高辛。\\\hline
帝黼,黃帝曾孫。起黃帝,至帝黼四世。號高辛。&&蟜極生高辛,為帝黼。&蟜極生高辛。高辛生放勛。&窮蟬生敬康。敬康生句望。&&高辛生党。&高辛生后稷,為周祖。\\\hline
帝堯。起黃帝,至黼子五世。號唐。&&&放勛為堯。&句望生蟜牛。蟜牛生瞽叟。&&镌為殷祖。&后稷生不窋。\\\hline
帝舜,黃帝玄孫之玄孫,號虞。&&&&瞽叟生重華,是為帝舜。&顓頊生鯀。鯀生文命。&镌生昭明。&不窋生鞠。\\\hline
帝禹,黃帝耳孫,號夏。&&&&&文命,是為禹。&昭明生相土。&鞠生公劉。\\\hline
帝啟,伐有扈,作甘誓。&&&&&&相土生昌若。&公劉生慶節。\\\hline
帝太康&&&&&&昌若生曹圉。曹圉生冥。&慶節生皇僕。皇僕生差弗。\\\hline
帝仲康,太康弟。&&&&&&冥生振。&差弗生毀渝。毀渝生公非。\\\hline
帝相&&&&&&振生微。微生報丁。&公非生高圉。高圉生亞圉。\\\hline
帝少康&&&&&&報丁生報乙。報乙生報丙。&亞圉生公祖類。\\\hline
帝予&&&&&&報丙生主壬。主壬生主癸。&公祖類生太王亶父。\\\hline
帝槐&&&&&&主癸生天乙,是為殷湯。&亶父生季歷。季歷生文王昌。益易卦。\\\hline
帝芒&&&&&&&文王昌生武王發。\\\hline
帝泄&&&&&&&\\\hline
帝不降&&&&&&&\\\hline
帝扃不降弟。&&&&&&&\\\hline
帝廑&&&&&&&\\\hline
帝孔甲,不降子。好鬼神,淫亂不好德,二龍去。&&&&&&&\\\hline
帝皋&&&&&&&\\\hline
帝發&&&&&&&\\\hline
帝履癸,是為桀。從禹至桀十七世。從黃帝至桀二十世。&&&&&&&\\\hline
殷湯代夏氏。從黃帝至湯十七世。&&&&&&&\\\hline
帝外丙,湯太子。太丁蚤卒,故立次弟外丙。&&&&&&&\\\hline
帝仲壬,外丙弟。&&&&&&&\\\hline
帝太甲,故太子太丁子。淫,伊尹放之桐宮。三年,悔過自責,伊尹乃迎之復位。&&&&&&&\\\hline
帝沃丁。伊尹卒。&&&&&&&\\\hline
帝太庚,沃丁弟。&&&&&&&\\\hline
帝小甲,太庚弟。殷道衰,諸侯或不至。&&&&&&&\\\hline
帝雍己,小甲弟。&&&&&&&\\\hline
帝太戊,雍己弟。以桑穀生,稱中宗。&&&&&&&\\\hline
帝中丁&&&&&&&\\\hline
帝外壬,中丁弟。&&&&&&&\\\hline
帝河亶甲,外壬弟。&&&&&&&\\\hline
帝祖乙&&&&&&&\\\hline
帝祖辛&&&&&&&\\\hline
帝沃甲,祖辛弟。&&&&&&&\\\hline
帝祖丁,祖辛子。&&&&&&&\\\hline
帝南庚,沃甲子。&&&&&&&\\\hline
帝陽甲,祖丁子。&&&&&&&\\\hline
帝盤庚,陽甲弟。徙河南。&&&&&&&\\\hline
帝小辛,盤庚弟。&&&&&&&\\\hline
帝小乙,小辛弟。&&&&&&&\\\hline
帝武丁。雉升鼎耳雊。得傅說。稱高宗。&&&&&&&\\\hline
帝祖庚&&&&&&&\\\hline
帝甲,祖庚弟。淫。&&&&&&&\\\hline
帝廩辛&&&&&&&\\\hline
帝庚丁,廩辛弟。殷徙河北。&&&&&&&\\\hline
帝武乙。慢神震死。&&&&&&&\\\hline
帝太丁&&&&&&&\\\hline
帝乙。殷益衰。&&&&&&&\\\hline
帝辛,是為紂。弒。從湯至紂二十九世。從黃帝至紂四十六世。&&&&&&&\\\hline
周武王代殷。從黃帝至武王十九世。&&&&&&&\\\hline
}

\biao{|p{7em}|p{5em}|p{5em}|p{4em}|p{4em}|p{5em}|p{5em}|p{4em}|p{4em}|p{4em}|p{4em}|p{4em}|}
{\hline
成王誦&魯周公旦,武王弟。初封。&齊太公尚,文王、武王師。初封。&晉唐叔虞,武王子。初封。&秦惡來,助紂。父飛廉,有力。&楚熊繹。繹父鬻熊,事文王。初封。&宋微子啟,紂庶兄。初封。&衛康叔,武王弟。初封。&陳胡公滿,舜之後。初封。&蔡叔度,武王弟。初封。&曹叔振鐸,武王弟。初封。&燕召公奭,周同姓。初封。\\\hline
康王釗刑錯四十餘年。&魯公伯禽&丁公呂伋&晉侯燮&女防&熊乂&微仲,啟弟。&康伯&申公&蔡仲&&九世至惠侯。\\\hline
昭王瑕南巡不返。不赴,諱之。&考公&乙公&武侯&旁皋&熊黮&宋公&孝伯&相公&蔡伯&太伯&\\\hline
穆王滿。作甫刑。荒服不至。&煬公,考公弟。&癸公&成侯&大几&熊勝&丁公&嗣伯&孝公&宮侯&仲君&\\\hline
恭王伊扈&幽公&哀公&厲侯&大駱&熊煬&湣公,丁公弟。&疌伯&慎公&厲侯&宮伯&\\\hline
懿王堅。周道衰,詩人作刺。&魏公&胡公&靖侯&非子&熊渠&煬公,湣公弟。&靖伯&幽公&武侯&孝伯&\\\hline
孝王方,懿王弟。&厲公&獻公弒胡公。&&秦侯&熊無康&厲公&貞伯&釐公&&夷伯&\\\hline
夷王燮,懿王子。&獻公,厲公弟。&武公&&公伯&熊鷙紅&釐公&頃侯&&&&\\\hline
厲王胡。以惡聞過亂,出奔,遂死於彘。&真公&&&秦仲&熊延,紅弟。&&釐侯&&&&\\\hline
共和,二伯行政。&武公,真公弟。&&&&熊勇&&&&&&\\\hline
}

張夫子問褚先生曰,詩言契、后稷皆無父而生。今案諸傳記咸言有父,父皆黃帝子也,得無與詩謬秋。

褚先生曰,不然。詩言契生於卵,后稷人跡者,欲見其有天命精誠之意耳。鬼神不能自成,須人而生,柰何無父而生乎。一言有父,一言無父,信以傳信,疑以傳疑,故兩言之。堯知契、稷皆賢人,天之所生,故封之契七十里,後十餘世至湯,王天下。堯知后稷子孫之後王也,故益封之百里,其後世且千歲,至文王而有天下。詩傳曰,湯之先為契,無父而生。契母與姊妹浴於玄丘水,有燕銜卵墮之,契母得,故含之,誤吞之,即生契。契生而賢,堯立為司徒,姓之曰子氏。子者茲,茲,益大也。詩人美而頌之曰殷社芒芒,天命玄鳥,降而生商。商者質,殷號也。文王之先為后稷,后稷亦無父而生。后稷母為姜嫄,出見大人蹟而履踐之,知於身,則生后稷。姜嫄以為無父,賤而棄之道中,牛羊避不踐也。抱之山中,山者養之。又捐之大澤,鳥覆席食之。姜嫄怪之,於是知其天子,乃取長之。堯知其賢才,立以為大農,姓之曰姬氏。姬者,本也。詩人美而頌之曰厥初生民,深修益成,而道后稷之始也。孔子曰,昔者堯命契為子氏,為有湯也。命后稷為姬氏,為有文王也。大王命季歷,明天瑞也。太伯之吳,遂生源也。天命難言,非聖人莫能見。舜、禹、契、后稷皆黃帝子孫也。黃帝策天命而治天下,德澤深後世,故其子孫皆復立為天子,是天之報有德也。人不知,以為閒從布衣匹夫起耳。夫布衣匹夫安能無故而起王天下乎。其有天命然。

黃帝後世何王天下之久遠邪。

曰,傳云天下之君王為萬夫之黔首請贖民之命者帝,有福萬世。黃帝是也。五政明則修禮義,因天時舉兵征伐而利者王,有福千世。蜀王,黃帝後世也,至今在漢西南五千里,常來朝降,輸獻於漢,非以其先之有德,澤流後世邪。行道德豈可以忽秋哉。人君王者舉而觀之。漢大將軍霍子孟名光者,亦黃帝後世也。此可為博聞遠見者言,固難為淺聞者說也。何以言之。古諸侯以國為姓。霍者,國名也。武王封弟叔處於霍,後世晉獻公滅霍公,後世為庶民,往來居平陽。平陽在河東,河東晉地,分為衛國。以詩言之,亦可為周世。周起后稷,后稷無父而生。以三代世傳言之,后稷有父名高辛,高辛,黃帝曾孫。黃帝終始傳曰,漢興百有餘年,有人不短不長,出自白燕之鄉,持天下之政,時有嬰兒主,欲行車。霍將軍者,本居平陽自白燕。臣為郎時,與方士考功會旗亭下,為臣言。豈不偉哉。