\chapter{建元以來侯者年表第八}

太史公曰,匈奴絕和親,攻當路塞,閩越擅伐,東甌請降。二夷交侵,當盛漢之隆,以此知功臣受封侔於祖考矣。何者。自詩書稱三代戎狄是膺,荊荼是徵,太史公曰,齊桓越燕伐山戎,武靈王以區區趙服單于,秦繆用百里霸西戎,吳楚之君以諸侯役百越。況乃以中國一統,明天子在上,兼文武,席卷四海,內輯億萬之眾,豈以晏然不為邊境征伐哉。自是後,遂出師北討彊胡,南誅勁越,將卒以次封矣。

\biao{|p{5em}|p{10em}|p{7em}|p{10em}|p{9em}|p{9em}|p{10em}|p{7em}|}
{
\hline
國名 & 侯功 & 元光 & 元朔 & 元狩 & 元鼎 & 元封 & 太初已後 \\ \hline
翕 & 匈奴相降,侯。元朔二年,屬車騎將軍,擊匈奴有功,益封。 & 三。四年七月壬午,侯趙信元年。 & 五。六年,侯信為前將軍擊匈奴,遇單于兵,敗,信降匈奴,國除。 &  &  &  &  \\ \hline
持裝 & 匈奴都尉降,侯。 & 六年後九月丙寅,侯樂。元年。 & 六 & 六 & 元年,侯樂死,無後,國除。 &  &  \\ \hline
親陽 & 匈奴相降,侯。 &  & 三。二年十月癸巳,侯月氏元年。五年,侯月氏坐亡斬,國除。 &  &  &  &  \\ \hline
若陽 & 匈奴相降,侯。 &  & 三。二年十月癸巳,侯猛元年。五年,侯猛坐亡斬,國除。 &  &  &  &  \\ \hline
長平 & 以元朔二年再以車騎將軍擊匈奴,取朔方、河南功侯。元朔五年,以大將軍擊匈奴,破右賢王,益封三千戶。 &  & 五。二年三月丙辰,烈侯衛青元年。 & 六 & 六 & 六 & 太初元年,今侯伉元年。 \\ \hline
平陵 & 以都尉從車騎將軍青擊匈奴功侯。以元朔五年,用遊擊將軍從大將軍,益封。 &  & 五。二年三月丙辰,侯蘇建元年。 & 六 & 六。六年,侯建為右將軍,與翕侯信俱敗,獨身脫來歸,當斬,贖,國除。 &  &  \\ \hline
岸頭 & 以都尉從車騎將軍青擊匈奴功侯。元朔六年,從大將軍,益封。 &  & 五。二年六月壬辰,侯張次公元年。 & 元年,次公坐與淮南王女姦,及受財物罪,國除。 &  &  &  \\ \hline
平津 & 以丞相詔所褒侯。 &  & 四。五年十一月乙丑,獻侯公孫弘元年。 & 二。四。三年,侯慶元年。 & 六 & 三。四年,侯慶坐為山陽太守有罪,國除。 &  \\ \hline
涉安 & 以匈奴單于太子降侯。 &  & 一。三年四月丙子,侯於單。元年。五月,卒,無後,國除。 &  &  &  &  \\ \hline
昌武 & 以匈奴王降侯。以昌武侯從驃騎將軍擊左賢王功,益封。 &  & 三。四年十月庚申,堅侯趙安稽元年。 & 六 & 六 & 一。五。二年,侯充國元年。 & 。太初元年,侯充國薨,亡後,國除。 \\ \hline
襄城 & 以匈奴相國降侯。 &  & 三。四年十月庚申,侯無龍元年。 & 六 & 六 & 六 & 一。太初二年,無龍從浞野侯戰死。二。三年,侯病已元年。 \\ \hline
南奅 & 以騎將軍從大將軍青擊匈奴得王功侯。太初二年,以丞相封為葛繹侯。 &  & 二。五年四月丁未,侯公孫賀元年。 & 六 & 四。五年,賀坐酎金,國除,絕,七歲。 &  & 十三。太初二年三月丁卯,封葛繹侯。征和二年,賀子敬聲有罪,國除。 \\ \hline
合騎 & 以護軍都尉三從大將軍擊匈奴,至右賢王庭,得王功侯。元朔六年益封。 &  & 二。五年四月丁未,侯公孫敖元年。 & 一。二年,侯敖將兵擊匈奴,與驃騎將軍期,後,畏懦,當斬,贖為庶人,國除。 &  &  &  \\ \hline
樂安 & 以輕車將軍再從大將軍青擊匈奴得王功侯。 &  & 二。五年四月丁未,侯李蔡元年。 & 四。五年,侯蔡以丞相盜孝景園神道壖地罪,自殺,國除。 &  &  &  \\ \hline
龍镪 & 以都尉從大將軍青擊匈奴得王功侯。元鼎六年,以橫海將軍擊東越功,為案道侯。 &  & 二。五年四月丁未,侯韓說元年。 & 六 & 四。五年,侯說坐酎金,國絕。二歲復侯。 & 六。元年五月丁卯,案道侯說元年。 & 十三。征和二年,子長代,有罪,絕。子曾復封為龍镪侯。 \\ \hline
隨成 & 以校尉三從大將軍青擊匈奴,攻農吾,先登石累,得王功侯。 &  & 二。五年四月乙卯,侯趙不虞元年。 & 三。三年,侯不虞坐為定襄都尉,匈奴敗太守,以聞非實,謾。國除。 &  &  &  \\ \hline
從平 & 以校尉三從大將軍青擊匈奴,至右賢王庭,數為鴈行上石山先登功侯。 &  & 二。五年四月乙卯,公孫戎奴元年。 & 一。二年,侯戎奴坐為上郡太守發兵擊匈奴,不以聞,謾,國除。 &  &  &  \\ \hline
涉軹 & 以校尉三從大將軍擊匈奴,至右賢王庭,得王,虜閼氏功侯。 &  & 二。五年四月丁未,侯李朔元年。 & 元年,侯朔有罪,國除。 &  &  &  \\ \hline
宜春 & 以父大將軍青破右賢王功侯。 &  & 二。五年四月丁未,侯衛伉元年。 & 六 & 元年,伉坐矯制不害,國除。 &  &  \\ \hline
陰安 & 以父大將軍青破右賢王功侯。 &  & 二。五年四月丁未,侯衛不疑元年。 & 六 & 四。五年,侯不疑坐酎金,國除。 &  &  \\ \hline
發干 & 以父大將軍青破右賢王功侯。 &  & 二。五年四月丁未,侯衛登元年。 & 六 & 四。五年,侯登坐酎金,國除。 &  &  \\ \hline
博望 & 以校尉從大將軍六年擊匈奴,知水道,及前使絕域大夏功侯。 &  & 一。六年三月甲辰,侯張騫元年。 & 一。二年,侯騫坐以將軍擊匈奴畏懦,當斬,贖,國除。 &  &  &  \\ \hline
冠軍 & 以嫖姚校尉再從大將軍,六年從大將軍擊匈奴,斬相國功侯。元狩二年,以驃騎將軍擊匈奴,至祁連,益封,迎渾邪王,益封,擊左右賢王,益封。 &  & 一。六年四月壬申,景桓侯霍去病元年。 & 六 & 六。元年,哀侯嬗元年。 & 元年,哀侯嬗薨,無後,國除。 &  \\ \hline
眾利 & 以上谷太守四從大將軍,六年擊匈奴,首虜千級以上功侯。 &  & 一。六年五月壬辰,侯郝賢。元年。 & 一。二年,侯賢坐為上谷太守入戍卒財物上計謾罪,國除。 &  &  &  \\ \hline
潦 & 以匈奴趙王降,侯。 &  &  & 一。元年七月壬午,悼侯趙王煖訾。元年。二年,煖訾死,無後,國除。 &  &  &  \\ \hline
宜冠 & 以校尉從驃騎將軍二年再出擊匈奴功侯。故匈奴歸義。 &  &  & 二。二年正月乙亥,侯高不識元年。四年,不識擊匈奴,戰軍功增首不以實,當斬,贖罪,國除。 &  &  &  \\ \hline
煇渠 & 以校尉從驃騎將軍二年再出擊匈奴,得王功侯。以校尉從驃騎將軍二年虜五王功,益封。故匈奴歸義。 &  &  & 五。二年二月乙丑,忠侯僕多。元年。 & 三。三。四年,侯電元年。 & 六 & 四 \\ \hline
從驃 & 以司馬再從驃騎將軍數深入匈奴,得兩王子騎將功侯。以匈河將軍元封三年擊樓蘭功,復侯。 &  &  & 五。二年五月丁丑,侯趙破奴元年。 & 四。五年,侯破奴坐酎金,國除。 & 浞野四。三年,侯破奴元年。 & 一。二年,侯破奴以浚稽將軍擊匈奴,失軍,為虜所得,國除。 \\ \hline
下麾 & 以匈奴王降侯。 &  &  & 五。二年六月乙亥,侯呼毒尼元年。 & 四。二。五年,煬侯伊即軒元年。 & 六 & 四 \\ \hline
漯陰 & 以匈奴渾邪王將眾十萬降侯,萬戶。 &  &  & 四。二年七月壬午,定侯渾邪元年。 & 六。元年,魏侯蘇元年。 & 五。五年,魏侯蘇薨,無後,國除。 &  \\ \hline
煇渠 & 以匈奴王降侯。 &  &  & 四。三年七月壬午,悼侯扁訾元年。 & 一。二年,侯扁訾死,無後,國除。 &  &  \\ \hline
河綦 & 以匈奴右王與渾邪降侯。 &  &  & 四。三年七月壬午,康侯烏犁元年。 & 二。四。三年,餘利鞮元年。 & 六 & 四 \\ \hline
常樂 & 以匈奴大當戶與渾邪降侯。 &  &  & 四。三年七月壬午,肥侯稠雕。元年。 & 六 & 六 & 二。太初三年,今侯廣漢元年。 \\ \hline
符離 & 以右北平太守從驃騎將軍四年擊右王,將重會期,首虜二千七百人功侯。 &  &  & 三。四年六月丁卯,侯路博德元年。 & 六 & 六 & 太初元年,侯路博德有罪,國除。 \\ \hline
壯 & 以匈奴歸義因淳王從驃騎將軍四年擊左王,以少破多,捕虜二千一百人功侯。 &  &  & 三。四年六月丁卯,侯復陸支元年。 & 二。四。三年,今侯偃元年。 & 六 & 四 \\ \hline
眾利 & 以匈奴歸義樓剸王。從驃騎將軍四年擊右王,手自劍合功侯。 &  &  & 三。四年六月丁卯,質侯伊即軒。元年 & 六 & 五。一。六年,今侯當時元年。 & 四 \\ \hline
湘成 & 以匈奴符離王降侯。 &  &  & 三。四年六月丁卯,侯敞屠洛元年。 & 四。五年,侯敞屠洛坐酎金,國除。 &  &  \\ \hline
義陽 & 以北地都尉從驃騎將軍四年擊左王,得王功侯。 &  &  & 三。四年六月丁卯,侯衛山元年。 & 六 & 六 & 四 \\ \hline
散 & 以匈奴都尉降侯。 &  &  & 三。四年六月丁卯,侯董荼吾。元年。 & 六 & 六 & 二。二。太初三年,今侯安漢元年。 \\ \hline
臧馬 & 以匈奴王降侯。 &  &  & 一。四年六月丁卯,康侯延年元年。五年,侯延年死,不得置後,國除。 &  &  &  \\ \hline
周子南君 & 以周後紹封。 &  &  &  & 三。四年十一月丁卯,侯姬嘉元年。 & 三。三。四年君買元年。 & 四 \\ \hline
樂通 & 以方術侯。 &  &  &  & 一。四年四月乙巳,侯五利將軍欒大元年。五年,侯大有罪,斬,國除。 &  &  \\ \hline
瞭 & 以匈奴歸義王降侯。 &  &  &  & 一。四年六月丙午,侯次公元年。五年,侯次公坐酎金,國除。 &  &  \\ \hline
術陽 & 以南越王兄越高昌侯。 &  &  &  & 一。四年,侯建德元年。五年,侯建德有罪,國除。 &  &  \\ \hline
龍亢 & 以校尉摎樂擊南越,死事,子侯。 &  &  &  & 二。五年三月壬午,侯廣德元年。 & 六。六年,侯廣德有罪誅,國除。 &  \\ \hline
成安 & 以校尉韓千秋擊南越死事,子侯。 &  &  &  & 二。五年三月壬子,侯延年元年。 & 六。六年,侯延年有罪,國除。 &  \\ \hline
昆 & 以屬國大且渠擊匈奴功侯。 &  &  &  & 二。五年五月戊戌,侯渠復累。元年。 & 六 & 四 \\ \hline
騏 & 以屬國騎擊匈奴,捕單于兄功侯。 &  &  &  & 二。五年六月壬子,侯駒幾元年。 & 六 & 四 \\ \hline
梁期 & 以屬國都尉五年閒出擊匈奴,得復累絺縵等功侯。 &  &  &  & 二。五年七月辛巳,侯任破胡元年。 & 六 & 四 \\ \hline
牧丘 & 以丞相及先人萬石積德謹行侯。 &  &  &  & 二。五年九月丁丑,恪侯石慶元年。 & 六 & 二。二。三年,侯德元年。 \\ \hline
瞭 & 以南越將降侯。 &  &  &  & 一。六年三月乙酉,侯畢取元年。 & 六 & 四 \\ \hline
將梁 & 以樓船將軍擊南越,椎鋒卻敵侯。 &  &  &  & 一。六年三月乙酉,侯楊僕元年。 & 三。四年,侯僕有罪,國除。 &  \\ \hline
安道 & 以南越揭陽令聞漢兵至自定降侯。 &  &  &  & 一。六年三月乙酉,侯揭陽令史定元年。 & 六 & 四 \\ \hline
隨桃 & 以南越蒼梧王聞漢兵至降侯。 &  &  &  & 一。六年四月癸亥,侯趙光元年。 & 六 & 四 \\ \hline
湘成 & 以南越桂林監聞漢兵破番禺,諭甌駱兵四十餘萬降侯。 &  &  &  & 一。六年五月壬申,侯監居翁。元年。 & 六 & 四 \\ \hline
海常 & 以伏波司馬捕得南越王建德功侯。 &  &  &  & 一。六年七月乙酉,莊侯蘇弘元年。 & 六 & 太初元年,侯弘死,無後,國除。 \\ \hline
北石 & 以故東越衍侯佐繇王斬餘善功侯。 &  &  &  &  & 六。元年正月壬午,侯吳陽元年。 & 三。太初四年,今侯首元年。 \\ \hline
下酈 & 以故甌駱左將斬西于王功侯。 &  &  &  &  & 六。元年四月丁酉,侯左將黃同元年。 & 四 \\ \hline
繚嫈 & 以故校尉從橫海將軍說擊東越功侯。 &  &  &  &  & 一。元年五月己卯,侯劉福元年。二年,侯福有罪,國除。 &  \\ \hline
蘌兒 & 以軍卒斬東越徇北將軍功侯。 &  &  &  &  & 六。元年閏月癸卯,莊侯轅終古元年。 & 太初元年,終古死,無後,國除。 \\ \hline
開陵 & 以故東越建成侯與繇王共斬東越王餘善功侯。 &  &  &  &  & 六。元年閏月癸卯,侯建成元年。 &  \\ \hline
臨蔡 & 以故南越郎聞漢兵破番禺,為伏波得南越相呂嘉功侯。 &  &  &  &  & 六。元年閏月癸卯,侯孫都元年。 &  \\ \hline
東成 & 以故東越繇王斬東越王餘善功侯,萬戶。 &  &  &  &  & 六。元年閏月癸卯,侯居服元年。 &  \\ \hline
無錫 & 以東越將軍漢兵至棄軍降侯。 &  &  &  &  & 六。元年,侯多軍元年。 &  \\ \hline
涉都 & 以父棄故南海守,漢兵至以城邑降,子侯。 &  &  &  &  & 六。元年中,侯嘉元年。 & 二。太初二年,侯嘉薨,無後,國除。 \\ \hline
平州 & 以朝鮮將漢兵至降侯。 &  &  &  &  & 一。三年四月丁卯,侯唊元年。四年,侯唊薨,無後,國除。 &  \\ \hline
荻苴 & 以朝鮮相漢兵至圍之降侯。 &  &  &  &  & 四。三年四月,侯朝鮮相韓陰元年。 &  \\ \hline
澅清 & 以朝鮮尼谿相使人殺其王右渠來降侯。 &  &  &  &  & 四。三年六月丙辰,侯朝鮮尼谿相參元年。 &  \\ \hline
騠茲 & 以小月氏若苴王將眾降侯。 &  &  &  &  & 三。四年十一月丁卯,侯稽谷姑。元年。 & 太初元年,侯稽谷姑薨,無後,國除。 \\ \hline
浩 & 以故中郎將將兵捕得車師王功侯。 &  &  &  &  & 一。四年正月甲申,侯王恢元年。四年四月,侯恢坐使酒泉矯制害,當死,贖,國除。封凡三月。 &  \\ \hline
瓡讘 & 以小月氏王將眾千騎降侯。 &  &  &  &  & 二。四年正月乙酉,侯扜者。元年。一。六年,侯勝元年。 & 四 \\ \hline
幾 & 以朝鮮王子漢兵圍朝鮮降侯。 &  &  &  &  & 二。四年三月癸未,侯張阍。歸義元年。六年,侯張阍使朝鮮,謀反,死,國除。 &  \\ \hline
涅陽 & 以朝鮮相路人,漢兵至,首先降,道死,其子侯。 &  &  &  &  & 三。四年三月壬寅,康侯子最元年。 & 二。太初二年,侯最死,無後,國除。 \\ \hline
}

右太史公本表。

\biao{|p{2em}|p{70em}|}{
\hline
當塗 & 魏不害,以圉守尉捕淮陽反者公孫勇等侯。 \\ \hline
蒲 & 蘇昌,以圉尉史捕淮陽反者公孫勇等侯。 \\ \hline
潦陽 & 江德,以園廄嗇夫共捕淮陽反者公孫勇等侯。 \\ \hline
富民 & 田千秋,家在長陵。以故高廟寢郎上書諫孝武曰,子弄父兵,罪當笞。父子之怒,自古有之。蚩尤畔父,黃帝涉江。上書至意,拜為大鴻臚。征和四年為丞相,封三千戶。至昭帝時病死,子順代立,為虎牙將軍,擊匈奴,不至質,誅死,國除。 \\ \hline}

右孝武封國名。

後進好事儒者褚先生曰,太史公記事盡於孝武之事,故復修記孝昭以來功臣侯者,編於左方,令後好事者得覽觀成敗長短絕世之適,得以自戒焉。當世之君子,行權合變,度時施宜,希世用事,以建功有土封侯,立名當世,豈不盛哉。觀其持滿守成之道,皆不謙讓,驕蹇爭權,喜揚聲譽,知進不知退,終以殺身滅國。以三得之,及身失之,不能傳功於後世,令恩德流子孫,豈不悲哉。夫龍雒侯曾為前將軍,世俗順善,厚重謹信,不與政事,退讓愛人。其先起於晉六卿之世。有土君國以來,為王侯,子孫相承不絕,歷年經世,以至于今,凡百餘歲,豈可與功臣及身失之者同日而語之哉。悲夫,後世其誡之。

\biao{|p{2em}|p{70em}|}{
\hline
博陸 & 霍光,家在平陽。以兄驃騎將軍故貴。前事武帝,覺捕得侍中謀反者馬何羅等功侯,三千戶。中輔幼主昭帝,為大將軍。謹信,用事擅治,尊為大司馬,益封邑萬戶。後事宣帝。歷事三主,天下信鄉之,益封二萬戶。子禹代立,謀反,族滅,國除。 \\ \hline
秺 & 金翁叔名日磾,以匈奴休屠王太子從渾邪王將眾五萬,降漢歸義,侍中,事武帝,覺捕侍中謀反者馬何羅等功侯,三千戶。中事昭帝,謹厚,益封三千戶。子弘代立,為奉車都尉,事宣帝。 \\ \hline
安陽 & 上官桀,家在隴西。以善騎射從軍。稍貴,事武帝,為左將軍。覺捕斬侍中謀反者馬何羅弟重合侯通功侯,三千戶。中事昭帝,與大將軍霍光爭權,因以謀反,族滅,國除。 \\ \hline
桑樂 & 上官安。以父桀為將軍故貴,侍中,事昭帝。安女為昭帝夫人,立為皇后故侯,三千戶。驕蹇,與大將軍霍光爭權,因以父子謀反,族滅,國除。 \\ \hline
富平 & 張安世,家在杜陵。以故御史大夫張湯子武帝時給事尚書,為尚書令。事昭帝,謹厚習事,為光祿勳右將軍。輔政十三年,無適過,侯,三千戶。及事宣帝,代霍光為大司馬,用事,益封萬六千戶。子延壽代立,為太僕,侍中。 \\ \hline
義陽 & 傅介子,家在北地。以從軍為郎,為平樂監。昭帝時,刺殺外國王,天子下詔書曰,平樂監傅介子使外國,殺樓蘭王,以直報怨,不煩師,有功,其以邑千三百戶封介子為義陽侯。子厲代立,爭財相告,有罪,國除。 \\ \hline
商利 & 王山,齊人也。故為丞相史,會騎將軍上官安謀反,山說安與俱入丞相,斬安。山以軍功為侯,三千戶。上書願治民,為代太守。為人所上書言,繫獄當死,會赦,出為庶人,國除。 \\ \hline
建平 & 杜延年。以故御史大夫杜周子給事大將軍幕府,發覺謀反者騎將軍上官安等罪,封為侯,邑二千七百戶,拜為太僕。元年,出為西河太守。五鳳三年,入為御史大夫。 \\ \hline
弋陽 & 任宮。以故上林尉捕格謀反者左將軍上官桀,殺之便門,封為侯,二千戶。後為太常,及行衛尉事。節儉謹信,以壽終,傳於子孫。 \\ \hline
宜城 & 燕倉。以故大將軍幕府軍吏發謀反者騎將軍上官安罪有功,封侯,邑二千戶。為汝南太守,有能名。 \\ \hline
宜春 & 王訢,家在齊。本小吏佐史,稍遷至右輔都尉。武帝數幸扶風郡,訢共置辦,拜為右扶風。至孝昭時,代桑弘羊為御史大夫。元鳳三年,代田千秋為丞相,封二千戶。立二年,為人所上書言暴,自殺,不殊。子代立,為屬國都尉。 \\ \hline
安平 & 楊敞,家在華陰。故給事大將軍幕府,稍遷至大司農,為御史大夫。元鳳六年,代王訢為丞相,封二千戶。立二年,病死。子賁代立,十三年病死。子翁君代立,為典屬國。三歲,以季父惲故出惡言,繫獄當死,得免,為庶人,國除。 \\ \hline}

右孝昭時所封國名。

\biao{|p{2em}|p{70em}|}{
\hline
陽平 & 蔡義,家在溫。故師受韓詩,為博士,給事大將軍幕府,為杜城門候。入侍中,授昭帝韓詩,為御史大夫。是時年八十,衰老,常兩人扶持乃能行。然公卿大臣議,以為為人主師,當以為相。以元平元年代楊敞為丞相,封二千戶。病死,絕無後,國除。 \\ \hline
扶陽 & 韋賢,家在魯。通詩、禮、尚書,為博士,授魯大儒,入侍中,為昭帝師,遷為光祿大夫,大鴻臚,長信少府。以為人主師,本始三年代蔡義為丞相,封扶陽侯,千八百戶。為丞相五歲,多恩,不習吏事,免相就第,病死。子玄成代立,為太常。坐祠廟騎,奪爵,為關內侯。 \\ \hline
平陵 & 范明友,家在隴西。以家世習外國事,使護西羌。事昭帝,拜為度遼將軍,擊烏桓功侯,二千戶。取霍光女為妻。地節四年,與諸霍子禹等謀反,族滅,國除。 \\ \hline
營平 & 趙充國。以隴西騎士從軍得官,侍中,事武帝。數將兵擊匈奴有功,為護軍都尉,侍中,事昭帝。昭帝崩,議立宣帝,決疑定策,以安宗廟功侯,封二千五百戶。 \\ \hline
陽成 & 田延年。以軍吏事昭帝,發覺上官桀謀反事,後留遲不得封,為大司農。本造廢昌邑王議立宣帝,決疑定策,以安宗廟功侯,二千七百戶。逢昭帝崩,方上事並急,因以盜都內錢三千萬。發覺,自殺,國除。 \\ \hline
平丘 & 王遷,家在衛。為尚書郎,習刀筆之文。侍中,事昭帝。帝崩,立宣帝,決疑定策,以安宗廟功侯,二千戶。為光祿大夫,秩中二千石。坐受諸侯王金錢財。漏洩中事,誅死,國除。 \\ \hline
樂成 & 霍山。山者,大將軍光兄子也。光未死時上書曰,臣兄驃騎將軍去病從軍有功,病死,賜謚景桓侯,絕無後,臣光願以所封東武陽邑三千五百戶分與山。天子許之,拜山為侯。後坐謀反,族滅,國除。 \\ \hline
冠軍 & 霍雲。以大將軍兄驃騎將軍適孫為侯。地節三年,天子下詔書曰驃騎將軍去病擊匈奴有功,封為冠軍侯。薨卒,子侯代立,病死無後。春秋之義,善善及子孫,其以邑三千戶封雲為冠軍侯。後坐謀反,族滅,國除。 \\ \hline
平恩 & 許廣漢,家昌邑。坐事下蠶室,獨有一女,嫁之。宣帝未立時,素與廣漢出入相通,卜相者言當大貴,以故廣漢施恩甚厚。地節三年,封為侯,邑三千戶。病死無後,國除。 \\ \hline
昌水 & 田廣明。故郎,為司馬,稍遷至南郡都尉、淮陽太守、鴻臚、左馮翊。昭帝崩,議廢昌邑王,立宣帝,決疑定策,以安宗廟。本始三年,封為侯,邑二千三百戶。為御史大夫。後為祁連將軍,擊匈奴,軍不至質,當死,自殺,國除。 \\ \hline
高平 & 魏相,家在濟陰。少學易,為府卒史,以賢良舉為茂陵令,遷河南太守。坐賊殺不辜,繫獄,當死,會赦,免為庶人。有詔守茂陵令,為楊州刺史,入為諫議大夫,復為河南太守,遷為大司農、御史大夫。地節三年,譖毀韋賢,代為丞相,封千五百戶。病死,長子賓代立,坐祠廟失侯。 \\ \hline
博望 & 許中翁。以平恩侯許廣漢弟封為侯,邑二千戶。亦故有私恩,為長樂衛尉。死,子延年代立。 \\ \hline
樂平 & 許翁孫。以平恩侯許廣漢少弟故為侯,封二千戶。拜為彊弩將軍,擊破西羌,還,更拜為大司馬、光祿勳。亦故有私恩,故得封。嗜酒好色,以早病死。子湯代立。 \\ \hline
將陵 & 史子回。以宣帝大母家封為侯,二千六百戶,與平臺侯昆弟行也。子回妻宜君,故成王孫,嫉妒,絞殺侍婢四十餘人,盜斷婦人初產子臂膝以為媚道。為人所上書言,論棄市。子回以外家故,不失侯。 \\ \hline
平臺 & 史子叔。以宣帝大母家封為侯,二千五百戶。衛太子時,史氏內一女於太子,嫁一女魯王,今見魯王亦史氏外孫也。外家有親,以故貴,數得賞賜。 \\ \hline
樂陵 & 史子長。以宣帝大母家貴,侍中,重厚忠信。以發覺霍氏謀反事,封三千五百戶。 \\ \hline
博成 & 張章,父故潁川人,為長安亭長。失官,之北闕上書,寄宿霍氏第舍,臥馬櫪閒,夜聞養馬奴相與語,言諸霍氏子孫欲謀反狀,因上書告反,為侯,封三千戶。 \\ \hline
都成 & 金安上,先故匈奴。以發覺故大將軍霍光子禹等謀反事有功,封侯,二千八百戶。安上者,奉車都尉秺侯從群子。行謹善,退讓以自持,欲傳功德於子孫。 \\ \hline
平通 & 楊惲,家在華陰,故丞相楊敞少子,任為郎。好士,自喜知人,居眾人中常與人顏色,以故高昌侯董忠引與屏語,言霍氏謀反狀,共發覺告反侯,二千戶,為光祿動。到五鳳四年,作為妖言,大逆罪腰斬,國除。 \\ \hline
高昌 & 董忠,父故潁川陽翟人,以習書詣長安。忠有材力,能騎射,用短兵,給事期門。與張章相習知,章告語忠霍禹謀反狀,忠以語常侍騎郎楊惲,共發覺告反,侯,二千戶。今為梟騎都尉,侍中。坐祠宗廟乘小車,奪百戶。 \\ \hline
爰戚 & 趙成。用發覺楚國事侯,二千三百戶。地節元年,楚王與廣陵王謀反,成發覺反狀,天子推恩廣德義,下詔書曰無治廣陵王,廣陵不變更。後復坐祝詛滅國,自殺,國除。今帝復立子為廣陵王。 \\ \hline
酇 & 地節三年,天子下詔書曰,朕聞漢之興,相國蕭何功第一,今絕無後,朕甚憐之,其以邑三千戶封蕭何玄孫建世為酇侯。 \\ \hline
平昌 & 王長君,家在趙國,常山廣望邑人也。衛太子時,嫁太子家,為太子男史皇孫為配,生子男,絕不聞聲問,行且四十餘歲,至今元康元年中,詔徵,立以為侯,封五千戶。宣帝舅父也。 \\ \hline
樂昌 & 王稚君,家在趙國,常山廣望邑人也。以宣帝舅父外家封為侯,邑五千戶。平昌侯王長君弟也。 \\ \hline
邛成 & 王奉光,家在房陵。以女立為宣帝皇后,故封千五百戶。言奉光初生時,夜見光其上,傳聞者以為當貴云。後果以女故為侯。 \\ \hline
安遠 & 鄭吉,家在會稽。以卒伍起從軍為郎,使護將弛刑士田渠梨。會匈奴單于死,國亂,相攻,日逐王將眾來降漢,先使語吉,吉將吏卒數百人往迎之。眾頗有欲還者,斬殺其渠率,遂與俱入漢。以軍功侯,二千戶。 \\ \hline
博陽 & 邴吉,家在魯。本以治獄為御史屬,給事大將軍幕府。常施舊恩宣帝,遷為御史大夫,封侯,二千戶。神爵二年,代魏相為丞相。立五歲,病死。子翁孟代立,為將軍,侍中。甘露元年,坐祠宗廟不乘大車而騎至廟門,有罪,奪爵,為關內侯。 \\ \hline
建成 & 黃霸,家在陽夏,以役使徙雲陽。以廉吏為河內守丞,遷為廷尉監,行丞相長史事。坐見知夏侯勝非詔書大不敬罪,久繫獄三歲,從勝學尚書。會赦,以賢良舉為揚州刺史,潁川太守。善化,男女異路,耕者讓畔,賜黃金百斤,秩中二千石。居潁川,入為太子傅,遷御史大夫。五鳳三年,代邴吉為丞相。封千八百戶。 \\ \hline
西平 & 于定國,家在東海。本以治獄給事為廷尉史,稍遷御史中丞。上書諫昌邑王,遷為光祿大夫,為廷尉。乃師受春秋,變道行化,謀厚愛人。遷為御史大夫,代黃霸為丞相。 \\ \hline
}

右孝宣時所封。

\biao{|p{2em}|p{70em}|}{
\hline
陽平 & 王稚君,家在魏郡。故丞相史。女為太子妃。太子立為帝,女為皇后,故侯,千二百戶。初元以來,方盛貴用事,游宦求官於京師者多得其力,未聞其有知略廣宣於國家也。 \\ \hline}