\chapter{惠景閒侯者年表第七}

太史公讀列封至便侯,曰:有以也夫!長沙王者,著令甲,稱其忠焉。昔高祖定天下,功臣非同姓疆土而王者八國。至孝惠時,唯獨長江全,禪五世,禪五世,以無嗣絕,意無過,為藩守職,信矣。故其澤流枝庶,毋功而侯者數人。及孝惠訖孝景閒五十載,追修高祖時遣功臣,及從代來,吳楚之勞,諸侯子弟若肺腑,外國歸義,封者九十有餘。咸表始終,當世仁義成功之著者也。

\biao{|p{2em}|p{12em}|p{12em}|p{12em}|p{12em}|p{8em}|p{10em}|p{4em}|}{
\hline
國名 & 侯功 & 孝惠七 & 高后八 & 孝文二十三 & 孝景十六 & 建元至元封六年三十六 & 太初已後 \\ \hline
便 & 長沙王子,侯,二千戶。 & 七。元年九月,頃侯吳淺元年。 & 八 & 二十二。一。後七年,恭侯信元年。 & 五。十一。前六年,侯廣志元年。 & 二十八。元鼎五年,侯千秋坐酎金,國除。 &  \\ \hline
軑 & 長沙相,侯,七百戶。 & 六。二年四月庚子,侯利倉元年。 & 二。六。三年,侯豨元年。 & 十五。八。十六年,侯彭祖元年。 & 十六 & 三十。元封元年,侯秩為東海太守,行過不請,擅發卒兵為衛,當斬,會赦,國除。 &  \\ \hline
平都 & 以齊將,高祖三年降,定齊,侯,千戶。 & 三。五年六月乙亥,孝侯劉到元年。 & 八 & 二。二十一。三年,侯成元年。 & 十四。後二年,侯成有罪,國除。 &  &  \\ \hline
}

右孝惠時三。

\biao{|p{2em}|p{12em}|p{12em}|p{12em}|p{12em}|}{
\hline
扶柳 & 高后姊長姁子,侯。 & 七。元年四月庚寅,侯呂平元年。八年,侯平坐呂氏事誅,國除。 &  &  \\ \hline
郊 & 呂后兄悼武王身佐高祖定天下,呂氏佐高祖治天下,天下大安,封武王少子產為郊侯。 & 五。元年四月辛卯,侯呂產元年。六年七月壬辰,產為呂王,國除。八年九月,產以呂王為漢相,謀為不善。大臣誅產,遂滅諸呂。 &  &  \\ \hline
南宮 & 以父越人為高祖騎將,從軍,以大中大夫侯。 & 七。元年四月丙寅,侯張買元年。八年,侯買坐呂氏事誅,國除。 &  &  \\ \hline
梧 & 以軍匠從起郟,入漢,後為少府,作長樂、未央宮,築長安城,先就,功侯,五百戶。 & 六。元年四月乙酉,齊侯陽成延元年。二。七年,敬侯去疾元年。 & 二十三 & 九。七。中三年,靖侯偃元年。 \\ \hline
平定 & 以卒從高祖起留,以家車吏入漢,以梟騎都尉擊項籍,得樓煩將功,用齊丞相侯。一云項涓。 & 八。元年四月乙酉,敬侯齊受元年。 & 一。四。二年,齊侯市人元年。十八。六年,恭侯應元年。 & 十六 \\ \hline
博成 & 以悼武王郎中,兵初起,從高祖起豐,攻雍丘,擊項籍,力戰,奉衛悼武王出滎陽,功侯。 & 三。元年四月乙酉,敬侯馮無擇元年。四。四年,侯代元年。八年,侯代坐呂氏事誅,國除。 &  &  \\ \hline
沛 & 呂后兄康侯少子,侯,奉呂宣王寢園。 & 七。元年四月乙酉,侯呂種元年。一。為不其侯。八年,侯種坐呂氏事誅,國除。 &  &  \\ \hline
襄成 & 孝惠子,侯。 & 一。元年四月辛卯,侯義元年。二年,侯義為常山王,國除。 &  &  \\ \hline
軹 & 孝惠子,侯。 & 三。元年四月辛卯,侯朝元年。四年,侯朝為常山王,國除。 &  &  \\ \hline
壺關 & 孝惠子,侯。 & 四。元年四月辛卯,侯武元年。五年,侯武為淮陽王,國除。 &  &  \\ \hline
沅陵 & 長沙嗣成王子,侯。 & 八。元年十一月壬申,頃侯吳陽元年。 & 十七。六。後二年,頃侯福元年。 & 十一。四。中五年,哀侯周元年。後三年,侯周薨,無後,國除。 \\ \hline
上邳 & 楚元王子,侯。 & 七。二年五月丙申,侯劉郢客元年。 & 一。二年,侯郢客為楚王,國除。 &  \\ \hline
朱虛 & 齊悼惠王子,侯。 & 七。二年五月丙申,侯劉章元年。 & 一。二年,侯章為城陽王,國除。 &  \\ \hline
昌平 & 孝惠子,侯。 & 三。四年二月癸未,侯太元年。七年,太為呂王,國除。 &  &  \\ \hline
贅其 & 呂后昆弟子,用淮陽丞相侯。 & 四。四年四月丙申,侯呂勝元年。八年,侯勝坐呂氏事誅,國除。 &  &  \\ \hline
中邑 & 以執矛從高祖入漢,以中尉破曹咎,用呂相侯,六百戶。 & 五。四年四月丙申,貞侯朱通元年。 & 十七。六。後二年,侯悼元年。 & 十五。後三年,侯悼有罪,國除。 \\ \hline
樂平 & 以隊卒從高祖起沛,屬皇訢,以郎擊陳餘,用衛尉侯,六百戶。 & 二。四年四月丙申,簡侯衛無擇元年。三。六年,恭侯勝元年。 & 二十三 & 十五。一。後三年,侯侈元年。 \\ \hline
山都 & 高祖五年為郎中柱下令,以衛將軍擊陳豨,用梁相侯。 & 五。四年四月丙申,貞侯王恬開元年。 & 三。二十。四年,惠侯中黃元年。 & 三。十三。四年,敬侯觸龍元年。 \\ \hline
松茲 & 兵初起,以舍人從起沛,以郎中入漢,還,得雍王邯家屬功,用常山丞相侯。 & 五。四年四月丙申,夷侯徐厲元年。 & 六。十七。七年,康侯悼元年。 & 十二。四。中六年,侯偃元年。 \\ \hline
成陶 & 以卒從高祖起單父,為呂氏舍人,度呂后淮之功,用河南守侯,五百戶。 & 五。四年四月丙申,夷侯周信元年。 & 十一。十二年,孝侯勃元年。三。十五年,侯勃有罪,國除。 &  \\ \hline
俞 & 以連敖從高祖破秦,入漢,以都尉定諸侯,功比朝陽侯。嬰死,子它襲功,用太中大夫侯。 & 四。四年四月丙申,侯呂它元年。。八年,侯它坐呂氏事誅,國除。 &  &  \\ \hline
滕 & 以舍人、郎中,十二歲,以都尉屯田霸上,用楚相侯。 & 四。四年四月丙申,侯呂更始元年。。八年,侯更始坐呂氏事誅,國除。 &  &  \\ \hline
醴陵 & 以卒從,漢王二年初起櫟陽,以卒吏擊項籍,為河內都尉,用長沙相侯,六百戶。 & 五。四年四月丙申,侯越元年。 & 三。四年,侯越有罪,國除。 &  \\ \hline
呂成 & 呂后昆弟子,侯。 & 四。四年四月丙申,侯呂忿元年。八年,侯忿坐呂氏事誅,國除。 &  &  \\ \hline
東牟 & 齊悼惠王子,侯。 & 三。六年四月丁酉,侯劉興居元年。 & 一。二年,侯興居為濟北王,國除。 &  \\ \hline
錘 & 呂肅王子,侯。 & 二。六年四月丁酉,侯呂通。元年。八年,侯通為燕王,坐呂氏事,國除。 &  &  \\ \hline
信都 & 以張敖、魯元太后子侯。 & 一。八年四月丁酉,侯張侈元年。 & 元年,侯侈有罪,國除。 &  \\ \hline
樂昌 & 以張敖、魯元太后子侯。 & 一。八年四月丁酉,侯張受元年。 & 元年,侯受有罪,國除。 &  \\ \hline
祝茲 & 呂后昆弟子,侯。 & 八年四月丁酉,侯呂榮元年。坐呂氏事誅,國除。 &  &  \\ \hline
建陵 & 以大謁者侯,宦者,多奇計。 & 八年四月丁酉,侯張澤元年。。九月,奪侯,國除。 &  &  \\ \hline
東平 & 以燕王呂通弟侯。 & 八年五月丙辰,侯呂莊元年。坐呂氏事誅,國除。 &  &  \\ \hline
}

右高后時三十一。

\biao{|p{2em}|p{12em}|p{12em}|p{12em}|p{12em}|}{
\hline
陽信 & 高祖十一年為郎。以典客奪趙王呂祿印,關殿門拒呂產等入,共尊立孝文,侯,二千戶。 & 十四。元年三月辛丑,侯劉揭。元年。九。十五年,侯中意元年。 & 五。六年,侯中意有罪,國除。 &  \\ \hline
軹 & 高祖十年為郎,從軍,十七歲為太中大夫,迎孝文代,用車騎將軍迎太后,侯,萬戶。薄太后弟。 & 十。元年四月乙巳,侯薄昭元年。十三。十一年,易侯戎奴元年。 & 十六 & 一。建元二年,侯梁元年。 \\ \hline
壯武 & 以家吏從高祖起山東,以都尉從守滎陽,食邑。以代中尉勸代王入,驂乘至代邸,王卒為帝,功侯,千四百戶。 & 二十三。元年四月辛亥,侯宋昌元年。 & 十一。中四年,侯昌奪侯,國除。 &  \\ \hline
清都 & 以齊哀王舅父侯。 & 五。元年四月辛未,侯駟鈞元年。前六年,鈞有罪,國除。 &  &  \\ \hline
周陽 & 以淮南厲王舅父侯。 & 五。元年四月辛未,侯趙兼元年。前六年,兼有罪,國除。 &  &  \\ \hline
樊 & 以睢陽令從高祖初起阿,以韓家子還定北地,用常山相侯,千二百戶。 & 十四。元年六月丙寅,侯蔡兼元年。九。十五年,康侯客元年。 & 九。七。中三年,恭侯平元年。 & 十三。元朔二年,侯辟方元年。十四。元鼎四年,侯辟方有罪,國除。 \\ \hline
管 & 齊悼惠王子,侯。 & 二。四年五月甲寅,恭侯劉罷軍元年。。十八。六年,侯戎奴元年。 & 二。三年,侯戎奴反,國除。 &  \\ \hline
瓜丘 & 齊悼惠王子。 & 十一。四年五月甲寅,侯劉寧國元年。九。十五年,侯偃元年。 & 二。三年,侯偃反,國除。 &  \\ \hline
營 & 齊悼惠王子,侯。 & 十。四年五月甲寅,平侯劉信都元年。十。十四年,侯廣元年。 & 二。三年,侯廣反,國除。 &  \\ \hline
楊虛 & 齊悼惠王子,侯 & 十二。四年五月甲寅,恭侯劉將廬元年。。十六年,侯將廬為齊王,有罪,國除。 &  &  \\ \hline
朸 & 齊悼惠王子,侯。 & 十二。四年五月甲寅,侯劉辟光元年。十六年,侯辟光為濟南王,國除。 &  &  \\ \hline
安都 & 齊悼惠王子,侯。 & 十二。四年五月甲寅,侯劉志元年。十六年,侯志為濟北王,國除。 &  &  \\ \hline
平昌 & 齊悼惠王子,侯。 & 十二。四年五月甲寅,侯劉卬元年。十六年,侯卬為膠西王,國除。 &  &  \\ \hline
武城 & 齊悼惠王子,侯。 & 十二。四年五月甲寅,侯劉賢元年。十六年,侯賢為菑川王,國除。 &  &  \\ \hline
白石 & 齊悼惠王子,侯。 & 十二。四年五月甲寅,侯劉雄渠元年。十六年,侯雄渠為膠東王,國除。 &  &  \\ \hline
波陵 & 以陽陵君侯。 & 五。七年三月甲寅,康侯魏駟元年。十二年,康侯魏駟薨,無後,國除。 &  &  \\ \hline
南𨜓 & 以信平君侯。 & 一。七年三月丙寅,侯起元年。。孝文時坐後父故奪爵級,關內侯。 &  &  \\ \hline
阜陵 & 以淮南厲王子侯。 & 八。八年五月丙午,侯劉安元年。十六年,安為淮南王,國除。 &  &  \\ \hline
安陽 & 以淮南厲王子侯。 & 八。八年五月丙午,侯勃元年。十六年,侯勃為衡山王,國除。 &  &  \\ \hline
陽周 & 以淮南厲王子侯。 & 八。八年五月丙午,侯劉賜元年。十六年,侯賜為廬江王,國除。 &  &  \\ \hline
東城 & 以淮南厲王子侯。 & 七。八年五月丙午,哀侯劉良元年。十五年,侯良薨,無後,國除。 &  &  \\ \hline
犁 & 以齊相召平子侯,千四百一十戶。 & 十一。十年四月癸丑,頃侯召奴元年。三。後五年,侯澤元年。 & 十六 & 十六。元朔五年,侯延元年。十九。元封六年,侯延坐不出持馬,斬,國除。 \\ \hline
缾 & 以北地都尉孫卬,匈奴入北地,力戰死事,子侯。 & 十。十四年三月丁巳,侯孫單元年。 & 二。前三年,侯單謀反,國除。 &  \\ \hline
弓高 & 以匈奴相國降,故韓王信孽子,侯,千二百三十七戶。 & 八。十六年六月丙子,莊侯韓頹當元年。 & 十六。前元年,侯則元年。 & 十六。元朔五年,侯則薨,無後,國除。 \\ \hline
襄成 & 以匈奴相國降侯,故韓王信太子之子,侯千四百三十二戶。 & 七。十六年六月丙子,哀侯韓嬰元年。一。後七年,侯澤之元年。 & 十六 & 十五。元朔四年,侯澤之坐詐病不從,不敬,國除。 \\ \hline
故安 & 孝文元年,舉淮陽守從高祖入漢功侯,食邑五百戶;用丞相侯,一千七百一十二戶。 & 五。後三年四月丁巳,節侯申屠嘉元年。 & 二。十四。前三年,恭侯蔑元年。 & 十九。元狩二年,清安侯臾元年。五。元鼎元年,臾坐為九江太守有罪,國除。 \\ \hline
章武 & 以孝文后弟侯,萬一千八百六十九戶。 & 一。後七年六月乙卯,景侯竇廣國元年。 & 六。十。前七年,恭侯完元年。 & 八。元光三年,侯常坐元年。十。元狩元年,侯常坐謀殺人未殺罪,國除。 \\ \hline
南皮 & 以孝文后兄竇長君子侯,六千四百六十戶。 & 一。後七年六月乙卯,侯竇彭祖元年。 & 十六 & 五。建元六年,夷侯良元年。五。元光五年,侯桑林元年。十八。元鼎五年,侯桑林坐酎金罪,國除。 \\ \hline
}

右孝文時二十九。

\biao{|p{2em}|p{12em}|p{12em}|p{12em}|p{12em}|}{
\hline
平陸 & 楚元王子,侯,三千二百六十七戶。 & 二。元年四月乙巳,。侯劉禮元年。三年,侯禮為楚王,國除。 &  &  \\ \hline
休 & 楚元王子,侯。 & 二。元年四月乙巳,侯富元年。三年,侯富以兄子戎為楚王反,富與家屬至長安北闕自歸,不能相教,上印綬。詔復王。後以平陸侯為楚王,更封富為紅侯。 &  &  \\ \hline
沈猶 & 楚元王子,侯,千三百八十戶。 & 十六。元年四月乙巳,夷侯劉穢元年。 & 四。建元五年,侯受元年。十八。元狩五年,侯受坐故為宗正聽謁不具宗室,不敬,國除。 &  \\ \hline
紅 & 楚元王子,侯,千七百五十戶。 & 四。三年四月乙巳,莊侯富元年。。一。前七年,悼侯澄元年。九。中元年,敬侯發元年。 & 十五。元朔四年,侯章元年。一。元朔五年,侯章薨,無後,國除。 &  \\ \hline
宛朐 & 楚元王子,侯。 & 二。元年四月乙巳,侯劉埶元年。。三年,侯埶反,國除。 &  &  \\ \hline
魏其 & 以大將軍屯滎陽,扞吳楚七國,侯,三千三百五十戶。 & 十四。三年六月乙巳,侯竇嬰元年。 & 九。建元元年為丞相,二歲免。元光四年,侯嬰坐爭灌夫事上書稱為先帝詔,矯制害,棄市,國除。 &  \\ \hline
棘樂 & 楚元王子,侯,戶千二百一十三。 & 十四。三年八月壬子,敬侯劉調元年。 & 一。建元二年,恭侯應元年。十一。元朔元年,侯慶元年。十六。元鼎五年,侯慶坐酎金,國除。 &  \\ \hline
俞 & 以將軍吳楚反擊齊有功。布故彭越舍人,越反時布使齊,還已梟越,布祭哭之,當亨,出忠言,高祖舍之。黥布反,布為都尉,侯,戶千八百。 & 六。六年四月丁卯,侯欒布元年。中五年,侯布薨。 & 十。元狩六年,侯賁坐為太常廟犧牲不如令,有罪,國除。 &  \\ \hline
建陵 & 以將軍擊吳楚功,用中尉侯,戶一千三百一十。 & 十一。六年四月丁卯,敬侯衛綰元年。 & 十。元光五年,侯信元年。十八。元鼎五年,侯信坐酎金,國除。 &  \\ \hline
建平 & 以將軍擊吳楚功,用江都相侯,戶三千一百五十。 & 十一。六年四月丁卯,哀侯程嘉元年。 & 七。元光二年,節侯橫元年。一。元光三年,侯回元年。一。元光四年,侯回薨,無後,國除。 &  \\ \hline
平曲 & 以將軍擊吳楚功,用隴西太守侯,戶三千二百二十。 & 五。六年四月己巳,侯公孫昆。。邪元年。中四年,侯昆邪有罪,國除。太僕賀父。 &  &  \\ \hline
江陽 & 以將軍擊吳楚功,用趙相侯,戶二千五百四十一。 & 四。六年四月壬申,康侯蘇嘉元年。。七。中三年,懿侯盧元年。 & 二。建元三年,侯明元年。十六。元朔六年,侯雕元年。十一。元鼎五年,侯雕坐酎金,國除。 &  \\ \hline
遽 & 以趙相建德,王遂反,建德不聽,死事,子侯,戶千九百七十。 & 六。中二年四月乙巳,侯橫。。元年。後二年,侯橫有罪,國除。 &  &  \\ \hline
新市 & 以趙內史王慎,王遂反,慎不聽,死事,子侯,戶一千十四。 & 五。中二年四月乙巳,侯王康元年。三。後元年,殤侯始昌元年。 & 九。元光四年,殤侯始昌為人所殺,國除。 &  \\ \hline
商陵 & 以楚太傅趙夷吾,王戊反,不聽,死事,子侯,千四十五戶。 & 八。中二年四月乙巳,侯趙周元年。 & 二十九。元鼎五年,侯周坐為丞相知列侯酎金輕,下廷尉,自殺,國除。 &  \\ \hline
山陽 & 以楚相張尚,王戊反,尚不聽,死事,子侯,戶千一百一十四。 & 八。中二年四月乙巳,侯張當居元年。 & 十六。元朔五年,侯當居坐為太常程博士弟子故不以實罪,國除。 &  \\ \hline
安陵 & 以匈奴王降侯,戶一千五百十七。 & 七。中三年十一月庚子,侯子軍元年。 & 五。建元六年,侯子軍薨,無後,國除。 &  \\ \hline
垣 & 以匈奴王降侯。 & 三。中三年十二月丁丑,侯賜元年。六年,賜死,不得及嗣。 &  &  \\ \hline
遒 & 以匈奴王降侯,戶五千五百六十九。 & 中三年十二月丁丑,侯隆彊。元年。不得隆彊嗣。 &  & 後元年四月甲辰,侯則坐使巫齊少君祠祝詛,大道無道,國除。 \\ \hline
容成 & 以匈奴王降侯,七百戶。 & 七。中三年十二月丁丑,侯唯徐盧。元年。 & 十四。建元元年,康侯綽元年。二十二。元朔三年,侯光元年。 & 十八。後二年,三月壬辰,侯光坐祠祝詛,國除。 \\ \hline
易 & 以匈奴王降侯。 & 六。中三年十二月丁丑,侯僕黥元年。後二年,侯僕黥薨,無嗣。 &  &  \\ \hline
范陽 & 以匈奴王降侯,戶千一百九十七。 & 七。中三年十二月丁丑,端侯代。。元年。 & 七。元光二年,懷侯德元年。二。元光四年,侯德薨,無後,國除。 &  \\ \hline
翕 & 以匈奴王降侯。 & 七。中三年十二月丁丑,侯邯鄲元年。 & 九。元光四年,侯邯鄲坐行來不請長信,不敬,國除。 &  \\ \hline
亞谷 & 以匈奴東胡王降,故燕王盧綰子侯,千五百戶。 & 二。中五年四月丁巳,簡侯它父。元年。三。後元年,安侯種元年。 & 十一。建元元年,康侯偏元年。二十五。元光六年,侯賀元年。 & 十五。征和二年七月辛巳,侯賀坐太子事,國除。 \\ \hline
隆慮 & 以長公主嫖子侯,戶四千一百二十六。 & 五。中五年五月丁丑,侯蟜元年。 & 二十四。元鼎元年,侯蟜坐母長公主薨未除服,姦,禽獸行,當死,自殺,國除。 &  \\ \hline
乘氏 & 以梁孝王子侯。 & 中五年五月丁卯,侯買元年。中六年,侯買嗣為梁王,國除。 &  &  \\ \hline
桓邑 & 以梁孝王子侯。 & 一。中五年五月丁卯,侯明元年。中六年,為濟川王,國除。 &  &  \\ \hline
蓋 & 以孝景后兄侯,戶二千八百九十。 & 五。中五年五月甲戌,靖侯王信元年。 & 二十。元狩三年,侯偃元年。八。元鼎五年,侯偃坐酎金,國除。 &  \\ \hline
塞 & 以御史大夫前將兵擊吳楚功侯,戶千四十六。 & 三。後元年八月,侯直不疑元年。 & 三。建元四年,侯相如元年。十二。元朔四年,侯堅元年。十三。元鼎五年,堅坐酎金,國除。 &  \\ \hline
武安 & 以孝景后同母弟侯,戶八千二百一十四。 & 一。後三年三月,侯田蚡元年。 & 九。元光四年,侯梧元年。五。元朔三年,侯梧坐衣襜褕入宮廷中,不敬,國除。 &  \\ \hline
周陽 & 以孝景后同母弟侯,戶六千二十六。 & 一。後三年三月,懿侯田勝元年。 & 十一。元光六年,侯彭祖元年。八。元狩二年,侯彭祖坐當歸與章侯宅不與罪,國除。 &  \\ \hline
}

右孝景時三十。