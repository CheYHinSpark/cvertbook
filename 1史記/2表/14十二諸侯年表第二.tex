\chapter{十二諸侯年表第二}

太史公讀春秋歷譜諜,至周厲王,未嘗不廢書而嘆也。曰,鳴呼,師摯見之矣。紂為象箸而箕子唏。周道缺,詩人本之衽席,關雎作。仁義陵遲,鹿鳴刺焉。及至厲王,以惡聞其過,公卿懼誅而禍作,厲王遂奔于彘,亂自京師始,而共和行政焉。是後或力政,彊乘弱,興師不請天子。然挾王室之義,以討伐為會盟主,政由五伯,諸侯恣行,淫侈不軌,賊臣篡子滋起矣。齊、晉、秦、楚其在成周微甚,封或百里或五十里。晉阻三河,齊負東海,楚介江淮,秦因雍州之固,四海迭興,更為伯主,文武所褒大封,皆威而服焉。是以孔子明王道,干七十餘君,莫能用,故西觀周室,論史記舊聞,興於魯而次春秋,上記隱,下至哀之獲麟,約其辭文,去其煩重,以制義法,王道備,人事浹。七十子之徒口受其傳指,為有所刺譏褒諱挹損之文辭不可以書見也。魯君子左丘明懼弟子人人異端,各安其意,失其真,故因孔子史記具論其語,成左氏春秋。鐸椒為楚威王傳,為王不能盡觀春秋,采取成敗,卒四十章,為鐸氏微。趙孝成王時,其相虞卿上采春秋,下觀近勢,亦著八篇,為虞氏春秋。呂不韋者,秦莊襄王相,亦上觀尚古,刪拾春秋,集六國時事,以為八覽、六論、十二紀,為呂氏春秋。及如荀卿、孟子、公孫固、韓非之徒,各往往捃摭春秋之文以著書,不同勝紀。漢相張蒼歷譜五德,上大夫董仲舒推春秋義,頗著文焉。

\biao{|p{2em}|p{5em}|p{5em}|p{5em}|p{4em}|p{4em}|p{4em}|p{4em}|p{4em}|p{4em}|p{4em}|p{4em}|p{4em}|p{4em}|p{4em}|}
{\hline
年&周&魯&齊&晉&秦&楚&宋&衛&陳&蔡&曹&鄭&燕&吳\\\hline
庚申&共和元年厲王子居召公宮,是為宣王。王少,大臣共和行政。&真公濞十五年,一云十四年&武公壽十年&靖侯宜臼十八年&秦仲四年&熊勇七年&釐公十八年&釐侯十四年&幽公寧十四年&武侯二十三年&夷伯二十四年&&惠侯二十四年&\\\hline
&二&十六&十一&晉釐侯司徒元年&五&八&十九&十五&十五&二十四&二十五&&二十五&\\\hline
&三&十七&十二&二&六&九&二十&十六&十六&二十五&二十六&&二十六&\\\hline
&四&十八&十三&三&七&十&二十一&十七&十七&二十六&二十七&&二十七&\\\hline
甲子&五&十九&十四&四&八&楚熊嚴元年&二十二&十八&十八&蔡夷侯元年&二十八&&二十八&\\\hline
&六&二十&十五&五&九&二&二十三&十九&十九&二&二十九&&二十九&\\\hline
&七&二十一&十六&六&十&三&二十四&二十&二十&三&三十&&三十&\\\hline
&八&二十二&十七&七&十一&四&二十五&二十一&二十一&四&曹幽伯彊元年&&三十一&\\\hline
&九&二十三&十八&八&十二&五&二十六&二十二&二十二&五&二&&三十二&\\\hline
&十&二十四&十九&九&十三&六&二十七&二十三&二十三&六&三&&三十三&\\\hline
&十一&二十五&二十&十&十四&七&二十八&二十四&陳釐公孝元年&七&四&&三十四&\\\hline
&十二&二十六&二十一&十一&十五&八&宋惠公鲒元年&二十五&二&八&五&&三十五&\\\hline
&十三&二十七&二十二&十二&十六&九&二&二十六&三&九&六&&三十六&\\\hline
&十四宣王即位,共和罷。&二十八&二十三&十三&十七&十&三&二十七&四&十&七&&三十七&\\\hline
甲戌&宣王元年&二十九&二十四&十四&十八&楚熊霜元年&四&二十八&五&十一&八&&三十八&\\\hline
&二&三十&二十五&十五&十九&二&五&二十九&六&十二&九&&燕釐侯莊元年&\\\hline
&三&魯武公敖元年&二十六&十六&二十&三&六&三十&七&十三&曹戴伯鮮元年&&二&\\\hline
&四&二&齊厲公無忌元年&十七&二十一&四&七&三十一&八&十四&二&&三&\\\hline
&五&三&二&十八&二十二&五&八&三十二&九&十五&三&&四&\\\hline
&六&四&三&晉獻侯籍元年&二十三&六&九&三十三&十&十六&四&&五&\\\hline
&七&五&四&二&秦莊公其元年&楚熊徇元年&十&三十四&十一&十七&五&&六&\\\hline
&八&六&五&三&二&二&十一&三十五&十二&十八&六&&七&\\\hline
&九&七&六&四&三&三&十二&三十六&十三&十九&七&&八&\\\hline
&十&八&七&五&四&四&十三&三十七&十四&二十&八&&九&\\\hline
甲申&十一&九&八&六&五&五&十四&三十八&十五&二十一&九&&十&\\\hline
&十二&十&九&七&六&六&十五&三十九&十六&二十二&十&&十一&\\\hline
&十三&魯懿公戲元年&齊文公赤元年&八&七&七&十六&四十&十七&二十三&十一&&十二&\\\hline
&十四&二&二&九&八&八&十七&四十一&十八&二十四&十二&&十三&\\\hline
&十五&三&三&十&九&九&十八&四十二&十九&二十五&十三&&十四&\\\hline
&十六&四&四&十一&十&十&十九&衛武公和元年&二十&二十六&十四&&十五&\\\hline
&十七&五&五&穆侯弗生元年&十一&十一&二十&二&二十一&二十七&十五&&十六&\\\hline
&十八&六&六&二&十二&十二&二十一&三&二十二&二十八&十六&&十七&\\\hline
&十九&七&七&三&十三&十三&二十二&四&二十三&蔡釐侯所事元年&十七&&十八&\\\hline
&二十&八&八&四取齊女為夫人。&十四&十四&二十三&五&二十四&二&十八&&十九&\\\hline
甲午&二十一&九&九&五&十五&十五&二十四&六&二十五&三&十九&&二十&\\\hline
&二十二&魯孝公稱元年,伯御立為君,稱為諸公子云。伯御,武公孫。&十&六&十六&十六&二十五&七&二十六&四&二十&鄭桓公友元年始封。周宣王母弟。&二十一&\\\hline
&二十三&二&十一&七以伐條生太子仇。&十七&十七&二十六&八&二十七&五&二十一&二&二十二&\\\hline
&二十四&三&十二&八&十八&十八&二十七&九&二十八&六&二十二&三&二十三&\\\hline
&二十五&四&齊成公說元年&九&十九&十九&二十八&十&二十九&七&二十三&四&二十四&\\\hline
&二十六&五&二&十以千畝戰。生仇弟成師。二子名反,君子譏之。後亂。&二十&二十&二十九&十一&三十&八&二十四&五&二十五&\\\hline
&二十七&六&三&十一&二十一&二十一&三十&十二&三十一&九&二十五&六&二十六&\\\hline
&二十八&七&四&十二&二十二&二十二&三十一宋惠公薨。&十三&三十二&十&二十六&七&二十七&\\\hline
&二十九&八&五&十三&二十三&楚熊鄂元年&宋戴公立。元年&十四&三十三&十一&二十七&八&二十八&\\\hline
&三十&九&六&十四&二十四&二&二&十五&三十四&十二&二十八&九&二十九&\\\hline
甲辰&三十一&十&七&十五&二十五&三&三&十六&三十五&十三&二十九&十&三十&\\\hline
&三十二&十一周宣王誅伯御,立其弟稱,是為孝公。&八&十六&二十六&四&四&十七&三十六&十四&三十&十一&三十一&\\\hline
&三十三&十二&九&十七&二十七&五&五&十八&陳武公靈元年&十五&曹惠伯雉元年&十二&三十二&\\\hline
&三十四&十三&齊莊公贖&十八&二十八&六&六&十九&二&十六&二&十三&三十三&\\\hline
&&&元年&&&&&&&&&&&\\\hline
&三十五&十四&二&十九&二十九&七&七&二十&三&十七&三&十四&三十四&\\\hline
&三十六&十五&三&二十&三十&八&八&二十一&四&十八&四&十五&三十五&\\\hline
&三十七&十六&四&二十一&三十一&九&九&二十二&五&十九&五&十六&三十六&\\\hline
&三十八&十七&五&二十二&三十二&楚若敖元年&十&二十三&六&二十&六&十七&燕頃侯元年&\\\hline
&三十九&十八&六&二十三&三十三&二&十一&二十四&七&二十一&七&十八&二&\\\hline
&四十&十九&七&二十四&三十四&三&十二&二十五&八&二十二&八&十九&三&\\\hline
甲寅&四十一&二十&八&二十五&三十五&四&十三&二十六&九&二十三&九&二十&四&\\\hline
&四十二&二十一&九&二十六&三十六&五&十四&二十七&十&二十四&十&二十一&五&\\\hline
&四十三&二十二&十&二十七穆侯卒,弟殤叔自立,太子仇出奔。&三十七&六&十五&二十八&十一&二十五&十一&二十二&六&\\\hline
&四十四&二十三&十一&晉殤叔元年&三十八&七&十六&二十九&十二&二十六&十二&二十三&七&\\\hline
&四十五&二十四&十二&二&三十九&八&十七&三十&十三&二十七&十三&二十四&八&\\\hline
&四十六&二十五&十三&三&四十&九&十八&三十一&十四&二十八&十四&二十五&九&\\\hline
&幽王元年&二十六&十四&四仇攻殺殤叔,立為文侯。&四十一&十&十九&三十二&十五&二十九&十五&二十六&十&\\\hline
&二三川震。&二十七&十五&晉文侯仇元年&四十二&十一&二十&三十三&陳夷公說元年&三十&十六&二十七&十一&\\\hline
&三王取褒姒。&二十八&十六&二&四十三&十二&二十一&三十四&二&三十一&十七&二十八&十二&\\\hline
&四&二十九&十七&三&四十四&十三&二十二&三十五&三&三十二&十八&二十九&十三&\\\hline
甲子&五&三十&十八&四&秦襄公元年&十四&二十三&三十六&陳平公燮元年&三十三&十九&三十&十四&\\\hline
&六&三十一&十九&五&二&十五&二十四&三十七&二&三十四&二十&三十一&十五&\\\hline
&七&三十二&二十&六&三&十六&二十五&三十八&三&三十五&二十一&三十二&十六&\\\hline
&八&三十三&二十一&七&四&十七&二十六&三十九&四&三十六&二十二&三十三&十七&\\\hline
&九&三十四&二十二&八&五&十八&二十七&四十&五&三十七&二十三&三十四&十八&\\\hline
&十&三十五&二十三&九&六&十九&二十八&四十一&六&三十八&二十四&三十五&十九&\\\hline
&十一幽王為犬戎所殺。&三十六&二十四&十&七始列為諸侯。&二十&二十九&四十二&七&三十九&二十五&三十六以幽王故,為犬戎所殺。&二十&\\\hline
&平王元年東徙雒邑。&三十七&二十五&十一&八初立西畤,祠白帝。&二十一&三十&四十三&八&四十&二十六&鄭武公滑突元年&二十一&\\\hline
&二&三十八&二十六&十二&九&二十二&三十一&四十四&九&四十一&二十七&二&二十二&\\\hline
&三&魯惠公弗湦&二十七&十三&十&二十三&三十二&四十五&十&四十二&二十八&三&二十三&\\\hline
&&元年&&&&&&&&&&&&\\\hline
甲戌&四&二&二十八&十四&十一&二十四&三十三&四十六&十一&四十三&二十九&四&二十四&\\\hline
&五&三&二十九&十五&十二伐戎至岐而死。&二十五&三十四&四十七&十二&四十四&三十&五&燕哀侯元年&\\\hline
&六&四&三十&十六&秦文公元年&二十六&宋武公司空元年&四十八&十三&四十五&三十一&六&二&\\\hline
&七&五&三十一&十七&二&二十七&二&四十九&十四&四十六&三十二&七&燕鄭侯元年&\\\hline
&八&六&三十二&十八&三&楚霄敖元年&三&五十&十五&四十七&三十三&八&二&\\\hline
&九&七&三十三&十九&四&二&四&五十一&十六&四十八&三十四&九&三&\\\hline
&十&八&三十四&二十&五&三&五&五十二&十七&蔡共侯興元年&三十五&十娶申侯女武姜。&四&\\\hline
&十一&九&三十五&二十一&六&四&六&五十三&十八&二&三十六&十一&五&\\\hline
&十二&十&三十六&二十二&七&五&七&五十四&十九&蔡戴侯元年&曹穆公元年&十二&六&\\\hline
&十三&十一&三十七&二十三&八&六&八&五十五&二十&二&二&十三&七&\\\hline
甲申&十四&十二&三十八&二十四&九&楚蚡冒元年&九&衛莊公楊元年&二十一&三&三&十四生莊公寤生。&八&\\\hline
&十五&十三&三十九&二十五&十作鄜畤。&二&十&二&二十二&四&曹桓公終生元年&十五&九&\\\hline
&十六&十四&四十&二十六&十一&三&十一&三&二十三&五&二&十六&十&\\\hline
&十七&十五&四十一&二十七&十二&四&十二&四&陳文公圉元年生桓公鮑、厲公他。他母蔡女。&六&三&十七生大叔段,母欲立段,公不聽。&十一&\\\hline
&十八&十六&四十二&二十八&十三&五&十三&五&二&七&四&十八&十二&\\\hline
&十九&十七&四十三&二十九&十四&六&十四&六&三&八&五&十九&十三&\\\hline
&二十&十八&四十四&三十&十五&七&十五&七&四&九&六&二十&十四&\\\hline
&二十一&十九&四十五&三十一&十六&八&十六&八&五&十&七&二十一&十五&\\\hline
&二十二&二十&四十六&三十二&十七&九&十七&九&六&蔡宣侯楷論元年&八&二十二&十六&\\\hline
&二十三&二十一&四十七&三十三&十八&十&十八生魯桓公母。&十&七&二&九&二十三&十七&\\\hline
甲午&二十四&二十二&四十八&三十四&十九作祠陳寶。&十一&宋宣公力元年&十一&八&三&十&二十四&十八&\\\hline
&二十五&二十三&四十九&三十五&二十&十二&二&十二&九&四&十一&二十五&十九&\\\hline
&二十六&二十四&五十&晉昭侯元年封季父成師于曲沃,曲沃大於國,君子譏曰,「晉人亂自曲沃始矣。」&二十一&十三&三&十三&十文公卒。&五&十二&二十六&二十&\\\hline
&二十七&二十五&五十一&二&二十二&十四&四&十四&陳桓公元年&六&十三&二十七&二十一&\\\hline
&二十八&二十六&五十二&三&二十三&十五&五&十五&二&七&十四&鄭莊公寤生元年祭仲相。&二十二&\\\hline
&二十九&二十七&五十三&四&二十四&十六&六&十六&三&八&十五&二&二十三&\\\hline
&三十&二十八&五十四&五&二十五&十七&七&十七愛妾子州吁,州吁好兵。&四&九&十六&三&二十四&\\\hline
&三十一&二十九&五十五&六&二十六&武王立。&八&十八&五&十&十七&四&二十五&\\\hline
&三十二&三十&五十六&潘父殺昭侯,納成師,不克。昭侯子立,是為孝侯。&二十七&二&九&十九&六&十一&十八&五&二十六&\\\hline
&三十三&三十一&五十七&二&二十八&三&十&二十&七&十二&十九&六&二十七&\\\hline
甲辰&三十四&三十二&五十八&三&二十九&四&十一&二十一&八&十三&二十&七&二十八&\\\hline
&三十五&三十三&五十九&四&三十&五&十二&二十二&九&十四&二十一&八&二十九&\\\hline
&三十六&三十四&六十&五&三十一&六&十三&二十三夫人無子,桓公立。&十&十五&二十二&九&三十&\\\hline
&三十七&三十五&六十一&六&三十二&七&十四&衛桓公完元年&十一&十六&二十三&十&三十一&\\\hline
&三十八&三十六&六十二&七&三十三&八&十五&二弟州吁驕,桓黜之,出奔。&十二&十七&二十四&十一&三十二&\\\hline
&三十九&三十七&六十三&八&三十四&九&十六&三&十三&十八&二十五&十二&三十三&\\\hline
&四十&三十八&六十四&九曲沃桓叔成師卒,子代立,為莊伯。&三十五&十&十七&四&十四&十九&二十六&十三&三十四&\\\hline
&四十一&三十九&齊釐公祿父元年&十&三十六&十一&十八&五&十五&二十&二十七&十四&三十五&\\\hline
&四十二&四十&二同母弟夷仲年生公孫毋知也。&十一&三十七&十二&十九公卒,命立弟和,為穆公。&六&十六&二十一&二十八&十五&三十六&\\\hline
&四十三&四十一&三&十二&三十八&十三&宋穆公和元年&七&十七&二十二&二十九&十六&燕穆侯元年&\\\hline
甲寅&四十四&四十二&四&十三&三十九&十四&二&八&十八&二十三&三十&十七&二&\\\hline
&四十五&四十三&五&十四&四十&十五&三&九&十九&二十四&三十一&十八&三&\\\hline
&四十六&四十四&六&十五&四十一&十六&四&十&二十&二十五&三十二&十九&四&\\\hline
&四十七&四十五&七&十六曲沃莊伯殺孝侯,晉人立孝侯子卻為鄂侯。&四十二&十七&五&十一&二十一&二十六&三十三&二十&五&\\\hline
&四十八&四十六&八&晉鄂侯卻元年曲沃強於晉。&四十三&十八&六&十二&二十二&二十七&三十四&二十一&六&\\\hline
&四十九&魯隱公息姑元年母聲子。&九&二&四十四&十九&七&十三&二十三&二十八&三十五&二十二段作亂,奔。&七&\\\hline
&五十&二&十&三&四十五&二十&八&十四&二十四&二十九&三十六&二十三公悔,思母不見,穿地相見。&八&\\\hline
&五十一&三二月,日蝕。&十一&四&四十六&二十一&九公屬孔父立殤公。馮奔鄭。&十五&二十五&三十&三十七&二十四侵周,取禾。&九&\\\hline
&桓王元年&四&十二&五&四十七&二十二&宋殤公與夷元年&十六州吁弒公自立。&二十六衛石碏來告,故執州吁。&三十一&三十八&二十五&十&\\\hline
&二使虢公伐晉之曲沃。&五公觀魚于棠,君子譏之。&十三&六鄂侯卒。曲沃莊伯復攻晉。立鄂侯子光為哀侯。&四十八&二十三&二鄭伐我。我伐鄭。&衛宣公晉元年共立之。討州吁。&二十七&三十二&三十九&二十六&十一&\\\hline
甲子&三&六鄭人來渝平。&十四&晉哀侯光元年&四十九&二十四&三&二&二十八&三十三&四十&二十七始朝王,王不禮。&十二&\\\hline
&四&七&十五&二莊伯卒,子稱立,為武公。&五十&二十五&四&三&二十九&三十四&四十一&二十八&十三&\\\hline
&五&八易許田,君子譏之。&十六&三&秦寧公元年&二十六&五&四&三十&三十五&四十二&二十九與魯祊,易許田。&十四&\\\hline
&六&九三月,大雨雹,電。&十七&四&二&二十七&六&五&三十一&蔡桓侯封人元年&四十三&三十&十五&\\\hline
&七&十&十八&五&三&二十八&七諸侯敗我。我師與衛人伐鄭。&六&三十二&二&四十四&三十一&十六&\\\hline
&八&十一大夫翬請殺桓公,求為相,公不聽,即殺公。&十九&六&四&二十九&八&七&三十三&三&四十五&三十二&十七&\\\hline
&九&魯桓公允元年母宋武公女,生手文為魯夫人。&二十&七&五&三十&九&八&三十四&四&四十六&三十三以璧加魯,易許田。&十八&\\\hline
&十&二宋賂以鼎,入於太廟,君子譏之。&二十一&八&六&三十一&華督見孔父妻好,悅之。華督殺孔父,及殺殤公。宋公馮元年華督為相。&九&三十五&五&四十七&三十四&燕宣侯元年&\\\hline
&十一&三翬迎女,齊侯送女,君子譏之。&二十二&晉小子元年&七&三十二&二&十&三十六&六&四十八&三十五&&\\\hline
&十二&四&二十三&二&八&三十三&三&十一&三十七&七&四十九&三十六&三&\\\hline
甲戌&十三伐鄭。&五&二十四&三&九&三十四&四&十二&三十八弟他殺太子免。代立,國亂,再赴。&八&五十&三十七伐周,傷王。&四&\\\hline
&十四&六&二十五山戎伐我。&曲沃武公殺小子。周伐曲沃,立晉哀侯弟湣為晉侯。晉侯湣元年&十&三十五侵隨,隨為善政,得止。&五&十三&陳厲公他元年&九&五十一&三十八太子忽救齊,齊將妻之。&五&\\\hline
&十五&七&二十六&二&十一&三十六&六&十四&二生敬仲完。周史卜完後世王齊。&十&五十二&三十九&六&\\\hline
&十六&八&二十七&三&十二&三十七伐隨,弗拔,但盟,罷兵。&七&十五&三&十一&五十三&四十&七&\\\hline
&十七&九&二十八&四&秦出子元年&三十八&八&十六&四&十二&五十四&四十一&八&\\\hline
&十八&十&二十九&五&二&三十九&九&十七&五&十三&五十五&四十二&九&\\\hline
&十九&十一&三十&六&三&四十&十執祭仲。&十八太子伋弟壽爭死。&六&十四&曹莊公射姑元年&四十三&十&\\\hline
&二十&十二&三十一&七&四&四十一&十一&十九&七公淫蔡,蔡殺公。&十五&二&鄭厲公突元年&十一&\\\hline
&二十一&十三&三十二釐公令毋知秩服如太子。&八&五&四十二&十二&衛惠公朔元年&陳莊公林元年桓公子。&十六&三&二&十二&\\\hline
&二十二&十四&三十三&九&六三父殺出子,立其兄武公。&四十三&十三&二&二&十七&四&三諸侯伐我,報宋故。&十三&\\\hline
甲申&二十三&十五天王求車,非禮。&齊襄公諸兒元年貶毋知秩服,毋知怨。&十&秦武公元年伐彭,至華山。&四十四&十四&三朔奔齊,立黔牟。&三&十八&五&四祭仲立忽,公出居櫟。&燕桓侯元年&\\\hline
&莊王元年生子穨。&十六公會曹,謀伐鄭。&二&十一&二&四十五&十五&衛黔牟元年&四&十九&六&鄭昭公忽元年忽母鄧女,祭仲取之。&二&\\\hline
&二有弟克。&十七日食,不書日,官失之。&三&十二&三&四十六&十六&二&五&二十&七&二渠彌殺昭公。&三&\\\hline
&三&十八公與夫人如齊,齊侯通焉,使彭生殺公於車上。&四殺魯桓公,誅彭生。&十三&四&四十七&十七&三&六&蔡哀侯獻舞元年&八&鄭子亹元年齊殺子亹,昭公弟。&四&\\\hline
&四周公欲殺王而立子克,王誅周公,克奔燕。&魯莊公同元年&五&十四&五&四十八&十八&四&七&二&九&鄭子嬰元年子亹之弟。&五&\\\hline
&五&二&六&十五&六&四十九&十九&五&陳宣公杵臼元年杵臼,莊公弟。&三&十&二&六&\\\hline
&六&三&七&十六&七&五十&宋湣公捷元年&六&二&四&十一&三&七&\\\hline
&七&四&八伐紀,去其都邑。&十七&八&五十一王伐,隨告夫人心動,王卒軍中。&二&七&三&五&十二&四&燕莊公元年&\\\hline
&八&五與齊伐衛,納惠公。&九&十八&九&楚文王貲元年始都郢。&三&八&四&六&十三&五&二&\\\hline
&九&六&十&十九&十&二伐申,過鄧,鄧甥曰楚可取,鄧侯不許。&四&九&五&七&十四&六&三&\\\hline
甲午&十&七星隕如雨,與雨偕。&十一&二十&十一&三&五&十齊立惠公,黔牟奔周。&六&八&十五&七&四&\\\hline
&十一&八子糾來奔,與管仲俱避毋知亂。&十二毋知殺君自立。&二十一&十二&四&六&衛惠公朔復入。十四年&七&九&十六&八&五&\\\hline
&十二&九魯欲與糾入,後小白,齊距魯,使生致管仲。&齊桓公小白元年春,齊殺毋知。&二十二&十三&五&七&十五&八&十&十七&九&六&\\\hline
&十三&十齊伐我,為糾故。&二&二十三&十四&六息夫人,陳女,過蔡,蔡不禮,惡之。楚伐蔡,獲哀侯以歸。&八&十六&九&十一楚虜我侯。&十八&十&七&\\\hline
&十四&十一臧文仲弔宋水。&三&二十四&十五&七&九宋大水,公自罪。魯使臧文仲來弔。&十七&十&十二&十九&十一&八&\\\hline
&十五&十二&四&二十五&十六&八&十萬殺君,仇牧有義。&十八&十一&十三&二十&十二&九&\\\hline
&釐王元年&十三曹沫劫桓公。反所亡地。&五與魯人會柯。&二十六&十七&九&宋桓公御說元年莊公子。&十九&十二&十四&二十一&十三&十&\\\hline
&二&十四&六&二十七&十八&十&二&二十&十三&十五&二十二&十四&十一&\\\hline
&三&十五&七始霸,會諸侯于鄄。&二十八曲沃武公滅晉侯湣,以寶獻周,周命武公為晉君,并其地。&十九&十一&三&二十一&十四&十六&二十三&鄭厲公元年厲公亡後十七歲復入。&十二&\\\hline
&四&十六&八&晉武公稱并晉,已立三十八年,不更元,因其元年。&二十葬雍,初以人從死。&十二伐鄧滅之。&四&二十二&十五&十七&二十四&二諸侯伐我。&十三&\\\hline
甲辰&五&十七&九&三十九武公卒,子詭諸立,為獻公。&秦德公元年武公弟。&十三&五&二十三&十六&十八&二十五&三&十四&\\\hline
&惠王元年取陳后。&十八&十&晉獻公詭諸元年&二初作伏,祠社,磔狗邑四門。&楚堵敖谡元年&六&二十四&十七&十九&二十六&四&十五&\\\hline
&二燕、衛伐王,王奔溫,立子穨。&十九&十一&二&秦宣公元年&二&七取衛女。文公弟。&二十五&十八&二十&二十七&五&十六伐王,王奔溫,立子穨。&\\\hline
&三&二十&十二&三&二&三&八&二十六&十九&蔡穆侯肸元年&二十八&六&十七鄭執我仲父。&\\\hline
&四誅穨,入惠王。&二十一&十三&四&三&四&九&二十七&二十&二&二十九&七救周亂,入王。&十八&\\\hline
&五太子母早死。惠后生叔帶。&二十二&十四陳完自陳來奔,田常始此也。&五伐驪戎,得姬。&四作密畤。&五弟惲殺堵敖自立。&十&二十八&二十一厲公子完奔齊。&三&三十&鄭文公捷元年&十九&\\\hline
&六&二十三公如齊觀社。&十五&六&五&楚成王惲元年&十一&二十九&二十二&四&三十一&二&二十&\\\hline
&七&二十四&十六&七&六&二&十二&三十&二十三&五&曹釐公夷元年&三&二十一&\\\hline
&八&二十五&十七&八盡殺故晉侯群公子。&七&三&十三&三十一&二十四&六&二&四&二十二&\\\hline
&九&二十六&十八&九始城絳都。&八&四&十四&衛懿公赤元年&二十五&七&三&五&二十三&\\\hline
甲寅&十賜齊侯命。&二十七&十九&十&九&五&十五&二&二十六&八&四&六&二十四&\\\hline
&十一&二十八&二十&十一&十&六&十六&三&二十七&九&五&七&二十五&\\\hline
&十二&二十九&二十一&十二太子申生居曲沃,重耳居蒲城,夷吾居屈。驪姬故。&十一&七&十七&四&二十八&十&六&八&二十六&\\\hline
&十三&三十&二十二&十三&十二&八&十八&五&二十九&十一&七&九&二十七&\\\hline
&十四&三十一&二十三伐山戎,為燕也。&十四&秦成公元年&九&十九&六&三十&十二&八&十&二十八&\\\hline
&十五&三十二莊公弟叔牙鴆死。慶父弒子般。季友奔陳,立湣公。&二十四&十五&二&十&二十&七&三十一&十三&九&十一&二十九&\\\hline
&十六&魯湣公開元年&二十五&十六滅魏、耿、霍。始封趙夙耿,畢萬魏,始此。&三&十一&二十一&八&三十二&十四&曹昭公元年&十二&三十&\\\hline
&十七&二慶父殺湣公。季友自陳立申,為釐公。殺慶父。&二十六&十七申生將軍,君子知其廢。&四&十二&二十二&翟伐我。公好鶴,士不戰,滅我國。國怨,惠公亂,滅其後,更立黔牟弟。衛戴公元年&三十三&十五&二&十三&三十一&\\\hline
&十八&魯釐公申元年哀姜喪自齊至。&二十七殺女弟魯莊公夫人,淫故。&十八&秦穆公任好元年&十三&二十三&衛文公燬元年戴公弟也。&三十四&十六&三&十四&三十二&\\\hline
&十九&二&二十八為衛築楚丘。救戎狄伐。&十九荀息以幣假道于虞以伐虢,滅下陽。&二&十四&二十四&二齊桓公率諸侯為我城楚丘。&三十五&十七&四&十五&三十三&\\\hline
甲子&二十&三&二十九與蔡姬共舟,蕩公,公怒,歸蔡姬。&二十&三&十五&二十五&三&三十六&十八以女故,齊伐我。&五&十六&燕襄公元年&\\\hline
&二十一&四&三十率諸侯伐蔡,蔡潰,遂伐楚,責包茅貢。&二十一申生以驪姬讒自殺。重耳奔蒲,夷吾奔屈。&四迎婦于晉。&十六齊伐我,至陘,使屈完盟。&二十六&四&三十七&十九&六&十七&二&\\\hline
&二十二&五&三十一&二十二滅虞、虢。重耳奔狄。&五&十七&二十七&五&三十八&二十&七&十八&三&\\\hline
&二十三&六&三十二率諸侯伐鄭。&二十三夷吾奔梁。&六&十八伐許,許君肉袒謝,楚從之。&二十八&六&三十九&二十一&八&十九&四&\\\hline
&二十四&七&三十三&二十四&七&十九&二十九&七&四十&二十二&九&二十&五&\\\hline
&二十五襄王立,畏太叔。&八&三十四&二十五伐翟,以重耳故。&八&二十&三十公疾,太子茲父讓兄目夷賢,公不聽。&八&四十一&二十三&曹共公元年&二十一&六&\\\hline
&襄王元年諸侯立王。&九齊率我伐晉亂,至高梁還。&三十五夏,會諸侯于葵丘。天子使宰孔賜胙,命無拜。&二十六公卒,立奚齊,里克殺之。及卓子。立夷吾。&九夷吾使郤芮賂,求入。&二十一&三十一公薨,未葬,齊桓會葵丘。&九&四十二&二十四&二&二十二&七&\\\hline
&二&十&三十六使隰朋立晉惠公。&晉惠公夷吾元年誅里克,倍秦約。&十丕鄭子豹亡來。&二十二&宋襄公茲父元年目夷相。&十&四十三&二十五&三&二十三&八&\\\hline
&三戎伐我,太叔帶召之。欲誅叔帶,叔帶奔齊。&十一&三十七&二&十一救王伐戎,戎去。&二十三伐黃。&二&十一&四十四&二十六&四&二十四有妾夢天與之蘭,生穆公蘭。&九&\\\hline
&四&十二&三十八使管仲平戎于周,欲以上卿禮,讓,受下卿。&三&十二&二十四&三&十二&四十五&二十七&五&二十五&十&\\\hline
甲戌&五&十三&三十九使仲孫請王,言叔帶,王怒。&四饑,請粟,秦與我。&十三丕豹欲無與,公不聽,輸晉粟,起雍至絳。&二十五&四&十三&陳穆公款元年&二十八&六&二十六&十一&\\\hline
&六&十四&四十&五秦饑,請粟,晉倍之。&十四&二十六滅六、英。&五&十四&二&二十九&七&二十七&十二&\\\hline
&七&十五五月,日有食之。不書,史官失之。&四十一&六秦虜惠公,復立之。&十五以盜食善馬士得破晉。&二十七&六&十五&三&蔡莊侯甲午元年&八&二十八&十三&\\\hline
&八&十六&四十二王以戎寇告齊,齊徵諸侯戍周。&七重耳聞管仲死,去翟之齊。&十六為河東置官司。&二十八&七隕五石。六鷁退飛,過我都。&十六&四&二&九&二十九&十四&\\\hline
&九&十七&四十三&八&十七&二十九&八&十七&五&三&十&三十&十五&\\\hline
&十&十八&齊孝公昭元年&九&十八&三十&九&十八&六&四&十一&三十一&十六&\\\hline
&十一&十九&二&十&十九滅梁。梁好城,不居,民罷,相驚,故亡。&三十一&十&十九&七&五&十二&三十二&十七&\\\hline
&十二&二十&三&十一&二十&三十二&十一&二十&八&六&十三&三十三&十八&\\\hline
&十三&二十一&四&十二&二十一&三十三執宋襄公,復歸之。&十二召楚盟。&二十一&九&七&十四&三十四&十九&\\\hline
&十四叔帶復歸於周。&二十二&五歸王弟帶。&十三太子圉質秦亡歸。&二十二&三十四&十三泓之戰,楚敗公。&二十二&十&八&十五&三十五君如楚,宋伐我。&二十&\\\hline
甲申&十五&二十三&六伐宋,以其不同盟。&十四圉立,為懷公。&二十三迎重耳於楚,厚禮之,妻之女。重耳願歸。&三十五重耳過,厚禮之。&十四公疾死泓戰。&二十三重耳從齊過,無禮。&十一&九&十六重耳過,無禮,僖負羈私善。&三十六重耳過,無禮,叔詹諫。&二十一&\\\hline
&十六王奔汜。汜,鄭地也。&二十四&七&晉文公元年誅子圉。魏武子為魏大夫,趙衰為原大夫。咎犯曰「求霸莫如內王。」&二十四以兵送重耳。&三十六&宋成公王臣元年&二十四&十二&十&十七&三十七&二十二&\\\hline
&十七晉納王。&二十五&八&二&二十五欲內王,軍河上。&三十七&二&二十五&十三&十一&十八&三十八&二十三&\\\hline
&十八&二十六&九&三宋服。&二十六&三十八&三倍楚親晉。&衛成公鄭元年&十四&十二&十九&三十九&二十四&\\\hline
&十九&二十七&十孝公薨,弟潘因衛公子開方殺孝公子,立潘。&四救宋,報曹、衛恥。&二十七&三十九使子玉伐宋。&四楚伐我,我告急於晉。&二&十五&十三&二十&四十&二十五&\\\hline
&二十王狩河陽。&二十八公如踐土會朝。&齊昭公潘元年會晉敗楚,朝周王。&五侵曹伐衛,取五鹿,執曹伯。諸侯敗楚而朝河陽,周命賜公土地。&二十八會晉伐楚朝周。&四十晉敗子玉于城濮。&五晉救我,楚兵去。&三晉伐我,取五鹿。公出奔,立公子瑕會晉朝,復歸衛。&十六會晉伐楚,朝周王。&十四會晉伐楚,朝周王。&二十一晉伐我,執公,復歸之。&四十一&二十六&\\\hline
&二十一&二十九&二&六&二十九&四十一&六&四晉以衛與宋。&陳共公朔元年&十五&二十二&四十二&二十七&\\\hline
&二十二&三十&三&七聽周歸衛成公。與秦圍鄭。&三十圍鄭,有言即去。&四十二&七&五周入成公,復衛。&二&十六&二十三&四十三秦、晉圍我,以晉故。&二十八&\\\hline
&二十三&三十一&四&八&三十一&四十三&八&六&三&十七&二十四&四十四&二十九&\\\hline
&二十四&三十二&五&九文公薨。&三十二將襲鄭,蹇叔曰不可。&四十四&九&七&四&十八&二十五&四十五文公薨。&三十&\\\hline
甲午&二十五&三十三僖公薨。&六狄侵我。&晉襄公驩元年破秦于殽。&三十三襲鄭,晉敗我殽。&四十五&十&八&五&十九&二十六&鄭穆公蘭元年秦襲我,弦高詐之。&三十一&\\\hline
&二十六&魯文公興元年&七&二伐衛,衛伐我。&三十四敗殽將亡歸,公復其官。&四十六王欲殺太子立職,太子恐,與傅潘崇殺王。王欲食熊蹯死,不聽。自立為王。&十一&九晉伐我,我伐晉。&六&二十&二十七&二&三十二&\\\hline
&二十七&二&八&三秦報我殽,敗于汪。&三十五伐晉報殽,敗我于汪。&楚穆王商臣元年以其太子宅賜崇,為相。&十二&十&七&二十一&二十八&三&三十三&\\\hline
&二十八&三公如晉。&九&四秦伐我,取王官,我不出。&三十六以孟明等伐晉,晉不敢出。&二晉伐我。&十三&十一&八&二十二&二十九&四&三十四&\\\hline
&二十九&四&十&五伐秦,圍邧、新城。&三十七晉伐我,圍邧、新城。&三滅江。&十四&十二公如晉。&九&二十三&三十&五&三十五&\\\hline
&三十&五&十一&六趙成子、欒貞子、霍伯、臼季皆卒。&三十八&四滅六、蓼。&十五&十三&十&二十四&三十一&六&三十六&\\\hline
&三十一&六&十二&七公卒。趙盾為太子少,欲更立君,恐誅,遂立太子為靈公。&三十九繆公薨。葬殉以人,從死者百七十人,君子譏之,故不言卒。&五&十六&十四&十一&二十五&三十二&七&三十七&\\\hline
&三十二&七&十三&晉靈公夷皋元年趙盾專政。&秦康公罃元年&六&十七公孫固殺成公。&十五&十二&二十六&三十三&八&三十八&\\\hline
&三十三襄王崩。&八王使衛來求金以葬,非禮。&十四&二秦伐我,取武城,報令孤之戰。&二&七&宋昭公杵臼元年襄公之子。&十六&十三&二十七&三十四&九&三十九&\\\hline
&頃王元年&九&十五&三率諸侯救鄭。&三&八伐鄭,以其服晉。&二&十七&十四&二十八&三十五&十楚伐我。&四十&\\\hline
甲辰&二&十&十六&四伐秦,拔少梁。秦取我北徵。&四晉伐我,取少梁。我伐晉取北徵。&九&三&十八&十五&二十九&曹文公壽元年&十一&燕桓公元年&\\\hline
&三&十一敗長翟于鹹而歸,得長翟。&十七&五&五&十&四敗長翟長丘。&十九&十六&三十&二&十二&二&\\\hline
&四&十二&十八&六秦取我羈馬。與秦戰河曲,秦師遁。&六伐晉,取羈馬。怒,與我大戰河曲。&十一&五&二十&十七&三十一&三&十三&三&\\\hline
&五&十三&十九&七得隨會。&七晉詐得隨會。&十二&六&二十一&十八&三十二&四&十四&四&\\\hline
&六頃王崩。公卿爭政,故不赴。&十四彗星入北斗,周史曰,七年,宋、齊、晉君死。&二十昭公卒。弟商人殺太子自立,是為懿公。&八趙盾以車八百乘納捷菑,平王室。&八&楚莊王侶元年&七&二十二&陳靈公平國元年&三十三&五&十五&五&\\\hline
&匡王元年&十五六月辛丑,日蝕,齊伐我。&齊懿公商人元年&九我入蔡。&九&二&八&二十三&二&三十四晉伐我。莊侯薨。&六齊入我郛。&十六&六&\\\hline
&二&十六&二不得民心。&十&十&三滅庸。&九襄夫人使衛伯殺昭公。弟鮑立。&二十四&三&蔡文侯申元年&七&十七&七&\\\hline
&三&十七齊伐我。&三伐魯。&十一率諸侯平宋。&十一&四&宋文公鮑元年昭公弟。晉率諸侯平我。&二十五&四&二&八&十八&八&\\\hline
&四&十八襄仲殺嫡,立庶子為宣公。&四公刖邴歜父而奪閻職妻,二人共殺公,立桓公子惠公。&十二&十二&五&二&二十六&五&三&九&十九&九&\\\hline
&五&魯宣公俀元年魯立宣公,不正,公室卑。&齊惠公元元年取魯濟西之田。&十三趙盾救陳、宋,伐鄭。&秦共公和元年&六伐宋、陳,以倍我服晉故。&三楚、鄭伐我,以我倍楚故也。&二十七&六&四&十&二十與楚侵陳,遂侵宋。晉使趙盾伐我,以倍晉故。&十&\\\hline
甲寅&六匡王崩。&二&二王子成父敗長翟。&十四趙穿殺靈公,趙盾使穿迎公子黑诈于周,立之。趙氏賜公族。&二&七&四華元以羊羹故陷於鄭。&二十八&七&五&十一&二十一與宋師戰,獲華元。&十一&\\\hline
&定王元年&三&三&晉成公黑诈元年伐鄭。&三&八伐陸渾,至雒,問鼎輕重。&五贖華元,亡歸。圍曹。&二十九&八&六&十二宋圍我。&二十二華元亡歸。&十二&\\\hline
&二&四&四&二&四&九若敖氏為亂,滅之。伐鄭。&六&三十&九&七&十三&鄭靈公夷元年公子歸生以黿故殺靈公。&十三&\\\hline
&三&五&五&三中行桓子荀林父救鄭,伐陳。&五&十&七&三十一&十楚伐鄭,與我平。晉中行桓子距楚,救鄭,伐我。&八&十四&鄭襄公堅元年靈公庶弟。楚伐我,晉來救。&十四&\\\hline
&四&六&六&四與衛侵陳。&秦桓公元年&十一&八&三十二與晉侵陳。&十一晉、衛侵我。&九&十五&二&十五&\\\hline
&五&七&七&五&二&十二&九&三十三&十二&十&十六&三&十六&\\\hline
&六&八七月,日蝕。&八&六與魯伐秦,獲秦諜,殺之絳市,六日而蘇。&三晉伐我,獲諜。&十三伐陳,滅舒蓼。&十&三十四&十三楚伐我。&十一&十七&四&燕宣公元年&\\\hline
&七&九&九&七使桓子伐楚。以諸侯師伐陳救鄭。成公薨。&四&十四伐鄭,晉郤缺救鄭,敗我。&十一&三十五&十四&十二&十八&五楚伐我,晉來救,敗楚師。&&\\\hline
&八&十四月,日蝕。&十公卒。崔杼有寵,高、國逐之,奔衛。&晉景公據元年與宋伐鄭。&五&十五&十二&衛穆公鸱元年齊崔杼來奔。&十五夏徵舒以其母辱,殺靈公。&十三&十九&六晉、宋、楚伐我。&三&\\\hline
&九&十一&齊頃公無野元年&二&六&十六率諸侯誅陳夏徵舒,立陳靈公子午。&十三&二&陳成公午元年靈公太子。&十四&二十&七&四&\\\hline
甲子&十&十二&二&三救鄭,為楚所敗河上。&七&十七圍鄭,鄭伯肉袒謝,釋之。&十四伐陳。&三&二&十五&二十一&八楚圍我,我卑辭以解。&五&\\\hline
&十一&十三&三&四&八&十八&十五&四&三&十六&二十二&九&六&\\\hline
&十二&十四&四&五伐鄭。&九&十九圍宋,為殺使者。&十六殺楚使者,楚圍我。&五&四&十七&二十三文公薨。&十晉伐我。&七&\\\hline
&十三&十五初稅畝。&五&六救宋,執解揚,有使節。秦伐我。&十&二十圍宋。五月,華元告子反以誠,楚罷。&十七華元告楚,楚去。&六&五&十八&曹宣公廬元年&十一佐楚伐宋,執解揚。&八&\\\hline
&十四&十六&六&七隨會滅赤翟。&十一&二十一&十八&七&六&十九&二&十二&九&\\\hline
&十五&十七日蝕。&七晉使郤克來齊,婦人笑之,克怒,歸去。&八使郤克使齊,婦人笑之,克怒歸。&十二&二十二&十九&八&七&二十文侯薨。&三&十三&十&\\\hline
&十六&十八宣公薨。&八晉伐敗我。&九伐齊,質子彊,兵罷。&十三&二十三莊王薨。&二十&九&八&蔡景侯固元年&四&十四&十一&\\\hline
&十七&魯成公黑肱元年春,齊取我隆。&九&十&十四&楚共王審元年&二十一&十&九&二&五&十五&十二&\\\hline
&十八&二與晉伐齊,齊歸我汶陽,竊與楚盟。&十晉郤克敗公於鞍,虜逢丑父。&十一與魯、曹敗齊。&十五&二秋,申公巫臣竊徵舒母奔晉,以為邢大夫。冬,伐衛、魯,救齊。&二十二&十一穆公薨。與諸侯敗齊,反侵地。楚伐我。&十&三&六&十六&十三&\\\hline
&十九&三會晉、宋、衛、曹伐鄭。&十一頃公如晉,欲王晉,晉不敢受。&十二始置六卿。率諸侯伐鄭。&十六&三&宋共公瑕元年&衛定公臧元年&十一&四&七伐鄭。&十七晉率諸侯伐我。&十四&\\\hline
甲戌&二十&四公如晉,晉不敬,公欲倍晉合於楚。&十二&十三魯公來,不敬。&十七&四子反救鄭。&二&二&十二&五&八&十八晉欒書取我氾。襄公薨。&十五&\\\hline
&二十一定王崩。&五&十三&十四梁山崩。伯宗隱其人而用其言。&十八&五伐鄭,倍我故也。鄭悼公來訟。&三&三&十三&六&九&鄭悼公費元年公如楚訟。&燕昭公元年&\\\hline
&簡王元年&六&十四&十五使欒書救鄭,遂侵蔡。&十九&六&四&四&十四&七晉侵我。&十&二悼公薨。楚伐我,晉使欒書來救。&二&吳壽夢元年\\\hline
&二&七&十五&十六以巫臣始通於吳而謀楚。&二十&七伐鄭。&五&五&十五&八&十一&鄭成公睔元年悼公弟也。楚伐我。&三&二臣巫來,謀伐楚。\\\hline
&三&八&十六&十七復趙武田邑。侵蔡。&二十一&八&六&六&十六&九晉伐我。&十二&二&四&三\\\hline
&四&九&十七頃公薨。&十八執鄭成公,伐鄭。秦伐我。&二十二伐晉。&九救鄭。冬,與晉成。&七&七&十七&十&十三&三與楚盟。公如晉,執公伐我。&五&四\\\hline
&五&十公如晉送葬,諱之。&齊靈公環元年&十九&二十三&十&八&八&十八&十一&十四&四晉率諸侯伐我。&六&五\\\hline
&六&十一&二&晉厲公壽曼元年&二十四與晉侯夾河盟,歸,倍盟。&十一&九&九&十九&十二&十五&五&七&六\\\hline
&七&十二&三&二&二十五&十二&十&十&二十&十三&十六&六&八&七\\\hline
&八&十三會晉伐秦。&四伐秦。&三伐秦至涇,敗之,獲其將成差。&二十六晉率諸侯伐我。&十三&十一晉率我伐秦。&十一&二十一&十四&十七晉率我伐秦。&七晉率我伐秦。&九&八\\\hline
甲申&九&十四&五&四&二十七&十四&十二&十二定公薨。&二十二&十五&曹成公負芻元年&八&十&九\\\hline
&十&十五始與吳通,會鍾離。&六&五三郤讒伯宗,殺之,伯宗好直諫。&秦景公元年&十五許畏鄭,請徙葉。&十三華元奔晉,復還。&衛獻公衎元年&二十三&十六&二晉執我公以歸。&九&十一&十與魯會鍾離。\\\hline
&十一&十六宣伯告晉,欲殺季文子,文子得以義脫。&七&六敗楚鄢陵。&二&十六救鄭,不利。子反醉,軍敗,殺子反歸。&宋平公成元年&二&二十四&十七&三&十倍晉盟楚,晉伐我,楚來救。&十二&十一\\\hline
&十二&十七&八&七&三&十七&二&三&二十五&十八&四&十一&十三昭公薨。&十二\\\hline
&十三&十八成公薨。&九&八欒書中行偃殺厲公,立襄公曾孫,為悼公。&四&十八為魚石伐宋彭城。&三楚伐彭城,封魚石。&四&二十六&十九&五&十二與楚伐宋。&燕武公元年&十三\\\hline
&十四簡王崩。&魯襄公午元年圍宋彭城。&十晉伐我,使太子光質於晉。&晉悼公元年圍宋彭城。&五&十九侵宋,救鄭。&四楚侵我,取犬丘。晉誅魚石,歸我彭城。&五圍宋彭城。&二十七&二十&六&十三晉伐敗我,兵次洧上,楚來救。&二&十四\\\hline
&靈王元年生有髭。&二會晉城虎牢。&十一&二率諸侯伐鄭,城虎牢。&六&二十&五&六&二十八&二十一&七&十四成公薨。晉率諸侯伐我。&三&十五\\\hline
&二&三&十二&三魏絳辱楊干。&七&二十一使子重伐吳,至衡山。使何忌侵陳。&六&七&二十九倍楚盟,楚侵我。&二十二&八&鄭釐公惲元年&四&十六楚伐我。\\\hline
&三&四公如晉。&十三&四魏絳說和戎、狄,狄朝晉。&八&二十二伐陳。&七&八&三十楚伐我。成公薨。&二十三&九&二&五&十七\\\hline
&四&五季文子卒。&十四&五&九&二十三伐陳。&八&九&陳哀公弱元年&二十四&十&三&六&十八\\\hline
甲午&五&六&十五&六&十&二十四&九&十&二&二十五&十一&四&七&十九\\\hline
&六&七&十六&七&十一&二十五圍陳。&十&十一&三楚圍我,為公亡歸。&二十六&十二&五子駟使賊夜殺釐公,詐以病卒赴諸侯。&八&二十\\\hline
&七&八公如晉。&十七&八&十二&二十六伐鄭。&十一&十二&四&二十七鄭侵我。&十三&鄭簡公嘉元年釐公子。&九&二十一\\\hline
&八&九與晉伐鄭,會河上,問公年十二,可冠,冠於衛。&十八與晉伐鄭。&九率齊、魯、宋、衛、曹伐鄭。秦伐我。&十三伐晉,楚為我援。&二十七伐鄭,師于武城,為秦。&十二晉率我伐鄭。&十三晉率我伐鄭。師曹鞭公幸妾。&五&二十八&十四晉率我伐鄭。&二誅子駟。晉率諸侯伐我,我與盟。楚怒,伐我。&十&二十二\\\hline
&九王叔奔晉。&十楚、鄭侵我西鄙。&十九令太子光、高厚會諸侯鍾離。&十率諸侯伐鄭。荀罃伐秦。&十四晉伐我。&二十八使子囊救鄭。&十三鄭伐我,衛來救。&十四救宋。&六&二十九&十五&三晉率諸侯伐我,楚來救。子孔作亂,子產攻之&十一&二十三\\\hline
&十&十一三桓分為三軍,各將軍。&二十&十一率諸侯伐鄭,秦敗我櫟。公曰,「吾用魏絳九合諸侯,」賜之樂。&十五我使庶長鮑伐晉救鄭,敗之櫟。&二十九與鄭伐宋。&十四楚、鄭伐我。&十五伐鄭。&七&三十&十六&四與楚伐宋,晉率諸侯伐我,秦來救。&十二&二十四\\\hline
&十一&十二公如晉。&二十一&十二&十六&三十&十五&十六&八&三十一&十七&五&十三&二十五壽夢卒。\\\hline
&十二&十三&二十二&十三&十七&三十一吳伐我,敗之。共王薨。&十六&十七&九&三十二&十八&六&十四&吳諸樊元年楚敗我。\\\hline
&十三&十四日蝕。&二十三衛獻公來奔。&十四率諸侯大夫伐秦,敗棫林。&十八晉諸侯大夫伐我,敗棫林。&楚康王昭元年共王太子出奔吳。&十七&十八孫文子攻公,公奔齊,立定公弟狄。&十&三十三&十九&七&十五&二季子讓位。楚伐我。\\\hline
&十四&十五日蝕。齊伐我。&二十四伐魯。&十五悼公薨。&十九&二&十八&衛殤公狄元年定公弟。&十一&三十四&二十&八&十六&三\\\hline
甲辰&十五&十六齊伐我。地震。齊復伐我北鄙。&二十五伐魯。&晉平公彪元年我敗楚于湛阪。&二十&三晉伐我,敗湛阪。&十九&二&十二&三十五&二十一&九&十七&四\\\hline
&十六&十七齊伐我北鄙。&二十六伐魯。&二&二十一&四&二十伐陳。&三伐曹。&十三宋伐我。&三十六&二十二衛伐我。&十&十八&五\\\hline
&十七&十八與晉伐齊。&二十七晉圍臨淄。晏嬰。&三率魯、宋、鄭、衛圍齊,大破之。&二十二&五伐鄭。&二十一晉率我伐齊。&四&十四&三十七&二十三成公薨。&十一晉率我圍齊。楚伐我。&十九武公薨。&六\\\hline
&十八&十九&二十八廢光,立子牙為太子。光與崔杼殺牙自立。晉、衛伐我。&四與衛伐齊。&二十三&六&二十二&五晉率我伐齊。&十五&三十八&曹武公勝元年&十二子產為卿。&燕文公元年&七\\\hline
&十九&二十日蝕。&齊莊公元年&五&二十四&七&二十三&六&十六&三十九&二&十三&二&八\\\hline
&二十&二十一公如晉。日再蝕。&二&六魯襄公來。殺羊舌虎。&二十五&八&二十四&七&十七&四十&三&十四&三&九\\\hline
&二十一&二十二孔子生。&三晉欒逞來奔,晏嬰曰「不如歸之」。&七欒逞奔齊。&二十六&九&二十五&八&十八&四十一&四&十五&四&十\\\hline
&二十二&二十三&四欲遣欒逞入曲沃伐晉,取朝歌。&八&二十七&十&二十六&九齊伐我。&十九&四十二&五&十六&五&十一\\\hline
&二十三&二十四侵齊。日再蝕。&五畏晉通楚,晏子謀。&九&二十八&十一與齊通。率陳、蔡伐鄭救齊。&二十七&十&二十楚率我伐鄭。&四十三楚率我伐鄭。&六&十七范宣子為政。我請伐陳。&六&十二\\\hline
&二十四&二十五齊伐我北鄙,以報孝伯之師。&六晉伐我,報朝歌。崔杼以莊公通其妻,殺之,立其弟,為景公。&十伐齊至高唐,報太行之役。&二十九公如晉,盟不結。&十二吳伐我,以報舟師之役,射殺吳王。&二十八&十一&二十一鄭伐我。&四十四&七&十八伐陳,入陳。&燕懿公元年&十三諸樊伐楚,迫巢門,傷射以薨。\\\hline
甲寅&二十五&二十六&齊景公杵臼元年如晉,請歸衛獻公。&十一誅衛殤公,復入獻公。&三十&十三率陳、蔡伐鄭。&二十九&十二齊、晉殺殤公,復內獻公。&二十二楚率我伐鄭。&四十五&八&十九楚率陳、蔡伐我。&二&吳餘祭元年\\\hline
&二十六&二十七日蝕。&二慶封欲專,誅崔氏,杼自殺。&十二&三十一&十四&三十&衛獻公衎後元年&二十三&四十六&九&二十&三&二\\\hline
&二十七&二十八公如楚。葬康王。&三冬,鮑、高、欒氏謀慶封,發兵攻慶封,慶封奔吳。&十三&三十二&十五康王薨。&三十一&二&二十四&四十七&十&二十一&四懿公薨。&三齊慶封來奔。\\\hline
&景王元年&二十九吳季札來觀周樂,盡知樂所為。&四吳季札來使,與晏嬰歡。&十四吳季札來,曰「晉政卒歸韓、魏、趙。」&三十三&楚熊郟敖元年&三十二&三&二十五&四十八&十一&二十二吳季札謂子產曰,政將歸子,子以禮,幸脫於厄矣。」&燕惠公元年齊高止來奔。&四守門閽殺餘祭。季札使諸侯。\\\hline
&二&三十&五&十五&三十四&二&三十三&衛襄公惡元年&二十六&四十九為太子取楚女,公通焉,太子殺公自立。&十二&二十三諸公子爭寵相殺,又欲殺子產,子成止之。&二&五\\\hline
&三&三十一襄公薨。&六&十六&三十五&三王季父圍為令尹。&三十四&二&二十七&蔡靈侯班元年&十三&二十四&三&六\\\hline
&四&魯昭公稠元年昭公年十九,有童心。&七&十七秦后子來奔。&三十六公弟后子奔晉,車千乘。&四令尹圍殺郟敖,自立為靈王。&三十五&三&二十八&二&十四&二十五&四&七\\\hline
&五&二公如晉,至河,晉謝還之。&八田無宇送女。&十八齊田無宇來送女。&三十七&楚靈王圍元年共王子,肘玉。&三十六&四&二十九&三&十五&二十六&五&八\\\hline
&六&三&九晏嬰使晉,見叔向,曰「齊政歸田氏。」叔向曰,「晉公室卑。」&十九&三十八&二&三十七&五&三十&四&十六&二十七夏,如晉。冬,如楚。&六公欲殺公卿立幸臣,公卿誅幸臣,公恐,出奔齊。&九\\\hline
&七&四稱病不會楚。&十&二十&三十九&三夏,合諸侯宋地,盟。伐吳朱方,誅慶封。冬,報我,取三城。&三十八&六稱病不會楚。&三十一&五&十七稱病不會楚。&二十八子產曰,三國不會。&七&十楚誅慶封。\\\hline
甲子&八&五&十一&二十一秦后子歸秦。&四十公卒。后子自晉歸。&四率諸侯伐吳。&三十九&七&三十二&六&十八&二十九&八&十一楚率諸侯伐我。\\\hline
&九&六&十二公如晉,請伐燕,入其君。&二十二齊景公來,請伐燕,入其君。&秦哀公元年&五伐吳,次乾谿。&四十&八&三十三&七&十九&三十&九齊伐我。&十二楚伐我,次乾谿。\\\hline
&十&七季武子卒。日蝕。&十三入燕君。&二十三入燕君。&二&六執芋尹亡人入章華。&四十一&九夫人姜氏無子。&三十四&八&二十&三十一&燕悼公元年惠公歸至卒。&十三\\\hline
&十一&八公如楚,楚留之。賀章華臺。&十四&二十四&三&七就章華臺,內亡人實之。滅陳。&四十二&衛靈公元年&三十五弟招作亂,哀公自殺。&九&二十一&三十二&二&十四\\\hline
&十二&九&十五&二十五&四&八弟棄疾將兵定陳。&四十三&二&陳惠公吳元年哀公孫也。楚來定我。&十&二十二&三十三&三&十五\\\hline
&十三&十&十六&二十六春,有星出婺女。七月,公薨。&五&九&四十四平公薨。&三&二&十一&二十三&三十四&四&十六\\\hline
&十四&十一&十七&晉昭公夷元年&六&十醉殺蔡侯,使棄疾圍之。棄疾居之,為蔡侯。&宋元公佐元年&四&三&十二靈侯如楚,楚殺之,使棄疾居之,為蔡侯。&二十四&三十五&五&十七\\\hline
&十五&十二朝晉至河,晉謝之歸。&十八公如晉。&二&七&十一王伐徐以恐吳,次乾谿。民罷於役,怨王。&二&五公如晉,朝嗣君。&四&蔡侯廬元年景侯子。&二十五&三十六公如晉。&六&吳餘昧元年\\\hline
&十六&十三&十九&三&八&十二棄疾作亂自立,靈王自殺。復陳、蔡。&三&六&五楚平王復陳,立惠公。&二楚平王復我,立景侯子廬。&二十六&鄭定公寧元年&七&二\\\hline
&十七&十四&二十&四&九&楚平王居元年共王子,抱玉。&四&七&六&三&二十七&二&燕共公元年&三\\\hline
甲戌&十八后太子卒。&十五日蝕。公如晉,晉留之葬,公恥之。&二十一&五&十&二王為太子取秦女,好,自取之。&五&八&七&四&曹平公須元年&三&二&四\\\hline
&十九&十六&二十二&六公卒。六卿彊,公室卑矣。&十一&三&六&九&八&五&二&四&三&吳僚元年\\\hline
&二十&十七五月朔,日蝕。彗星見辰。&二十三&晉頃公去疾元年&十二&四與吳戰。&七&十&九&六&三&五火,欲禳之,子產曰,不如脩德。&四&二與楚戰。\\\hline
&二十一&十八&二十四&二&十三&五&八火。&十一火。&十火。&七&四平公薨。&六火。&五共公薨。&三\\\hline
&二十二&十九地震。&二十五&三&十四&六&九&十二&十一&八&曹悼公午元年&七&燕平公元年&四\\\hline
&二十三&二十齊景公與晏子狩,入魯問禮。&二十六獵魯界,因入魯。&四&十五&七誅伍奢、尚,太子建奔宋,伍胥奔吳。&十公毋信。詐殺諸公子。楚太子建來奔,見亂,之鄭。&十三&十二&九平侯薨。靈侯孫東國殺平侯子而自立。&二&八楚太子建從宋來奔。&二&五伍員來奔。\\\hline
&二十四&二十一公如晉至河,晉謝之,歸。日蝕。&二十七&五&十六&八蔡侯來奔。&十一&十四&十三&蔡悼侯東國元年奔楚。&三&九&三&六\\\hline
&二十五&二十二日蝕。&二十八&六周室亂,公平亂,立敬王。&十七&九&十二&十五&十四&二&四&十&四&七\\\hline
&敬王元年&二十三地震。&二十九&七&十八&十吳伐敗我。&十三&十六&十五吳敗我兵,取胡、沈。&三&五&十一楚建作亂,殺之。&五&八公子光敗楚。\\\hline
&二&二十四⒧鵒來巢。&三十&八&十九&十一吳卑梁人爭桑,伐取我鍾離。&十四&十七&十六&蔡昭侯申元年悼侯弟。&六&十二公如晉,請內王。&六&九\\\hline
甲申&三&二十五公欲誅季氏,三桓氏攻公,公出居鄆。&三十一&九&二十&十二&十五&十八&十七&二&七&十三&七&十\\\hline
&四&二十六齊取我鄆以處公。&三十二彗星見。晏子曰,「田氏有德於齊,可畏。」&十知櫟、趙鞅內王於王城。&二十一&十三欲立子西,子西不肯。秦女子立,為昭王。&宋景公頭曼元年&十九&十八&三&八&十四&八&十一\\\hline
&五&二十七&三十三&十一&二十二&楚昭王珍元年誅無忌以說眾。&二&二十&十九&四&九&十五&九&十二公子光使專諸殺僚,自立。\\\hline
&六&二十八公如晉,求入,晉弗聽,處之乾侯。&三十四&十二六卿誅公族,分其邑。各使其子為大夫。&二十三&二&三&二十一&二十&五&曹襄公元年&十六&十&吳闔閭元年\\\hline
&七&二十九公自乾侯如鄆。齊侯曰,主君。公恥之,復之乾侯。&三十五&十三&二十四&三&四&二十二&二十一&六&二&鄭獻公蠆元年&十一&二\\\hline
&八&三十&三十六&十四頃公薨。&二十五&四吳三公子來奔,封以扞吳。&五&二十三&二十二&七&三&二&十二&三三公子奔楚。\\\hline
&九&三十一日蝕。&三十七&晉定公午元年&二十六&五吳伐我六、潛。&六&二十四&二十三&八&四&三&十三&四伐楚六、潛。\\\hline
&十晉使諸侯為我築城。&三十二公卒乾侯。&三十八&二率諸侯為周築城。&二十七&六&七&二十五&二十四&九&五平公弟通殺襄公自立。&四&十四&五\\\hline
&十一&魯定公宋元年昭公喪自乾侯至。&三十九&三&二十八&七囊瓦伐吳,敗我豫章。蔡侯來朝。&八&二十六&二十五&十朝楚,以裘故留。&曹隱公元年&五&十五&六楚伐我,迎擊,敗之,取楚之居巢。\\\hline
&十二&二&四十&四&二十九&八&九&二十七&二十六&十一&二&六&十六&七\\\hline
甲午&十三&三&四十一&五&三十&九蔡昭侯留三歲,得裘,故歸。&十&二十八&二十七&十二與子常裘,得歸,如晉,請伐楚。&三&七&十七&八\\\hline
&十四與晉率諸侯侵楚。&四&四十二&六周與我率諸侯侵楚。&三十一楚包胥請救。&十吳、蔡伐我,入郢,昭王亡。伍子胥鞭平王墓。&十一&二十九與蔡爭長。&二十八&十三與衛爭長。楚侵我,吳與我伐楚,入郢。&四&八&十八&九與蔡伐楚,入郢。\\\hline
&十五&五陽虎執季桓子,與盟,釋之。日蝕。&四十三&七&三十二&十一秦救至,吳去,昭王復入。&十二&三十&陳懷公柳元年&十四&曹靖公路元年&九&十九&十\\\hline
&十六王子朝之徒作亂故,王奔晉。&六&四十四&八&三十三&十二吳伐我番,楚恐,徙鄀。&十三&三十一&二&十五&二&十魯侵我。&燕簡公元年&十一伐楚,取番。\\\hline
&十七劉子迎王,晉入王。&七齊伐我。&四十五侵衛。伐魯。&九入周敬王。&三十四&十三&十四&三十二齊侵我。&三&十六&三&十一&二&十二\\\hline
&十八&八陽虎欲伐三桓,三桓攻陽虎,虎奔陽關。&四十六魯伐我。我伐魯。&十伐衛。&三十五&十四子西為民泣,民亦泣,蔡昭侯恐。&十五&三十三晉、魯侵伐我。&四公如吳,吳留之,因死吳。&十七&四靖公薨。&十二&三&十三陳懷公來,留之,死於吳。\\\hline
&十九&九伐陽虎,虎奔齊。&四十七囚陽虎,虎奔晉。&十一陽虎來奔。&三十六哀公薨。&十五&十六陽虎來奔。&三十四&陳湣公越元年&十八&曹伯陽元年&十三獻公薨。&四&十四\\\hline
&二十&十公會齊侯於夾谷。孔子相。齊歸我地。&四十八&十二&秦惠公元年彗星見。&十六&十七&三十五&二&十九&二&鄭聲公勝元年鄭益弱。&五&十五\\\hline
&二十一&十一&四十九&十三&二生躁公、懷公、簡公。&十七&十八&三十六&三&二十&三國人有夢眾君子立社宮,謀亡曹,振鐸請待公孫彊,許之。&二&六&十六\\\hline
&二十二&十二齊來歸女樂,季桓子受之,孔子行。&五十遺魯女樂。&十四&三&十八&十九&三十七伐曹。&四&二十一&四衛伐我。&三&七&十七\\\hline
甲辰&二十三&十三&五十一&十五趙鞅伐范、中行。&四&十九&二十&三十八孔子來,祿之如魯。&五&二十二&五&四&八&十八\\\hline
&二十四&十四&五十二&十六&五&二十&二十一&三十九太子蒯聵出奔。&六孔子來。&二十三&六公孫彊好射,獻鴈,君使為司城,夢者子行。&五子產卒。&九&十九伐越,敗我,傷闔閭指,以死。\\\hline
&二十五&十五定公薨。日蝕。&五十三&十七&六&二十一滅胡。以吳敗,我倍之。&二十二鄭伐我。&四十&七&二十四&七&六伐宋。&十&吳王夫差元年\\\hline
&二十六&魯哀公將元年&五十四伐晉。&十八趙鞅圍范、中行朝歌。齊、衛伐我。&七&二十二率諸侯圍蔡。&二十三&四十一伐晉。&八吳伐我。&二十五楚伐我,以吳怨故。&八&七&十一&二伐越。\\\hline
&二十七&二&五十五輸范、中行氏粟。&十九趙鞅圍范、中行,鄭來救,我敗之。&八&二十三&二十四&四十二靈公薨。蒯聵子輒立。晉納太子蒯聵于戚。&九&二十六畏楚,私召吳人,乞遷于州來,州來近吳。&九&八救范、中行氏,與趙鞅戰於鐵,敗我師。&十二&三\\\hline
&二十八&三地震。&五十六&二十&九&二十四&二十五孔子過宋,桓魋惡之。&衛出公輒元年&十&二十七&十宋伐我。&九&燕獻公元年&四\\\hline
&二十九&四&五十七乞救范氏。&二十一趙鞅拔邯鄲、柏人,有之。&十惠公薨。&二十五&二十六&二&十一&二十八大夫共誅昭侯。&十一&十&二&五\\\hline
&三十&五&五十八景公薨。立嬖姬子為太子。&二十二趙鞅敗范中行,中行奔齊。伐衛。&秦悼公元年&二十六&二十七&三晉伐我,救范氏故。&十二&蔡成侯朔元年&十二&十一&三&六\\\hline
&三十一&六&齊晏孺子元年田乞詐立陽生,殺孺子。&二十三&二&二十七救陳,王死城父。&二十八伐曹。&四&十三吳伐我,楚來救。&二&十三宋伐我。&十二&四&七伐陳。\\\hline
&三十二&七公會吳王于繒。吳徵百牢,季康子使子貢謝之。&齊悼公陽生元年&二十四侵衛。&三&楚惠王章元年&二十九侵鄭,圍曹。&五晉侵我。&十四&三&十四宋圍我,鄭救我。&十三&五&八魯會我繒。\\\hline
甲寅&三十三&八吳為邾伐我,至城下,盟而去。齊取我三邑。&二伐魯,取三邑。&二十五&四&二子西召建子勝於吳,為白公。&三十曹倍我,我滅之。&六&十五&四&十五宋滅曹,虜伯陽。&十四&六&九伐魯。\\\hline
&三十四&九&三&二十六&五&三伐陳,陳與吳故。&三十一鄭圍我,敗之于雍丘。&七&十六倍楚,與吳成。&五&&十五圍宋,敗我師雍丘,伐我。&七&十\\\hline
&三十五&十與吳伐齊。&四吳、魯伐我。鮑子殺悼公,齊人立其子壬為簡公。&二十七使趙鞅伐齊。&六&四伐陳。&三十二伐鄭。&八孔子自陳來。&十七&六&&十六&八&十一與魯伐齊救陳。誅五員。\\\hline
&三十六&十一齊伐我。冉有言,故迎孔子,孔子歸。&齊簡公元年魯與吳敗我。&二十八&七&五&三十三&九孔子歸魯。&十八&七&&十七&九&十二與魯敗齊。\\\hline
&三十七&十二與吳會橐皋。用田賦。&二&二十九&八&六白公勝數請子西伐鄭,以父怨故。&三十四&十公如晉,與吳會橐皋。&十九&八&&十八宋伐我。&十&十三與魯會橐皋。\\\hline
&三十八&十三與吳會黃池。&三&三十與吳會黃池,爭長。&九&七伐陳。&三十五鄭敗我師。&十一&二十&九&&十九敗宋師。&十一&十四與晉會黃池。\\\hline
&三十九&十四西狩獲麟。衛出公來奔。&四田常殺簡公,立其弟驁,為平公,常相之,專國權。&三十一&十&八&三十六&十二父蒯聵入,輒出亡。&二十一&十&&二十&十二&十五\\\hline
&四十&十五子服景伯使齊,子貢為介,齊歸我侵地。&齊平公驁元年景公孫也。齊自是稱田氏。&三十二&十一&九&三十七熒惑守心,子韋曰「善。」&衛莊公蒯聵元年&二十二&十一&&二十一&十三&十六\\\hline
&四十一&十六孔子卒。&二&三十三&十二&十白公勝殺令尹子西,攻惠王。葉公攻白公,白公自殺。惠王復國。&三十八&二&二十三楚滅陳,殺湣公。&十二&&二十二&十四&十七\\\hline
&四十二&十七&三&三十四&十三&十一&三十九&三莊公辱戎州人,戎州人與趙簡子攻莊公,出奔。&&十三&&二十三&十五&十八越敗我。\\\hline
甲子&四十三敬王崩。&十八二十七卒。&四二十五卒。&三十五三十七卒。&十四卒,子厲共公立。&十二五十七卒。&四十六十四卒。&衛君起元年石傅逐起出,輒復入。&&十四十九卒&&二十四三十八卒。&十六二十八卒。&十九二十三卒。\\\hline
}

太史公曰,儒者斷其義,馳說者騁其辭,不務綜其終始,歷人取其年月,數家隆於神運,譜諜獨記世謚,其辭略,欲一觀諸要難。於是譜十二諸侯,自共和訖孔子,表見春秋、國語學者所譏盛衰大指著于篇,為成學治古文者要刪焉。