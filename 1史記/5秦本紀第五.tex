\chapter{秦本紀第五}

秦之先,帝顓頊之苗裔孫曰女修。女修織,玄鳥隕卵,女修吞之,生子大業。大業取少典之子,曰女華。女華生大費,與禹平水土。已成,帝錫玄圭。禹受曰,非予能成,亦大費為輔。帝舜曰,咨爾費,贊禹功,其賜爾皁游。爾後嗣將大出。乃妻之姚姓之玉女。大費拜受,佐舜調馴鳥獸,鳥獸多馴服,是為柏翳。舜賜姓嬴氏。

大費生子二人,一曰大廉,實鳥俗氏,二曰若木,實費氏。其玄孫曰費昌,子孫或在中國,或在夷狄。費昌當夏桀之時,去夏歸商,為湯御,以敗桀於鳴條。大廉玄孫曰孟戲、中衍,鳥身人言。帝太戊聞而卜之使御,吉,遂致使御而妻之。自太戊以下,中衍之後,遂世有功,以佐殷國,故嬴姓多顯,遂為諸侯。

其玄孫曰中潏,在西戎,保西垂。生蜚廉。蜚廉生惡來。惡來有力,蜚廉善走,父子俱以材力事殷紂。周武王之伐紂,并殺惡來。是時蜚廉為紂石北方,還,無所報,為壇霍太山而報,得石棺,銘曰帝令處父不與殷亂,賜爾石棺以華氏。死,遂葬於霍太山。蜚廉復有子曰季勝。季勝生孟增。孟增幸於周成王,是為宅皋狼。皋狼生衡父,衡父生造父。造父以善御幸於周繆王,得驥、溫驪、驊騮、騄耳之駟,西巡狩,樂而忘歸。徐偃王作亂,造父為繆王御,長驅歸周,一日千里以救亂。繆王以趙城封造父,造父族由此為趙氏。自蜚廉生季勝已下五世至造父,別居趙。趙衰其後也。惡來革者,蜚廉子也,蚤死。有子曰女防。女防生旁皋,旁皋生太幾,太幾生大駱,大駱生非子。以造父之寵,皆蒙趙城,姓趙氏。

非子居犬丘,好馬及畜,善養息之。犬丘人言之周孝王,孝王召使主馬于汧渭之閒,馬大蕃息。孝王欲以為大駱適嗣。申侯之女為大駱妻,生子成為適。申侯乃言孝王曰,昔我先酈山之女,為戎胥軒妻,生中潏,以親故歸周,保西垂,西垂以其故和睦。今我復與大駱妻,生適子成。申駱重婚,西戎皆服,所以為王。王其圖之。於是孝王曰,昔伯翳為舜主畜,畜多息,故有土,賜姓嬴。今其後世亦為朕息馬,朕其分土為附庸。邑之秦,使復續嬴氏祀,號曰秦嬴。亦不廢申侯之女子為駱適者,以和西戎。

秦嬴生秦侯。秦侯立十年,卒。生公伯。公伯立三年,卒。生秦仲。

秦仲立三年,周厲王無道,諸侯或叛之。西戎反王室,滅犬丘大駱之族。周宣王即位,乃以秦仲為大夫,誅西戎。西戎殺秦仲。秦仲立二十三年,死於戎。有子五人,其長者曰莊公。周宣王乃召莊公昆弟五人,與兵七千人,使伐西戎,破之。於是復予秦仲後,及其先大駱地犬丘并有之,為西垂大夫。

莊公居其故西犬丘,生子三人,其長男世父。世父曰,戎殺我大父仲,我非殺戎王則不敢入邑。遂將擊戎,讓其弟襄公。襄公為太子。莊公立四十四年,卒,太子襄公代立。襄公元年,以女弟繆嬴為豐王妻。襄公二年,戎圍犬丘,世父世父擊之,為戎人所虜。歲餘,復歸世父。七年春,周幽王用褒姒廢太子,立褒姒子為適,數欺諸侯,諸侯叛之。西戎犬戎與申侯伐周,殺幽王酈山下。而秦襄公將兵救周,戰甚力,有功。周避犬戎難,東徙雒邑,襄公以兵送周平王。平王封襄公為諸侯,賜之岐以西之地。曰,戎無道,侵奪我岐、豐之地,秦能攻逐戎,即有其地。與誓,封爵之。襄公於是始國,與諸侯通使聘享之禮,乃用騮駒、黃牛、羝羊各三,祠上帝西畤。十二年,伐戎而至岐,卒。生文公。

文公元年,居西垂宮。三年,文公以兵七百人東獵。四年,至汧渭之會。曰,昔周邑我先秦嬴於此,後卒獲為諸侯。乃卜居之,占曰吉,即營邑之。十年,初為鄜畤,用三牢。十三年,初有史以紀事,民多化者。十六年,文公以兵伐戎,戎敗走。於是文公遂收周餘民有之,地至岐,岐以東獻之周。十九年,得陳寶。二十年,法初有三族之罪。二十七年,伐南山大梓,豐大特。四十八年,文公太子卒,賜謚為竫公。竫公之長子為太子,是文公孫也。五十年,文公卒,葬西山。竫公子立,是為寧公。

寧公二年,公徙居平陽。遣兵伐蕩社。三年,與亳戰,亳王奔戎,遂滅蕩社。四年,魯公子翬弒其君隱公。十二年,伐蕩氏,取之。寧公生十歲立,立十二年卒,葬西山。生子三人,長男武公為太子。武公弟德公,同母魯姬子。生出子。寧公卒,大庶長弗忌、威壘、三父廢太子而立出子為君。出子六年,三父等復共令人賊殺出子。出子生五歲立,立六年卒。三父等乃復立故太子武公。

武公元年,伐彭戲氏,至于華山下,居平陽封宮。三年,誅三父等而夷三族,以其殺出子也。鄭高渠瞇殺其君昭公。十年,伐邽、冀戎,初縣之。十一年,初縣杜、鄭。滅小虢。

十三年,齊人管至父、連稱等殺其君襄公而立公孫無知。晉滅霍、魏、耿。齊雍廩殺無知、管至父等而立齊桓公。齊、晉為彊國。

十九年,晉曲沃始為晉侯。齊桓公伯於鄄。

二十年,武公卒,葬雍平陽。初以人從死,從死者六十六人。有子一人,名曰白,白不立,封平陽。立其弟德公。

德公元年,初居雍城大鄭宮。以犧三百牢祠鄜畤。卜居雍。後子孫飲馬於河。梁伯、芮伯來朝。二年,初伏,以狗御蠱。德公生三十三歲而立,立二年卒。生子三人,長子宣公,中子成公,少子穆公。長子宣公立。

宣公元年,衛、燕伐周,出惠王,立王子穨。三年,鄭伯、虢叔殺子穨而入惠王。四年,作密畤。與晉戰河陽,勝之。十二年,宣公卒。生子九人,莫立,立其弟成公。

成公元年,梁伯、芮伯來朝。齊桓公伐山戎,次于孤竹。

成公立四年卒。子七人,莫立,立其弟繆公。

繆公任好元年,自將伐茅津,勝之。四年,迎婦於晉,晉太子申生姊也。其歲,齊桓公伐楚,至邵陵。

五年,晉獻公滅虞、虢,虜虞君與其大夫百里傒,以璧馬賂於虞故也。既虜百里傒,以為秦繆公夫人媵於秦。百里傒亡秦走宛,楚鄙人執之。繆公聞百里傒賢,欲重贖之,恐楚人不與,乃使人謂楚曰,吾媵臣百里傒在焉,請以五羖羊皮贖之。。楚人遂許與之。當是時,百里傒年已七十餘。繆公釋其囚,與語國事。謝曰,臣亡國之臣,何足問。繆公曰,虞君不用子,故亡,非子罪也。固問,語三日,繆公大說,授之國政,號曰五羖大夫。百里傒讓曰,臣不及臣友蹇叔,蹇叔賢而世莫知。臣常游困於齊而乞食铚人,蹇叔收臣。臣因而欲事齊君無知,蹇叔止臣,臣得脫齊難,遂之周。周王子穨好牛,臣以養牛干之。及穨欲用臣,蹇叔止臣,臣去,得不誅。事虞君,蹇叔止臣。臣知虞君不用臣,臣誠私利祿爵,且留。再用其言,得脫,一不用,及虞君難,是以知其賢。於是繆公使人厚幣迎蹇叔,以為上大夫。

秋,繆公自將伐晉,戰於河曲。晉驪姬作亂,太子申生死新城,重耳、夷吾出奔。

九年,齊桓公會諸侯於葵丘。

晉獻公卒。立驪姬子奚齊,其臣里克殺奚齊。荀息立卓子,克又殺卓子及荀息。夷吾使人請秦,求入晉。於是繆公許之,使百里傒將兵送夷吾。夷吾謂曰,誠得立,請割晉之河西八城與秦。及至,已立,而使丕鄭謝秦,背約不與河西城,而殺裏克。丕鄭聞之,恐,因與繆公謀曰,晉人不欲夷吾,實欲重耳。今背秦約而殺裏克,皆呂甥、郤芮之計也。願君以利急召呂、郤,呂、郤至,則更入重耳便。繆公許之,使人與丕鄭歸,召呂、郤。呂、郤等疑丕鄭有閒,乃言夷吾殺丕鄭。丕鄭子丕豹奔秦,說繆公曰,晉君無道,百姓不親,可伐也。繆公曰,百姓茍不便,何故能誅其大臣。能誅其大臣,此其調也。不聽,而陰用豹。

十二年,齊管仲、隰朋死。

晉旱,來請粟。丕豹說繆公勿與,因其饑而伐之。繆公問公孫支,支曰,饑穰更事耳,不可不與。問百里傒,傒曰,夷吾得罪於君,其百姓何罪。於是用百里傒、公孫支言,卒與之粟。以船漕車轉,自雍相望至絳。

十四年,秦饑,請粟於晉。晉君謀之群臣。虢射曰,因其饑伐之,可有大功。晉君從之。十五年,興兵將攻秦。繆公發兵,使丕豹將,自往擊之。九月壬戌,與晉惠公夷吾合戰於韓地。晉君棄其軍,與秦爭利,還而馬騺。繆公與麾下馳追之,不能得晉君,反為晉軍所圍。晉擊繆公,繆公傷。於是岐下食善馬者三百人馳冒晉軍,晉軍解圍,遂脫繆公而反生得晉君。初,繆公亡善馬,岐下野人共得而食之者三百餘人,吏逐得,欲法之。繆公曰,君子不以畜產害人。吾聞食善馬肉不飲酒,傷人。乃皆賜酒而赦之。三百人者聞秦擊晉,皆求從,從而見繆公窘,亦皆推鋒爭死,以報食馬之德。於是繆公虜晉君以歸,令於國,齊宿,吾將以晉君祠上帝。周天子聞之,曰晉我同姓,為請晉君。夷吾姊亦為繆公夫人,夫人聞之,乃衰绖跣,曰,妾兄弟不能相救,以辱君命。繆公曰,我得晉君以為功,今天子為請,夫人是憂。乃與晉君盟,許歸之,更舍上舍,而饋之七牢。十一月,歸晉君夷吾,夷吾獻其河西地,使太子圉為質於秦。秦妻子圉以宗女。是時秦地東至河。

十八年,齊桓公卒。二十年,秦滅梁、芮。

二十二年,晉公子圉聞晉君病,曰,梁,我母家也,而秦滅之。我兄弟多,即君百歲後,秦必留我,而晉輕,亦更立他子。子圉乃亡歸晉。二十三年,晉惠公卒,子圉立為君。秦怨圉亡去,乃迎晉公子重耳於楚,而妻以故子圉妻。重耳初謝,後乃受。繆公益禮厚遇之。二十四年春,秦使人告晉大臣,欲入重耳。晉許之,於是使人送重耳。二月,重耳立為晉君,是為文公。文公使人殺子圉。子圉是為懷公。

其秋,周襄王弟帶以翟伐王,王出居鄭。二十五年,周王使人告難於晉、秦。秦繆公將兵助晉文公入襄王,殺王弟帶。二十八年,晉文公敗楚於城濮。三十年,繆公助晉文公圍鄭。鄭使人言繆公曰,亡鄭厚晉,於晉而得矣,而秦未有利。晉之彊,秦之憂也。繆公乃罷兵歸。晉亦罷。三十二年冬,晉文公卒。

鄭人有賣鄭於秦曰,我主其城門,鄭可襲也。繆公問蹇叔、百里傒,對曰,徑數國千里而襲人,希有得利者。且人賣鄭,庸知我國人不有以我情告鄭者乎。不可。繆公曰,子不知也,吾已決矣。遂發兵,使百里傒子孟明視,蹇叔子西乞術及白乙丙將兵。行日,百里傒、蹇叔二人哭之。繆公聞,怒曰,孤發兵而子沮哭吾軍,何也。二老曰,臣非敢沮君軍。軍行,臣子與往,臣老,遲還恐不相見,故哭耳。二老退,謂其子曰,汝軍即敗,必於殽阨矣。三十三年春,秦兵遂東,更晉地,過周北門。周王孫滿曰,秦師無禮,不敗何待。兵至滑,鄭販賣賈人弦高,持十二牛將賣之周,見秦兵,恐死虜,因獻其牛,曰,聞大國將誅鄭,鄭君謹修守御備,使臣以牛十二勞軍士。秦三將軍相謂曰,將襲鄭,鄭今已覺之,往無及已。滅滑。滑,晉之邊邑也。

當是時,晉文公喪尚未葬。太子襄公怒曰,秦侮我孤,因喪破我滑。遂墨衰绖,發兵遮秦兵於殽,擊之,大破秦軍,無一人得脫者。虜秦三將以歸。文公夫人,秦女也,為秦三囚將請曰,繆公之怨此三人入於骨髓,願令此三人歸,令我君得自快烹之。晉君許之,歸秦三將。三將至,繆公素服郊迎,向三人哭曰,孤以不用百里傒、蹇叔言以辱三子,三子何罪乎。子其悉心雪恥,毋怠。遂復三人官秩如故,愈益厚之。

三十四年,楚太子商臣弒其父成王代立。

繆公於是復使孟明視等將兵伐晉,戰于彭衙。秦不利,引兵歸。

戎王使由余於秦。由余,其先晉人也,亡入戎,能晉言。聞繆公賢,故使由余觀秦。秦繆公示以宮室、積聚。由余曰,使鬼為之,則勞神矣。使人為之,亦苦民矣。繆公怪之,問曰,中國以詩書禮樂法度為政,然尚時亂,今戎夷無此,何以為治,不亦難乎。由余笑曰,此乃中國所以亂也。夫自上聖黃帝作為禮樂法度,身以先之,僅以小治。及其後世,日以驕淫。阻法度之威,以責督於下,下罷極則以仁義怨望於上,上下交爭怨而相篡弒,至於滅宗,皆以此類也。夫戎夷不然。上含淳德以遇其下,下懷忠信以事其上,一國之政猶一身之治,不知所以治,此真聖人之治也。於是繆公退而問內史廖曰,孤聞鄰國有聖人,敵國之憂也。今由余賢,寡人之害,將奈之何。內史廖曰,戎王處辟匿,未聞中國之聲。君試遺其女樂,以奪其志,為由余請,以疏其閒,留而莫遣,以失其期。戎王怪之,必疑由余。君臣有閒,乃可虜也。且戎王好樂,必怠於政。繆公曰,善。因與由余曲席而坐,傳器而食,問其地形與其兵勢盡察,而後令內史廖以女樂二八遺戎王。戎王受而說之,終年不還。於是秦乃歸由余。由余數諫不聽,繆公又數使人閒要由余,由余遂去降秦。繆公以客禮禮之,問伐戎之形。

三十六年,繆公復益厚孟明等,使將兵伐晉,渡河焚船,大敗晉人,取王官及鄗,以報殽之役。晉人皆城守不敢出。於是繆公乃自茅津渡河,封殽中尸,為發喪,哭之三日。乃誓於軍曰,嗟士卒。聽無譁,余誓告汝。古之人謀黃髪番番,則無所過。以申思不用蹇叔、百里傒之謀,故作此誓,令後世以記余過。君子聞之,皆為垂涕,曰,嗟乎。秦繆公之與人周也,卒得孟明之慶。

三十七年,秦用由余謀伐戎王,益國十二,開地千里,遂霸西戎。天子使召公過賀繆公以金鼓。三十九年,繆公卒,葬雍。從死者百七十七人,秦之良臣子輿氏三人名曰奄息、仲行、鍼虎,亦在從死之中。秦人哀之,為作歌黃鳥之詩。君子曰,秦繆公廣地益國,東服彊晉,西霸戎夷,然不為諸侯盟主,亦宜哉。死而棄民,收其良臣而從死。且先王崩,尚猶遺德垂法,況奪之善人良臣百姓所哀者乎。是以知秦不能復東征也。繆公子四十人,其太子嵤代立,是為康公。

康公元年。往歲繆公之卒,晉襄公亦卒,襄公之弟名雍,秦出也,在秦。晉趙盾欲立之,使隨會來迎雍,秦以兵送至令狐。晉立襄公子而反擊秦師,秦師敗,隨會來奔。二年,秦伐晉,取武城,報令狐之役。四年,晉伐秦,取少梁。六年,秦伐晉,取羈馬。戰於河曲,大敗晉軍。晉人患隨會在秦為亂,乃使魏讎餘詳反,合謀會,詐而得會,會遂歸晉。康公立十二年卒,子共公立。

共公二年,晉趙穿弒其君靈公。三年,楚莊王彊,北兵至雒,問周鼎。共公立五年卒,子桓公立。

桓公三年,晉敗我一將。十年,楚莊王服鄭,北敗晉兵於河上。當是之時,楚霸,為會盟合諸侯。二十四年,晉厲公初立,與秦桓公夾河而盟。歸而秦倍盟,與翟合謀擊晉。二十六年,晉率諸侯伐秦,秦軍敗走,追至涇而還。桓公立二十七年卒,子景公立。

景公四年,晉欒書弒其君厲公。十五年,救鄭,敗晉兵於櫟。是時晉悼公為盟主。十八年,晉悼公彊,數會諸侯,率以伐秦,敗秦軍。秦軍走,晉兵追之,遂渡涇,至棫林而還。二十七年,景公如晉,與平公盟,已而背之。三十六年,楚公子圍弒其君而自立,是為靈王。景公母弟後子鍼有寵,景公母弟富,或譖之,恐誅,乃奔晉,車重千乘。晉平公曰,后子富如此,何以自亡。對曰,秦公無道,畏誅,欲待其後世乃歸。三十九年,楚靈王彊,會諸侯於申,為盟主,殺齊慶封。景公立四十年卒,子哀公立。后子復來歸秦。

哀公八年,楚公子棄疾弒靈王而自立,是為平王。十一年,楚平王來求秦女為太子建妻。至國,女好而自娶之。十五年,楚平王欲誅建,建亡,伍子胥奔吳。晉公室卑而六卿彊,欲內相攻,是以久秦晉不相攻。三十一年,吳王闔閭與伍子胥伐楚,楚王亡奔隨,吳遂入郢。楚大夫申包胥來告急,七日不食,日夜哭泣。於是秦乃發五百乘救楚,敗吳師。吳師歸,楚昭王乃得復入郢。哀公立三十六年卒。太子夷公,夷公蚤死,不得立,立夷公子,是為惠公。

惠公元年,孔子行魯相事。五年,晉卿中行、范氏反晉,晉使智氏、趙簡子攻之,范、中行氏亡奔齊。惠公立十年卒,子悼公立。

悼公二年,齊臣田乞弒其君孺子,立其兄陽生,是為悼公。六年,吳敗齊師。齊人弒悼公,立其子簡公。九年,晉定公與吳王夫差盟,爭長於黃池,卒先吳。吳彊,陵中國。十二年,齊田常弒簡公,立其弟平公,常相之。十三年,楚滅陳。秦悼公立十四年卒,子厲共公立。孔子以悼公十二年卒。

厲共公二年,蜀人來賂。十六年,塹河旁。以兵二萬伐大荔,取其王城。二十一年,初縣頻陽。晉取武成。二十四年,晉亂,殺智伯,分其國與趙、韓、魏。二十五年,智開與邑人來奔。三十三年,伐義渠,虜其王。三十四年,日食。厲共公卒,子躁公立。

躁公二年,南鄭反。十三年,義渠來伐,至渭南。十四年,躁公卒,立其弟懷公。

懷公四年,庶長朝與大臣圍懷公,懷公自殺。懷公太子曰昭子,蚤死,大臣乃立太子昭子之子,是為靈公。靈公,懷公孫也。

靈公六年,晉城少梁,秦擊之。十三年,城籍姑。靈公卒,子獻公不得立,立靈公季父悼子,是為簡公。簡公,昭子之弟而懷公子也。

簡公六年,令吏初帶劍。塹洛。城重泉。十六年卒,子惠公立。

惠公十二年,子出子生。十三年,伐蜀,取南鄭。惠公卒,出子立。

出子二年,庶長改迎靈公之子獻公于河西而立之。殺出子及其母,沈之淵旁。秦以往者數易君,君臣乖亂,故晉復彊,奪秦河西地。

獻公元年,止從死。二年,城櫟陽。四年正月庚寅,孝公生。十一年,周太史儋見獻公曰,周故與秦國合而別,別五百歲復合,合七十七歲而霸王出。十六年,桃冬花。十八年,雨金櫟陽。二十一年,與晉戰於石門,斬首六萬,天子賀以黼黻。二十三年,與魏晉戰少梁,虜其將公孫痤。二十四年,獻公卒,子孝公立,年已二十一歲矣。

孝公元年,河山以東彊國六,與齊威、楚宣、魏惠、燕悼、韓哀、趙成侯并。淮泗之閒小國十餘。楚、魏與秦接界。魏筑長城,自鄭濱洛以北,有上郡。楚自漢中,南有巴、黔中。周室微,諸侯力政,爭相併。秦僻在雍州,不與中國諸侯之會盟,夷翟遇之。孝公於是布惠,振孤寡,招戰士,明功賞。下令國中曰,昔我繆公自岐雍之閒,修德行武,東平晉亂,以河為界,西霸戎翟,廣地千里,天子致伯,諸侯畢賀,為後世開業,甚光美。會往者厲、躁、簡公、出子之不寧,國家內憂,未遑外事,三晉攻奪我先君河西地,諸侯卑秦、醜莫大焉。獻公即位,鎮撫邊境,徙治櫟陽,且欲東伐,復繆公之故地,修繆公之政令。寡人思念先君之意,常痛於心。賓客群臣有能出奇計彊秦者,吾且尊官,與之分土。於是乃出兵東圍陜城,西斬戎之獂王。

衛鞅聞是令下,西入秦,因景監求見孝公。

二年,天子致胙。

三年,衛鞅說孝公變法修刑,內務耕稼,外勸戰死之賞罰,孝公善之。甘龍、杜摯等弗然,相與爭之。卒用鞅法,百姓苦之,居三年,百姓便之。乃拜鞅為左庶長。其事在商君語中。

七年,與魏惠王會杜平。八年,與魏戰元裏,有功。十年,衛鞅為大良造,將兵圍魏安邑,降之。十二年,作為咸陽,筑冀闕,秦徙都之。并諸小鄉聚,集為大縣,縣一令,四十一縣。為田開阡陌。東地渡洛。十四年,初為賦。十九年,天子致伯。二十年,諸侯畢賀。秦使公子少官率師會諸侯逢澤,朝天子。

二十一年,齊敗魏馬陵。

二十二年,衛鞅擊魏,虜魏公子卬。封鞅為列侯,號商君。

二十四年,與晉戰雁門,虜其將魏錯。

孝公卒,子惠文君立。是歲,誅衛鞅。鞅之初為秦施法,法不行,太子犯禁。鞅曰,法之不行,自於貴戚。君必欲行法,先於太子。太子不可黥,黥其傅師。於是法大用,秦人治。及孝公卒,太子立,宗室多怨鞅,鞅亡,因以為反,而卒車裂以徇秦國。

惠文君元年,楚、韓、趙、蜀人來朝。二年,天子賀。三年,王冠。四年,天子致文武胙。齊、魏為王。

五年,陰晉人犀首為大良造。六年,魏納陰晉,陰晉更名寧秦。七年,公子卬與魏戰,虜其將龍賈,斬首八萬。八年,魏納河西地。九年,渡河,取汾陰、皮氏。與魏王會應。圍焦,降之。十年,張儀相秦。魏納上郡十五縣。十一年,縣義渠。歸魏焦、曲沃。義渠君為臣。更名少梁曰夏陽。十二年,初臘。十三年四月戊午,魏君為王,韓亦為王。使張儀伐取陜,出其人與魏。

十四年,更為元年。二年,張儀與齊、楚大臣會齧桑。三年,韓、魏太子來朝。張儀相魏。五年,王游至北河。七年,樂池相秦。韓、趙、魏、燕、齊帥匈奴共攻秦。秦使庶長疾與戰修魚,虜其將申差,敗趙公子渴、韓太子奐,斬首八萬二千。八年,張儀復相秦。九年,司馬錯伐蜀,滅之。伐取趙中都、西陽十年,韓太子蒼來質。伐取韓石章。伐敗趙將泥。伐取義渠二十五城。十一年,摢裏疾攻魏焦,降之。敗韓岸門,斬首萬,其將犀首走。公子通封於蜀。燕君讓其臣子之。十二年,王與梁王會臨晉。庶長疾攻趙,虜趙將莊。張儀相楚。十三年,庶長章擊楚於丹陽,虜其將屈丐,斬首八萬,又攻楚漢中,取地六百里,置漢中郡。楚圍雍氏,秦使庶長疾助韓而東攻齊,到滿助魏攻燕。十四年,伐楚,取召陵。丹、犁臣,蜀相壯殺蜀侯來降。

惠王卒,子武王立。韓、魏、齊、楚、越皆賓從。

武王元年,與魏惠王會臨晉。誅蜀相壯。張儀、魏章皆東出之魏。伐義渠、丹、犁。二年,初置丞相,摢裏疾、甘茂為左右丞相。張儀死於魏。三年,與韓襄王會臨晉外。南公揭卒,摢裏疾相韓。武王謂甘茂曰,寡人欲容車通三川,窺周室,死不恨矣。其秋,使甘茂、庶長封伐宜陽。四年,拔宜陽,斬首六萬。涉河,城武遂。魏太子來朝。武王有力好戲,力士任鄙、烏獲、孟說皆至大官。王與孟說舉鼎,絕臏。八月,武王死。族孟說。武王取魏女為后,無子。立異母弟,是為昭襄王。昭襄母楚人,姓羋氏,號宣太后。武王死時,昭襄王為質於燕,燕人送歸,得立。

昭襄王元年,嚴君疾為相。甘茂出之魏。二年,彗星見。庶長壯與大臣、諸侯、公子為逆,皆誅,及惠文后皆不得良死。悼武王后出歸魏。三年,王冠。與楚王會黃棘,與楚上庸。四年,取蒲阪。彗星見。五年,魏王來朝應亭,復與魏蒲阪。六年,蜀侯煇反,司馬錯定蜀。庶長奐伐楚,斬首二萬。涇陽君質於齊。日食,晝晦。七年,拔新城。摢裏子卒。八年,使將軍羋戎攻楚,取新市。齊使章子,魏使公孫喜,韓使暴鳶共攻楚方城,取唐眛。趙破中山,其君亡,竟死齊。魏公子勁、韓公子長為諸侯。九年,孟嘗君薛文來相秦。奐攻楚,取八城,殺其將景快。十年,楚懷王入朝秦,秦留之。薛文以金受免。樓緩為丞相。十一年,齊、韓、魏、趙、宋、中山五國共攻秦,至鹽氏而還。秦與韓、魏河北及封陵以和。彗星見。楚懷王走之趙,趙不受,還之秦,即死,歸葬。十二年,樓緩免,穰侯魏冉為相。予楚粟五萬石。

十三年,向壽伐韓,取武始。左更白起攻新城。五大夫禮出亡奔魏。任鄙為漢中守。十四年,左更白起攻韓、魏於伊闕,斬首二十四萬,虜公孫喜,拔五城。十五年,大良造白起攻魏,取垣,復予之。攻楚,取宛。十六年,左更錯取軹及鄧。冉免,封公子市宛,公子悝鄧,魏冉陶,為諸侯。十七年,城陽君入朝,及東周君來朝。秦以垣為蒲阪、皮氏。王之宜陽。十八年,錯攻垣、河雍,決橋取之。十九年,王為西帝,齊為東帝,皆復去之。呂禮來自歸。齊破宋,宋王在魏,死溫。任鄙卒。二十年,王之漢中,又之上郡、北河。二十一年,錯攻魏河內。魏獻安邑,秦出其人,募徙河東賜爵,赦罪人遷之。涇陽君封宛。二十二年,蒙武伐齊。河東為九縣。與楚王會宛。與趙王會中陽。二十三年,尉斯離與三晉、燕伐齊,破之濟西。王與魏王會宜陽,與韓王會新城。二十四年,與楚王會鄢,又會穰。秦取魏安城,至大梁,燕、趙救之,秦軍去。魏冉免相。二十五年,拔趙二城。與韓王會新城,與魏王會新明邑。二十六年,赦罪人遷之穰。侯冉復相。二十七年,錯攻楚。赦罪人遷之南陽。白起攻趙,取代光狼城。又使司馬錯發隴西,因蜀攻楚黔中,拔之。二十八年,大良造白起攻楚,取鄢、鄧,赦罪人遷之。二十九年,大良造白起攻楚,取郢為南郡,楚王走。周君來。王與楚王會襄陵。白起為武安君。三十年,蜀守若伐楚,取巫郡,及江南為黔中郡。三十一年,白起伐魏,取兩城。楚人反我江南。三十二年,相穰侯攻魏,至大梁,破暴鳶,斬首四萬,鳶走,魏入三縣請和。三十三年,客卿胡傷陽攻魏卷、蔡陽、長社,取之。擊芒卯華陽,破之,斬首十五萬。魏入南陽以和。三十四年,秦與魏、韓上庸地為一郡,南陽免臣遷居之。三十五年,佐韓、魏、楚伐燕。初置南陽郡。三十六年,客卿灶攻齊,取剛、壽,予穰侯。三十八年,中更胡傷陽攻趙閼與,不能取。四十年,悼太子死魏,歸葬芷陽。四十一年夏,攻魏,取邢丘、懷。四十二年,安國君為太子。十月,宣太后薨,葬芷陽酈山。九月,穰侯出之陶。四十三年,武安君白起攻韓,拔九城,斬首五萬。四十四年,攻韓南郡陽,取之。四十五年,五大夫賁攻韓,取十城。葉陽君悝出之國,未至而死。四十七年,秦攻韓上黨,上黨降趙,秦因攻趙,趙發兵擊秦,相距。秦使武安君白起擊,大破趙於長平,四十餘萬盡殺之。四十八年十月,韓獻垣雍。秦軍分為三軍。武安君歸。王龁將伐趙武安皮牢,拔之。司馬梗北定太原,盡有韓上黨。正月,兵罷,復守上黨。其十月,五大夫陵攻趙邯鄲。四十九年正月,益發卒佐陵。陵戰不善,免,王龁代將。其十月,將軍張唐攻魏,為蔡尉捐弗守,還斬之。五十年十月,武安君白起有罪,為士伍,遷陰密。張唐攻鄭,拔之。十二月,益發卒軍汾城旁。武安君白起有罪,死。龁攻邯鄲,不拔,去,還奔汾軍。二月餘攻晉軍,斬首六千,晉楚流死河二萬人。攻汾城,即從唐拔寧新中,寧新中更名安陽。初作河橋。

五十一年,將軍摎攻韓,取陽城、負黍,斬首四萬。攻趙,取二十餘縣,首虜九萬。西周君背秦,與諸侯約從,將天下銳兵出伊闕攻秦,令秦毋得通陽城。於是秦使將軍摎攻西周。西周君走來自歸,頓首受罪,盡獻其邑三十六城,口三萬。秦王受獻,歸其君於周。五十二年,周民東亡,其器九鼎入秦。周初亡。

五十三年,天下來賓。魏後,秦使摎伐魏,取吳城。韓王入朝,魏委國聽令。五十四年,王郊見上帝於雍。五十六年秋,昭襄王卒,子孝文王立。尊唐八子為唐太后,而合其葬於先王。韓王衰绖入弔祠,諸侯皆使其將相來弔祠,視喪事。

孝文王元年,赦罪人,修先王功臣,褒厚親戚,弛苑囿。孝文王除喪,十月己亥即位,三日辛醜卒,子莊襄王立。

莊襄王元年,大赦罪人,修先王功臣,施德厚骨肉而布惠於民。東周君與諸侯謀秦,秦使相國呂不韋誅之,盡入其國。秦不絕其祀,以陽人地賜周君,奉其祭祀。使蒙驁伐韓,韓獻成皋、鞏。秦界至大梁,初置三川郡。二年,使蒙驁攻趙,定太原。三年,蒙驁攻魏高都、汲,拔之。攻趙榆次、新城、狼孟,取三十七城。四月日食。四年王龁攻上黨。初置太原郡。魏將無忌率五國兵擊秦,秦卻於河外。蒙驁敗,解而去。五月丙午,莊襄王卒,子政立,是為秦始皇帝。

秦王政立二十六年,初并天下為三十六郡,號為始皇帝。始皇帝五十一年而崩,子胡亥立,是為二世皇帝。三年,諸侯并起叛秦,趙高殺二世,立子嬰。子嬰立月餘,諸侯誅之,遂滅秦。其語在始皇本紀中。

太史公曰,秦之先為嬴姓。其後分封,以國為姓,有徐氏、郯氏、莒氏、終黎氏、運奄氏、菟裘氏、將梁氏、黃氏、江氏、修魚氏、白冥氏、蜚廉氏、秦氏。然秦以其先造父封趙城,為趙氏。