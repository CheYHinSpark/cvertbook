\chapter{蕭相國世家第二十三}

蕭相國何者,沛豐人也。以文無害為沛主吏掾。

高祖為布衣時,何數以吏事護高祖。高祖為亭長,常左右之。高祖以吏繇咸陽,吏皆送奉錢三,何獨以五。

秦御史監郡者與從事,常辨之。何乃給泗水卒史事,第一。秦御史欲入言徵何,何固請,得毋行。

及高祖起為沛公,何常為丞督事。沛公至咸陽,諸將皆爭走金帛財物之府分之,何獨先入收秦丞相御史律令圖書藏之。沛公為漢王,以何為丞相。項王與諸侯屠燒咸陽而去。漢王所以具知天下阸塞,戶口多少,彊弱之處,民所疾苦者,以何具得秦圖書也。何進言韓信,漢王以信為大將軍。語在淮陰侯事中。

漢王引兵東定三秦,何以丞相留收巴蜀,填撫諭告,使給軍食。漢二年,漢王與諸侯擊楚,何守關中,侍太子,治櫟陽。為法令約束,立宗廟社稷宮室縣邑,輒奏上,可,許以從事,即不及奏上,輒以便宜施行,上來以聞。關中事計戶口轉漕給軍,漢王數失軍遁去,何常興關中卒,輒補缺。上以此專屬任何關中事。

漢三年,漢王與項羽相距京索之閒,上數使使勞苦丞相。鮑生謂丞相曰,王暴衣露蓋,數使使勞苦君者,有疑君心也。為君計,莫若遣君子孫昆弟能勝兵者悉詣軍所,上必益信君。於是何從其計,漢王大說。

漢五年,既殺項羽,定天下,論功行封。群臣爭功,歲餘功不決。高祖以蕭何功最盛,封為酂侯,所食邑多。功臣皆曰,臣等身被堅執銳,多者百餘戰,少者數十合,攻城略地,大小镑有差。今蕭何未嘗有汗馬之勞,徒持文墨議論,不戰,顧反居臣等上,何也。高帝曰,諸君知獵乎。曰,知之。知獵狗乎。曰,知之。高帝曰,夫獵,追殺獸兔者狗也,而發蹤指示獸處者人也。今諸君徒能得走獸耳,功狗也。至如蕭何,發蹤指示,功人也。且諸君獨以身隨我,多者兩三人。今蕭何舉宗數十人皆隨我,功不可忘也。群臣皆莫敢言。

列侯畢已受封,及奏位次,皆曰,平陽侯曹參身被七十創,攻城略地,功最多,宜第一。上已橈功臣,多封蕭何,至位次未有以復難之,然心欲何第一。關內侯鄂君進曰,群臣議皆誤。夫曹參雖有野戰略地之功,此特一時之事。夫上與楚相距五歲,常失軍亡眾,逃身遁者數矣。然蕭何常從關中遣軍補其處,非上所詔令召,而數萬眾會上之乏絕者數矣。夫漢與楚相守滎陽數年,軍無見糧,蕭何轉漕關中,給食不乏。陛下雖數亡山東,蕭何常全關中以待陛下,此萬世之功也。今雖亡曹參等百數,何缺於漢。漢得之不必待以全。柰何欲以一旦之功而加萬世之功哉。蕭何第一,曹參次之。高祖曰,善。於是乃令蕭何第一,賜帶劍履上殿,入朝不趨。

上曰,吾聞進賢受上賞。蕭何功雖高,得鄂君乃益明。於是因鄂君故所食關內侯邑封為安平侯。是日,悉封何父子兄弟十餘人,皆有食邑。乃益封何二千戶,以帝嘗繇咸陽時何送我獨贏錢二也。

漢十一年,陳豨反,高祖自將,至邯鄲。未罷,淮陰侯謀反關中,呂后用蕭何計,誅淮陰侯,語在淮陰事中。上已聞淮陰侯誅,使使拜丞相何為相國,益封五千戶,令卒五百人一都尉為相國衛。諸君皆賀,召平獨弔。召平者,故秦東陵侯。秦破,為布衣,貧,種瓜於長安城東,瓜美,故世俗謂之東陵瓜,從召平以為名也。召平謂相國曰,禍自此始矣。上暴露於外而君守於中,非被矢石之事而益君封置衛者,以今者淮陰侯新反於中,疑君心矣。夫置衛衛君,非以寵君也。願君讓封勿受,悉以家私財佐軍,則上心說。相國從其計,高帝乃大喜。

漢十二年秋,黥布反,上自將擊之,數使使問相國何為。相國為上在軍,乃拊循勉力百姓,悉以所有佐軍,如陳豨時。客有說相國曰,君滅族不久矣。夫君位為相國,功第一,可復加哉。然君初入關中,得百姓心,十餘年矣,皆附君,常復孳孳得民和。上所為數問君者,畏君傾動關中。今君胡不多買田地,賤貰貸以自汙。上心乃安。於是相國從其計,上乃大說。

上罷布軍歸,民道遮行上書,言相國賤彊買民田宅數千萬。上至,相國謁。上笑曰,夫相國乃利民。民所上書皆以與相國,曰,君自謝民。相國因為民請曰,長安地狹,上林中多空地,棄,願令民得入田,毋收槁為禽獸食。上大怒曰,相國多受賈人財物,乃為請吾苑。乃下相國廷尉,械系之。數日,王衛尉侍,前問曰,相國何大罪,陛下系之暴也。上曰,吾聞李斯相秦皇帝,有善歸主,有惡自與。今相國多受賈豎金而為民請吾苑,以自媚於民,故系治之。王衛尉曰,夫職事茍有便於民而請之,真宰相事,陛下柰何乃疑相國受賈人錢乎。且陛下距楚數歲,陳豨、黥布反,陛下自將而往,當是時,相國守關中,搖足則關以西非陛下有也。相國不以此時為利,今乃利賈人之金乎。且秦以不聞其過亡天下,李斯之分過,又何足法哉。陛下何疑宰相之淺也。高帝不懌。是日,使使持節赦出相國。相國年老,素恭謹,入,徒跣謝。高帝曰,相國休矣。相國為民請苑,吾不許,我不過為桀紂主,而相國為賢相。吾故系相國,欲令百姓聞吾過也。

何素不與曹參相能,及何病,孝惠自臨視相國病,因問曰,君即百歲後,誰可代君者。對曰,知臣莫如主。孝惠曰,曹參何如。何頓首曰,帝得之矣。臣死不恨矣。

何置田宅必居窮處,為家不治垣屋。曰,後世賢,師吾儉,不賢,毋為勢家所奪。

孝惠二年,相國何卒,謚為文終侯。

後嗣以罪失侯者四世,絕,天子輒復求何後,封續酂侯,功臣莫得比焉。

太史公曰,蕭相國何於秦時為刀筆吏,錄錄未有奇節。及漢興,依日月之末光,何謹守管籥,因民之疾秦法,順流與之更始。淮陰、黥布等皆以誅滅,而何之勳爛焉。位冠群臣,聲施後世,與閎夭、散宜生等爭烈矣。