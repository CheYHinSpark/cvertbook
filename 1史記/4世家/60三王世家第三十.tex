\chapter{三王世家第三十}

大司馬臣去病昧死再拜上疏皇帝陛下,陛下過聽,使臣去病待罪行閒。宜專邊塞之思慮,暴骸中野無以報,乃敢惟他議以干用事者,誠見陛下憂勞天下,哀憐百姓以自忘,虧膳貶樂,損郎員。皇子賴天,能勝衣趨拜,至今無號位師傅官。陛下恭讓不恤,群臣私望,不敢越職而言。臣竊不勝犬馬心,昧死願陛下詔有司,因盛夏吉時定皇子位。唯陛下幸察。臣去病昧死再拜以聞皇帝陛下。三月乙亥,御史臣光守尚書令奏未央宮。制曰,下御史。

六年三月戊申朔,乙亥,御史臣光守尚書令、丞非,下御史書到,言,丞相臣青翟、御史大夫臣湯、太常臣充、大行令臣息、太子少傅臣安行宗正事昧死上言,大司馬去病上疏曰,陛下過聽,使臣去病待罪行閒。宜專邊塞之思慮,暴骸中野無以報,乃敢惟他議以干用事者,誠見陛下憂勞天下,哀憐百姓以自忘,虧膳貶樂,損郎員。皇子賴天,能勝衣趨拜,至今無號位師傅官。陛下恭讓不恤,群臣私望,不敢越職而言。臣竊不勝犬馬心,昧死願陛下詔有司,因盛夏吉時定皇子位。唯願陛下幸察。制曰下御史。臣謹與中二千石、二千石臣賀等議,古者裂地立國,并建諸侯以承天于,所以尊宗廟重社稷也。今臣去病上疏,不忘其職,因以宣恩,乃道天子卑讓自貶以勞天下,慮皇子未有號位。臣青翟、臣湯等宜奉義遵職,愚憧而不逮事。方今盛夏吉時,臣青翟、臣湯等昧死請立皇子臣閎、臣旦、臣胥為諸侯王。昧死請所立國名。

制曰,蓋聞周封八百,姬姓并列,或子、男、附庸。禮支子不祭。云并建諸侯所以重社稷,朕無聞焉。且天非為君生民也。朕之不德,海內未洽,乃以未教成者彊君連城,即股肱何勸。其更議以列侯家之。

三月丙子,奏未央宮。丞相臣青翟、御史大夫臣湯昧死言,臣謹與列侯臣嬰齊、中二千石二千石臣賀、諫大夫博士臣安等議曰,伏聞周封八百,姬姓并列,奉承天子。康叔以祖考顯,而伯禽以周公立,咸為建國諸侯,以相傅為輔。百官奉憲,各遵其職,而國統備矣。竊以為并建諸侯所以重社稷者,四海諸侯各以其職奉貢祭。支子不得奉祭宗祖,禮也。封建使守藩國,帝王所以扶德施化。陛下奉承天統,明開聖緒,尊賢顯功,興滅繼絕。續蕭文終之後於酂,褒厲群臣平津侯等。昭六親之序,明天施之屬,使諸侯王封君得推私恩分子弟戶邑,錫號尊建百有餘國。而家皇子為列侯,則尊卑相踰,列位失序,不可以垂統於萬世。臣請立臣閎、臣旦、臣胥為諸侯王。三月丙子,奏未央宮。

制曰,康叔親屬有十而獨尊者,褒有德也。周公祭天命郊,故魯有白牡、騂剛之牲。群公不毛,賢不肖差也。高山仰之,景行向之,朕甚慕焉。所以抑未成,家以列侯可。

四月戊寅,奏未央宮。丞相臣青翟、御史大夫臣湯昧死言,臣青翟等與列侯、吏二千石、諫大夫、博士臣慶等議,昧死奏請立皇子為諸侯王。制曰,康叔親屬有十而獨尊者,褒有德也。周公祭天命郊,故魯有白牡、騂剛之牲。群公不毛,賢不肖差也。高山仰之,景行向之,朕甚慕焉。所以抑未成,家以列侯可。臣青翟、臣湯、博士臣將行等伏聞康叔親屬有十,武王繼體,周公輔成王,其八人皆以祖考之尊建為大國。康叔之年幼,周公在三公之位,而伯禽據國於魯,蓋爵命之時,未至成人。康叔後捍祿父之難,伯禽殄淮夷之亂。昔五帝異制,周爵五等,春秋三等,皆因時而序尊卑。高皇帝撥亂世反諸正,昭至德,定海內,封建諸侯,爵位二等。皇子或在繦緥而立為諸侯王,奉承天子,為萬世法則,不可易。陛下躬親仁義,體行聖德,表裏文武。顯慈孝之行,廣賢能之路。內褒有德,外討彊暴。極臨北海,西溱月氏,匈奴、西域,舉國奉師。輿械之費,不賦於民。虛御府之藏以賞元戎,開禁倉以振貧窮,減戍卒之半。百蠻之君,靡不鄉風,承流稱意。遠方殊俗,重譯而朝,澤及方外。故珍獸至,嘉穀興,天應甚彰。今諸侯支子封至諸侯王,而家皇子為列侯,臣青翟、臣湯等竊伏孰計之,皆以為尊卑失序,使天下失望,不可。臣請立臣閎、臣旦、臣胥為諸侯王。四月癸未,奏未央宮,留中不下。

丞相臣青翟、太仆臣賀、行御史大夫事太常臣充、太子少傅臣安行宗正事昧死言,臣青翟等前奏大司馬臣去病上疏言,皇子未有號位,臣謹與御史大夫臣湯、中二千石、二千石、諫大夫、博士臣慶等昧死請立皇子臣閎等為諸侯王。陛下讓文武,躬自切,及皇子未教。群臣之議,儒者稱其術,或誖其心。陛下固辭弗許,家皇子為列侯。臣青翟等竊與列侯臣壽成等二十七人議,皆曰以為尊卑失序。高皇帝建天下,為漢太祖,王子孫,廣支輔。先帝法則弗改,所以宣至尊也。臣請令史官擇吉日,具禮儀上,御史奏輿地圖,他皆如前故事。制曰,可。

四月丙申,奏未央宮。太仆臣賀行御史大夫事昧死言,太常臣充言卜入四月二十八日乙巳,可立諸侯王。臣昧死奏輿地圖,請所立國名。禮儀別奏。臣昧死請。

制曰,立皇子閎為齊王,旦為燕王,胥為廣陵王。

四月丁酉,奏未央宮。六年四月戊寅朔,癸卯,御史大夫湯下丞相,丞相下中二千石,二千石下郡太守、諸侯相,丞書從事下當用者。如律令。

維六年四月乙巳,皇帝使御史大夫湯廟立子閎為齊王。曰,於戲,小子閎,受茲青社。朕承祖考,維稽古建爾國家,封于東土,世為漢藩輔。於戲念哉。抱朕之詔,惟命不于常。人之好德,克明顯光。義之不圖,俾君子怠。悉爾心,允執其中,天祿永終。厥有愆不臧,乃凶于而國,害于爾躬。於戲,保國艾民,可不敬與。王其戒之。

右齊王策。

維六年四月乙巳,皇帝使御史大夫湯廟立子旦為燕王。曰,於戲,小子旦,受茲玄社。朕承祖考,維稽古,建爾國家,封于北土,世為漢藩輔。於戲。葷粥氏虐老獸心,侵犯寇盜,加以姦巧邊萌。於戲。朕命將率徂征厥罪,萬夫長,千夫長,三十有二君皆來,降期奔師。葷粥徙域,北州以綏。悉爾心,毋作怨,毋卹德,毋乃廢備。非教士不得從徵。於戲,保國艾民,可不敬與。王其戒之。

右燕王策。

維六年四月乙巳,皇帝使御史大夫湯廟立子胥為廣陵王。曰,於戲,小子胥,受茲赤社。朕承祖考,維稽古建爾國家,封于南土,世為漢藩輔。古人有言曰,大江之南,五湖之閒,其人輕心。楊州保疆,三代要服,不及以政。於戲。悉爾心,戰戰兢兢,乃惠乃順,毋侗好軼,毋邇宵人,維法維則。書云,臣不作威,不作福,靡有後羞。於戲,保國艾民,可不敬與。王其戒之。

右廣陵王策。

太史公曰,古人有言曰愛之欲其富,親之欲其貴。故王者壃土建國,封立子弟,所以褒親親,序骨肉,尊先祖,貴支體,廣同姓於天下也。是以形勢彊而王室安。自古至今,所由來久矣。非有異也,故弗論箸也。燕齊之事,無足采者。然封立三王,天子恭讓,群臣守義,文辭爛然,甚可觀也,是以附之世家。

褚先生曰,臣幸得以文學為侍郎,好覽觀太史公之列傳。傳中稱三王世家文辭可觀,求其世家終不能得。竊從長老好故事者取其封策書,編列其事而傳之,令後世得觀賢主之指意。

蓋聞孝武帝之時,同日而俱拜三子為王,封一子於齊,一子於廣陵,一子於燕。各因子才力智能,及土地之剛柔,人民之輕重,為作策以申戒之。謂王,世為漢藩輔,保國治民,可不敬與。王其戒之。夫賢主所作,固非淺聞者所能知,非博聞彊記君子者所不能究竟其意。至其次序分絕,文字之上下,簡之參差長短,皆有意,人莫之能知。謹論次其真草詔書,編于左方。令覽者自通其意而解說之。

王夫人者,趙人也,與衛夫人并幸武帝,而生子閎。閎且立為王時,其母病,武帝自臨問之。曰,子當為王,欲安所置之。王夫人曰,陛下在,妾又何等可言者。帝曰,雖然,意所欲,欲於何所王之。王夫人曰,願置之雒陽。武帝曰,雒陽有武庫敖倉,天下衝阸,漢國之大都也。先帝以來,無子王於雒陽者。去雒陽,餘盡可。王夫人不應。武帝曰,關東之國無大於齊者。齊東負海而城郭大,古時獨臨菑中十萬戶,天下膏腴地莫盛於齊者矣。王夫人以手擊頭,謝曰,幸甚。王夫人死而帝痛之,使使者拜之曰,皇帝謹使使太中大夫明奉璧一,賜夫人為齊王太后。子閎王齊,年少,無有子,立,不幸早死,國絕,為郡。天下稱齊不宜王云。

所謂受此土者,諸侯王始封者必受土於天子之社,歸立之以為國社,以歲時祠之。春秋大傳曰,天子之國有泰社。東方青,南方赤,西方白,北方黑,上方黃。故將封於東方者取青土,封於南方者取赤土,封於西方者取白土,封於北方者取黑土,封於上方者取黃土。各取其色物,裹以白茅,封以為社。此始受封於天子者也。此之為主土。主土者,立社而奉之也。朕承祖考,祖者先也,考者父也。維稽古,維者度也,念也,稽者當也,當順古之道也。

齊地多變詐,不習於禮義,故戒之曰恭朕之詔,唯命不可為常。人之好德,能明顯光。不圖於義,使君子怠慢。悉若心,信執其中,天祿長終。有過不善,乃凶于而國,而害于若身。齊王之國,左右維持以禮義,不幸中年早夭。然全身無過,如其策意。

傳曰青采出於藍,而質青於藍者,教使然也。遠哉賢主,昭然獨見,誡齊王以慎內,誡燕王以無作怨,無修德,誡廣陵王以慎外,無作威與福。

夫廣陵在吳越之地,其民精而輕,故誡之曰江湖之閒,其人輕心。楊州葆疆,三代之時,迫要使從中國俗服,不大及以政教,以意御之而已。無侗好佚,無邇宵人,維法是則。無長好佚樂馳騁弋獵淫康,而近小人。常念法度,則無羞辱矣。三江、五湖有魚鹽之利,銅山之富,天下所仰。故誡之曰臣不作福者,勿使行財幣,厚賞賜,以立聲譽,為四方所歸也。又曰臣不作威者,勿使因輕以倍義也。

會孝武帝崩,孝昭帝初立,先朝廣陵王胥,厚賞賜金錢財幣,直三千餘萬,益地百里,邑萬戶。

會昭帝崩,宣帝初立,緣恩行義,以本始元年中,裂漢地,盡以封廣陵王胥四子,一子為朝陽侯,一子為平曲侯,一子為南利侯,最愛少子弘,立以為高密王。

其後胥果作威福,通楚王使者。楚王宣言曰,我先元王,高帝少弟也,封三十二城。今地邑益少,我欲與廣陵王共發兵云。立廣陵王為上,我復王楚三十二城,如元王時。事發覺,公卿有司請行罰誅。天子以骨肉之故,不忍致法於胥,下詔書無治廣陵王,獨誅首惡楚王。傳曰蓬生麻中,不扶自直,白沙在泥中,與之皆黑者,土地教化使之然也。其後胥復祝詛謀反,自殺,國除。

燕土墝埆,北迫匈奴,其人民勇而少慮,故誡之曰葷粥氏無有孝行而禽獸心,以竊盜侵犯邊民。朕詔將軍往征其罪,萬夫長,千夫長,三十有二君皆來,降旗奔師。葷粥徙域遠處,北州以安矣。悉若心,無作怨者,勿使從俗以怨望也。無俷德者,勿使王背德也。無廢備者,無乏武備,常備匈奴也。非教士不得從徵者,言非習禮義不得在於側也。

會武帝年老長,而太子不幸薨,未有所立,而旦使來上書,請身入宿衛於長安。孝武見其書,擊地,怒曰,生子當置之齊魯禮義之鄉,乃置之燕趙,果有爭心,不讓之端見矣。於是使使即斬其使者於闕下。

會武帝崩,昭帝初立,旦果作怨而望大臣。自以長子當立,與齊王子劉澤等謀為叛逆,出言曰,我安得弟在者。今立者乃大將軍子也。欲發兵。事發覺,當誅。昭帝緣恩寬忍,抑案不揚。公卿使大臣請,遣宗正與太中大夫公戶滿意、御史二人,偕往使燕,風喻之。到燕,各異日,更見責王。宗正者,主宗室諸劉屬籍,先見王,為列陳道昭帝實武帝子狀。侍御史乃復見王,責之以正法,問,王欲發兵罪名明白,當坐之。漢家有正法,王犯纖介小罪過,即行法直斷耳,安能寬王。驚動以文法。王意益下,心恐。公戶滿意習於經術,最後見王,稱引古今通義,國家大禮,文章爾雅。謂王曰,古者天子必內有異姓大夫,所以正骨肉也,外有同姓大夫,所以正異族也。周公輔成王,誅其兩弟,故治。武帝在時,尚能寬王。今昭帝始立,年幼,富於春秋,未臨政,委任大臣。古者誅罰不阿親戚,故天下治。方今大臣輔政,奉法直行,無敢所阿,恐不能寬王。王可自謹,無自令身死國滅,為天下笑。於是燕王旦乃恐懼服罪,叩頭謝過。大臣欲和合骨肉,難傷之以法。

其後旦復與左將軍上官桀等謀反,宣言曰我次太子,太子不在,我當立,大臣共抑我云云。大將軍光輔政,與公卿大臣議曰,燕王旦不改過悔正,行惡不變。於是修法直斷,行罰誅。旦自殺,國除,如其策指。有司請誅旦妻子。孝昭以骨肉之親,不忍致法,寬赦旦妻子,免為庶人。傳曰蘭根與白芷,漸之滫中,君子不近,庶人不服者,所以漸然也。

宣帝初立,推恩宣德,以本始元年中盡復封燕王旦兩子,一子為安定侯,立燕故太子建為廣陽王,以奉燕王祭祀。