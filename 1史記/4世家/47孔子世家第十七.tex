\chapter{孔子世家第十七}

孔子生魯昌平鄉陬邑。其先宋人也,曰孔防叔。防叔生伯夏,伯夏生叔梁紇。紇與顏氏女野合而生孔子,禱於尼丘得孔子。魯襄公二十二年而孔子生。生而首上圩頂,故因名曰丘云。字仲尼,姓孔氏。

丘生而叔梁紇死,葬於防山。防山在魯東,由是孔子疑其父墓處,母諱之也。孔子為兒嬉戲,常陳俎豆,設禮容。孔子母死,乃殯五父之衢,蓋其慎也。郰人輓父之母誨孔子父墓,然後往合葬於防焉。

孔子要絰,季氏饗士,孔子與往。陽虎絀曰,季氏饗士,非敢饗子也。孔子由是退。

孔子年十七,魯大夫孟釐子病且死,誡其嗣懿子曰,孔丘,聖人之後,滅於宋。其祖弗父何始有宋而嗣讓厲公。及正考父佐戴、武、宣公,三命茲益恭,故鼎銘云,一命而僂,再命而傴,三命而俯,循墻而走,亦莫敢余侮。饘於是,粥於是,以餬余口。其恭如是。吾聞聖人之後,雖不當世,必有達者。今孔丘年少好禮,其達者歟。吾即沒,若必師之。及釐子卒,懿子與魯人南宮敬叔往學禮焉。是歲,季武子卒,平子代立。

孔子貧且賤。及長,嘗為季氏史,料量平,嘗為司職吏而畜蕃息。由是為司空。已而去魯,斥乎齊,逐乎宋、衛,困於陳蔡之間,於是反魯。孔子長九尺有六寸,人皆謂之長人而異之。魯復善待,由是反魯。

魯南宮敬叔言魯君曰,請與孔子適周。魯君與之一乘車,兩馬,一豎子俱,適周問禮,蓋見老子云。辭去,而老子送之曰,吾聞富貴者送人以財,仁人者送人以言。吾不能富貴,竊仁人之號,送子以言,曰,聰明深察而近於死者,好議人者也。博辯廣大危其身者,發人之惡者也。為人子者毋以有己,為人臣者毋以有己。孔子自周反于魯,弟子稍益進焉。

是時也,晉平公淫,六卿擅權,東伐諸侯,楚靈王兵彊,陵轢中國,齊大而近於魯。魯小弱,附於楚則晉怒,附於晉則楚來伐,不備於齊,齊師侵魯。

魯昭公之二十年,而孔子蓋年三十矣。齊景公與晏嬰來適魯,景公問孔子曰,昔秦穆公國小處辟,其霸何也。對曰,秦,國雖小,其志大,處雖辟,行中正。身舉五羖,爵之大夫,起纍紲之中,與語三日,授之以政。以此取之,雖王可也,其霸小矣。景公說。

孔子年三十五,而季平子與郈昭伯以鬬雞故得罪魯昭公,昭公率師擊平子,平子與孟氏、叔孫氏三家共攻昭公,昭公師敗,奔於齊,齊處昭公乾侯。其後頃之,魯亂。孔子適齊,為高昭子家臣,欲以通乎景公。與齊太師語樂,聞韶音,學之,三月不知肉味,齊人稱之。

景公問政孔子,孔子曰,君君,臣臣,父父,子子。景公曰,善哉。信如君不君,臣不臣,父不父,子不子,雖有粟,吾豈得而食諸。他日又復問政於孔子,孔子曰,政在節財。景公說,將欲以尼谿田封孔子。晏嬰進曰,夫儒者滑稽而不可軌法,倨傲自順,不可以為下,崇喪遂哀,破產厚葬,不可以為俗,游說乞貸,不可以為國。自大賢之息,周室既衰,禮樂缺有間。今孔子盛容飾,繁登降之禮,趨詳之節,累世不能殫其學,當年不能究其禮。君欲用之以移齊俗,非所以先細民也。後景公敬見孔子,不問其禮。異日,景公止孔子曰,奉子以季氏,吾不能。以季孟之間待之。齊大夫欲害孔子,孔子聞之。景公曰,吾老矣,弗能用也。孔子遂行,反乎魯。

孔子年四十二,魯昭公卒於乾侯,定公立。定公立五年,夏,季平子卒,桓子嗣立。季桓子穿井得土缶,中若羊,問仲尼云得狗。仲尼曰,以丘所聞,羊也。丘聞之,木石之怪夔、罔閬,水之怪龍、罔象,土之怪墳羊。

吳伐越,墮會稽,得骨節專車。吳使使問仲尼,骨何者最大。仲尼曰,禹致群神於會稽山,防風氏後至,禹殺而戮之,其節專車,此為大矣。吳客曰,誰為神。仲尼曰,山川之神足以綱紀天下,其守為神,社稷為公侯,皆屬於王者。客曰,防風何守。仲尼曰,汪罔氏之君守封、禺之山,為釐姓。在虞、夏、商為汪罔,於周為長翟,今謂之大人。客曰,人長幾何。仲尼曰,僬僥氏三尺,短之至也。長者不過十之,數之極也。於是吳客曰,善哉聖人。

桓子嬖臣曰仲梁懷,與陽虎有隙。陽虎欲逐懷,公山不狃止之。其秋,懷益驕,陽虎執懷。桓子怒,陽虎因囚桓子,與盟而醳之。陽虎由此益輕季氏。季氏亦僭於公室,陪臣執國政,是以魯自大夫以下皆僭離於正道。故孔子不仕,退而脩詩書禮樂,弟子彌眾,至自遠方,莫不受業焉。

定公八年,公山不狃不得意於季氏,因陽虎為亂,欲廢三桓之適,更立其庶孽陽虎素所善者,遂執季桓子。桓子詐之,得脫。定公九年,陽虎不勝,奔于齊。是時孔子年五十。

公山不狃以費畔季氏,使人召孔子。孔子循道彌久,溫溫無所試,莫能己用,曰,蓋周文武起豐鎬而王,今費雖小,儻庶幾乎。欲往。子路不說,止孔子。孔子曰,夫召我者豈徒哉。如用我,其為東周乎。然亦卒不行。

其後定公以孔子為中都宰,一年,四方皆則之。由中都宰為司空,由司空為大司寇。

定公十年春,及齊平。夏,齊大夫黎鉏言於景公曰,魯用孔丘,其勢危齊。乃使使告魯為好會,會於夾谷。魯定公且以乘車好往。孔子攝相事,曰,臣聞有文事者必有武備,有武事者必有文備。古者諸侯出疆,必具官以從。請具左右司馬。定公曰,諾。具左右司馬。會齊侯夾谷,為壇位,土階三等,以會遇之禮相見,揖讓而登。獻酬之禮畢,齊有司趨而進曰,請奏四方之樂。景公曰,諾。於是旍旄羽袚矛戟劍撥鼓噪而至。孔子趨而進,歷階而登,不盡一等,舉袂而言曰,吾兩君為好會,夷狄之樂何為於此。請命有司。有司卻之,不去,則左右視晏子與景公。景公心怍,麾而去之。有頃,齊有司趨而進曰,請奏宮中之樂。景公曰,諾。優倡侏儒為戲而前。孔子趨而進,歷階而登,不盡一等,曰,匹夫而營惑諸侯者罪當誅。請命有司。有司加法焉,手足異處。景公懼而動,知義不若,歸而大恐,告其群臣曰,魯以君子之道輔其君,而子獨以夷狄之道教寡人,使得罪於魯君,為之奈何。有司進對曰,君子有過則謝以質,小人有過則謝以文。君若悼之,則謝以質。於是齊侯乃歸所侵魯之鄆、汶陽、龜陰之田以謝過。

定公十三年夏,孔子言於定公曰,臣無藏甲,大夫毋百雉之城。使仲由為季氏宰,將墮三都。於是叔孫氏先墮郈。季氏將墮費,公山不狃、叔孫輒率費人襲魯。公與三子入于季氏之宮,登武子之臺。費人攻之,弗克,入及公側。孔子命申句須、樂頎下伐之,費人北。國人追之,敗諸姑蔑。二子奔齊,遂墮費。將墮成,公斂處父謂孟孫曰,墮成,齊人必至于北門。且成,孟氏之保鄣,無成是無孟氏也。我將弗墮。十二月,公圍成,弗克。

定公十四年,孔子年五十六,由大司寇行攝相事,有喜色。門人曰,聞君子禍至不懼,福至不喜。孔子曰,有是言也。不曰樂其以貴下人乎。於是誅魯大夫亂政者少正卯。與聞國政三月,粥羔豚者弗飾賈,男女行者別於塗,塗不拾遺,四方之客至乎邑者不求有司,皆予之以歸。

齊人聞而懼,曰,孔子為政必霸,霸則吾地近焉,我之為先并矣。盍致地焉。黎鉏曰,請先嘗沮之,沮之而不可則致地,庸遲乎。於是選齊國中女子好者八十人,皆衣文衣而舞康樂,文馬三十駟,遺魯君。陳女樂文馬於魯城南高門外,季桓子微服往觀再三,將受,乃語魯君為周道游,往觀終日,怠於政事。子路曰,夫子可以行矣。孔子曰,魯今且郊,如致膰乎大夫,則吾猶可以止。桓子卒受齊女樂,三日不聽政,郊,又不致膰俎於大夫。孔子遂行,宿乎屯。而師己送,曰,夫子則非罪。孔子曰,吾歌可夫。歌曰,彼婦之口,可以出走,彼婦之謁,可以死敗。蓋優哉游哉,維以卒歲。師己反,桓子曰,孔子亦何言。師己以實告。桓子喟然嘆曰,夫子罪我以群婢故也夫。

孔子遂適衛,主於子路妻兄顏濁鄒家。衛靈公問孔子,居魯得祿幾何。對曰,奉粟六萬。衛人亦致粟六萬。居頃之,或譖孔子於衛靈公。靈公使公孫余假一出一入。孔子恐獲罪焉,居十月,去衛。

將適陳,過匡,顏刻為僕,以其策指之曰,昔吾入此,由彼缺也。匡人聞之,以為魯之陽虎。陽虎嘗暴匡人,匡人於是遂止孔子。孔子狀類陽虎,拘焉五日,顏淵後,子曰,吾以汝為死矣。顏淵曰,子在,回何敢死。匡人拘孔子益急,弟子懼。孔子曰,文王既沒,文不在茲乎。天之將喪斯文也,後死者不得與于斯文也。天之未喪斯文也,匡人其如予何。孔子使從者為甯武子臣於衛,然後得去。

去即過蒲。月餘,反乎衛,主蘧伯玉家。靈公夫人有南子者,使人謂孔子曰,四方之君子不辱欲與寡君為兄弟者,必見寡小君。寡小君願見。孔子辭謝,不得已而見之。夫人在絺帷中。孔子入門,北面稽首。夫人自帷中再拜,環珮玉聲璆然。孔子曰,吾鄉為弗見,見之禮答焉。子路不說。孔子矢之曰,予所不者,天厭之。天厭之。居衛月餘,靈公與夫人同車,宦者雍渠參乘,出,使孔子為次乘,招搖市過之。孔子曰,吾未見好德如好色者也。於是醜之,去衛,過曹。是歲,魯定公卒。

孔子去曹適宋,與弟子習禮大樹下。宋司馬桓魋欲殺孔子,拔其樹。孔子去。弟子曰,可以速矣。孔子曰,天生德於予,桓魋其如予何。

孔子適鄭,與弟子相失,孔子獨立郭東門。鄭人或謂子貢曰,東門有人,其顙似堯,其項類皋陶,其肩類子產,然自要以下不及禹三寸。纍纍若喪家之狗。子貢以實告孔子。孔子欣然笑曰,形狀,末也。而謂似喪家之狗,然哉。然哉。

孔子遂至陳,主於司城貞子家。歲餘,吳王夫差伐陳,取三邑而去。趙鞅伐朝歌。楚圍蔡,蔡遷于吳。吳敗越王句踐會稽。

有隼集于陳廷而死,楛矢貫之,石砮,矢長尺有咫。陳湣公使使問仲尼。仲尼曰,隼來遠矣,此肅慎之矢也。昔武王克商,通道九夷百蠻,使各以其方賄來貢,使無忘職業。於是肅慎貢楛矢石砮,長尺有咫。先王欲昭其令德,以肅慎矢分大姬,配虞胡公而封諸陳。分同姓以珍玉,展親,分異姓以遠職,使無忘服。故分陳以肅慎矢。試求之故府,果得之。

孔子居陳三歲,會晉楚爭彊,更伐陳,及吳侵陳,陳常被寇。孔子曰,歸與。歸與。吾黨之小子狂簡,進取不忘其初。於是孔子去陳。

過蒲,會公叔氏以蒲畔,蒲人止孔子。弟子有公良孺者,以私車五乘從孔子。其為人長賢,有勇力,謂曰,吾昔從夫子遇難於匡,今又遇難於此,命也已。吾與夫子再罹難,寧鬬而死。鬬甚疾。蒲人懼,謂孔子曰,苟毋適衛,吾出子。與之盟,出孔子東門。孔子遂適衛。子貢曰,盟可負耶。孔子曰,要盟也,神不聽。

衛靈公聞孔子來,喜,郊迎。問曰,蒲可伐乎。對曰,可。靈公曰,吾大夫以為不可。今蒲,衛之所以待晉楚也,以衛伐之,無乃不可乎。孔子曰,其男子有死之志,婦人有保西河之志。吾所伐者不過四五人。靈公曰,善。然不伐蒲。

靈公老,怠於政,不用孔子。孔子喟然歎曰,苟有用我者,朞月而已,三年有成。孔子行。

佛肸為中牟宰。趙簡子攻范、中行,伐中牟。佛肸畔,使人召孔子。孔子欲往。子路曰,由聞諸夫子,其身親為不善者,君子不入也。今佛肸親以中牟畔,子欲往,如之何。孔子曰,有是言也。不曰堅乎,磨而不磷,不曰白乎,涅而不淄。我豈匏瓜也哉,焉能系而不食。

孔子擊磬。有荷蕢而過門者,曰,有心哉,擊磬乎。硜硜乎,莫己知也夫而已矣。

孔子學鼓琴師襄子,十日不進。師襄子曰,可以益矣。孔子曰,丘已習其曲矣,未得其數也。有間,曰,已習其數,可以益矣。孔子曰,丘未得其志也。有間,曰,已習其志,可以益矣。孔子曰,丘未得其為人也。有間,曰有所穆然深思焉,有所怡然高望而遠志焉。曰,丘得其為人,黯然而黑,幾然而長,眼如望羊,如王四國,非文王其誰能為此也。師襄子辟席再拜,曰,師蓋云文王操也。

孔子既不得用於衛,將西見趙簡子。至於河而聞竇鳴犢、舜華之死也,臨河而嘆曰,美哉水,洋洋乎。丘之不濟此,命也夫。子貢趨而進曰,敢問何謂也。孔子曰,竇鳴犢,舜華,晉國之賢大夫也。趙簡子未得志之時,須此兩人而后從政,及其已得志,殺之乃從政。丘聞之也,刳胎殺夭,則麒麟不至郊,竭澤涸漁,則蛟龍不合陰陽,覆巢毀卵,則鳳皇不翔。何則。君子諱傷其類也。夫鳥獸之於不義也尚知辟之,而況乎丘哉。乃還息乎陬鄉,作為陬操以哀之。而反乎衛,入主蘧伯玉家。

他日,靈公問兵陳。孔子曰,俎豆之事則嘗聞之,軍旅之事未之學也。明日,與孔子語,見蜚鴈,仰視之,色不在孔子。孔子遂行,復如陳。

夏,衛靈公卒,立孫輒,是為衛出公。六月,趙鞅內太子蒯聵于戚。陽虎使太子絻,八人衰絰,偽自衛迎者,哭而入,遂居焉。冬,蔡遷于州來。是歲魯哀公三年,而孔子年六十矣。齊助衛圍戚,以衛太子蒯聵在故也。

夏,魯桓釐廟燔,南宮敬叔救火。孔子在陳,聞之,曰,災必於桓釐廟乎。已而果然。

秋,季桓子病,輦而見魯城,喟然嘆曰,昔此國幾興矣,以吾獲罪於孔子,故不興也。顧謂其嗣康子曰,我即死,若必相魯,相魯,必召仲尼。後數日,桓子卒,康子代立。已葬,欲召仲尼。公之魚曰,昔吾先君用之不終,終為諸侯笑。今又用之,不能終,是再為諸侯笑。康子曰,則誰召而可。曰,必召冉求。於是使使召冉求。冉求將行,孔子曰,魯人召求,非小用之,將大用之也。是日,孔子曰,歸乎歸乎。吾黨之小子狂簡,斐然成章,吾不知所以裁之。子貢知孔子思歸,送冉求,因誡曰即用,以孔子為招云。

冉求既去,明年,孔子自陳遷于蔡。蔡昭公將如吳,吳召之也。前昭公欺其臣遷州來,後將往,大夫懼復遷,公孫翩射殺昭公。楚侵蔡。秋,齊景公卒。

明年,孔子自蔡如葉。葉公問政,孔子曰,政在來遠附邇。他日,葉公問孔子於子路,子路不對。孔子聞之,曰,由,爾何不對曰其為人也,學道不倦,誨人不厭,發憤忘食,樂以忘憂,不知老之將至云爾。

去葉,反于蔡。長沮、桀溺耦而耕,孔子以為隱者,使子路問津焉。長沮曰,彼執輿者為誰。子路曰,為孔丘。曰,是魯孔丘與。曰,然。曰,是知津矣。桀溺謂子路曰,子為誰。曰,為仲由。曰,子,孔丘之徒與。曰,然。桀溺曰,悠悠者天下皆是也,而誰以易之。且與其從辟人之士,豈若從辟世之士哉。耰而不輟。子路以告孔子,孔子憮然曰,鳥獸不可與同群。天下有道,丘不與易也。

他日,子路行,遇荷蓧丈人,曰,子見夫子乎。丈人曰,四體不勤,五穀不分,孰為夫子。植其杖而芸。子路以告,孔子曰,隱者也。復往,則亡。

孔子遷于蔡三歲,吳伐陳。楚救陳,軍于城父。聞孔子在陳蔡之間,楚使人聘孔子。孔子將往拜禮,陳蔡大夫謀曰,孔子賢者,所刺譏皆中諸侯之疾。今者久留陳蔡之間,諸大夫所設行皆非仲尼之意。今楚,大國也,來聘孔子。孔子用於楚,則陳蔡用事大夫危矣。於是乃相與發徒役圍孔子於野。不得行,絕糧。從者病,莫能興。孔子講誦弦歌不衰。子路慍見曰,君子亦有窮乎。孔子曰,君子固窮,小人窮斯濫矣。

子貢色作。孔子曰,賜,爾以予為多學而識之者與。曰,然。非與。孔子曰,非也。予一以貫之。

孔子知弟子有慍心,乃召子路而問曰,詩云匪兕匪虎,率彼曠野。吾道非邪。吾何為於此。子路曰,意者吾未仁邪。人之不我信也。意者吾未知邪。人之不我行也。孔子曰,有是乎。由,譬使仁者而必信,安有伯夷、叔齊。使知者而必行,安有王子比干。

子路出,子貢入見。孔子曰,賜,詩云匪兕匪虎,率彼曠野。吾道非邪。吾何為於此。子貢曰,夫子之道至大也,故天下莫能容夫子。夫子蓋少貶焉。孔子曰,賜,良農能稼而不能為穡,良工能巧而不能為順。君子能脩其道,綱而紀之,統而理之,而不能為容。今爾不脩爾道而求為容。賜,而志不遠矣。

子貢出,顏回入見。孔子曰,回,詩云匪兕匪虎,率彼曠野。吾道非邪。吾何為於此。顏回曰,夫子之道至大,故天下莫能容。雖然,夫子推而行之,不容何病,不容然後見君子。夫道之不修也,是吾醜也。夫道既已大修而不用,是有國者之醜也。不容何病,不容然後見君子。孔子欣然而笑曰,有是哉顏氏之子。使爾多財,吾為爾宰。

於是使子貢至楚。楚昭王興師迎孔子,然後得免。

昭王將以書社地七百里封孔子。楚令尹子西曰,王之使使諸侯有如子貢者乎。曰,無有。王之輔相有如顏回者乎。曰,無有。王之將率有如子路者乎。曰,無有。王之官尹有如宰予者乎。曰,無有。且楚之祖封於周,號為子男五十里。今孔丘述三五之法,明周召之業,王若用之,則楚安得世世堂堂方數千里乎。夫文王在豐,武王在鎬,百里之君卒王天下。今孔丘得據土壤,賢弟子為佐,非楚之福也。昭王乃止。其秋,楚昭王卒于城父。

楚狂接輿歌而過孔子曰,鳳兮。鳳兮。何德之衰。往者不可諫兮,來者猶可追也。已而,已而。今之從政者殆而。孔子下,欲與之言。趨而去,弗得與之言。

於是孔子自楚反乎衛。是歲也,孔子年六十三,而魯哀公六年也。

其明年,吳與魯會繒,徵百牢。太宰嚭召季康子。康子使子貢往,然後得已。

孔子曰,魯衛之政,兄弟也。是時,衛君輒父不得立,在外,諸侯數以為讓。而孔子弟子多仕於衛,衛君欲得孔子為政。子路曰,衛君待子而為政,子將奚先。孔子曰,必也正名乎。子路曰,有是哉,子之迂也。何其正也。孔子曰,野哉由也。夫名不正則言不順,言不順則事不成,事不成則禮樂不興,禮樂不興則刑罰不中,刑罰不中則民無所錯手足矣。夫君子為之必可名,言之必可行。君子於其言,無所苟而已矣。

其明年,冉有為季氏將師,與齊戰於郎,克之。季康子曰,子之於軍旅,學之乎。性之乎。冉有曰,學之於孔子。季康子曰,孔子何如人哉。對曰,用之有名,播之百姓,質諸鬼神而無憾。求之至於此道,雖累千社,夫子不利也。康子曰,我欲召之,可乎。對曰,欲召之,則毋以小人固之,則可矣。而衛孔文子將攻太叔,問策於仲尼。仲尼辭不知,退而命載而行,曰,鳥能擇木,木豈能擇鳥乎。文子固止。會季康子逐公華、公賓、公林,以幣迎孔子,孔子歸魯。

孔子之去魯凡十四歲而反乎魯。

魯哀公問政,對曰,政在選臣。季康子問政,曰,舉直錯諸枉,則枉者直。康子患盜,孔子曰,苟子之不欲,雖賞之不竊。然魯終不能用孔子,孔子亦不求仕。

孔子之時,周室微而禮樂廢,詩書缺。追跡三代之禮,序書傳,上紀唐虞之際,下至秦繆,編次其事。曰,夏禮吾能言之,杞不足徵也。殷禮吾能言之,宋不足徵也。足,則吾能徵之矣。觀殷夏所損益,曰,後雖百世可知也,以一文一質。周監二代,郁郁乎文哉。吾從周。故書傳、禮記自孔氏。

孔子語魯大師,樂其可知也。始作翕如,縱之純如,皦如,繹如也,以成。吾自衛反魯,然後樂正,雅頌各得其所。

古者詩三千餘篇,及至孔子,去其重,取可施於禮義,上采契后稷,中述殷周之盛,至幽厲之缺,始於衽席,故曰關雎之亂以為風始,鹿鳴為小雅始,文王為大雅始,清廟為頌始。三百五篇孔子皆弦歌之,以求合韶武雅頌之音。禮樂自此可得而述,以備王道,成六藝。

孔子晚而喜易,序彖、繫、象、說卦、文言。讀易,韋編三絕。曰,假我數年,若是,我於易則彬彬矣。

孔子以詩書禮樂教,弟子蓋三千焉,身通六藝者七十有二人。如顏濁鄒之徒,頗受業者甚眾。

孔子以四教,文,行,忠,信。絕四,毋意,毋必,毋固,毋我。所慎,齊,戰,疾。子罕言利與命與仁。不憤不啟,舉一隅不以三隅反,則弗復也。

其於鄉黨,恂恂似不能言者。其於宗廟朝廷,辯辯言,唯謹爾。朝,與上大夫言,誾誾如也,與下大夫言,侃侃如也。

入公門,鞠躬如也,趨進,翼如也。君召使儐,色勃如也。君命召,不俟駕行矣。

魚餒,肉敗,割不正,不食。席不正,不坐。食於有喪者之側,未嘗飽也。

是日哭,則不歌。見齊衰、瞽者,雖童子必變。

三人行,必得我師。德之不脩,學之不講,聞義不能徙,不善不能改,是吾憂也。使人歌,善,則使復之,然后和之。

子不語,怪,力,亂,神。

子貢曰,夫子之文章,可得聞也。夫子言天道與性命,弗可得聞也已。顏淵喟然嘆曰,仰之彌高,鑽之彌堅。瞻之在前,忽焉在後。夫子循循然善誘人,博我以文,約我以禮,欲罷不能。既竭我才,如有所立,卓爾。雖欲從之,蔑由也已。達巷黨人童子曰,大哉孔子,博學而無所成名。子聞之曰,我何執。執御乎。執射乎。我執御矣。牢曰,子云不試,故藝。

魯哀公十四年春,狩大野。叔孫氏車子鉏商獲獸,以為不祥。仲尼視之,曰,麟也。取之。曰,河不出圖,雒不出書,吾已矣夫。顏淵死,孔子曰,天喪予。及西狩見麟,曰,吾道窮矣。喟然嘆曰,莫知我夫。子貢曰,何為莫知子。子曰,不怨天,不尤人,下學而上達,知我者其天乎。

不降其志,不辱其身,伯夷、叔齊乎。謂柳下惠、少連降志辱身矣。謂虞仲、夷逸隱居放言,行中清,廢中權。我則異於是,無可無不可。

子曰,弗乎弗乎,君子病沒世而名不稱焉。吾道不行矣,吾何以自見於後世哉。乃因史記作春秋,上至隱公,下訖哀公十四年,十二公。據魯,親周,故殷,運之三代。約其文辭而指博。故吳楚之君自稱王,而春秋貶之曰子,踐土之會實召周天子,而春秋諱之曰天王狩於河陽,推此類以繩當世。貶損之義,後有王者舉而開之。春秋之義行,則天下亂臣賊子懼焉。

孔子在位聽訟,文辭有可與人共者,弗獨有也。至於為春秋,筆則筆,削則削,子夏之徒不能贊一辭。弟子受春秋,孔子曰,後世知丘者以春秋,而罪丘者亦以春秋。

明歲,子路死於衛。孔子病,子貢請見。孔子方負杖逍遙於門,曰,賜,汝來何其晚也。孔子因歎,歌曰,太山壞乎。梁柱摧乎。哲人萎乎。因以涕下。謂子貢曰,天下無道久矣,莫能宗予。夏人殯於東階,周人於西階,殷人兩柱閒。昨暮予夢坐奠兩柱之閒,予始殷人也。後七日卒。

孔子年七十三,以魯哀公十六年四月己丑卒。

哀公誄之曰,旻天不弔,不愸遺一老,俾屏余一人以在位,煢煢余在疚。嗚呼哀哉。尼父,毋自律。子貢曰,君其不沒於魯乎。夫子之言曰,禮失則昬,名失則愆。失志為昬,失所為愆。生不能用,死而誄之,非禮也。稱余一人,非名也。

孔子葬魯城北泗上,弟子皆服三年。三年心喪畢,相訣而去,則哭,各復盡哀,或復留。唯子貢廬於冢上,凡六年,然後去。弟子及魯人往從冢而家者百有餘室,因命曰孔里。魯世世相傳以歲時奉祠孔子冢,而諸儒亦講禮鄉飲大射於孔子冢。孔子冢大一頃。故所居堂弟子內,後世因廟藏孔子衣冠琴車書,至于漢二百餘年不絕。高皇帝過魯,以太牢祠焉。諸侯卿相至,常先謁然後從政。

孔子生鯉,字伯魚。伯魚年五十,先孔子死。

伯魚生伋,字子思,年六十二。嘗困於宋。子思作中庸。

子思生白,字子上,年四十七。子上生求,字子家,年四十五。子家生箕,字子京,年四十六。子京生穿,字子高,年五十一。子高生子慎,年五十七,嘗為魏相。

子慎生鮒,年五十七,為陳王涉博士,死於陳下。

鮒弟子襄,年五十七。嘗為孝惠皇帝博士,遷為長沙太守。長九尺六寸。

子襄生忠,年五十七。忠生武,武生延年及安國。安國為今皇帝博士,至臨淮太守,蚤卒。安國生卬,卬生驩。

太史公曰,詩有之,高山仰止,景行行止。雖不能至,然心鄉往之。余讀孔氏書,想見其為人。適魯,觀仲尼廟堂車服禮器,諸生以時習禮其家,余祗回留之不能去云。天下君王至於賢人眾矣,當時則榮,沒則已焉。孔子布衣,傳十餘世,學者宗之。自天子王侯,中國言六藝者折中於夫子,可謂至聖矣。