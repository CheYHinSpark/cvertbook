\chapter{管蔡世家第五}

管叔鮮、蔡叔度者,周文王子而武王弟也。武王同母兄弟十人。母曰太姒,文王正妃也。其長子曰伯邑考,次曰武王發,次曰管叔鮮,次曰周公旦,次曰蔡叔度,次曰曹叔振鐸,次曰成叔武,次曰霍叔處,次曰康叔封,次曰冉季載。冉季載最少。同母昆弟十人,唯發、旦賢,左右輔文王,故文王舍伯邑考而以發為太子。及文王崩而發立,是為武王。伯邑考既已前卒矣。

武王已克殷紂,平天下,封功臣昆弟。於是封叔鮮於管,封叔度於蔡,二人相紂子武庚祿父,治殷遺民。封叔旦於魯而相周,為周公。封叔振鐸於曹,封叔武於成,封叔處於霍。康叔封、冉季載皆少,未得封。

武王既崩,成王少,周公旦專王室。管叔、蔡叔疑周公之為不利於成王,乃挾武庚以作亂。周公旦承成王命伐誅武庚,殺管叔,而放蔡叔,遷之,與車十乘,徒七十人從。而分殷餘民為二,其一封微子啟於宋,以續殷祀,其一封康叔為衛君,是為衛康叔。封季載於冉。冉季、康叔皆有馴行,於是周公舉康叔為周司寇,冉季為周司空,以佐成王治,皆有令名於天下。

蔡叔度既遷而死。其子曰胡,胡乃改行,率德馴善。周公聞之,而舉胡以為魯卿士,魯國治。於是周公言於成王,復封胡於蔡,以奉蔡叔之祀,是為蔡仲。餘五叔皆就國,無為天子吏者。

蔡仲卒,子蔡伯荒立。蔡伯荒卒,子宮侯立。宮侯卒,子厲侯立。厲侯卒,子武侯立。武侯之時,周厲王失國,奔彘,共和行政,諸侯多叛周。

武侯卒,子夷侯立。夷侯十一年,周宣王即位。二十八年,夷侯卒,子釐侯所事立。

釐侯三十九年,周幽王為犬戎所殺,周室卑而東徙。秦始得列為諸侯。

四十八年,釐侯卒,子共侯興立。共侯二年卒,子戴侯立。戴侯十年卒,子宣侯措父立。

宣侯二十八年,魯隱公初立。三十五年,宣侯卒,子桓侯封人立。桓侯三年,魯弒其君隱公。二十年,桓侯卒,弟哀侯獻舞立。

哀侯十一年,初,哀侯娶陳,息侯亦娶陳。息夫人將歸,過蔡,蔡侯不敬。息侯怒,請楚文王,來伐我,我求救於蔡,蔡必來,楚因擊之,可以有功。楚文王從之,虜蔡哀侯以歸。哀侯留九歲,死於楚。凡立二十年卒。蔡人立其子肸,是為繆侯。

繆侯以其女弟為齊桓公夫人。十八年,齊桓公與蔡女戲船中,夫人蕩舟,桓公止之,不止,公怒,歸蔡女而不絕也。蔡侯怒,嫁其弟。齊桓公怒,伐蔡,蔡潰,遂虜繆侯,南至楚邵陵。已而諸侯為蔡謝齊,齊侯歸蔡侯。二十九年,繆侯卒,子莊侯甲午立。

莊侯三年,齊桓公卒。十四年,晉文公敗楚於城濮。二十年,楚太子商臣弒其父成王代立。二十五年,秦穆公卒。三十三年,楚莊王即位。三十四年,莊侯卒,子文侯申立。

文侯十四年,楚莊王伐陳,殺夏徵舒。十五年,楚圍鄭,鄭降楚,楚復醳之。二十年,文侯卒,子景侯固立。

景侯元年,楚莊王卒。四十九年,景侯為太子般娶婦於楚,而景侯通焉。太子弒景侯而自立,是為靈侯。

靈侯二年,楚公子圍弒其王郟敖而自立,為靈王。九年,陳司徒招弒其君哀公。楚使公子棄疾滅陳而有之。十二年,楚靈王以靈侯弒其父,誘蔡靈侯于申,伏甲飲之,醉而殺之,刑其士卒七十人。令公子棄疾圍蔡。十一月,滅蔡,使棄疾為蔡公。

楚滅蔡三歲,楚公子棄疾弒其君靈王代立,為平王。平王乃求蔡景侯少子廬,立之,是為平侯。是年,楚亦復立陳。楚平王初立,欲親諸侯,故復立陳、蔡後。

平侯九年卒,靈侯般之孫東國攻平侯子而自立,是為悼侯。悼侯父曰隱太子友。隱太子友者,靈侯之太子,平侯立而殺隱太子,故平侯卒而隱太子之子東國攻平侯子而代立,是為悼侯。悼侯三年卒,弟昭侯申立。

昭侯十年,朝楚昭王,持美裘二,獻其一於昭王而自衣其一。楚相子常欲之,不與。子常讒蔡侯,留之楚三年。蔡侯知之,乃獻其裘於子常,子常受之,乃言歸蔡侯。蔡侯歸而之晉,請與晉伐楚。

十三年春,與衛靈公會邵陵。蔡侯私於周萇弘以求長於衛,衛使史言康叔之功德,乃長衛。夏,為晉滅沈,楚怒,攻蔡。蔡昭侯使其子為質於吳,以共伐楚。冬,與吳王闔閭遂破楚入郢。蔡怨子常,子常恐,奔鄭。十四年,吳去而楚昭王復國。十六年,楚令尹為其民泣以謀蔡,蔡昭侯懼。二十六年,孔子如蔡。楚昭王伐蔡,蔡恐,告急於吳。吳為蔡遠,約遷以自近,易以相救,昭侯私許,不與大夫計。吳人來救蔡,因遷蔡于州來。二十八年,昭侯將朝于吳,大夫恐其復遷,乃令賊利殺昭侯,已而誅賊利以解過,而立昭侯子朔,是為成侯。

成侯四年,宋滅曹。十年,齊田常弒其君簡公。十三年,楚滅陳。十九年,成侯卒,子聲侯產立。聲侯十五年卒,子元侯立。元侯六年卒,子侯齊立。

侯齊四年,楚惠王滅蔡,蔡侯齊亡,蔡遂絕祀。後陳滅三十三年。

伯邑考,其後不知所封。武王發,其後為周,有本紀言。管叔鮮作亂誅死,無後。周公旦,其後為魯,有世家言。蔡叔度,其後為蔡,有世家言。曹叔振鐸,有後為曹,有世家言。成叔武,其後世無所見。霍叔處,其後晉獻公時滅霍。康叔封,其後為衛,有世家言。冉季載,其後世無所見。

太史公曰,管蔡作亂,無足載者。然周武王崩,成王少,天下既疑,賴同母之弟成叔、冉季之屬十人為輔拂,是以諸侯卒宗周,故附之世家言。

曹叔振鐸者,周武王弟也。武王已克殷紂,封叔振鐸於曹。

叔振鐸卒,子太伯脾立。太伯卒,子仲君平立。仲君平卒,子宮伯侯立。宮伯侯卒,子孝伯云立。孝伯云卒,子夷伯喜立。

夷伯二十三年,周厲王奔于彘。

三十年卒,弟幽伯彊立。幽伯九年,弟蘇殺幽伯代立,是為戴伯。戴伯元年,周宣王已立三歲。三十年,戴伯卒,子惠伯兕立。

惠伯二十五年,周幽王為犬戎所殺,因東徙,益卑,諸侯畔之。秦始列為諸侯。

三十六,惠伯卒,子石甫立,其弟武殺之代立,是為繆公。繆公三年卒,子桓公終生立。

桓公三十五年,魯隱公立。四十五年,魯弒其君隱公。四十六年,宋華父督弒其君殤公,及孔父。五十五年,桓公卒,子莊公夕姑立。

莊公二十三年,齊桓公始霸。

三十一年,莊公卒,子釐公夷立。釐公九年卒,子昭公班立。昭公六年,齊桓公敗蔡,遂至楚召陵。九年,昭公卒,子共公襄立。

共公十六年,初,晉公子重耳其亡過曹,曹君無禮,欲觀其駢脅。釐負羈諫,不聽,私善於重耳。二十一年,晉文公重耳伐曹,虜共公以歸,令軍毋入釐負羈之宗族閭。或說晉文公曰,昔齊桓公會諸侯,復異姓,今君囚曹君,滅同姓,何以令於諸侯。晉乃復歸共公。

二十五年,晉文公卒。三十五年,共公卒,子文公壽立。文公二十三年卒,子宣公彊立。宣公十七年卒,弟成公負芻立。

成公三年,晉厲公伐曹,虜成公以歸,已復釋之。五年,晉欒書、中行偃使程滑弒其君厲公。二十三年,成公卒,子武公勝立。武公二十六年,楚公子棄疾弒其君靈王代立。二十七年,武公卒,子平公須立。平公四年卒,子悼公午立。是歲,宋、衛、陳、鄭皆火。

悼公八年,宋景公立。九年,悼公朝于宋,宋囚之,曹立其弟野,是為聲公。悼公死於宋,歸葬。

聲公五年,平公弟通弒聲公代立,是為隱公。隱公四年,聲公弟露弒隱公代立,是為靖公。靖公四年卒,子伯陽立。

伯陽三年,國人有夢眾君子立于社宮,謀欲亡曹,曹叔振鐸止之,請待公孫彊,許之。旦,求之曹,無此人。夢者戒其子曰,我亡,爾聞公孫彊為政,必去曹,無離曹禍。及伯陽即位,好田弋之事。六年,曹野人公孫彊亦好田弋,獲白鴈而獻之,且言田弋之說,因訪政事。伯陽大說之,有寵,使為司城以聽政。夢者之子乃亡去。

公孫彊言霸說於曹伯。十四年,曹伯從之,乃背晉干宋。宋景公伐之,晉人不救。十五年,宋滅曹,執曹伯陽及公孫彊以歸而殺之。曹遂絕其祀。

太史公曰,余尋曹共公之不用僖負羈,乃乘軒者三百人,知唯德之不建。及振鐸之夢,豈不欲引曹之祀者哉。如公孫彊不修厥政,叔鐸之祀忽諸。