\chapter{齊悼惠王世家第二十二}

齊悼惠王劉肥者,高祖長庶男也。其母外婦也,曰曹氏。高祖六年,立肥為齊王,食七十城,諸民能齊言者皆予齊王。

齊王,孝惠帝兄也。孝惠帝二年,齊王入朝。惠帝與齊王燕飲,亢禮如家人。呂太后怒,且誅齊王。齊王懼不得脫,乃用其內史勳計,獻城陽郡,以為魯元公主湯沐邑。呂太后喜,乃得辭就國。

悼惠王即位十三年,以惠帝六年卒。子襄立,是為哀王。

哀王元年,孝惠帝崩,呂太后稱制,天下事皆決於高后。二年,高后立其兄子酈侯呂臺為呂王,割齊之濟南郡為呂王奉邑。

哀王三年,其弟章入宿衛於漢,呂太后封為朱虛侯,以呂祿女妻之。後四年,封章弟興居為東牟侯,皆宿衛長安中。

哀王八年,高后割齊瑯邪郡立營陵侯劉澤為瑯邪王。

其明年,趙王友入朝,幽死于邸。三趙王皆廢。高后立諸呂諸呂為三王,擅權用事。

朱虛侯年二十,有氣力,忿劉氏不得職。嘗入待高后燕飲,高后令朱虛侯劉章為酒吏。章自請曰,臣,將種也,請得以軍法行酒。高后曰,可。酒酣,章進飲歌舞。已而曰,請為太后言耕田歌。高后兒子畜之,笑曰,顧而父知田耳。若生而為王子,安知田乎。章曰,臣知之。太后曰,試為我言田。章曰,深耕穊種,立苗欲疏,非其種者,鉏而去之。呂后默然。頃之,諸呂有一人醉,亡酒,章追,拔劍斬之,而還報曰,有亡酒一人,臣謹行法斬之。太后左右皆大驚。業已許其軍法,無以罪也。因罷。自是之後,諸呂憚朱虛侯,雖大臣皆依朱虛侯,劉氏為益彊。

其明年,高后崩。趙王呂祿為上將軍,呂王產為相國,皆居長安中,聚兵以威大臣,欲為亂。朱虛侯章以呂祿女為婦,知其謀,乃使人陰出告其兄齊王,欲令發兵西,朱虛侯、東牟侯為內應,以誅諸呂,因立齊王為帝。

齊王既聞此計,乃與其舅父駟鈞、郎中令祝午、中尉魏勃陰謀發兵。齊相召平聞之,乃發卒衛王宮。魏勃紿召平曰,王欲發兵,非有漢虎符驗也。而相君圍王,固善。勃請為君將兵衛衛王。召平信之,乃使魏勃將兵圍王宮。勃既將兵,使圍相府。召平曰,嗟乎。道家之言當斷不斷,反受其亂,乃是也。遂自殺。於是齊王以駟鈞為相,魏勃為將軍,祝午為內史,悉發國中兵。使祝午東詐瑯邪王曰,呂氏作亂,齊王發兵欲西誅之。齊王自以兒子,年少,不習兵革之事,願舉國委大王。大王自高帝將也,習戰事。齊王不敢離兵,使臣請大王幸之臨菑見齊王計事,并將齊兵以西平關中之亂。瑯邪王信之,以為然,迺馳見齊王。齊王與魏勃等因留瑯邪王,而使祝午盡發瑯邪國而并將其兵。

瑯邪王劉澤既見欺,不得反國,乃說齊王曰,齊悼惠王高皇帝長子,推本言之,而大王高皇帝適長孫也,當立。今諸大臣狐疑未有所定,而澤於劉氏最為長年,大臣固待澤決計。今大王留臣無為也,不如使我入關計事。齊王以為然,乃益具車送瑯邪王。

瑯邪王既行,齊遂舉兵西攻呂國之濟南。於是齊哀王遺諸侯王書曰,高帝平定天下,王諸子弟,悼惠王於齊。悼惠王薨,惠帝使留侯張良立臣為齊王。惠帝崩,高后用事,春秋高,聽諸呂擅廢高帝所立,又殺三趙王,滅梁、燕、趙以王諸呂,分齊國為四。忠臣進諫,上惑亂不聽。今高后崩,皇帝春秋富,未能治天下,固恃大臣諸侯。今諸呂又擅自尊官,聚兵嚴威,劫列侯忠臣,矯制以令天下,宗廟所以危。今寡人率兵入誅不當為王者。

漢聞齊發兵而西,相國呂產乃遣大將軍灌嬰東擊之。灌嬰至滎陽,乃謀曰,諸呂將兵居關中,欲危劉氏而自立。我今破齊還報,是益呂氏資也。乃留兵屯滎陽,使使喻齊王及諸侯,與連和,以待呂氏之變而共誅之。齊王聞之,乃西取其故濟南郡,亦屯兵於齊西界以待約。

呂祿、呂產欲作亂關中,朱虛侯與太尉勃、丞相平等誅之。朱虛侯首先斬呂產,於是太尉勃等乃得盡誅諸呂。而瑯邪王亦從齊至長安。

大臣議欲立齊王,而瑯邪王及大臣曰,齊王母家駟鈞,惡戾,虎而冠者也。方以呂氏故幾亂天下,今又立齊王,是欲復為呂氏也。代王母家薄氏,君子長者,且代王又親高帝子,於今見在,且最為長。以子則順,以善人則大臣安。於是大臣乃謀迎立代王,而遣朱虛侯以誅呂氏事告齊王,令罷兵。

灌嬰在滎陽,聞魏勃本教齊王反,既誅呂氏,罷齊兵,使使召責問魏勃。勃曰,失火之家,豈暇先言大人而後救火乎。因退立,股戰而栗,恐不能言者,終無他語。灌將軍熟視笑曰,人謂魏勃勇,妄庸人耳,何能為乎。乃罷魏勃。魏勃父以善鼓琴見秦皇帝。及魏勃少時,欲求見齊相曹參,家貧無以自通,乃常獨早夜埽齊相舍人門外。相舍人怪之,以為物,而伺之,得勃。勃曰,願見相君,無因,故為子埽,欲以求見。於是舍人見勃曹參,因以為舍人。一為參御,言事,參以為賢,言之齊悼惠王。悼惠王召見,則拜為內史。始,悼惠王得自置二千石。及悼惠王卒而哀王立,勃用事,重於齊相。

王既罷兵歸,而代王來立,是為孝文帝。

孝文帝元年,盡以高后時所割齊之城陽、瑯邪、濟南郡復與齊,而徙瑯邪王王燕,益封朱虛侯、東牟侯各二千戶。

是歲,齊哀王卒,太子則立,是為文王。

齊文王元年,漢以齊之城陽郡立朱虛侯為城陽王,以齊濟北郡立東牟侯為濟北王。

二年,濟北王反,漢誅殺之,地入于漢。

後二年,孝文帝盡封齊悼惠王子罷軍等七人皆為列侯。

齊文王立十四年卒,無子,國除,地入于漢。

後一歲,孝文帝以所封悼惠王子分齊為王,齊孝王將閭以悼惠王子楊虛侯為齊王。故齊別郡盡以王悼惠王子,子志為濟北王,子辟光為濟南王,子賢為菑川王,子卬為膠西王,子雄渠為膠東王,與城陽、齊凡七王。

齊孝王十一年,吳王濞、楚王戊反,興兵西,告諸侯曰將誅漢賊臣晁錯以安宗廟。膠西、膠東、菑川、濟南皆擅發兵應吳楚。欲與齊,齊孝王狐疑,城守不聽,三國兵共圍齊。齊王使路中大夫告於天子。天子復令路中大夫還告齊王,善堅守,吾兵今破吳楚矣。路中大夫至,三國兵圍臨菑數重,無從入。三國將劫與路中大夫盟,曰,若反言漢已破矣,齊趣下三國,不且見屠。路中大夫既許之,至城下,望見齊王,曰,漢已發兵百萬,使太尉周亞夫擊破吳楚,方引兵救齊,齊必堅守無下。三國將誅路中大夫。

齊初圍急,陰與三國通謀,約未定,會聞路中大夫從漢來,喜,及其大臣乃復勸王毋下三國。居無何,漢將欒布、平陽侯等兵至齊,擊破三國兵,解齊圍。已而復聞齊初與三國有謀,將欲移兵伐齊。齊孝王懼,乃飲藥自殺。景帝聞之,以為齊首善,以迫劫有謀,非其罪也,乃立孝王太子壽為齊王,是為懿王,續齊後。而膠西、膠東、濟南、菑川王咸誅滅,地入于漢。徙濟北王王菑川。齊懿王立二十二年卒,子次景立,是為厲王。

齊厲王,其母曰紀太后。太后取其弟紀氏女為厲王后。王不愛紀氏女。太后欲其家重寵,令其長女紀翁主入王宮,正其後宮,毋令得近王,欲令愛紀氏女。王因與其姊翁主姦。

齊有宦者徐甲,入事漢皇太后。皇太后有愛女曰修成君,修成君非劉氏,太后憐之。修成君有女名娥,太后欲嫁之於諸侯,宦者甲乃請使齊,必令王上書請娥。皇太后喜,使甲之齊。是時齊人主父偃知甲之使齊以取后事,亦因謂甲,即事成,幸言偃女願得充王後宮。甲既至齊,風以此事。紀太后大怒,曰,王有后,後宮具備。且甲,齊貧人,急乃為宦者,入事漢,無補益,乃欲亂吾王家。且主父偃何為者。乃欲以女充後宮。徐甲大窮,還報皇太后曰,王已願尚娥,然有一害,恐如燕王。燕王者,與其子昆弟姦,新坐以死,亡國,故以燕感太后。太后曰,無復言嫁女齊事。事浸潯不得聞於天子。主父偃由此亦與齊有卻。

主父偃方幸於天子,用事,因言,齊臨菑十萬戶,市租千金,人眾殷富,巨於長安,此非天子親弟愛子不得王此。今齊王於親屬益疏。乃從容言,呂太后時齊欲反,吳楚時孝王幾為亂。今聞齊王與其姊亂。於是天子乃拜主父偃為齊相,且正其事。主父偃既至齊,乃急治王後宮宦者為王通於姊翁主所者,令其辭證皆引王。王年少,懼大罪為吏所執誅,乃飲藥自殺。絕無後。

是時趙王懼主父偃一出廢齊,恐其漸疏骨肉,乃上書言偃受金及輕重之短。天子亦既囚偃。公孫弘言,齊王以憂死毋後,國入漢,非誅偃無以塞天下之望。遂誅偃。

齊厲王立五年死,毋後,國入于漢。

齊悼惠王後尚有二國,城陽及菑川。菑川地比齊。天子憐齊,為悼惠王冢園在郡,割臨菑東環悼惠王冢園邑盡以予菑川,以奉悼惠王祭祀。

城陽景王章,齊悼惠王子,以朱虛侯與大臣共誅諸呂,而章身首先斬相國呂王產於未央宮。孝文帝既立,益封章二千戶,賜金千斤。孝文二年,以齊之城陽郡立章為城陽王。立二年卒,子喜立,是為共王。

共王八年,徙王淮南。四年,復還王城陽。凡三十三年卒,子建延立,是為頃王。

頃王二十六年卒,子義立,是為敬王。敬王九年卒,子武立,是為惠王。惠王十一年卒,子順立,是為荒王。荒王四十六年卒,子恢立,是為戴王。戴王八年卒,子景立,至建始三年,十五歲,卒。

濟北王興居,齊悼惠王子,以東牟侯助大臣誅諸呂,功少。及文帝從代來,興居曰,請與太仆嬰入清宮。廢少帝,共與大臣尊立孝文帝。

孝文帝二年,以齊之濟北郡立興居為濟北王,與城陽王俱立。立二年,反。始大臣誅呂氏時,朱虛侯功尤大,許盡以趙地王朱虛侯,盡以梁地王東牟侯。及孝文帝立,聞朱虛、東牟之初欲立齊王,故絀其功。及二年,王諸子,乃割齊二郡以王章、興居。章、興居自以失職奪功。章死,而興居聞匈奴大入漢,漢多發兵,使丞相灌嬰擊之,文帝親幸太原,以為天子自擊胡,遂發兵反於濟北。天子聞之,罷丞相及行兵,皆歸長安。使棘蒲侯柴將軍擊破虜濟北王,王自殺,地入于漢,為郡。

後十三年,文帝十六年,復以齊悼惠王子安都侯志為濟北王。十一年,吳楚反時,志堅守,不與諸侯合謀。吳楚已平,徙志王菑川。

濟南王辟光,齊悼惠王子,以勒侯孝文十六年為濟南王。十一年,與吳楚反。漢擊破,殺辟光,以濟南為郡,地入于漢。

菑川王賢,齊悼惠王子,以武城侯文帝十六年為菑川王。十一年,與吳楚反,漢擊破,殺賢。

天子因徙濟北王志王菑川。志亦齊悼惠王子,以安都侯王濟北。菑川王反,毋後,乃徙濟北王王菑川。凡立三十五年卒,謚為懿王。子建代立,是為靖王。二十年卒,子遺代立,是為頃王。三十六年卒,子終古立,是為思王。二十八年卒,子尚立,是為孝王。五年卒,子橫立,至建始三年,十一歲,卒。

膠西王卬,齊悼惠王子,以昌平侯文帝十六年為膠西王。十一年,與吳楚反。漢擊破,殺卬,地入于漢,為膠西郡。

膠東王雄渠,齊悼惠王子,以白石侯文帝十六年為膠東王。十一年,與吳楚反,漢擊破,殺雄渠,地入于漢,為膠東郡。太史公曰,諸侯大國無過齊悼惠王。以海內初定,子弟少,激秦之無尺土封,故大封同姓,以填萬民之心。及後分裂,固其理也。