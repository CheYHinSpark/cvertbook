\chapter{魏世家第十四}

魏之先,畢公高之後也。畢公高與周同姓。武王之伐紂,而高封於畢,於是為畢姓。其後絕封,為庶人,或在中國,或在夷狄。其苗裔曰畢萬,事晉獻公。

獻公之十六年,趙夙為御,畢萬為右,以伐霍、耿、魏,滅之。以耿封趙夙,以魏封畢萬,為大夫。卜偃曰,畢萬之後必大矣,萬,滿數也,魏,大名也。以是始賞,天開之矣,天子曰兆民,諸侯曰萬民。今命之大,以從滿數,其必有眾。初,畢萬卜事晉,遇屯之比。辛廖占之,曰,吉。屯固比入,吉孰大焉,其必蕃昌。

畢萬封十一年,晉獻公卒,四子爭更立,晉亂。而畢萬之世彌大,從其國名為魏氏。生武子。魏武子以魏諸子事晉公子重耳。晉獻公之二十一年,武子從重耳出亡。十九年反,重耳立為晉文公,而令魏武子襲魏氏之後封,列為大夫,治於魏。生悼子。

魏悼子徙治霍。生魏絳。

魏絳事晉悼公。悼公三年,會諸侯。悼公弟楊干亂行,魏絳僇辱楊干。悼公怒曰,合諸侯以為榮,今辱吾弟。將誅魏絳。或說悼公,悼公止。卒任魏絳政,使和戎、翟,戎、翟親附。悼公之十一年,曰,自吾用魏絳,八年之中,九合諸侯,戎、翟和,子之力也。賜之樂,三讓,然後受之。徙治安邑。魏絳卒,謚為昭子。生魏嬴。嬴生魏獻子。

獻子事晉昭公。昭公卒而六卿彊,公室卑。

晉頃公之十二年,韓宣子老,魏獻子為國政。晉宗室祁氏、羊舌氏相惡,六卿誅之,盡取其邑為十縣,六卿各令其子為之大夫。獻子與趙簡子、中行文子、范獻子并為晉卿。

其後十四歲而孔子相魯。後四歲,趙簡子以晉陽之亂也,而與韓、魏共攻范、中行氏。魏獻子生魏侈。魏侈與趙鞅共攻范、中行氏。

魏侈之孫曰魏桓子,與韓康子、趙襄子共伐滅知伯,分其地。

桓子之孫曰文侯都。魏文侯元年,秦靈公之元年也。與韓武子、趙桓子、周威王同時。

六年,城少梁。十三年,使子擊圍繁、龐,出其民。十六年,伐秦,筑臨晉元裏。

十七年,伐中山,使子擊守之,趙倉唐傅之。子擊逢文侯之師田子方於朝歌,引車避,下謁。田子方不為禮。子擊因問曰,富貴者驕人乎。且貧賤者驕人乎。子方曰,亦貧賤者驕人耳。夫諸侯而驕人則失其國,大夫而驕人則失其家。貧賤者,行不合,言不用,則去之楚、越,若脫屣然,柰何其同之哉。子擊不懌而去。西攻秦,至鄭而還,筑雒陰、合陽。

二十二年,魏、趙、韓列為諸侯。

二十四年,秦伐我,至陽狐。

二十五年,子擊生子罃。

文侯受子夏經藝,客段干木,過其閭,未嘗不軾也。秦嘗欲伐魏,或曰,魏君賢人是禮,國人稱仁,上下和合,未可圖也。文侯由此得譽於諸侯。

任西門豹守鄴,而河內稱治。

魏文侯謂李克曰,先生嘗教寡人曰家貧則思良妻,國亂則思良相。今所置非成則璜,二子何如。李克對曰,臣聞之,卑不謀尊,疏不謀戚。臣在闕門之外,不敢當命。文侯曰,先生臨事勿讓。李克曰,君不察故也。居視其所親,富視其所與,達視其所舉,窮視其所不為,貧視其所不取,五者足以定之矣,何待克哉。文侯曰,先生就舍,寡人之相定矣。李克趨而出,過翟璜之家。翟璜曰,今者聞君召先生而卜相,果誰為之。李克曰,魏成子為相矣。翟璜忿然作色曰,以耳目之所睹記,臣何負於魏成子。西河之守,臣之所進也。君內以鄴為憂,臣進西門豹。君謀欲伐中山,臣進樂羊。中山以拔,無使守之,臣進先生。君之子無傅,臣進屈侯鮒。臣何以負於魏成子。李克曰,且子之言克於子之君者,豈將比周以求大官哉。君問而置相非成則璜,二子何如。克對曰,君不察故也。居視其所親,富視其所與,達視其所舉,窮視其所不為,貧視其所不取,五者足以定之矣,何待克哉。是以知魏成子之為相也。且子安得與魏成子比乎。魏成子以食祿千鐘,什九在外,什一在內,是以東得卜子夏、田子方、段干木。此三人者,君皆師之。子之所進五人者,君皆臣之。子惡得與魏成子比也。翟璜逡巡再拜曰,璜,鄙人也,失對,願卒為弟子。

二十六年,虢山崩,壅河。

三十二年,伐鄭。城酸棗。敗秦于注。三十五年,齊伐取我襄陵。三十六年,秦侵我陰晉。

三十八年,伐秦,敗我武下,得其將識。是歲,文侯卒,子擊立,是為武侯。

魏武侯元年,趙敬侯初立,公子朔為亂,不勝,奔魏,與魏襲邯鄲,魏敗而去。

二年,城安邑、王垣。

七年,伐齊,至桑丘。九年,翟敗我于澮。使吳起伐齊,至靈丘。齊威王初立。

十一年,與韓、趙三分晉地,滅其後。

十三年,秦獻公縣櫟陽。十五年,敗趙北藺。

十六年,伐楚,取魯陽。武侯卒,子罃立,是為惠王。

惠王元年,初,武侯卒也,子罃與公中緩爭為太子。公孫頎自宋入趙,自趙入韓,謂韓懿侯曰,魏罃與公中緩爭為太子,君亦聞之乎。今魏罃得王錯,挾上黨,固半國也。因而除之,破魏必矣,不可失也。懿侯說,乃與趙成侯合軍并兵以伐魏,戰于濁澤,魏氏大敗,魏君圍。趙謂韓曰,除魏君,立公中緩,割地而退,我且利。韓曰,不可。殺魏君,人必曰暴,割地而退,人必曰貪。不如兩分之。魏分為兩,不彊於宋、衛,則我終無魏之患矣。趙不聽。韓不說,以其少卒夜去。惠王之所以身不死,國不分者,二家謀不和也。若從一家之謀,則魏必分矣。故曰君終無適子,其國可破也。

二年,魏敗韓于馬陵,敗趙于懷。三年,齊敗我觀。五年,與韓會宅陽。城武堵。為秦所敗。六年,伐取宋儀臺。九年,伐敗韓于澮。與秦戰少梁,虜我將公孫痤,取龐。秦獻公卒,子孝公立。

十年,伐取趙皮牢。彗星見。十二年,星晝墜,有聲。

十四年,與趙會鄗。十五年,魯、衛、宋、鄭君來朝。十六年,與秦孝公會杜平。侵宋黃池,宋復取之。

十七年,與秦戰元裏,秦取我少梁。圍趙邯鄲。十八年,拔邯鄲。趙請救于齊,齊使田忌、孫臏救趙,敗魏桂陵。

十九年,諸侯圍我襄陵。筑長城,塞固陽。

二十年,歸趙邯鄲,與盟漳水上。二十一年,與秦會彤。趙成侯卒。二十八年,齊威王卒。中山君相魏。

三十年,魏伐趙,趙告急齊。齊宣王用孫子計,救趙擊魏。魏遂大興師,使龐涓將,而令太子申為上將軍。過外黃,外黃徐子謂太子曰,臣有百戰百勝之術。太子曰,可得聞乎。客曰,固願效之。曰,太子自將攻齊,大勝并莒,則富不過有魏,貴不益為王。若戰不勝齊,則萬世無魏矣。此臣之百戰百勝之術也。太子曰,諾,請必從公之言而還矣。客曰,太子雖欲還,不得矣。彼勸太子戰攻,欲啜汁者眾。太子雖欲還,恐不得矣。太子因欲還,其御曰,將出而還,與北同。太子果與齊人戰,敗於馬陵。齊虜魏太子申,殺將軍涓,軍遂大破。

三十一年,秦、趙、齊共伐我,秦將商君詐我將軍公子卬而襲奪其軍,破之。秦用商君,東地至河,而齊、趙數破我,安邑近秦,於是徙治大梁。以公子赫為太子。

三十三年,秦孝公卒,商君亡秦歸魏,魏怒,不入。三十五年,與齊宣王會平阿南。

惠王數被於軍旅,卑禮厚幣以招賢者。鄒衍、淳于髡、孟軻皆至梁。梁惠王曰,寡人不佞,兵三折於外,太子虜,上將死,國以空虛,以羞先君宗廟社稷,寡人甚丑之,叟不遠千里,辱幸至獘邑之廷,將何利吾國。孟軻曰,君不可以言利若是。夫君欲利則大夫欲利,大夫欲利則庶人欲利,上下爭利,國則危矣。為人君,仁義而已矣,何以利為。

三十六年,復與齊王會甄。是歲,惠王卒,子襄王立。

襄王元年,與諸侯會徐州,相王也。追尊父惠王為王。

五年,秦敗我龍賈軍四萬五千于雕陰,圍我焦、曲沃。予秦河西之地。

六年,與秦會應。秦取我汾陰、皮氏、焦。魏伐楚,敗之陘山。七年,魏盡入上郡于秦。秦降我蒲陽。八年,秦歸我焦、曲沃。

十二年,楚敗我襄陵。諸侯執政與秦相張儀會齧桑。十三年,張儀相魏。魏有女子化為丈夫。秦取我曲沃、平周。

十六年,襄王卒,子哀王立。張儀復歸秦。

哀王元年,五國共攻秦,不勝而去。

二年,齊敗我觀津。五年,秦使樗里子伐取我曲沃,走犀首岸門。六年,秦來立公子政為太子。與秦會臨晉。七年,攻齊。與秦伐燕。

八年,伐衛,拔列城二。衛君患之。如耳約斬趙,趙分而為二,所以不亡者,魏為從主也。今衛已迫亡,將西請事於秦。與其以秦醳衛,不如以魏醳衛,見衛君曰,請罷魏兵,免成陵君可乎。衛君曰,先生果能,孤請世世以衛事先生。如耳見成陵君曰,昔者魏伐趙,斷羊腸,拔閼與,衛之德魏必終無窮。成陵君曰,諾。如耳見魏王曰,臣有謁於衛。衛故周室之別也,其稱小國,多寶器。今國迫於難而寶器不出者,其心以為攻衛醳衛不以王為主,故寶器雖出必不入於王也。臣竊料之,先言醳衛者必受衛者也。如耳出,成陵君入,以其言見魏王。魏王聽其說,罷其兵,免成陵君,終身不見。

九年,與秦王會臨晉。張儀、魏章皆歸于魏。魏相田需死,楚害張儀、犀首、薛公。楚相昭魚謂蘇代曰,田需死,吾恐張儀、犀首、薛公有一人相魏者也。代曰,然相者欲誰而君便之。昭魚曰,吾欲太子之自相也。代曰,請為君北,必相之。昭魚曰,柰何。對曰,君其為梁王,代請說君。昭魚曰,柰何。對曰,代也從楚來,昭魚甚憂,曰,田需死,吾恐張儀、犀首、薛公有一人相魏者也。代曰,梁王,長主也,必不相張儀。張儀相,必右秦而左魏。犀首相,必右韓而左魏。薛公相,必右齊而左魏。梁王,長主也,必不便也。王曰,然則寡人孰相。代曰,莫若太子之自相。太子之自相,是三人者皆以太子為非常相也,皆將務以其國事魏,欲得丞相璽也。以魏之彊,而三萬乘之國輔之,魏必安矣。故曰莫若太子之自相也。遂北見梁王,以此告之。太子果相魏。

十年,張儀死。十一年,與秦武王會應。十二年,太子朝於秦。秦來伐我皮氏,未拔而解。十四年,秦來歸武王后。十六年,秦拔我蒲反、陽晉、封陵。十七年,與秦會臨晉。秦予我蒲反。十八年,與秦伐楚。二十一年,與齊、韓共敗秦軍函谷。

二十三年,秦復予我河外及封陵為和。哀王卒,子昭王立。

昭王元年,秦拔我襄城。二年,與秦戰,我不利。三年,佐韓攻秦,秦將白起敗我軍伊闕二十四萬。六年,予秦河東地方四百里。芒卯以詐重。七年,秦拔我城大小六十一。八年,秦昭王為西帝,齊湣王為東帝,月餘,皆復稱王歸帝。九年,秦拔我新垣、曲陽之城。

十年,齊滅宋,宋王死我溫。十二年,與秦、趙、韓、燕共伐齊,敗之濟西,湣王出亡。燕獨入臨菑。與秦王會西周。

十三年,秦拔我安城。兵到大梁,去。十八年,秦拔郢,楚王徙陳。

十九年,昭王卒,子安釐王立。

安釐王元年,秦拔我兩城。二年,又拔我二城,軍大梁下,韓來救,予秦溫以和。三年,秦拔我四城,斬首四萬。四年,秦破我及韓、趙,殺十五萬人,走我將芒卯。魏將段干子請予秦南陽以和。蘇代謂魏王曰,欲璽者段干子也,欲地者秦也。今王使欲地者制璽,使欲璽者制地,魏氏地不盡則不知已。且夫以地事秦,譬猶抱薪救火,薪不盡,火不滅。王曰,是則然也。雖然,事始已行,不可更矣。對曰,王獨不見夫博之所以貴梟者,便則食,不便則止矣。今王曰事始已行,不可更,是何王之用智不如用梟也。

九年,秦拔我懷。十年,秦太子外質於魏死。十一年,秦拔我郪丘。

秦昭王謂左右曰,今時韓、魏與始孰彊。對曰,不如始彊。王曰,今時如耳、魏齊與孟嘗、芒卯孰賢。對曰,不如。王曰,以孟嘗、芒卯之賢,率彊韓、魏以攻秦,猶無柰寡人何也。今以無能之如耳、魏齊而率弱韓、魏以伐秦,其無柰寡人何亦明矣。左右皆曰,甚然。中旗馮琴而對曰,王之料天下過矣。當晉六卿之時,知氏最彊,滅范、中行,又率韓、魏之兵以圍趙襄子於晉陽,決晉水以灌晉陽之城,不湛者三版。知伯行水,魏桓子御,韓康子為參乘。知伯曰,吾始不知水之可以亡人之國也,乃今知之。汾水可以灌安邑,絳水可以灌平陽。魏桓子肘韓康子,韓康子履魏桓子,肘足接於車上,而知氏地分,身死國亡,為天下笑。今秦兵雖彊,不能過知氏,韓、魏雖弱,尚賢其在晉陽之下也。此方其用肘足之時也,願王之勿易也。於是秦王恐。

齊、楚相約而攻魏,魏使人求救於秦,冠蓋相望也,而秦救不至。魏人有唐雎者,年九十餘矣,謂魏王曰,老臣請西說秦王,令兵先臣出。魏王再拜,遂約車而遣之。唐雎到,入見秦王。秦王曰,丈人芒然乃遠至此,甚苦矣。夫魏之來求救數矣,寡人知魏之急已。唐雎對曰,大王已知魏之急而救不發者,臣竊以為用策之臣無任矣。夫魏,一萬乘之國也,然所以西面而事秦,稱東藩,受冠帶,祠春秋者,以秦之彊足以為與也。今齊、楚之兵已合於魏郊矣,而秦救不發,亦將賴其未急也。使之大急,彼且割地而約從,王尚何救焉。必待其急而救之,是失一東藩之魏而彊二敵之齊、楚,則王何利焉。於是秦昭王遽為發兵救魏。魏氏復定。

趙使人謂魏王曰,為我殺范痤,吾請獻七十里之地。魏王曰,諾。使吏捕之,圍而未殺。痤因上屋騎危,謂使者曰,與其以死痤市,不如以生痤市。有如痤死,趙不予王地,則王將柰何。故不若與先定割地,然後殺痤。魏王曰,善。痤因上書信陵君曰,痤,故魏之免相也,趙以地殺痤而魏王聽之,有如彊秦亦將襲趙之欲,則君且柰何。信陵君言於王而出之。

魏王以秦救之故,欲親秦而伐韓,以求故地。無忌謂魏王曰,

秦與戎翟同俗,有虎狼之心,貪戾好利無信,不識禮義德行。茍有利焉,不顧親戚兄弟,若禽獸耳,此天下之所識也,非有所施厚積德也。故太后母也,而以憂死,穰侯舅也,功莫大焉,而竟逐之,兩弟無罪,而再奪之國。此於親戚若此,而況於仇讎之國乎。今王與秦共伐韓而益近秦患,臣甚惑之。而王不識則不明,群臣莫以聞則不忠。

今韓氏以一女子奉一弱主,內有大亂,外交彊秦魏之兵,王以為不亡乎。韓亡,秦有鄭地,與大梁鄴,王以為安乎。王欲得故地,今負彊秦之親,王以為利乎。

秦非無事之國也,韓亡之後必將更事,更事必就易與利,就易與利必不伐楚與趙矣。是何也。夫越山踰河,絕韓上黨而攻彊趙,是復閼與之事,秦必不為也。若道河內,倍鄴、朝歌,絕漳滏水,與趙兵決於邯鄲之郊,是知伯之禍也,秦又不敢。伐楚,道涉谷,行三千里。而攻冥阸之塞,所行甚遠,所攻甚難,秦又不為也。若道河外,倍大梁,右上蔡、召陵,與楚兵決於陳郊,秦又不敢。故曰秦必不伐楚與趙矣,又不攻衛與齊矣。

夫韓亡之後,兵出之日,非魏無攻已。秦固有懷、茅、邢丘,城垝津以臨河內,河內共、汲必危,有鄭地,得垣雍,決熒澤水灌大梁,大梁必亡。王之使者出過而惡安陵氏於秦,秦之欲誅之久矣。秦葉陽、昆陽與舞陽鄰,聽使者之惡之,隨安陵氏而亡之,繞舞陽之北,以東臨許,南國必危,國無害乎。

夫憎韓不愛安陵氏可也,夫不患秦之不愛南國非也。異日者,秦在河西晉,國去梁千里,有河山以闌之,有周韓以閒之。從林鄉軍以至于今,秦七攻魏,五入囿中,邊城盡拔,文臺墮,垂都焚,林木伐,麋鹿盡,而國繼以圍。又長驅梁北,東至陶衛之郊,北至平監。所亡於秦者,山南山北,河外河內,大縣數十,名都數百。秦乃在河西晉,去梁千里,而禍若是矣,又況於使秦無韓,有鄭地,無河山而闌之,無周韓而閒之,去大梁百里,禍必由此矣。

異日者,從之不成也,楚、魏疑而韓不可得也。今韓受兵三年,秦橈之以講,識亡不聽,投質於趙,請為天下鴈行頓刃,楚、趙必集兵,皆識秦之欲無窮也,非盡亡天下之國而臣海內,必不休矣。是故臣願以從事王,王速受楚趙之約,而挾韓之質以存韓,而求故地,韓必效之。此士民不勞而故地得,其功多於與秦共伐韓,而又與彊秦鄰之禍也。

夫存韓安魏而利天下,此亦王之天時已。通韓上黨於共、甯,使道安成,出入賦之,是魏重質韓以其上黨也。今有其賦,足以富國。韓必德魏愛魏重魏畏魏,韓必不敢反魏,是韓則魏之縣也。魏得韓以為縣,衛、大梁、河外必安矣。今不存韓,二周、安陵必危,楚、趙大破,衛、齊甚畏,天下西鄉而馳秦入朝而為臣不久矣。

二十年,秦圍邯鄲,信陵君無忌矯奪將軍晉鄙兵以救趙,趙得全。無忌因留趙。二十六年,秦昭王卒。

三十年,無忌歸魏,率五國兵攻秦,敗之河外,走蒙驁。魏太子增質於秦,秦怒,欲囚魏太子增。或為增謂秦王曰,公孫喜固謂魏相曰請以魏疾擊秦,秦王怒,必囚增。魏王又怒,擊秦,秦必傷。今王囚增,是喜之計中也。故不若貴增而合魏,以疑之於齊、韓。秦乃止增。

三十一年,秦王政初立。

三十四年,安釐王卒,太子增立,是為景湣王。信陵君無忌卒。

景湣王元年,秦拔我二十城,以為秦東郡。二年,秦拔我朝歌。衛徙野王。三年,秦拔我汲。五年,秦拔我垣、蒲陽、衍。十五年,景湣王卒,子王假立。

王假元年,燕太子丹使荊軻刺秦王,秦王覺之。

三年,秦灌大梁,虜王假,遂滅魏以為郡縣。

太史公曰,吾適故大梁之墟,墟中人曰,秦之破梁,引河溝而灌大梁,三月城壞,王請降,遂滅魏。說者皆曰魏以不用信陵君故,國削弱至於亡,余以為不然。天方令秦平海內,其業未成,魏雖得阿衡之佐,曷益乎。