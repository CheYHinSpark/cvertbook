\chapter{燕召公世家第四}

召公奭與周同姓,姓姬氏。周武王之滅紂,封召公於北燕。

其在成王時,召王為三公,自陜以西,召公主之,自陜以東,周公主之。成王既幼,周公攝政,當國踐祚,召公疑之,作君奭。君奭不說周公。周公乃稱湯時有伊尹,假于皇天,在太戊時,則有若伊陟、臣扈,假于上帝,巫咸治王家,在祖乙時,則有若巫賢,在武丁時,則有若甘般,率維茲有陳,保乂有殷。於是召公乃說。

召公之治西方,甚得兆民和。召公巡行鄉邑,有棠樹,決獄政事其下,自侯伯至庶人各得其所,無失職者。召公卒,而民人思召公之政,懷棠樹不敢伐,哥詠之,作甘棠之詩。

自召公已下九世至惠侯。燕惠侯當周厲王奔彘,共和之時。

惠侯卒,子釐侯立。是歲,周宣王初即位。釐侯二十一年,鄭桓公初封於鄭。三十六年,釐侯卒,子頃侯立。

頃侯二十年,周幽王淫亂,為犬戎所弒。秦始列為諸侯。

二十四年,頃侯卒,子哀侯立。哀侯二年卒,子鄭侯立。鄭侯三十六年卒,子繆侯立。

繆侯七年,而魯隱公元年也。十八年卒,子宣侯立。宣侯十三年卒,子桓侯立。桓侯七年卒,子莊公立。

莊公十二年,齊桓公始霸。十六年,與宋、衛共伐周惠王,惠王出奔溫,立惠王弟穨為周王。十七年,鄭執燕仲父而內惠王于周。二十七年,山戎來侵我,齊桓公救燕,遂北伐山戎而還。燕君送齊桓公出境,桓公因割燕所至地予燕,使燕共貢天子,如成周時職,使燕復修召公之法。三十三年卒,子襄公立。

襄公二十六年,晉文公為踐土之會,稱伯。三十一年,秦師敗于殽。三十七年,秦穆公卒。四十年,襄公卒,桓公立。

桓公十六年卒,宣公立。宣公十五年卒,昭公立。昭公十三年卒,武公立。是歲晉滅三郤大夫。

武公十九年卒,文公立。文公六年卒,懿公立。懿公元年,齊崔杼弒其君莊公。四年卒,子惠公立。

惠公元年,齊高止來奔。六年,惠公多寵姬,公欲去諸大夫而立寵姬宋,大夫共誅姬宋,惠公懼,奔齊。四年,齊高偃如晉,請共伐燕,入其君。晉平公許,與齊伐燕,入惠公。惠公至燕而死。燕立悼公。

悼公七年卒,共公立。共公五年卒,平公立。晉公室卑,六卿始彊大。平公十八年,吳王闔閭破楚入郢。十九年卒,簡公立。簡公十二年卒,獻公立。晉趙鞅圍范、中行於朝歌。獻公十二年,齊田常弒其君簡公。十四年,孔子卒。二十八年,獻公卒,孝公立。

孝公十二年,韓、魏、趙滅知伯,分其地,三晉彊。

十五年,孝公卒,成公立。成公十六年卒,湣公立。湣公三十一年卒,釐公立。是歲,三晉列為諸侯。

釐公三十年,伐敗齊于林營。釐公卒,桓公立。桓公十一年卒,文公立。是歲,秦獻公卒。秦益彊。

文公十九年,齊威王卒。二十八年,蘇秦始來見,說文公。文公予車馬金帛以至趙,趙肅侯用之。因約六國,為從長。秦惠王以其女為燕太子婦。

二十九年,文公卒,太子立,是為易王。

易王初立,齊宣王因燕喪伐我,取十城,蘇秦說齊,使復歸燕十城。十年,燕君為王。蘇秦與燕文公夫人私通,懼誅,乃說王使齊為反閒,欲以亂齊。易王立十二年卒,子燕噲立。

燕噲既立,齊人殺蘇秦。蘇秦之在燕,與其相子之為婚,而蘇代與子之交。及蘇秦死,而齊宣王復用蘇代。燕噲三年,與楚、三晉攻秦,不勝而還。子之相燕,貴重,主斷。蘇代為齊使於燕,燕王問曰,齊王奚如。對曰,必不霸。燕王曰,何也。對曰,不信其臣。蘇代欲以激燕王以尊子之也。於是燕王大信子之。子之因遺蘇代百金,而聽其所使。

鹿毛壽謂燕王,不如以國讓相子之。人之謂堯賢者,以其讓天下於許由,許由不受,有讓天下之名而實不失天下。今王以國讓於子之,子之必不敢受,是王與堯同行也。燕王因屬國於子之,子之大重。或曰,禹薦益,已而以啟人為吏。及老,而以啟人為不足任乎天下,傳之於益。已而啟與交黨攻益,奪之。天下謂禹名傳天下於益,已而實令啟自取之。今王言屬國於子之,而吏無非太子人者,是名屬子之而實太子用事也。王因收印自三百石吏已上而效之子之。子之南面行王事,而噲老不聽政,顧為臣,國事皆決於子之。

三年,國大亂,百姓恫恐。將軍市被與太子平謀,將攻子之。諸將謂齊湣王曰,因而赴之,破燕必矣。齊王因令人謂燕太子平曰,寡人聞太子之義,將廢私而立公,飭君臣之義,明父子之位。寡人之國小,不足以為先後。雖然,則唯太子所以令之。太子因要黨聚眾,將軍市被圍公宮,攻子之,不克。將軍市被及百姓反攻太子平,將軍市被死,以徇。因搆難數月,死者數萬,眾人恫恐,百姓離志。孟軻謂齊王曰,今伐燕,此文、武之時,不可失也。王因令章子將五都之兵,以因北地之眾以伐燕。士卒不戰,城門不閉,燕君噲死,齊大勝。燕子之亡二年,而燕人共立太子平,是為燕昭王。

燕昭王於破燕之後即位,卑身厚幣以招賢者。謂郭隗曰,齊因孤之國亂而襲破燕,孤極知燕小力少,不足以報。然誠得賢士以共國,以雪先王之恥,孤之願也。先生視可者,得身事之。郭隗曰,王必欲致士,先從隗始。況賢於隗者,豈遠千里哉。於是昭王為隗改筑宮而師事之。樂毅自魏往,鄒衍自齊往,劇辛自趙往,士爭趨燕。燕王噲死問孤,與百姓同甘苦。

二十八年,燕國殷富,士卒樂軼輕戰,於是遂以樂毅為上將軍,與秦、楚、三晉合謀以伐齊。齊兵敗,湣王出亡於外。燕兵獨追北,入至臨淄,盡取齊寶,燒其宮室宗廟。齊城之不下者,獨唯聊、莒、即墨,其餘皆屬燕,六歲。

武成王七年,齊田單伐我,拔中陽。十三年,秦敗趙於長平四十餘萬。十四年,武成王卒,子孝王立。

孝王元年,秦圍邯鄲者解去。三年卒,子今王喜立。

今王喜四年,秦昭王卒。燕王命相栗腹約歡趙,以五百金為趙王酒。還報燕王曰,趙王壯者皆死長平,其孤未壯,可伐也。王召昌國君樂閒問之。對曰,趙四戰之國,其民習兵,不可伐。王曰,吾以五而伐一。對曰,不可。燕王怒,群臣皆以為可。卒起二軍,車二千乘,栗腹將而攻鄗,卿秦攻代。唯獨大夫將渠謂燕王曰,與人通關約交,以五百金飲人之王,使者報而反攻之,不祥,兵無成功。燕王不聽,自將偏軍隨之。將渠引燕王綬止之曰,王必無自往,往無成功。王蹵之以足。將渠泣曰,臣非以自為,為王也。燕軍至宋子,趙使廉頗將,擊破栗腹於鄗。樂乘破卿秦於代。樂閒奔趙。廉頗逐之五百餘里,圍其國。燕人請和,趙人不許,必令將渠處和。燕相將渠以處和。趙聽將渠,解燕圍。

六年,秦滅東周,置三川郡。七年,秦拔趙榆次三十七城,秦置大原郡。九年,秦王政初即位。十年,趙使廉頗將攻繁陽,拔之。趙孝成王卒,悼襄王立。使樂乘代廉頗,廉頗不聽,攻樂乘,樂乘走,廉頗奔大梁。十二年,趙使李牧攻燕,拔武遂、方城。劇辛故居趙,與龐煖善,已而亡走燕。燕見趙數困于秦,而廉頗去,令龐煖將也,欲因趙獘攻之。問劇辛,辛曰,龐煖易與耳。燕使劇辛將擊趙,趙使龐煖擊之,取燕軍二萬,殺劇辛。秦拔魏二十城,置東郡。十九年,秦拔趙之鄴九城。趙悼襄王卒。二十三年,太子丹質於秦,亡歸燕。二十五年,秦虜滅韓王安,置潁川郡。二十七年,秦虜趙王遷,滅趙。趙公子嘉自立為代王。

燕見秦且滅六國,秦兵臨易水,禍且至燕。太子丹陰養壯士二十人,使荊軻獻督亢地圖於秦,因襲刺秦王。秦王覺,殺軻,使將軍王翦擊燕。二十九年,秦攻拔我薊,燕王亡,徙居遼東,斬丹以獻秦。三十年,秦滅魏。

三十三年,秦拔遼東,虜燕王喜,卒滅燕。是歲,秦將王賁亦虜代王嘉。

太史公曰,召公奭可謂仁矣。笆棠且思之,況其人乎。燕外迫蠻貉,內措齊、晉,崎嶇彊國之閒,最為弱小,幾滅者數矣。然社稷血食者八九百歲,於姬姓獨後亡,豈非召公之烈邪。