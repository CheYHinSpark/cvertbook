\chapter{絳侯周勃世家第二十七}

絳侯周勃者,沛人也。其先卷人,徙沛。勃以織薄曲為生,常為人吹簫給喪事,材官引彊。

高祖之為沛公初起,勃以中涓從攻胡陵,下方與。方與反,與戰,卻適。攻豐。擊秦軍碭東。還軍留及蕭。復攻碭,破之。下下邑,先登。賜爵五大夫。攻蒙、虞,取之。擊章邯車騎,殿。定魏地。攻爰戚、東緡,以往至栗,取之。攻齧桑,先登。擊秦軍阿下,破之。追至濮陽,下甄城。攻都關、定陶,襲取宛朐,得單父令。夜襲取臨濟,攻張,以前至卷,破之。擊李由軍雍丘下。攻開封,先至城下為多。後章邯破殺項梁,沛公與項羽引兵東如碭。自初起沛還至碭,一歲二月。楚懷王封沛公號安武侯,為碭郡長。沛公拜勃為虎賁令,以令從沛公定魏地。攻東郡尉於城武,破之。擊王離軍,破之。攻長社,先登。攻潁陽、緱氏,絕河津。擊趙賁軍尸北。南攻南陽守齮,破武關、峣關。破秦軍於藍田,至咸陽,滅秦。

項羽至,以沛公為漢王。漢王賜勃爵為威武侯。從入漢中,拜為將軍。還定三秦,至秦,賜食邑懷德。攻槐裏、好畤,最。擊趙賁、內史保於咸陽,最。北攻漆。擊章平、姚卬軍。西定汧。還下郿、頻陽。圍章邯廢丘。破西丞。擊盜巴軍,破之。攻上邽。東守峣關。轉擊項籍。攻曲逆,最。還守敖倉,追項籍。籍已死,因東定楚地泗水、東海郡,凡得二十二縣。還守雒陽、櫟陽,賜與潁陰侯共食鐘離。以將軍從高帝反者燕王臧荼,破之易下。所將卒當馳道為多。賜爵列侯,剖符世世勿絕。食絳八千一百八十戶,號絳侯。

以將軍從高帝擊反韓王信於代,降下霍人。以前至武泉,擊胡騎,破之武泉北。轉攻韓信軍銅鞮,破之。還,降太原六城。擊韓信胡騎晉陽下,破之,下晉陽。後擊韓信軍於硰石,破之,追北八十里。還攻樓煩三城,因擊胡騎平城下,所將卒當馳道為多。勃遷為太尉。

擊陳豨,屠馬邑。所將卒斬豨將軍乘馬絺。擊韓信、陳豨、趙利軍於樓煩,破之。得豨將宋最、鴈門守圂。因轉攻得雲中守遬、丞相箕肆、將勳。定鴈門郡十七縣,雲中郡十二縣。因復擊豨靈丘,破之,斬豨,得豨丞相程縱、將軍陳武、都尉高肆。定代郡九縣。

燕王盧綰反,勃以相國代樊噲將,擊下薊,得綰大將抵、丞相偃、守陘、太尉弱、御史大夫施,屠渾都。破綰軍上蘭,復擊破綰軍沮陽。追至長城,定上谷十二縣,右北平十六縣,遼西、遼東二十九縣,漁陽二十二縣。最從高帝得相國一人,丞相二人,將軍、二千石各三人,別破軍二,下城三,定郡五,縣七十九,得丞相、大將各一人。

勃為人木彊敦厚,高帝以為可屬大事。勃不好文學,每召諸生說士,東鄉坐而責之,趣為我語。其椎少文如此。

勃既定燕而歸,高祖已崩矣,以列侯事孝惠帝。孝惠帝六年,置太尉官,以勃為太尉。十歲,高后崩。呂祿以趙王為漢上將軍,呂產以呂王為漢相國,秉漢權,欲危劉氏。勃為太尉,不得入軍門。陳平為丞相,不得任事。於是勃與平謀,卒誅諸呂而立孝文皇帝。其語在呂后、孝文事中。

文帝既立,以勃為右丞相,賜金五千斤,食邑萬戶。居月餘,人或說勃曰,君既誅諸呂,立代王,威震天下,而君受厚賞,處尊位,以寵,久之即禍及身矣。勃懼,亦自危,乃謝請歸相印。上許之。歲餘,丞相平卒,上復以勃為丞相。十餘月,上曰,前日吾詔列侯就國,或未能行,丞相吾所重,其率先之。乃免相就國。

歲餘,每河東守尉行縣至絳,絳侯勃自畏恐誅,常被甲,令家人持兵以見之。其後人有上書告勃欲反,下廷尉。廷尉下其事長安,逮捕勃治之。勃恐,不知置辭。吏稍侵辱之。勃以千金與獄吏,獄吏乃書牘背示之,曰以公主為證。公主者,孝文帝女也,勃太子勝之尚之,故獄吏教引為證。勃之益封受賜,盡以予薄昭。及系急,薄昭為言薄太后,太后亦以為無反事。文帝朝,太后以冒絮提文帝,曰,絳侯綰皇帝璽,將兵於北軍,不以此時反,今居一小縣,顧欲反邪。文帝既見絳侯獄辭,乃謝曰,吏事方驗而出之。於是使使持節赦絳侯,復爵邑。絳侯既出,曰,吾嘗將百萬軍,然安知獄吏之貴乎。

絳侯復就國。孝文帝十一年卒,謚為武侯。子勝之代侯。六歲,尚公主,不相中,坐殺人,國除。絕一歲,文帝乃擇絳侯勃子賢者河內守亞夫,封為條侯,續絳侯後。

條侯亞夫自未侯為河內守時,許負相之,曰,君後三歲而侯。侯八歲為將相,持國秉,貴重矣,於人臣無兩。其後九歲而君餓死。亞夫笑曰,臣之兄已代父侯矣,有如卒,子當代,亞夫何說侯乎。然既已貴如負言,又何說餓死。指示我。許負指其口曰,有從理入口,此餓死法也。居三歲,其兄絳侯勝之有罪,孝文帝擇絳侯子賢者,皆推亞夫,乃封亞夫為條侯,續絳侯後。

文帝之後六年,匈奴大入邊。乃以宗正劉禮為將軍,軍霸上,祝茲侯徐厲為將軍,軍棘門,以河內守亞夫為將軍,軍細柳,以備胡。上自勞軍。至霸上及棘門軍,直馳入,將以下騎送迎。已而之細柳軍,軍士吏被甲,銳兵刃,彀弓弩,持滿。天子先驅至,不得入。先驅曰,天子且至。軍門都尉曰,將軍令曰軍中聞將軍令,不聞天子之詔。居無何,上至,又不得入。於是上乃使使持節詔將軍,吾欲入勞軍。亞夫乃傳言開壁門。壁門士吏謂從屬車騎曰,將軍約,軍中不得驅馳。於是天子乃按轡徐行。至營,將軍亞夫持兵揖曰,介胄之士不拜,請以軍禮見。天子為動,改容式車。使人稱謝,皇帝敬勞將軍。成禮而去。既出軍門,群臣皆驚。文帝曰,嗟乎,此真將軍矣。曩者霸上、棘門軍,若兒戲耳,其將固可襲而虜也。至於亞夫,可得而犯邪。稱善者久之。月餘,三軍皆罷。乃拜亞夫為中尉。

孝文且崩時,誡太子曰,即有緩急,周亞夫真可任將兵。文帝崩,拜亞夫為車騎將軍。

孝景三年,吳楚反。亞夫以中尉為太尉,東擊吳楚。因自請上曰,楚兵剽輕,難與爭鋒。願以梁委之,絕其糧道,乃可制。上許之。

太尉既會兵滎陽,吳方攻梁,梁急,請救。太尉引兵東北走昌邑,深壁而守。梁日使使請太尉,太尉守便宜,不肯往。梁上書言景帝,景帝使使詔救梁。太尉不奉詔,堅壁不出,而使輕騎兵弓高侯等絕吳楚兵後食道。吳兵乏糧,饑,數欲挑戰,終不出。夜,軍中驚,內相攻擊擾亂,至於太尉帳下。太尉終臥不起。頃之,復定。後吳奔壁東南陬,太尉使備西北。已而其精兵果奔西北,不得入。吳兵既餓,乃引而去。太尉出精兵追擊,大破之。吳王濞棄其軍,而與壯士數千人亡走,保於江南丹徒。漢兵因乘勝,遂盡虜之,降其兵,購吳王千金。月餘,越人斬吳王頭以告。凡相攻守三月,而吳楚破平。於是諸將乃以太尉計謀為是。由此梁孝王與太尉有卻。

歸,復置太尉官。五歲,遷為丞相,景帝甚重之。景帝廢栗太子,丞相固爭之,不得。景帝由此疏之。而梁孝王每朝,常與太后言條侯之短。

竇太后曰,皇后兄王信可侯也。景帝讓曰,始南皮、章武侯先帝不侯,及臣即位乃侯之。信未得封也。竇太后曰,人主各以時行耳。自竇長君在時,竟不得侯,死後乃封其子彭祖顧得侯。吾甚恨之。帝趣侯信也。景帝曰,請得與丞相議之。丞相議之,亞夫曰,高皇帝約非劉氏不得王,非有功不得侯。不如約,天下共擊之。今信雖皇后兄,無功,侯之,非約也。景帝默然而止。

其後匈奴王唯徐盧等五人降,景帝欲侯之以勸後。丞相亞夫曰,彼背其主降陛下,陛下侯之,則何以責人臣不守節者乎。景帝曰,丞相議不可用。乃悉封唯徐盧等為列侯。亞夫因謝病。景帝中三年,以病免相。

頃之,景帝居禁中,召條侯,賜食。獨置大胾,無切肉,又不置櫡。條侯心不平,顧謂尚席取櫡。景帝視而笑曰,此不足君所乎。條侯免冠謝。上起,條侯因趨出。景帝以目送之,曰,此怏怏者非少主臣也。

居無何,條侯子為父買工官尚方甲楯五百被可以葬者。取庸苦之,不予錢。庸知其盜買縣官器,怒而上變告子,事連汙條侯。書既聞上,上下吏。吏簿責條侯,條侯不對。景帝罵之曰,吾不用也。召詣廷尉。廷尉責曰,君侯欲反邪。亞夫曰,臣所買器,乃葬器也,何謂反邪。吏曰,君侯縱不反地上,即欲反地下耳。吏侵之益急。初,吏捕條侯,條侯欲自殺,夫人止之,以故不得死,遂入廷尉。因不食五日,嘔血而死。國除。

絕一歲,景帝乃更封絳侯勃他子堅為平曲侯,續絳侯後。十九年卒,謚為共侯。子建德代侯,十三年,為太子太傅。坐酎金不善,元鼎五年,有罪,國除。

條侯果餓死。死後,景帝乃封王信為蓋侯。

太史公曰,絳侯周勃始為布衣時,鄙樸人也,才能不過凡庸。及從高祖定天下,在將相位,諸呂欲作亂,勃匡國家難,復之乎正。雖伊尹、周公,何以加哉。亞夫之用兵,持威重,執堅刃,穰苴曷有加焉。足己而不學,守節不遜,終以窮困。悲夫。