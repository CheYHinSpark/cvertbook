\chapter{留侯世家第二十五}

留侯張良者,其先韓人也。大父開地,相韓昭侯、宣惠王、襄哀王。父平,相釐王、悼惠王。悼惠王二十三年,平卒。卒二十歲,秦滅韓。良年少,未宦事韓。韓破,良家僮三百人,弟死不葬,悉以家財求客刺秦王,為韓報仇,以大父、父五世相韓故。

良嘗學禮淮陽。東見倉海君。得力士,為鐵椎重百二十斤。秦皇帝東游,良與客狙擊秦皇帝博浪沙中,誤中副車。秦皇帝大怒,大索天下,求賊甚急,為張良故也。良乃更名姓,亡匿下邳。

良嘗閒從容步游下邳圯上,有一老父,衣褐,至良所,直墮其履圯下,顧謂良曰,孺子,下取履。良鄂然,欲毆之。為其老,彊忍,下取履。父曰,履我。良業為取履,因長跪履之。父以足受,笑而去。良殊大驚,隨目之。父去里所,復還,曰,孺子可教矣。後五日平明,與我會此。良因怪之,跪曰,諾。五日平明,良往。父已先在,怒曰,與老人期,後,何也。去,曰,後五日早會。五日雞鳴,良往。父又先在,復怒曰,後,何也。去,曰,後五日復早來。五日,良夜未半往。有頃,父亦來,喜曰,當如是。出一編書,曰,讀此則為王者師矣。後十年興。十三年孺子見我濟北,穀城山下黃石即我矣。遂去,無他言,不復見。旦日視其書,乃太公兵法也。良因異之,常習誦讀之。

居下邳,為任俠。項伯常殺人,從良匿。

後十年,陳涉等起兵,良亦聚少年百餘人。景駒自立為楚假王,在留。良欲往從之,道還沛公。沛公將數千人,略地下邳西,遂屬焉。沛公拜良為廄將。良數以太公兵法說沛公,沛公善之,常用其策。良為他人者,皆不省。良曰,沛公殆天授。故遂從之,不去見景駒。

及沛公之薛,見項梁。項梁立楚懷王。良乃說項梁曰,君已立楚後,而韓諸公子橫陽君成賢,可立為王,益樹黨。項梁使良求韓成,立以為韓王。以良為韓申徒,與韓王將千餘人西略韓地,得數城,秦輒復取之,往來為游兵潁川。

沛公之從雒陽南出轘轅,良引兵從沛公,下韓十餘城,擊破楊熊軍。沛公乃令韓王成留守陽翟,與良俱南,攻下宛,西入武關。沛公欲以兵二萬人擊秦嶢下軍,良說曰,秦兵尚彊,未可輕。臣聞其將屠者子,賈豎易動以利。願沛公且留壁,使人先行,為五萬人具食,益為張旗幟諸山上,為疑兵,令酈食其持重寶啗秦將。秦將果畔,欲連和俱西襲咸陽,沛公欲聽之。良曰,此獨其將欲叛耳,恐士卒不從。不從必危,不如因其解擊之。沛公乃引兵擊秦軍,大破之。逐北至藍田,再戰,秦兵竟敗。遂至咸陽,秦王子嬰降沛公。

沛公入秦宮,宮室帷帳狗馬重寶婦女以千數,意欲留居之。樊噲諫沛公出舍,沛公不聽。良曰,夫秦為無道,故沛公得至此。夫為天下除殘賊,宜縞素為資。今始入秦,即安其樂,此所謂助桀為虐。且忠言逆耳利於行,毒藥苦口利於病,願沛公聽樊噲言。沛公乃還軍霸上。

項羽至鴻門下,欲擊沛公,項伯乃夜馳入沛公軍,私見張良,欲與俱去。良曰,臣為韓王送沛公,今事有急,亡去不義。乃具以語沛公。沛公大驚,曰,為將奈何。良曰,沛公誠欲倍項羽邪。沛公曰,鯫生教我距關無內諸侯,秦地可盡王,故聽之。良曰,沛公自度能卻項羽乎。沛公默然良久,曰,固不能也。今為奈何。良乃固要項伯。項伯見沛公。沛公與飲為壽,結賓婚。令項伯具言沛公不敢倍項羽,所以距關者,備他盜也。及見項羽後解,語在項羽事中。

漢元年正月,沛公為漢王,王巴蜀。漢王賜良金百鎰,珠二斗,良具以獻項伯。漢王亦因令良厚遺項伯,使請漢中地。項王乃許之,遂得漢中地。漢王之國,良送至襃中,遣良歸韓。良因說漢王曰,王何不燒絕所過棧道,示天下無還心,以固項王意。乃使良還。行,燒絕棧道。

良至韓,韓王成以良從漢王故,項王不遣成之國,從與俱東。良說項王曰,漢王燒絕棧道,無還心矣。乃以齊王田榮反,書告項王。項王以此無西憂漢心,而發兵北擊齊。

項王竟不肯遣韓王,乃以為侯,又殺之彭城。良亡,間行歸漢王,漢王亦已還定三秦矣。復以良為成信侯,從東擊楚。至彭城,漢敗而還。至下邑,漢王下馬踞鞍而問曰,吾欲捐關以東等棄之,誰可與共功者。良進曰,九江王黥布,楚梟將,與項王有郄,彭越與齊王田榮反梁地,此兩人可急使。而漢王之將獨韓信可屬大事,當一面。即欲捐之,捐之此三人,則楚可破也。漢王乃遣隨何說九江王布,而使人連彭越。及魏王豹反,使韓信將兵擊之,因舉燕、代、齊、趙。然卒破楚者,此三人力也。

張良多病,未嘗特將也,常為畫策,時時從漢王。

漢三年,項羽急圍漢王滎陽,漢王恐憂,與酈食其謀橈楚權。食其曰,昔湯伐桀,封其後於杞。武王伐紂,封其後於宋。今秦失德棄義,侵伐諸侯社稷,滅六國之後,使無立錐之地。陛下誠能復立六國後世,畢已受印,此其君臣百姓必皆戴陛下之德,莫不鄉風慕義,願為臣妾。德義已行,陛下南鄉稱霸,楚必斂衽而朝。漢王曰,善。趣刻印,先生因行佩之矣。

食其未行,張良從外來謁。漢王方食,曰,子房前。客有為我計橈楚權者。其以酈生語告,曰,於子房何如。良曰,誰為陛下畫此計者。陛下事去矣。漢王曰,何哉。張良對曰,臣請藉前箸為大王籌之。曰,昔者湯伐桀而封其後於杞者,度能制桀之死命也。今陛下能制項籍之死命乎。曰,未能也。其不可一也。武王伐紂封其後於宋者,度能得紂之頭也。今陛下能得項籍之頭乎。曰,未能也。其不可二也。武王入殷,表商容之閭,釋箕子之拘,封比干之墓。今陛下能封聖人之墓,表賢者之閭,式智者之門乎。曰,未能也。其不可三也。發鉅橋之粟,散鹿臺之錢,以賜貧窮。今陛下能散府庫以賜貧窮乎。曰,未能也。其不可四矣。殷事已畢,偃革為軒,倒置干戈,覆以虎皮,以示天下不復用兵。今陛下能偃武行文,不復用兵乎。曰,未能也。其不可五矣。休馬華山之陽,示以無所為。今陛下能休馬無所用乎。曰,未能也。其不可六矣。放牛桃林之陰,以示不復輸積。今陛下能放牛不復輸積乎。曰,未能也。其不可七矣。且天下游士離其親戚,棄墳墓,去故舊,從陛下游者,徒欲日夜望咫尺之地。今復六國,立韓、魏、燕、趙、齊、楚之後,天下游士各歸事其主,從其親戚,反其故舊墳墓,陛下與誰取天下乎。其不可八矣。且夫楚唯無彊,六國立者復橈而從之,陛下焉得而臣之。誠用客之謀,陛下事去矣。漢王輟食吐哺,罵曰,豎儒,幾敗而公事。令趣銷印。

漢四年,韓信破齊而欲自立為齊王,漢王怒。張良說漢王,漢王使良授齊王信印,語在淮陰事中。

其秋,漢王追楚至陽夏南,戰不利而壁固陵,諸侯期不至。良說漢王,漢王用其計,諸侯皆至。語在項籍事中。

漢六年正月,封功臣。良未嘗有戰鬬功,高帝曰,運籌策帷帳中,決勝千里外,子房功也。自擇齊三萬戶。良曰,始臣起下邳,與上會留,此天以臣授陛下。陛下用臣計,幸而時中,臣願封留足矣,不敢當三萬戶。乃封張良為留侯,與蕭何等俱封。

六年上已封大功臣二十餘人,其餘日夜爭功不決,未得行封。上在雒陽南宮,從複道望見諸將往往相與坐沙中語。上曰,此何語。留侯曰,陛下不知乎。此謀反耳。上曰,天下屬安定,何故反乎。留侯曰,陛下起布衣,以此屬取天下,今陛下為天子,而所封皆蕭、曹故人所親愛,而所誅者皆生平所仇怨。今軍吏計功,以天下不足遍封,此屬畏陛下不能盡封,恐又見疑平生過失及誅,故即相聚謀反耳。上乃憂曰,為之奈何。留侯曰,上平生所憎,群臣所共知,誰最甚者。上曰,雍齒與我故,數嘗窘辱我。我欲殺之,為其功多,故不忍。留侯曰,今急先封雍齒以示群臣,群臣見雍齒封,則人人自堅矣。於是上乃置酒,封雍齒為什方侯,而急趣丞相、御史定功行封。群臣罷酒,皆喜曰,雍齒尚為侯,我屬無患矣。

劉敬說高帝曰,都關中。上疑之。左右大臣皆山東人,多勸上都雒陽,雒陽東有成皋,西有殽黽,倍河,向伊雒,其固亦足恃。留侯曰,雒陽雖有此固,其中小,不過數百里,田地薄,四面受敵,此非用武之國也。夫關中左殽函,右隴蜀,沃野千里,南有巴蜀之饒,北有胡苑之利,阻三面而守,獨以一面東制諸侯。諸侯安定,河渭漕輓天下,西給京師,諸侯有變,順流而下,足以委輸。此所謂金城千里,天府之國也,劉敬說是也。於是高帝即日駕,西都關中。

留侯從入關。留侯性多病,即道引不食穀,杜門不出歲餘。

上欲廢太子,立戚夫人子趙王如意。大臣多諫爭,未能得堅決者也。呂后恐,不知所為。人或謂呂后曰,留侯善畫計筴,上信用之。呂后乃使建成侯呂澤劫留侯,曰,君常為上謀臣,今上欲易太子,君安得高枕而臥乎。留侯曰,始上數在困急之中,幸用臣筴。今天下安定,以愛欲易太子,骨肉之間,雖臣等百餘人何益。呂澤彊要曰,為我畫計。留侯曰,此難以口舌爭也。顧上有不能致者,天下有四人。四人者年老矣,皆以為上慢侮人,故逃匿山中,義不為漢臣。然上高此四人。今公誠能無愛金玉璧帛,令太子為書,卑辭安車,因使辯士固請,宜來。來,以為客,時時從入朝,令上見之,則必異而問之。問之,上知此四人賢,則一助也。於是呂后令呂澤使人奉太子書,卑辭厚禮,迎此四人。四人至,客建成侯所。

漢十一年,黥布反,上病,欲使太子將,往擊之。四人相謂曰,凡來者,將以存太子。太子將兵,事危矣。乃說建成侯曰,太子將兵,有功則位不益太子,無功還,則從此受禍矣。且太子所與俱諸將,皆嘗與上定天下梟將也,今使太子將之,此無異使羊將狼也,皆不肯為盡力,其無功必矣。臣聞母愛者子抱,今戚夫人日夜待御,趙王如意常抱居前,上曰終不使不肖子居愛子之上,明乎其代太子位必矣。君何不急請呂后承間為上泣言,黥布,天下猛將也,善用兵,今諸將皆陛下故等夷,乃令太子將此屬,無異使羊將狼,莫肯為用,且使布聞之,則鼓行而西耳。上雖病,彊載輜車,臥而護之,諸將不敢不盡力。上雖苦,為妻子自彊。於是呂澤立夜見呂后,呂后承間為上泣涕而言,如四人意。上曰,吾惟豎子固不足遣,而公自行耳。於是上自將兵而東,群臣居守,皆送至灞上。留侯病,自彊起,至曲郵,見上曰,臣宜從,病甚。楚人剽疾,願上無與楚人爭鋒。因說上曰,令太子為將軍,監關中兵。上曰,子房雖病,彊臥而傅太子。是時叔孫通為太傅,留侯行少傅事。

漢十二年,上從擊破布軍歸,疾益甚,愈欲易太子。留侯諫,不聽,因疾不視事。叔孫太傅稱說引古今,以死爭太子。上詳許之,猶欲易之。及燕,置酒,太子侍。四人從太子,年皆八十有餘,鬚眉皓白,衣冠甚偉。上怪之,問曰,彼何為者。四人前對,各言名姓,曰東園公,角里先生,綺里季,夏黃公。上乃大驚,曰,吾求公數歲,公辟逃我,今公何自從吾兒游乎。四人皆曰,陛下輕士善罵,臣等義不受辱,故恐而亡匿。竊聞太子為人仁孝,恭敬愛士,天下莫不延頸欲為太子死者,故臣等來耳。上曰,煩公幸卒調護太子。

四人為壽已畢,趨去。上目送之,召戚夫人指示四人者曰,我欲易之,彼四人輔之,羽翼已成,難動矣。呂后真而主矣。戚夫人泣,上曰,為我楚舞,吾為若楚歌。歌曰,鴻鴈高飛,一舉千里。羽翮已就,橫絕四海。橫絕四海,當可奈何。雖有矰繳,尚安所施。歌數闋,戚夫人噓唏流涕,上起去,罷酒。竟不易太子者,留侯本招此四人之力也。

留侯從上擊代,出奇計馬邑下,及立蕭何相國,所與上從容言天下事甚眾,非天下所以存亡,故不著。留侯乃稱曰,家世相韓,及韓滅,不愛萬金之資,為韓報讐彊秦,天下振動。今以三寸舌為帝者師,封萬戶,位列侯,此布衣之極,於良足矣。願棄人間事,欲從赤松子游耳。乃學辟穀,道引輕身。會高帝崩,呂后德留侯,乃彊食之,曰,人生一世間,如白駒過隙,何至自苦如此乎。留侯不得已,彊聽而食。

後八年卒,謚為文成侯。子不疑代侯。

子房始所見下邳圯上老父與太公書者,後十三年從高帝過濟北,果見穀城山下黃石,取而葆祠之。留侯死,并葬黃石。每上冢伏臘,祠黃石。

留侯不疑,孝文帝五年坐不敬,國除。

太史公曰,學者多言無鬼神,然言有物。至如留侯所見老父予書,亦可怪矣。高祖離困者數矣,而留侯常有功力焉,豈可謂非天乎。上曰,夫運籌筴帷帳之中,決勝千里外,吾不如子房。余以為其人計魁梧奇偉,至見其圖,狀貌如婦人好女。蓋孔子曰,以貌取人,失之子羽。留侯亦云。