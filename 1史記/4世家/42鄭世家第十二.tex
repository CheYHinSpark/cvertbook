\chapter{鄭世家第十二}

鄭桓公友者,周厲王少子而宣王庶弟也。宣王立二十二年,友初封于鄭。封三十三歲,百姓皆便愛之。幽王以為司徒。和集周民,周民皆說,河雒之閒,人便思之。為司徒一歲,幽王以褒后故,王室治多邪,諸侯或畔之。於是桓公問太史伯曰,王室多故,予安逃死乎。太史伯對曰,獨雒之東土,河濟之南可居。公曰,何以。對曰,地近虢、鄶,虢、鄶之君貪而好利,百姓不附。今公為司徒,民皆愛公,公誠請居之,虢、鄶之君見公方用事,輕分公地。公誠居之,虢、鄶之民皆公之民也。公曰,吾欲南之江上,何如。對曰,昔祝融為高辛氏火正,其功大矣,而其於周未有興者,楚其後也。周衰,楚必興。興,非鄭之利也。公曰,吾欲居西方,何如。對曰,其民貪而好利,難久居。公曰,周衰,何國興者。對曰,齊、秦、晉、楚乎。夫齊,姜姓,伯夷之後也,伯夷佐堯典禮。秦,嬴姓,伯翳之後也,伯翳佐舜懷柔百物。及楚之先,皆嘗有功於天下。而周武王克紂後,成王封叔虞于唐,其地阻險,以此有德與周衰并,亦必興矣。桓公曰,善。於是卒言王,東徙其民雒東,而虢、鄶果獻十邑,竟國之。

二歲,犬戎殺幽王於驪山下,并殺桓公。鄭人共立其子掘突,是為武公。

武公十年,娶申侯女為夫人,曰武姜。生太子寤生,生之難,及生,夫人弗愛。後生少子叔段,段生易,夫人愛之。二十七年,武公疾。夫人請公,欲立段為太子,公弗聽。是歲,武公卒,寤生立,是為莊公。

莊公元年,封弟段於京,號太叔。祭仲曰,京大於國,非所以封庶也。莊公曰,武姜欲之,我弗敢奪也。段至京,繕治甲兵,與其母武姜謀襲鄭。二十二年,段果襲鄭,武姜為內應。莊公發兵伐段,段走。伐京,京人畔段,段出走鄢。鄢潰,段出奔共。於是莊公遷其母武姜於城潁,誓言曰,不至黃泉,毋相見也。居歲餘,已悔思母。潁谷之考叔有獻於公,公賜食。考叔曰,臣有母,請君食賜臣母。莊公曰,我甚思母,惡負盟,柰何。考叔曰,穿地至黃泉,則相見矣。於是遂從之,見母。

二十四年,宋繆公卒,公子馮奔鄭。鄭侵周地,取禾。二十五年,衛州吁弒其君桓公自立,與宋伐鄭,以馮故也。二十七年,始朝周桓王。桓王怒其取禾,弗禮也。二十九年,莊公怒周弗禮,與魯易祊、許田。三十三年,宋殺孔父。三十七年,莊公不朝周,周桓王率陳、蔡、虢、衛伐鄭。莊公與祭仲、高渠彌發兵自救,王師大敗。祝聸射中王臂。祝聸請從之,鄭伯止之,曰,犯長且難之,況敢陵天子乎。乃止。夜令祭仲問王疾。

三十八年,北戎伐齊,齊使求救,鄭遣太子忽將兵救齊。齊釐公欲妻之,忽謝曰,我小國,非齊敵也。時祭仲與俱,勸使取之,曰,君多內寵,太子無大援將不立,三公子皆君也。所謂三公子者,太子忽,其弟突,次弟子亹也。

四十三年,鄭莊公卒。初,祭仲甚有寵於莊公,莊公使為卿,公使娶鄧女,生太子忽,故祭仲立之,是為昭公。

莊公又娶宋雍氏女,生厲公突。雍氏有寵於宋。宋莊公聞祭仲之立忽,乃使人誘召祭仲而執之,曰,不立突,將死。亦執突以求賂焉。祭仲許宋,與宋盟。以突歸,立之。昭公忽聞祭仲以宋要立其弟突,九月丁亥,忽出奔衛。己亥,突至鄭,立,是為厲公。

厲公四年,祭仲專國政。厲公患之,陰使其婿雍糾欲殺祭仲。糾妻,祭仲女也,知之,謂其母曰,父與夫孰親。母曰,父一而已,人盡夫也。女乃告祭仲,祭仲反殺雍糾,戮之於市。厲公無柰祭仲何,怒糾曰,謀及婦人,死固宜哉。夏,厲公出居邊邑櫟。祭仲迎昭公忽,六月乙亥,復入鄭,即位。

秋,鄭厲公突因櫟人殺其大夫單伯,遂居之。諸侯聞厲公出奔,伐鄭,弗克而去。宋頗予厲公兵,自守於櫟,鄭以故亦不伐櫟。

昭公二年,自昭公為太子時,父莊公欲以高渠彌為卿,太子忽惡之,莊公弗聽,卒用渠彌為卿。及昭公即位,懼其殺己,冬十月辛卯,渠彌與昭公出獵,射殺昭公於野。祭仲與渠彌不敢入厲公,乃更立昭公弟子亹為君,是為子亹也,無謚號。

子亹元年七月,齊襄公會諸侯於首止,鄭子亹往會,高渠彌相,從,祭仲稱疾不行。所以然者,子亹自齊襄公為公子之時,嘗會鬬,相仇,及會諸侯,祭仲請子亹無行。子亹曰,齊彊,而厲公居櫟,即不往,是率諸侯伐我,內厲公。我不如往,往何遽必辱,且又何至是。卒行。於是祭仲恐齊并殺之,故稱疾。子亹至,不謝齊侯,齊侯怒,遂伏甲而殺子亹。高渠彌亡歸,歸與祭仲謀,召子亹弟公子嬰於陳而立之,是為鄭子。是歲,齊襄公使彭生醉拉殺魯桓公。

鄭子八年,齊人管至父等作亂,弒其君襄公。十二年,宋人長萬弒其君湣公。鄭祭仲死。

十四年,故鄭亡厲公突在櫟者使人誘劫鄭大夫甫假,要以求入。假曰,捨我,我為君殺鄭子而入君。厲公與盟,乃捨之。六月甲子,假殺鄭子及其二子而迎厲公突,突自櫟復入即位。初,內蛇與外蛇鬬於鄭南門中,內蛇死。居六年,厲公果復入。入而讓其伯父原曰,我亡國外居,伯父無意入我,亦甚矣。原曰,事君無二心,人臣之職也。原知罪矣。遂自殺。厲公於是謂甫假曰,子之事君有二心矣。遂誅之。假曰,重德不報,誠然哉。

厲公突後元年,齊桓公始霸。

五年,燕、衛與周惠王弟穨伐王,王出奔溫,立弟穨為王。六年,惠王告急鄭,厲公發兵擊周王子穨,弗勝,於是與周惠王歸,王居于櫟。七年春,鄭厲公與虢叔襲殺王子穨而入惠王于周。

秋,厲公卒,子文公踕立。厲公初立四歲,亡居櫟,居櫟十七歲,復入,立七歲,與亡凡二十八年。

二十四年,文公之賤妾曰燕姞,夢天與之蘭,曰,余為伯鯈。余,爾祖也。以是為而子,蘭有國香。以夢告文公,文公幸之,而予之草蘭為符。遂生子,名曰蘭。

三十六年,晉公子重耳過,文公弗禮。文公弟叔詹曰,重耳賢,且又同姓,窮而過君,不可無禮。文公曰,諸侯亡公子過者多矣,安能盡禮之。詹曰,君如弗禮,遂殺之,弗殺,使即反國,為鄭憂矣。文公弗聽。

三十七年春,晉公子重耳反國,立,是為文公。秋,鄭入滑,滑聽命,已而反與衛,於是鄭伐滑。周襄王使伯服請滑。鄭文公怨惠王之亡在櫟,而文公父厲公入之,而惠王不賜厲公爵祿,又怨襄王之與衛滑,故不聽襄王請而囚伯服。王怒,與翟人伐鄭,弗克。冬,翟攻伐襄王,襄王出奔鄭,鄭文公居王于氾。三十八年,晉文公入襄王成周。

四十一年,助楚擊晉。自晉文公之過無禮,故背晉助楚。四十三年,晉文公與秦穆公共圍鄭,討其助楚攻晉者,及文公過時之無禮也。初,鄭文公有三夫人,寵子五人,皆以罪蚤死。公怒,溉逐群公子。子蘭奔晉,從晉文公圍鄭。時蘭事晉文公甚謹,愛幸之,乃私於晉,以求入鄭為太子。晉於是欲得叔詹為僇。鄭文公恐,不敢謂叔詹言。詹聞,言於鄭君曰,臣謂君,君不聽臣,晉卒為患。然晉所以圍鄭,以詹,詹死而赦鄭國,詹之願也。乃自殺。鄭人以詹尸與晉。晉文公曰,必欲一見鄭君,辱之而去。鄭人患之,乃使人私於秦曰,破鄭益晉,非秦之利也。秦兵罷。晉文公欲入蘭為太子,以告鄭。鄭大夫石癸曰,吾聞姞姓乃后稷之元妃,其後當有興者。子蘭母,其後也。且夫人子盡已死,餘庶子無如蘭賢。今圍急,晉以為請,利孰大焉。遂許晉,與盟,而卒立子蘭為太子,晉兵乃罷去。

四十五年,文公卒,子蘭立,是為繆公。

繆公元年春,秦繆公使三將將兵欲襲鄭,至滑,逢鄭賈人弦高詐以十二牛勞軍,故秦兵不至而還,晉敗之於崤。初,往年鄭文公之卒也,鄭司城繒賀以鄭情賣之,秦兵故來。三年,鄭發兵從晉伐秦,敗秦兵於汪。

往年楚太子商臣弒其父成王代立。二十一年,與宋華元伐鄭。華元殺羊食士,不與其御羊斟,怒以馳鄭,鄭囚華元。宋贖華元,元亦亡去。晉使趙穿以兵伐鄭。

靈公元年春,楚獻黿於靈公。子家、子公將朝靈公,子公之食指動,謂子家曰,佗日指動,必食異物。及入,見靈公進黿羹,子公笑曰,果然。靈公問其笑故,具告靈公。靈公召之,獨弗予羹。子公怒,染其指,嘗之而出。公怒,欲殺子公。子公與子家謀先。夏,弒靈公。鄭人欲立靈公弟去疾,去疾讓曰,必以賢,則去疾不肖,必以順,則公子堅長。堅者,靈公庶弟,去疾之兄也。於是乃立子堅,是為襄公。

襄公立,將盡去繆氏。繆氏者,殺靈公、子公之族家也。去疾曰,必去繆氏,我將去之。乃止。皆以為大夫。

襄公元年,楚怒鄭受宋賂縱華元,伐鄭。鄭背楚,與晉親。五年,楚復伐鄭,晉來救之。六年,子家卒,國人復逐其族,以其弒靈公也。

七年,鄭與晉盟鄢陵。八年,楚莊王以鄭與晉盟,來伐,圍鄭三月,鄭以城降楚。楚王入自皇門,鄭襄公肉袒掔羊以迎,曰,孤不能事邊邑,使君王懷怒以及獘邑,孤之罪也。敢不惟命是聽。君王遷之江南,及以賜諸侯,亦惟命是聽。若君王不忘厲、宣王,桓、武公,哀不忍絕其社稷,錫不毛之地,使復得改事君王,孤之願也,然非所敢望也。敢布腹心,惟命是聽。莊王為卻三十里而後舍。楚群臣曰,自郢至此,士大夫亦久勞矣。今得國捨之,何如。莊王曰,所為伐,伐不服也。今已服,尚何求乎。卒去。晉聞楚之伐鄭,發兵救鄭。其來持兩端,故遲,比至河,楚兵已去。晉將率或欲渡,或欲還,卒渡河。莊王聞,還擊晉。鄭反助楚,大破晉軍於河上。十年,晉來伐鄭,以其反晉而親楚也。

十一年,楚莊王伐宋,宋告急于晉。晉景公欲發兵救宋,伯宗諫晉君曰,天方開楚,未可伐也。乃求壯士得霍人解揚,字子虎,誆楚,令宋毋降。過鄭,鄭與楚親,乃執解揚而獻楚。楚王厚賜與約,使反其言,令宋趣降,三要乃許。於是楚登解揚樓車,令呼宋。遂負楚約而致其晉君命曰,晉方悉國兵以救宋,宋雖急,慎毋降楚,晉兵今至矣。楚莊王大怒,將殺之。解揚曰,君能制命為義,臣能承命為信。受吾君命以出,有死無隕。莊王曰,若之許我,已而背之,其信安在。解揚曰,所以許王,欲以成吾君命也。將死,顧謂楚軍曰,為人臣無忘盡忠得死者。楚王諸弟皆諫王赦之,於是赦解揚使歸。晉爵之為上卿。

十八年,襄公卒,子悼公沸立。

悼公元年,鄦公惡鄭於楚,悼公使弟睔於楚自訟。訟不直,楚囚睔。於是鄭悼公來與晉平,遂親。睔私於楚子反,子反言歸睔於鄭。

二年,楚伐鄭,晉兵來救。是歲,悼公卒,立其弟睔,是為成公。

成公三年,楚共王曰鄭成公孤有德焉,使人來與盟。成公私與盟。秋,成公朝晉,晉曰鄭私平於楚,執之。使欒書伐鄭。四年春,鄭患晉圍,公子如乃立成公庶兄繻為君。其四月,晉聞鄭立君,乃歸成公。鄭人聞成公歸,亦殺君繻,迎成公。晉兵去。

十年,背晉盟,盟於楚。晉厲公怒,發兵伐鄭。楚共王救鄭。晉楚戰鄢陵,楚兵敗,晉射傷楚共王目,俱罷而去。十三年,晉悼公伐鄭,兵於洧上。鄭城守,晉亦去。

十四年,成公卒,子惲立。是為釐公。

釐公五年,鄭相子駟朝釐公,釐公不禮。子駟怒,使廚人藥殺釐公,赴諸侯曰釐公暴病卒。立釐公子嘉,嘉時年五歲,是為簡公。

簡公元年,諸公子謀欲誅相子駟,子駟覺之,反盡誅諸公子。二年,晉伐鄭,鄭與盟,晉去。冬,又與楚盟。子駟畏誅,故兩親晉、楚。三年,相子駟欲自立為君,公子子孔使尉止殺相子駟而代之。子孔又欲自立。子產曰,子駟為不可,誅之,今又效之,是亂無時息也。於是子孔從之而相鄭簡公。

四年,晉怒鄭與楚盟,伐鄭,鄭與盟。楚共王救鄭,敗晉兵。簡公欲與晉平,楚又囚鄭使者。

十二年,簡公怒相子孔專國權,誅之,而以子產為卿。十九年,簡公如晉請衛君還,而封子產以六邑。子產讓,受其三邑。二十二年,吳使延陵季子於鄭,見子產如舊交,謂子產曰,鄭之執政者侈,難將至,政將及子。子為政,必以禮,不然,鄭將敗。子產厚遇季子。二十三年,諸公子爭寵相殺,又欲殺子產。公子或諫曰,子產仁人,鄭所以存者子產也,勿殺。乃止。

二十五年,鄭使子產於晉,問平公疾。平公曰,卜而曰實沈、臺駘為祟,史官莫知,敢問。對曰,高辛氏有二子,長曰閼伯,季曰實沈,居曠林,不相能也,日操干戈以相征伐。后帝弗臧,遷閼伯于商丘,主辰,商人是因,故辰為商星。遷實沈于大夏,主參,唐人是因,服事夏、商,其季世曰唐叔虞。當武王邑姜方娠大叔,夢帝謂己,余命而子曰虞,乃與之唐,屬之參而蕃育其子孫。及生有文在其掌曰虞,遂以命之。及成王滅唐而國大叔焉。故參為晉星。由是觀之,則實沈,參神也。昔金天氏有裔子曰昧,為玄冥師,生允格、臺駘。臺駘能業其官,宣汾、洮,障大澤,以處太原。帝用嘉之,國之汾川。沈、姒、蓐、黃實守其祀。今晉主汾川而滅之。由是觀之,則臺駘,汾、洮神也。然是二者不害君身。山川之神,則水旱之菑禜之,日月星辰之神,則雪霜風雨不時禜之,若君疾,飲食哀樂女色所生也。平公及叔向曰,善,博物君子也。厚為之禮於子產。

二十七年夏,鄭簡公朝晉。冬,畏楚靈王之彊,又朝楚,子產從。二十八年,鄭君病,使子產會諸侯,與楚靈王盟於申,誅齊慶封。

三十六年,簡公卒,子定公寧立。秋,定公朝晉昭公。

定公元年,楚公子棄疾弒其君靈王而自立,為平王。欲行德諸侯。歸靈王所侵鄭地于鄭。

四年,晉昭公卒,其六卿彊,公室卑。子產謂韓宣子曰,為政必以德,毋忘所以立。六年,鄭火,公欲禳之。子產曰,不如修德。

八年,楚太子建來奔。十年,太子建與晉謀襲鄭。鄭殺建,建子勝奔吳。

十一年,定公如晉。晉與鄭謀,誅周亂臣,入敬王于周。

十三年,定公卒,子獻公蠆立。獻公十三年卒,子聲公勝立。當是時,晉六卿彊,侵奪鄭,鄭遂弱。

聲公五年,鄭相子產卒,鄭人皆哭泣,悲之如亡親戚。子產者,鄭成公少子也。為人仁愛人,事君忠厚。孔子嘗過鄭,與子產如兄弟云。及聞子產死,孔子為泣曰,古之遺愛也。

八年,晉范、中行氏反晉,告急於鄭,鄭救之。晉伐鄭,敗鄭軍於鐵。

十四年,宋景公滅曹。二十年,齊田常弒其君簡公,而常相於齊。二十二年,楚惠王滅陳。孔子卒。

三十六年,晉知伯伐鄭,取九邑。

三十七年,聲公卒,子哀公易立。哀公八年,鄭人弒哀公而立聲公弟丑,是為共公。共公三年,三晉滅知伯。三十一年,共公卒,子幽公已立。幽公元年,韓武子伐鄭,殺幽公。鄭人立幽公弟駘,是為繻公。

繻公十五年,韓景侯伐鄭,取雍丘。鄭城京。

十六年,鄭伐韓,敗韓兵於負黍。二十年,韓、趙、魏列為諸侯。二十三年,鄭圍韓之陽翟。

二十五年,鄭君殺其相子陽。二十七,子陽之黨共弒繻公駘而立幽公弟乙為君,是為鄭君。鄭君乙立二年,鄭負黍反,復歸韓。十一年,韓伐鄭,取陽城。

二十一年,韓哀侯滅鄭,并其國。

太史公曰,語有之,以權利合者,權利盡而交疏,甫瑕是也。甫瑕雖以劫殺鄭子內厲公,厲公終背而殺之,此與晉之裏克何異。守節如荀息,身死而不能存奚齊。變所從來,亦多故矣。