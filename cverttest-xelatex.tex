\documentclass[twocolumn]{cvertbook}

\usepackage{hyperref}

%\nopunct    %啟用無標點模式

\judou      %启用句读模式,默认为不启用

%\setlength{\vertbookoffset}{1cm}    %设置装订线距离,默认为1cm

\begin{document}

\tableofparts

\tableofcontents

\part{史記}

\chapter{五帝本紀第一}

黃帝者,少典之子,姓公孫,名曰軒轅。生而神靈,弱而能言,幼而徇齊,長而敦敏,成而聰明。

軒轅之時,神農氏世衰。諸侯相侵伐,暴虐百姓,而神農氏弗能征。於是軒轅乃習用干戈,以征不享,諸侯咸來賓從。而蚩尤最為暴,莫能伐。炎帝欲侵陵諸侯,諸侯咸歸軒轅。軒轅乃修德振兵,治五氣,藝五種,撫萬民,度四方,教熊羆貔貅貙虎,以與炎帝戰於阪泉之野。三戰然後得其志。蚩尤作亂,不用帝命。於是黃帝乃徵師諸侯,與蚩尤戰於涿鹿之野,遂禽殺蚩尤。而諸侯咸尊軒轅為天子,代神農氏,是為黃帝。天下有不順者,黃帝從而征之,平者去之,披山通道,未嘗寧居。

東至于海,登丸山,及岱宗。西至于空桐,登雞頭。南至于江,登熊、湘。北逐葷粥,合符釜山,而邑于涿鹿之阿。遷徙往來無常處,以師兵為營衛。官名皆以雲命,為雲師。置左右大監,監于萬國。萬國和,而鬼神山川封禪與為多焉。獲寶鼎,迎日推筴。舉風后、力牧、常先、大鴻以治民。順天地之紀,幽明之占,死生之說,存亡之難。時播百穀草木,淳化鳥獸蟲蛾,旁羅日月星辰水波土石金玉,勞勤心力耳目,節用水火材物。有土德之瑞,故號黃帝。

黃帝二十五子,其得姓者十四人。

黃帝居于軒轅之丘,而娶于西陵之女,是為嫘祖為黃帝正妃,生二子,其後皆有天下,其一曰玄囂,是為青陽,青陽降居江水,其二曰昌意,降居若水。昌意娶蜀山氏女,曰昌僕,生高陽,高陽有聖德焉。黃帝崩,葬橋山。其孫昌意之子高陽立,是為帝顓頊也。

帝顓頊高陽者,黃帝之孫而昌意之子也。靜淵以有謀,疏通而知事,養材以任地,載時以象天,依鬼神以制義,治氣以教化,絜誠以祭祀。北至于幽陵,南至于交阯,西至于流沙,東至于蟠木。動靜之物,大小之神,日月所照,莫不砥屬。

帝顓頊生子曰窮蟬。顓頊崩,而玄囂之孫高辛立,是為帝嚳。

帝嚳高辛者,黃帝之曾孫也。高辛父曰蟜極,蟜極父曰玄囂,玄囂父曰黃帝。自玄囂與蟜極皆不得在位,至高辛即帝位。高辛於顓頊為族子。

高辛生而神靈,自言其名。普施利物,不於其身。聰以知遠,明以察微。順天之義,知民之急。仁而威,惠而信,修身而天下服。取地之財而節用之,撫教萬民而利誨之,歷日月而迎送之,明鬼神而敬事之。其色郁郁,其德嶷嶷。其動也時,其服也士。帝嚳溉執中而遍天下,日月所照,風雨所至,莫不從服。

帝嚳娶陳鋒氏女,生放勛。娶娵訾氏女,生摯。帝嚳崩,而摯代立。帝摯立,不善,而弟放勛立,是為帝堯。

帝堯者,放勛。其仁如天,其知如神。就之如日,望之如雲。富而不驕,貴而不舒。黃收純衣,彤車乘白馬。能明馴德,以親九族。九族既睦,便章百姓。百姓昭明,合和萬國。

乃命羲、和,敬順昊天,數法日月星辰,敬授民時。分命羲仲,居郁夷,曰暘谷。敬道日出,便程東作。日中,星鳥,以殷中春。其民析,鳥獸字微。申命羲叔,居南交。便程南為,敬致。日永,星火,以正中夏。其民因,鳥獸希革。申命和仲,居西土,曰昧谷。敬道日入,便程西成。夜中,星虛,以正中秋。其民夷易,鳥獸毛毨。申命和叔,居北方,曰幽都。便在伏物。日短,星昴,以正中冬。其民燠,鳥獸氄毛。歲三百六十六日,以閏月正四時。信飭百官,眾功皆興。

堯曰,誰可順此事。放齊曰,嗣子丹朱開明。堯曰,吁。頑凶,不用。堯又曰,誰可者。讙兜曰,共工旁聚布功,可用。堯曰,共工善言,其用僻,似恭漫天,不可。堯又曰,嗟,四嶽,湯湯洪水滔天,浩浩懷山襄陵,下民其憂,有能使治者。皆曰鯀可。堯曰,鯀負命毀族,不可。嶽曰,異哉,試不可用而已。堯於是聽嶽用鯀。九歲,功用不成。

堯曰,嗟。四嶽,朕在位七十載,汝能庸命,踐朕位。嶽應曰,鄙德忝帝位。堯曰,悉舉貴戚及疏遠隱匿者。眾皆言於堯曰,有矜在民閒,曰虞舜。堯曰,然,朕聞之。其何如。嶽曰,盲者子。父頑,母嚚,弟傲,能和以孝,烝烝治,不至姦。堯曰,吾其試哉。於是堯妻之二女,觀其德於二女。舜飭下二女於媯汭,如婦禮。

堯善之,乃使舜慎和五典,五典能從。乃遍入百官,百官時序。賓於四門,四門穆穆,諸侯遠方賓客皆敬。堯使舜入山林川澤,暴風雷雨,舜行不迷。堯以為聖,召舜曰,女謀事至而言可績,三年矣。女登帝位。舜讓於德不懌。正月上日,舜受終於文祖。文祖者,堯大祖也。

於是帝堯老,命舜攝行天子之政,以觀天命。舜乃在璿璣玉衡,以齊七政。遂類于上帝,禋于六宗,望于山川,辯于群神。揖五瑞,擇吉月日,見四嶽諸牧,班瑞。歲二月,東巡狩,至於岱宗,祡,望秩於山川。遂見東方君長,合時月正日,同律度量衡,修五禮五玉三帛二生一死為摯,如五器,卒乃復。五月,南巡狩,八月,西巡狩,十一月,北巡狩,皆如初。歸,至于祖禰廟,用特牛禮。五歲一巡狩,群后四朝。遍告以言,明試以功,車服以庸。肇十有二州,決川。象以典刑,流宥五刑,鞭作官刑,撲作教刑,金作贖刑。眚災過,赦,怙終賊,刑。欽哉,欽哉,惟刑之靜哉。

讙兜進言共工,堯曰不可而試之工師,共工果淫辟。四嶽舉鯀治鴻水,堯以為不可,嶽彊請試之,試之而無功,故百姓不便。三苗在江淮、荊州數為亂。於是舜歸而言於帝,請流共工於幽陵,以變北狄,放讙兜於崇山,以變南蠻,遷三苗於三危,以變西戎,殛鯀於羽山,以變東夷,四罪而天下咸服。

堯立七十年得舜,二十年而老,令舜攝行天子之政,薦之於天。堯辟位凡二十八年而崩。百姓悲哀,如喪父母。三年,四方莫舉樂,以思堯。堯知子丹朱之不肖,不足授天下,於是乃權授舜。授舜,則天下得其利而丹朱病,授丹朱,則天下病而丹朱得其利。堯曰,終不以天下之病而利一人,而卒授舜以天下。堯崩,三年之喪畢,舜讓辟丹朱於南河之南。諸侯朝覲者不之丹朱而之舜,獄訟者不之丹朱而之舜,謳歌者不謳歌丹朱而謳歌舜。舜曰,天也,夫而後之中國踐天子位焉,是為帝舜。

虞舜者,名曰重華。重華父曰瞽叟,瞽叟父曰橋牛,橋牛父曰句望,句望父曰敬康,敬康父曰窮蟬,窮蟬父曰帝顓頊,顓頊父曰昌意,以至舜七世矣。自從窮蟬以至帝舜,皆微為庶人。

舜父瞽叟盲,而舜母死,瞽叟更娶妻而生象,象傲。瞽叟愛後妻子,常欲殺舜,舜避逃,及有小過,則受罪。順事父及後母與弟,日以篤謹,匪有解。

舜,冀州之人也。舜耕歷山,漁雷澤,陶河濱,作什器於壽丘,就時於負夏。舜父瞽叟頑,母嚚,弟象傲,皆欲殺舜。舜順適不失子道,兄弟孝慈。欲殺,不可得,即求,嘗在側。

舜年二十以孝聞。三十而帝堯問可用者,四嶽咸薦虞舜,曰可。於是堯乃以二女妻舜以觀其內,使九男與處以觀其外。舜居媯汭,內行彌謹。堯二女不敢以貴驕事舜親戚,甚有婦道。堯九男皆益篤。舜耕歷山,歷山之人皆讓畔,漁雷澤,雷澤上人皆讓居,陶河濱,河濱器皆不苦窳。一年而所居成聚,二年成邑,三年成都。堯乃賜舜絺衣,與琴,為筑倉廩,予牛羊。瞽叟尚復欲殺之,使舜上涂廩,瞽叟從下縱火焚廩。舜乃以兩笠自捍而下,去,得不死。後瞽叟又使舜穿井,舜穿井為匿空旁出。舜既入深,瞽叟與象共下土實井,舜從匿空出,去。瞽叟、象喜,以舜為已死。象曰,本謀者象。象與其父母分,於是曰,舜妻堯二女,與琴,象取之。牛羊倉廩予父母。象乃止舜宮居,鼓其琴。舜往見之。象鄂不懌,曰,我思舜正郁陶。舜曰,然,爾其庶矣。舜復事瞽叟愛弟彌謹。於是堯乃試舜五典百官,皆治。

昔高陽氏有才子八人,世得其利,謂之八愷。高辛氏有才子八人,世謂之八元。此十六族者,世濟其美,不隕其名。至於堯,堯未能舉。舜舉八愷,使主后土,以揆百事,莫不時序。舉八元,使布五教于四方,父義,母慈,兄友,弟恭,子孝,內平外成。

昔帝鴻氏有不才子,掩義隱賊,好行凶慝,天下謂之渾沌。少暤氏有不才子,毀信惡忠,崇飾惡言,天下謂之窮奇。顓頊氏有不才子,不可教訓,不知話言,天下謂之梼杌。此三族世憂之。至于堯,堯未能去。縉云氏有不才子,貪于飲食,冒于貨賄,天下謂之饕餮。天下惡之,比之三凶。舜賓於四門,乃流四凶族,遷于四裔,以御螭魅,於是四門辟,言毋凶人也。

舜入于大麓,烈風雷雨不迷,堯乃知舜之足授天下。堯老,使舜攝行天子政,巡狩。舜得舉用事二十年,而堯使攝政。攝政八年而堯崩。三年喪畢,讓丹朱,天下歸舜。而禹、皋陶、契、后稷、伯夷、夔、龍、倕、益、彭祖自堯時而皆舉用,未有分職。於是舜乃至於文祖,謀于四嶽,辟四門,明通四方耳目,命十二牧論帝德,行厚德,遠佞人,則蠻夷率服。舜謂四嶽曰,有能奮庸美堯之事者,使居官相事。皆曰,伯禹為司空,可美帝功。舜曰,嗟,然。禹,汝平水土,維是勉哉。禹拜稽首,讓於稷、契與皋陶。舜曰,然,往矣。舜曰,棄,黎民始饑,汝后稷播時百穀。舜曰,契,百姓不親,五品不馴,汝為司徒,而敬敷五教,在寬。舜曰,皋陶,蠻夷猾夏,寇賊姦軌,汝作士,五刑有服,五服三就,五流有度,五度三居,維明能信。舜曰,誰能馴予工。皆曰垂可。於是以垂為共工。舜曰,誰能馴予上下草木鳥獸。皆曰益可。於是以益為朕虞。益拜稽首,讓于諸臣朱虎、熊羆。舜曰,往矣,汝諧。遂以朱虎、熊羆為佐。舜曰,嗟。四嶽,有能典朕三禮。皆曰伯夷可。舜曰,嗟。伯夷,以汝為秩宗,夙夜維敬,直哉維靜絜。伯夷讓夔、龍。舜曰,然。以夔為典樂,教稚子,直而溫,寬而栗,剛而毋虐,簡而毋傲,詩言意,歌長言,聲依永,律和聲,八音能諧,毋相奪倫,神人以和。夔曰,於。予擊石拊石,百獸率舞。舜曰,龍,朕畏忌讒說殄偽,振驚朕眾,命汝為納言,夙夜出入朕命,惟信。舜曰,嗟。女二十有二人,敬哉,惟時相天事。三歲一考功,三考絀陟,遠近眾功咸興。分北三苗。

此二十二人咸成厥功,皋陶為大理,平,民各伏得其實,伯夷主禮,上下咸讓,垂主工師,百工致功,益主虞,山澤辟,棄主稷,百穀時茂,契主司徒,百姓親和,龍主賓客,遠人至,十二牧行而九州莫敢辟違,唯禹之功為大,披九山,通九澤,決九河,定九州,各以其職來貢,不失厥宜。方五千里,至于荒服。南撫交阯、北發,西戎、析枝、渠廋、氐、羌,北山戎、發、息慎,東長、鳥夷,四海之內咸戴帝舜之功。於是禹乃興九招之樂,致異物,鳳皇來翔。天下明德皆自虞帝始。

舜年二十以孝聞,年三十堯舉之,年五十攝行天子事,年五十八堯崩,年六十一代堯踐帝位。踐帝位三十九年,南巡狩,崩於蒼梧之野。葬於江南九疑,是為零陵。舜之踐帝位,載天子旗,往朝父瞽叟,夔夔唯謹,如子道。封弟象為諸侯。舜子商均亦不肖,舜乃豫薦禹於天。十七年而崩。三年喪畢,禹亦乃讓舜子,如舜讓堯子。諸侯歸之,然後禹踐天子位。堯子丹朱,舜子商均,皆有疆土,以奉先祀。服其服,禮樂如之。以客見天子,天子弗臣,示不敢專也。

自黃帝至舜、禹,皆同姓而異其國號,以章明德。故黃帝為有熊,帝顓頊為高陽,帝嚳為高辛,帝堯為陶唐,帝舜為有虞。帝禹為夏后而別氏,姓姒氏。契為商,姓子氏。棄為周,姓姬氏。

太史公曰,學者多稱五帝,尚矣。然尚書獨載堯以來,而百家言黃帝,其文不雅馴,薦紳先生難言之。孔子所傳宰予問五帝德及帝系姓,儒者或不傳。余嘗西至空桐,北過涿鹿,東漸於海,南浮江淮矣,至長老皆各往往稱黃帝、堯、舜之處,風教固殊焉,總之不離古文者近是。予觀春秋、國語,其發明五帝德、帝系姓章矣,顧弟弗深考,其所表見皆不虛。書缺有閒矣,其軼乃時時見於他說。非好學深思,心知其意,固難為淺見寡聞道也。余并論次,擇其言尤雅者,故著為本紀書首。

\end{document}